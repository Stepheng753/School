\documentclass[11pt]{article}
\usepackage[margin = 1in]{geometry}
\usepackage{amsmath}
\usepackage{amssymb}
\usepackage{amsthm}
\usepackage{array}
\usepackage{multicol}
\usepackage{graphicx}
\usepackage{enumitem}
\usepackage{pdfpages}
\usepackage{url}
\usepackage[parfill]{parskip}
\usepackage{listings}
\usepackage{caption}
\usepackage{subcaption}
\usepackage[utf8]{inputenc}
% ------ Codoing Packages ------ %

\usepackage{xcolor}
\definecolor{codegreen}{rgb}{0,0.6,0}
\definecolor{codegray}{rgb}{0.5,0.5,0.5}
\definecolor{codepurple}{rgb}{0.58,0,0.82}
\definecolor{backcolour}{rgb}{0.95,0.95,0.92}
\lstdefinestyle{mystyle}{
	backgroundcolor=\color{backcolour},
	commentstyle=\color{codegreen},
	keywordstyle=\color{magenta},
	numberstyle=\tiny\color{codegray},
	stringstyle=\color{codepurple},
	basicstyle=\ttfamily\footnotesize,
	breakatwhitespace=false,
	breaklines=true,
	captionpos=b,
	keepspaces=true,
	numbers=left,
	numbersep=5pt,
	showspaces=false,
	showstringspaces=false,
	showtabs=false,
	tabsize=2
}
\lstset{style=mystyle}

% ------ Codoing Packages ------ %

\newcommand{\skipline}{\vspace{\baselineskip}}
\newcommand{\spacer}{\noalign{\medskip}}
\newcommand{~}{\sim}
\newcommand{\qrarrow}{\quad \rightarrow \quad}
\newcommand{\qqrarrow}{\qquad \rightarrow \qquad}
\newcommand{\partiald}[2]{\frac{\partial #1}{\partial #2}}
\newenvironment{problem}[1]{\textbf{Problem #1: }}{\newpage}
\newenvironment{alist}{\begin{enumerate}[label=(\alph*)]}{\end{enumerate}}
\newenvironment{rlist}{\begin{enumerate}[label=(\roman*)]}{\end{enumerate}}
\newenvironment{nlist}{\begin{enumerate}[label=(\arabic*)]}{\end{enumerate}}

\begin{document}

	\begin{center}
		\textbf{Homework 1} \\
		\textbf{Math Modeling} \\
		\textbf{Math 636} \\
		\textbf{Stephen Giang RedID: 823184070} \\
		\skipline \skipline
	\end{center}

    \begin{problem}{1}
        A ball is thrown directly upward from the surface of the Earth. Assuming that the maximum height reached is a monomial function of the acceleration due to gravity, the mass of the ball and the initial velocity, use dimensional analysis to approximate the expression for the maximum height reached.
    \end{problem}

    \begin{problem}{2}
        Non-dimensionalize the following equations:
        \begin{alist}
            \item \[\frac{dy}{dt} = ry(1 - \frac{y}{K}) \text{, where $r$ and $K$ are constant}\]
            \item \[\frac{dy}{dt} = sy(a - y)(y - b) \text{, where $s$, $a$, $b$ are constants}\]
        \end{alist}
    \end{problem}

    \begin{problem}{3}
        For a given model equation
        \[\frac{dN(t)}{dt} = r_BN(t)\left[1 - \frac{N(t)}{K_B}\right] - B\frac{N(t)^2}{A^2 + N(t)^2}\]
        perform dimensional analysis to reduce the equation to the form
        \[\frac{du}{dr} = ru\left(1 - \frac{u}{q}\right) - \frac{u^2}{1 + u^2}\]
    \end{problem}

    \begin{problem}{4}
        For the damp pendulum equation
        \[\ddot{\theta} + \alpha \dot{\theta} + \beta \sin \gamma \theta = 0, \quad \theta(0) = 0, \quad \dot{\theta}(0) = 1 \]
        find suitable rescaling for the following cases:
        \begin{alist}
            \item \[\alpha, \gamma = \mathcal{O}, \quad \beta >> 1\]
            \item \[\alpha, \beta = \mathcal{O}, \quad \gamma >> 1\]
            \item \[\alpha ~ \beta\gamma ~ 1/\gamma >> 1\]
        \end{alist}
    \end{problem}

    \begin{problem}{5}
        Consider the following model equation:
        \[\partiald{y}{t} = D\partiald{^2 y}{x^2} + \gamma y^3\]
        \[y(x,0) = 0, \quad 0 < x < \infty\]
        \[y(0,t) = y_0, \quad t > 0, \quad y(\infty, t) = 0, \quad t > 0\]
        \begin{alist}
            \item Write the dimensions of each variable, parameter, and term in the equation.
            \item Non-dimensionalize the equation.
            \item Analyze the non-dimensionalized parameters to find out what parameters to choose to make (relative) small diffusion or (relative) large diffusion.
        \end{alist}
    \end{problem}

    \begin{problem}{6}
        According to the radioactive decay law, the per capita decay rate of the amount $A(t)$ of $C^{14}$ (Carbon 14) is $-\lambda$. Suppose that an archeologist excavates a bone and measures its content for $C^{14}$. If the result is 25\% of the carbon present in bones of a living organism, what can be said about the age of the bone? The half-life of $C^{14}$ is 5730 years
    \end{problem}
\end{document}
