\documentclass[11pt]{article}
\usepackage[margin = 1in]{geometry}
\usepackage{amsmath}
\usepackage{amssymb}
\usepackage{amsthm}
\usepackage{array}
\usepackage{multicol}
\usepackage{graphicx}
\usepackage{enumitem}
\usepackage{pdfpages}
\usepackage{url}
\usepackage[parfill]{parskip}
\usepackage{listings}
\usepackage{caption}
\usepackage{subcaption}
\usepackage[utf8]{inputenc}
% ------ Codoing Packages ------ %

\usepackage{xcolor}
\definecolor{codegreen}{rgb}{0,0.6,0}
\definecolor{codegray}{rgb}{0.5,0.5,0.5}
\definecolor{codepurple}{rgb}{0.58,0,0.82}
\definecolor{backcolour}{rgb}{0.95,0.95,0.92}
\lstdefinestyle{mystyle}{
	backgroundcolor=\color{backcolour},
	commentstyle=\color{codegreen},
	keywordstyle=\color{magenta},
	numberstyle=\tiny\color{codegray},
	stringstyle=\color{codepurple},
	basicstyle=\ttfamily\footnotesize,
	breakatwhitespace=false,
	breaklines=true,
	captionpos=b,
	keepspaces=true,
	numbers=left,
	numbersep=5pt,
	showspaces=false,
	showstringspaces=false,
	showtabs=false,
	tabsize=2
}
\lstset{style=mystyle}

% ------ Codoing Packages ------ %

\newcommand{\skipline}{\vspace{\baselineskip}}
\newcommand{\spacer}{\noalign{\medskip}}
\newcommand{~}{\sim}
\newcommand{\qrarrow}{\quad \rightarrow \quad}
\newcommand{\qqrarrow}{\qquad \rightarrow \qquad}
\newcommand{\partiald}[2]{\frac{\partial #1}{\partial #2}}
\newenvironment{problem}[1]{\textbf{Problem #1: }}{\newpage}
\newenvironment{alist}{\begin{enumerate}[label=(\alph*)]}{\end{enumerate}}
\newenvironment{rlist}{\begin{enumerate}[label=(\roman*)]}{\end{enumerate}}
\newenvironment{nlist}{\begin{enumerate}[label=(\arabic*)]}{\end{enumerate}}

\begin{document}

	\begin{center}
		\textbf{Homework 1} \\
		\textbf{Math Modeling} \\
		\textbf{Math 636} \\
		\textbf{Stephen Giang RedID: 823184070} \\
		\skipline \skipline
	\end{center}

    \begin{problem}{1}
        A ball is thrown directly upward from the surface of the Earth. Assuming that the maximum height reached is a monomial function of the acceleration due to gravity, the mass of the ball and the initial velocity, use dimensional analysis to approximate the expression for the maximum height reached.
        \\ \\
        Let the maximum height, $h_m$, be a function of the acceleration due to gravity, the mass of the ball and the initial velocity such that:
        \[[h_m] = [g^a m^b v_0^c]\]
        \begin{align*}
            [h_m] &= [g^a m^b v_0^c] \\
            L &= \left(\frac{L}{T^2}\right)^a (M^b) \left(\frac{L}{T}\right)^c \\
            L &= \frac{L^a}{T^{2a}} (M^b) \frac{L^c}{T^c} \\
            L &= \frac{L^{a + c}}{T^{2a + c}} (M^b)
        \end{align*}
        where $g$ is the acceleration due to gravity, $m$ is the mass of the ball, $v_0$ is the initial velocity, $L$ represents length, $T$ represents time, $M$ represents mass
        \\ \\
        Looking at the above equation, we get the following system of equations:
        \[
        \begin{cases}
            b = 0 \\
            2a + c = 0 \\
            a + c = 1
        \end{cases}
        \]
        Simple algebra shows the following:
        \begin{align*}
            c &= -2a \\
            a - 2a &= 1 \\
            a &= -1 \\
            c &= 2 \\
            b &= 0
        \end{align*}
        Such that the maximum height reached is a monomial function of the acceleration due to gravity, the mass of the ball and the initial velocity is:
        \[[h_m] = [g^{-1} v_0^2] \qqrarrow h_m(g, v_0, h_0) = \frac{h_0v_0^2}{g}\]
    \end{problem}

    \begin{problem}{2}
        Non-dimensionalize the following equations:
        \begin{alist}
            \item \[\frac{dy}{dt} = ry(1 - \frac{y}{K}) \text{, where $r$ and $K$ are constant}\]
            Let the following be true:
            \[t = [t]t^* \qquad y = [y]y^*\]
            Now we can start to non-dimensionalize the equation:
            \begin{align*}
                \frac{dy}{dt} &= \frac{dy}{dy^*}\frac{dy^*}{dt^*}\frac{dt^*}{dt} \\
                r[y]y^*(1 - \frac{[y]y^*}{K}) &= [y]\frac{dy^*}{dt^*}\frac{1}{[t]} \\
                r[t]y^*(1 - \frac{[y]y^*}{K}) &= \frac{dy^*}{dt^*} \\
                r[t]y^* - \frac{r[t][y](y^*)^2}{K} &= \frac{dy^*}{dt^*}
            \end{align*}
            Now to cancel out the dimension scales, we can set the following to be true:
            \[r = \frac{1}{[t]} \qquad K = [y]\]
            such that we get:
            \[y^* - {y^*}^2 = \frac{dy^*}{dt^*}\]
            \item \[\frac{dy}{dt} = sy(a - y)(y - b) \text{, where $s$, $a$, $b$ are constants}\]
            Let the following be true:
            \[t = [t]t^* \qquad y = [y]y^*\]
            Now we can start to non-dimensionalize the equation:
            \begin{align*}
                \frac{dy}{dt} &= \frac{dy}{dy^*}\frac{dy^*}{dt^*}\frac{dt^*}{dt} \\
                s[y]y^*(a - [y]y^*)([y]y^* - b) &= [y]\frac{dy^*}{dt^*}\frac{1}{[t]} \\
                s[t]y^*(a - [y]y^*)([y]y^* - b) &= \frac{dy^*}{dt^*} \\
                -s[t]y^*([y]y^* - a)([y]y^* - b) &= \frac{dy^*}{dt^*} \\
                -s[t]y^*(([y]y^*)^2 - (a+b)[y]y^* + ab) &= \frac{dy^*}{dt^*} \\
                -s[t][y]^2{y^*}^3 + s[t](a+b)[y]{y^*}^2 -s[t]y^*ab &= \frac{dy^*}{dt^*}
            \end{align*}
            Now to cancel out the dimension scales, we can set the following to be true:
            \[s = \frac{1}{[t][y]^2} \qquad a+b = \frac{1}{s[t][y]} = [y] \qquad ab = \frac{1}{s[t]} = [y]^2\]
            such that:
            \[\frac{dy^*}{dt^*} = -{y^*}^3 + {y^*}^2 -y^*\]
        \end{alist}
    \end{problem}

    \begin{problem}{3}
        For a given model equation
        \[\frac{dN(t)}{dt} = r_BN(t)\left[1 - \frac{N(t)}{K_B}\right] - B\frac{N(t)^2}{A^2 + N(t)^2}\]
        perform dimensional analysis to reduce the equation to the form
        \[\frac{du}{d\tau} = ru\left(1 - \frac{u}{q}\right) - \frac{u^2}{1 + u^2}\]
        First we let the following be true:
        \[N = [N]u \qquad t = [t]\tau\]
        such that we get:
        \begin{align*}
            \frac{dN(t)}{dt} &= \frac{dN}{du}\frac{du}{d\tau}\frac{d\tau}{dt} \\
            r_B[N]u\left[1 - \frac{[N]u}{K_B}\right] - B\frac{[N]^2u^2}{A^2 + [N]^2u^2} &= \frac{[N]}{[t]}\frac{du}{d\tau} \\
            r_B[t]u\left[1 - \frac{[N]u}{K_B}\right] - B\frac{[t]u^2}{A^2 + [N]^2u^2} &= \frac{du}{d\tau} \\
            (r_B[t])u\left[1 - \frac{u}{(K_B / [N])}\right] - \frac{B[t]}{[N]^2}\frac{u^2}{(A/[N])^2 + u^2} &= \frac{du}{d\tau}
        \end{align*}
        Now to cancel out the dimension scales, we can set the following to be true:
        \[r_B = \frac{r}{[t]} \qquad K_B = [N]q \qquad B = \frac{[N]^2}{[t]} \qquad A = [N]\]
        such that we get:
        \[\frac{du}{d\tau} = ru\left(1 - \frac{u}{q}\right) - \frac{u^2}{1 + u^2}\]
    \end{problem}

    \begin{problem}{4}
        For the damp pendulum equation
        \[\ddot{\theta} + \alpha \dot{\theta} + \beta \sin \gamma \theta = 0, \quad \theta(0) = 0, \quad \dot{\theta}(0) = 1 \]
        find suitable rescaling for the following cases:
        \begin{alist}
            \item \[\alpha, \gamma = \mathcal{O}(1), \quad \beta >> 1\]
            Because the damp pendulum equation is already dimensionalized, we can rewrite the equation as:
            \[\frac{d^2\theta}{{dt^*}^2} + \alpha\frac{d\theta}{dt^*} + \beta \sin \gamma \theta  = 0, \qquad \theta(t^* = 0) = 0 \qquad \frac{d\theta(0)}{dt^*} = 1\]
            To rescale, we just set the following:
            \[t^* = [t^*]\tau, \qquad \frac{d\theta}{dt^*} = \frac{d\theta}{d\tau}\frac{d\tau}{dt^*} = \frac{1}{[t^*]}\frac{d\theta}{d\tau}, \qquad \frac{d^2\theta}{{dt^*}^2} = \frac{d}{d\tau}\frac{d\theta}{dt^*}\frac{d\tau}{dt^*} = \frac{d}{d\tau}\left(\frac{1}{[t^*]}\frac{d\theta}{d\tau}\right)\frac{1}{[t^*]} = \frac{1}{[t^*]^2}\frac{d^2\theta}{d\tau^2}\]
            Substituting we get:
            \begin{align*}
                \frac{1}{[t^*]^2}\frac{d^2\theta}{d\tau^2} + \frac{\alpha}{[t^*]}\frac{d\theta}{d\tau} + \beta \sin \gamma \theta &= 0 \\
                \frac{d^2\theta}{d\tau^2} + \alpha[t^*]\frac{d\theta}{d\tau} + [t^*]^2\beta \sin \gamma \theta &= 0
            \end{align*}
            Thus, we get that:
            \[[t^*] = \frac{1}{\sqrt{\beta}}\]
            such that:
            \[\frac{d^2\theta}{d\tau^2} + \eta\frac{d\theta}{d\tau} + \sin \gamma \theta = 0, \qquad \eta = \frac{\alpha}{\sqrt{\beta}}\]
            and
            \[\theta(0) = 0 \qquad \frac{d\theta(0)}{dt^*} = \sqrt{\beta}\frac{d\theta(0)}{d\tau} = 1 \qrarrow \frac{d\theta(0)}{d\tau} = \frac{1}{\sqrt{\beta}}\]
            \item \[\alpha, \beta = \mathcal{O}(1), \quad \gamma >> 1\]
            To rescale, we just set the following:
            \[t^* = [t^*]\tau, \theta^* = [\theta^*]\tilde{\theta}, \qquad \frac{d\theta}{dt^*} = \frac{d\theta}{d\tilde{\theta}}\frac{d\tilde{\theta}}{d\tau}\frac{d\tau}{dt^*} = \frac{[\theta]}{[t^*]}\frac{d\tilde{\theta}}{d\tau}, \qquad \frac{d^2\theta}{{dt^*}^2} = \frac{[\theta]}{[t^*]^2}\frac{d\tilde{\theta}}{d\tau}\]
            Substituting we get:
            \[\frac{[\theta]}{[t^*]^2}\frac{d^2\tilde{\theta}}{d\tau^2} + \frac{\alpha [\theta]}{[t^*]}\frac{d\tilde{\theta}}{d\tau} + \beta \sin \gamma [\theta]\tilde{\theta} = 0\]
            Thus we get that:
            \[[\theta] = \frac{1}{\gamma} \qquad [t^*] = \frac{1}{\sqrt{\gamma}}\]
            such that:
            \[\frac{d^2\tilde{\theta}}{d\tau^2} + \eta \frac{d\tilde{\theta}}{d\tau} + \beta \sin \tilde{\theta} = 0, \qquad \eta = \frac{\alpha}{\gamma}\]
            and
            \[\theta(0) = 0 \qquad \frac{d\theta(0)}{dt^*} = \frac{1}{\sqrt{\gamma}}\frac{d\theta(0)}{d\tau} = 1 \qrarrow \frac{d\theta(0)}{d\tau} = \sqrt{\gamma}\]
            \item \[\alpha ~ \beta\gamma ~ 1/\gamma >> 1\]
            We can let the following to be true:
            \[t^* = [t^*]\tau, \theta^* = [\theta^*]\tilde{\theta}, \qquad \frac{d\theta}{dt^*} = \frac{d\theta}{d\tilde{\theta}}\frac{d\tilde{\theta}}{d\tau}\frac{d\tau}{dt^*} = \frac{[\theta]}{[t^*]}\frac{d\tilde{\theta}}{d\tau}, \qquad \frac{d^2\theta}{{dt^*}^2} = \frac{[\theta]}{[t^*]^2}\frac{d\tilde{\theta}}{d\tau}\]
            Such that, as seen from the previous part:
            \begin{align*}
                \frac{[\theta]}{[t^*]^2}\frac{d^2\tilde{\theta}}{d\tau^2} + \frac{\alpha [\theta]}{[t^*]}\frac{d\tilde{\theta}}{d\tau} + \beta \sin \gamma [\theta]\tilde{\theta} &= 0 \\
                \frac{1}{[t^*]^2}\frac{d^2\tilde{\theta}}{d\tau^2} + \frac{\alpha}{[t^*]}\frac{d\tilde{\theta}}{d\tau} + \frac{\beta}{[\theta]} \sin \gamma [\theta]\tilde{\theta} &= 0
            \end{align*}
            From here, we can set the following:
            \[[\theta] = \frac{\beta}{\gamma} \qquad [t^*] = \alpha\]
            such that we get:
            \[\frac{1}{\alpha^2}\frac{d^2\tilde{\theta}}{d\tau^2} + \frac{d\tilde{\theta}}{d\tau} + \gamma \sin \beta\tilde{\theta} = 0\]
            and
            \[\theta(0) = 0 \qquad \frac{d\theta(0)}{dt^*} = \frac{\beta}{\alpha \gamma}\frac{d\theta(0)}{d\tau} = 1 \qrarrow \frac{d\theta(0)}{d\tau} = \frac{\alpha \gamma}{\beta}\]
        \end{alist}
    \end{problem}

    \begin{problem}{5}
        Consider the following model equation:
        \[\partiald{y}{t} = D\partiald{^2 y}{x^2} + \gamma y^3\]
        \[y(x,0) = 0, \quad 0 < x < \infty\]
        \[y(0,t) = y_0, \quad t > 0, \quad y(\infty, t) = 0, \quad t > 0\]
        \begin{alist}
            \item Write the dimensions of each variable, parameter, and term in the equation.
            \\ \\
            Notice the following dimensions:
            \[y = [y] \qquad t = [t] \qquad \frac{dy}{dt} = \frac{[y]}{[t]} \qquad \partiald{^2 y}{x^2} = \frac{[y]}{[x]^2} \]
            Substituting the following, we get:
            \[\frac{[y]}{[t]} = D\frac{[y]}{[x]^2} + \gamma [y]^3\]
            Thus, we can see the following parameters are:
            \[D = \frac{[x]^2}{[t]} \qquad \gamma = \frac{1}{[t][y]^2}\]
            \item Non-dimensionalize the equation.
            \\ \\
            Taking what we did from earlier, we can add to it and say:
            \[y = [y]y^* \qquad t = [t]t^* \qquad \frac{dy}{dt} = \frac{[y]}{[t]}\frac{dy^*}{dt^*} \qquad \partiald{^2 y}{x^2} = \frac{[y]}{[x]^2}\frac{dy^*}{d{x^*}^2}\]
            \begin{align*}
                \frac{[y]}{[t]}\frac{dy^*}{dt^*} &= \frac{[x]^2}{[t]}\frac{[y]}{[x]^2}\frac{dy^*}{d{x^*}^2} + \frac{1}{[t][y]^2}[y]^3{y^*}^3 \\
                \frac{[y]}{[t]}\frac{dy^*}{dt^*} &= \frac{[y]}{[t]}\frac{dy^*}{d{x^*}^2} + \frac{[y]}{[t]}{y^*}^3 \\
                \frac{dy^*}{dt^*} &= \frac{dy^*}{d{x^*}^2} + {y^*}^3
            \end{align*}

            \item Analyze the non-dimensionalized parameters to find out what parameters to choose to make (relative) small diffusion or (relative) large diffusion.
            \\ \\
            \[\lim_{t \rightarrow \pm\infty} D = \lim_{t \rightarrow \pm\infty} \gamma = \lim_{y \rightarrow \pm\infty} \gamma = 0 \qquad \lim_{x \rightarrow \pm \infty} = \infty\]
        \end{alist}
    \end{problem}

    \begin{problem}{6}
        According to the radioactive decay law, the per capita decay rate of the amount $A(t)$ of $C^{14}$ (Carbon 14) is $-\lambda$. Suppose that an archeologist excavates a bone and measures its content for $C^{14}$. If the result is 25\% of the carbon present in bones of a living organism, what can be said about the age of the bone? The half-life of $C^{14}$ is 5730 years.
        \\ \\
        The following equation can be modeled:
        \begin{align*}
            \frac{dA}{dt} &= -\lambda A \\
            \int \frac{1}{A}\,dA &= \int -\lambda \,dt \\
            \ln |A| &= e^{-\lambda t + C} \\
            A = A_0e^{-\lambda t}
        \end{align*}
        To find $\lambda$, we use the half-life, $t_{1/2} = 5730$:
        \begin{align*}
            \frac{A_0}{2} &= A_0e^{-\lambda t_{1/2}} \\
            \frac{1}{2} &= e^{-\lambda t_{1/2}} \\
            \ln \frac{1}{2} &= -\lambda t_{1/2} \\
            \frac{-1}{t_{1/2}} \ln \frac{1}{2} &= \lambda
        \end{align*}
        such that we get:
        \[\lambda = \frac{-1}{t_{1/2}} \ln \frac{1}{2} = \frac{-1}{5730} \ln \frac{1}{2} = 0.000120968094339\]
        To find the age of the bone, we get:
        \[t = \frac{-1}{\lambda}\ln \frac{0.25A_0}{A_0} = 11460\]
    \end{problem}
\end{document}
