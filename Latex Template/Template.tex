\documentclass[11pt]{article}
\usepackage[margin = 1in]{geometry}
\usepackage{amsmath}
\usepackage{amssymb}
\usepackage{amsthm}
\usepackage{graphicx}
\usepackage{enumitem}
\usepackage{url}
\usepackage[parfill]{parskip}
\usepackage{listings}
\usepackage{caption}
\usepackage{subcaption}
\usepackage[utf8]{inputenc}
% ------ Codoing Packages ------ %

\usepackage{xcolor}
\definecolor{codegreen}{rgb}{0,0.6,0}
\definecolor{codegray}{rgb}{0.5,0.5,0.5}
\definecolor{codepurple}{rgb}{0.58,0,0.82}
\definecolor{backcolour}{rgb}{0.95,0.95,0.92}
\lstdefinestyle{mystyle}{
	backgroundcolor=\color{backcolour},   
	commentstyle=\color{codegreen},
	keywordstyle=\color{magenta},
	numberstyle=\tiny\color{codegray},
	stringstyle=\color{codepurple},
	basicstyle=\ttfamily\footnotesize,
	breakatwhitespace=false,         
	breaklines=true,                 
	captionpos=b,                    
	keepspaces=true,                 
	numbers=left,                    
	numbersep=5pt,                  
	showspaces=false,                
	showstringspaces=false,
	showtabs=false,                  
	tabsize=2
}
\lstset{style=mystyle}

% ------ Codoing Packages ------ %

\newcommand{\skipline}{\vspace{\baselineskip}}
\newcommand{\spacer}{\noalign{\medskip}}
\newcommand{~}{\sim}
\newcommand{\qrarrow}{\quad \rightarrow \quad}
\newcommand{\qqrarrow}{\qquad \rightarrow \qquad}
\newcommand{\partiald}[2]{\frac{\partial #1}{\partial #2}}
\newenvironment{problem}[1]{\textbf{Problem #1: }}{\newpage}


\begin{document}
	
	\begin{center}
		\textbf{Title} \\
		\textbf{Subject} \\
		\textbf{Class Number} \\
		\textbf{First Last Name RedID: 0123456789} \\
		\skipline \skipline
	\end{center}

	\begin{problem}{1}
		This begin problem command, I created.  It simply creates the word 'Problem' followed by the number specified and then bolds it.  It also makes a new page so the next problem will have its own page.
		\\ \\ - "Double backslash" means new line \\ \\
		This is inline made mode - $x^2 + 2x + 1$, whereas this is display math mode:
		\[x^2 + 2x + 1\]
		This is how we align equations:
		\begin{align*}
			x^2 + 2x + 1 &= (x + 1)^2 \\
			&= 0 \\
			x &= -1 
		\end{align*}
		This is piecewise functions:
		\[f(x) = \begin{cases}
			x^2 + 2x + 1 & x < 0 \\
			(x + 1)^2 & x \geq 0
		\end{cases}\]
		\\ \\
		This is how to include images and graphs: \\
		\begin{figure}[h!]
			\centering
			\includegraphics[width=.5\textwidth]{lorem.png} 
		\end{figure}
		\newpage
		This is how we created a bullet list:
		\begin{enumerate}[label = (\alph*)]
			\item Item 1
			\item Item 2
		\end{enumerate}
		This is how to create better looking parentheses:
		\[(\frac{x^2}{2}) = \left(\frac{x^2}{2}\right)\]
		This is how to make like spaces, and big spaces, and arrows:
		\[x \quad y \qquad z \rightarrow xyz\]
		This is how to make a big space followed by an arrow then another big space:
		\[x*y*z \qqrarrow xyz\]
		This is limits, derivatives, integrals, series:
		\[\lim\limits_{x \rightarrow 0} \frac{1}{x} \qqrarrow \frac{dy}{dx} = \frac{1}{x} \qqrarrow \int_{0}^{L}f(x) = 0 \qqrarrow e^x = 1 + x + \frac{x^2}{2} + \frac{x^3}{6} + \cdots + \frac{x^n}{n!} \] 
		\\ \\
		\textbf{This is bolding} \\ \\
		\textbf{\boldmath This is bolding with inline math $x^2 + 2x + 1$} \\ \\
		This is bolding with display math:
		\[\boldsymbol{x^2 + 2x + 1}\]
		\\ \\
		This is how to include code:
		\begin{lstlisting}[language=Python]
def main(last_num):
	for i in range(1, last_num + 1):
		if i % 15 == 0:
			print(i, ': FizzBuzz')
		elif i % 3 == 0:
			print(i, ': Fizz')
		elif i % 5 == 0:
			print(i, ': Buzz')
		else:
			print(i)
			
def other(last_num):
	for i in range(1, last_num + 1):
		print(i, end=": ")
		if i % 3 == 0:
			print('Fizz', end="")
		if i % 5 == 0:
			print('Buzz', end="")    
		print()
			
if __name__ == '__main__':
other(1000)
		\end{lstlisting}
	\end{problem}


\end{document}
