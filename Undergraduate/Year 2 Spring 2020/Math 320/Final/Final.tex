\documentclass[12pt]{article}
\usepackage[margin = 1in]{geometry}
\usepackage{amsmath}
\usepackage{amssymb}
\usepackage{amsthm}
\usepackage{graphicx}
\usepackage{subfig}
\usepackage{enumitem}

\begin{document}
	
	\begin{center}
		\textbf{Final} \\
		\textbf{Abstract Algebra} \\
		\textbf{Math 320} \\
		\textbf{Stephen Giang} \\
	\end{center}

\noindent \textbf{Problem 1: }Let $T$ be the set of real $2 \times 2$ matrices with determinant $1$:
	$$
	T = \left\{ 
	\begin{pmatrix}
		a & b \\
		c & d
	\end{pmatrix}: a,b,c,d \in \mathbb{R}; ad-bc = 1 \right\}.
	$$
Prove or disprove: under the usual matrix addition and multiplication, T is a ring:
\\ \\
\begin{proof}[Disproof]
	Notice the following:
		$$
		a = \begin{pmatrix}
			1 & 0 \\ 0 & 1
		\end{pmatrix},
		b = \begin{pmatrix}
			0 & 1 \\ -1 & 0
		\end{pmatrix} \in T, \qquad
		\begin{pmatrix}
		1 & 0 \\ 0 & 1
		\end{pmatrix} + 
		\begin{pmatrix}
		0 & 1 \\ -1 & 0
		\end{pmatrix} = 
		\begin{pmatrix}
		1 & 1 \\ -1 & 1
		\end{pmatrix} \not \in T
		$$
	Notice that $a$ and $b$ are both in $T$ as their determinant is equal to 1 and they contain real elements.  However, their sum is not in $T$ as their sum's determinant is equal to 2.  This shows that $T$ is not closed under addition. \textbf{Thus T is NOT a ring}\\
\end{proof}
\newpage

\noindent \textbf{Problem 2: }Determine if the following rings are fields. If the ring is a field, explain why.
If the ring is not a field, explain why, and provide a zero divisor of the ring:
	\begin{enumerate}[label = (\alph*)]
		\item $\mathbb{Q}[x]/(x^8 - 169x^6 + 52x^3 - 104x + 39)$
		\\ \\
		Notice that we can prove that $f(x) = x^8 - 169x^6 + 52x^3 - 104x + 39$ is irreducible in $\mathbb{Q}[x]$ using Eisenstein's Criterion.  Notice the following:
			\begin{enumerate}[label = (\alph*)]
				\item 13 is Prime 
				\item $13| -169, \qquad 13|52, \qquad 13|-104, \qquad 13|39$
				\item $13 \nmid 1, \qquad 13^2 \nmid 39$
			\end{enumerate}
		Thus by Eisenstein's Criterion, $f(x)$ is irreducible. Now by Theorem 5.10, because we proved that $f(x)$ is irreducible, we know that \textbf{(a) is a field.}
		\item $\mathbb{Q}[x]/(7x^3 + 25x + 51)$
		\\ \\
		Notice that we can prove that $f(x) = 7x^3 + 25x + 51$ is irreducible in $\mathbb{Q}[x]$ by using Theorem 4.25.  We can choose a prime number, 2, which doesn't divide 7. Now if we prove that $f(x)$ is irreducible in $\mathbb{Z}_2[x]$, then it will be irreducible in $\mathbb{Q}[x]$.
		\\ \\
		We can rewrite $f(x) \in \mathbb{Z}_2[x]$ as $x^3 + x + 1$. 
		Because the degree of $f(x)$ is 3 and its leading coefficient is 1, its factors are polynomials of degree 2 with its roots. And notice that the only numbers in $\mathbb{Z}_2$ are 0 and 1:
			\begin{align*}
				f(0) &= [1] \not = [0] \\
				f(1) &= [1] \not = [0]
			\end{align*}
		Because $f(x)$ is irreducible in $\mathbb{Z}_2[x]$, it is also irreducible in $\mathbb{Q}[x]$. Now by Theorem 5.10, because we proved that $f(x)$ is irreducible, we know that \textbf{(b) is a field.}  
		\item $\mathbb{Z}_5[x]/(x^3 - 3)$
		\\ \\
		Notice that $x^3 - 3$ has a root in $\mathbb{Z}_5$. If we let $f(x) = x^3 - 3$, then $f(2) = 8 - 3 = [5] = [0]$.  Thus proving that $x - 2$ is a zero divisor, showing that \textbf{(c) is not a field.}
		\item $\mathbb{Z}_7[x]/(x^7 + 1)$
		\\ \\
		Notice that $x^7 + 1$ has a root in $\mathbb{Z}_7$. If we let $f(x) = x^7 + 1$, then $f(6) = 279936 + 1 = [279937] = [39991][7] = [0]$.  Thus proving that $x - 6$ is a zero divisor, showing that \textbf{(d) is not a field.}
	\end{enumerate}

\newpage

\noindent \textbf{Problem 3: }Explain why $x^2$ does not divide $x - 5$ in $\mathbb{Q}[x]$.
\\ \\
By definition, "A polynomial with coefficients in $\mathbb{R}$ is an expression of the form:
	$$
	a_0 + a_1x + ... + a_nx^n
	$$
where n is a non-negative integer and $a_i \in \mathbb{R}$". So let $x^2$ divide $x-5$ such that
	$$
	x - 5 = x^2a(x)
	$$
where $a(x)$ is a polynomial in $\mathbb{Q}[x]$.  By Theorem 4.2, we can see the following:
	$$
	1 = deg[x-5] = deg[x^2a(x)] = deg[x^2] + deg[a(x)] = 2 + deg[a(x)]
	$$
So we can see that $deg[a(x)]$ has to be $-1$, which would contradict the definition of polynomial as it must have non-negative exponents.  Thus \textbf{\boldmath $x^2$ does not divide $x - 5$ in $\mathbb{Q}[x]$}


\newpage

\noindent\textbf{Problem 4: }: Let $F$ be a field and suppose $f(x) \in F[x]$ is a polynomial of degree 5. Prove
that if $f(x)$ has no factors in $F[x]$ of degree 3, then $f(x)$ has no factors in $F[x]$ of degree 2.
\\ \\
	\begin{proof}
		Let $f(x) \in F[x]$ be a polynomial of degree 5. Let $f(x)$ have no factors in $F[x]$ of degree 3.
		\\ \\
		By Theorem 4.2, the degrees of each pair of factors of $f(x)$ have to have a sum of 5. Also notice that by definition of polynomials, all of the exponents of the factors have to be non-negative integers.
		\\ \\
		Because for all factors with degree 2, they have to be paired with a factor of 3 because $2+3=5$. Finally, because $f(x)$ has no factors in $F[x]$ of degree 3, then we can conclude that \textbf{\boldmath $f(x)$ has no factors in $F[x]$ of degree 2} \\
	\end{proof}

\newpage

\noindent \textbf{Problem 5: }Let A be the following ring:
	$$
	A = \left\{
	\begin{pmatrix}
		a & b \\ b & a
	\end{pmatrix}: a,b \in \mathbb{Q}\right\}
	$$
\begin{enumerate}[label = (\alph*)]
	\item Prove that $\mathbb{Q}[x]/(x^2 - 1) \cong A$
	\begin{proof}
		Let $f:\mathbb{Q}[x]/(x^2 - 1) \rightarrow A$ such that $f(ax + b) = \begin{pmatrix}
			a & b \\ b & a
		\end{pmatrix}$.
		\\ \\
		Notice the following:
			$$
			f(ax + b) =
			\begin{pmatrix}
				a & b \\ b & a
			\end{pmatrix} = 
			\begin{pmatrix}
				c & d \\ d & c
			\end{pmatrix} = 
			f(cx + d)
			$$
		Thus the only way for this to be true is if $a=c$ and $b = d$, thus proving that f is injective.
		\\ \\
		Notice that for all matrices in A, 
		$
		\begin{pmatrix}
			a & b \\ b & a
		\end{pmatrix}
		$, it can be written as a function, $f(ax + b)$, thus showing that f is surjective.
		\\ \\
		Notice the following homomorphic properties:
			\begin{align*}
				f(ax + b) + f(cx + d) &= 
				\begin{pmatrix}
				a & b \\ b & a
				\end{pmatrix} +
				\begin{pmatrix}
				c & d \\ d & c
				\end{pmatrix} = 
				\begin{pmatrix}
				a+c & b+d \\ b+d & a+c
				\end{pmatrix} \\
				&= f((a+c)x + (b+d)) = f((ax+b) + (cx + d)) \\
				f(ax+b)f(cx + d) &= 
				\begin{pmatrix}
				a & b \\ b & a
				\end{pmatrix}
				\begin{pmatrix}
				c & d \\ d & c
				\end{pmatrix} = 
				\begin{pmatrix}
				ac + bd & ad + bc \\ ad + bc & ac + bd
				\end{pmatrix} \\
				&= f((ac+bd)x + (ad + bc)) = f(adx^2 + acx + bdx + bc) \\
				&= f((ax + b)(dx + c)) = f((ax + b)(cx + d))
			\end{align*}
		Because we are in $\mathbb{Q}[x]/(x^2 - 1)$, the following is true: $[x^2] = [1]$ and $[x] = 1$.  Thus allowing us to say $ad = adx^2$ and $dx + c = cx + d$
		\\ \\
		\textbf{\boldmath Because $f:\mathbb{Q}[x]/(x^2 - 1) \rightarrow A$ is bijective and satisfies the homomorphic properties, $\mathbb{Q}[x]/(x^2 - 1) \cong A$} \\
	\end{proof}
	\newpage
	\item Let $R, S$ be rings, and $f : R \rightarrow S$ be a ring homomorphism. Show that if $a, b \in R $ and
	$a \cdot b = 0_R$, then $f(a) \cdot f(b) = 0_S$.
	\\ \\
	Because $f$ is a ring homomorphism, then the following is true for all $a,b \in R$
		$$
		f(a \cdot b) = f(a) \cdot f(b)
		$$ 
	Let $a \cdot b = 0_R$, such that the following is true:
		$$
		f(a \cdot b) = f(0_R) = f(a) \cdot f(b) 
		$$
	By Theorem 3.10, we know that for all homomorphisms, $f(0_R) = 0_S$, such that we get the following:
		\boldmath
		$$
		f(a \cdot b) = f(0_R) = 0_S = f(a) \cdot f(b)
		$$
		\unboldmath 
	\\ 
	\item Use part (b) to find two zero divisors in A.
	\\ \\
	Notice that in $\mathbb{Q}[x]/(x^2 - 1)$, $x-1$ and $x+1$ are both zero divisors.  So we have $(x-1)(x+1) = [0]$.  If we take the same $f(ax + b) = \begin{pmatrix}
		a & b \\ b & a
	\end{pmatrix}$ from part (a), notice the following:
		\begin{align*}
			f((x-1) \cdot (x+1)) &= f(x-1) \cdot f(x+1) = f([0]) \\
			&=\begin{pmatrix}
				1 & -1 \\ -1 & 1
			\end{pmatrix}
			\begin{pmatrix}
				1 & 1 \\ 1 & 1
			\end{pmatrix}
			=\begin{pmatrix}
				0 & 0 \\ 0 & 0 
			\end{pmatrix} = 0_A
		\end{align*}
	\textbf{\boldmath Thus the zero divisors of A are 
	$\begin{pmatrix}
		1 & -1 \\ -1 & 1
	\end{pmatrix}, \text{ and }
	\begin{pmatrix}
		1 & 1 \\ 1 & 1
	\end{pmatrix}$ }
	
\end{enumerate}

\newpage

\noindent \textbf{Problem 6: }Consider the set of lower-triangular matrices with integer coefficients:
	$$
	R = \left\{
	\begin{pmatrix}
		a & 0 \\ b & c
	\end{pmatrix}: a,b,c \in \mathbb{Z}\right\}
	$$
You may assume that $R$ is a ring (with the usual matrix addition and multiplication).
	\begin{enumerate}[label = (\alph*)]
		\item Prove that the following subset $I$ of $R$ is an ideal in $R$:
			$$
			I = \left\{
			\begin{pmatrix}
				0 & 0 \\ b & 0
 			\end{pmatrix}: b \in \mathbb{Z}\right\}
			$$
		Notice that $0_R \in I$ by letting 
		$
		X = \begin{pmatrix}
			0 & 0 \\ x & 0
		\end{pmatrix} \in I
		$ with x = 0:
			$$
			0_R =
			\begin{pmatrix}
				0 & 0 \\ 0 & 0
			\end{pmatrix} = X
			\in I
			$$
		Notice that $I$ is closed under subtraction, and let 
		$
		X = \begin{pmatrix}
		0 & 0 \\ x & 0
		\end{pmatrix}, 
		Y = \begin{pmatrix}
		0 & 0 \\ y & 0
		\end{pmatrix} \in I
		$: 
		$$
		X - Y = 
		\begin{pmatrix}
			0 & 0 \\ x & 0
 		\end{pmatrix} - 
 		\begin{pmatrix}
 			0 & 0 \\ y & 0
 		\end{pmatrix} = 
 		\begin{pmatrix}
 		0 & 0 \\ (x-y) & 0
 		\end{pmatrix} \in I
		$$
		Notice that $I$ satisfies the absorption property, and let 
		$
		X = \begin{pmatrix}
			0 & 0 \\ x & 0
		\end{pmatrix} \in I
		$ and \\
		$
		Z = \begin{pmatrix}
			z_1 & 0 \\ z_2 & z_3
		\end{pmatrix} \in R
		$
		\begin{align*}
		XZ &= 
		\begin{pmatrix}
		0 & 0 \\ x & 0
		\end{pmatrix}
		\begin{pmatrix}
		z_1 & 0 \\ z_2 & z_3
		\end{pmatrix} = 
		\begin{pmatrix}
			0 & 0 \\ xz_1 & 0
		\end{pmatrix} \in I \\
		ZX &= 
		\begin{pmatrix}
		z_1 & 0 \\ z_2 & z_3
		\end{pmatrix}
		\begin{pmatrix}
		0 & 0 \\ x & 0
		\end{pmatrix}= 
		\begin{pmatrix}
		0 & 0 \\ xz_3 & 0
		\end{pmatrix} \in I
		\end{align*}
		\textbf{\boldmath Thus $I$ of $R$ is an ideal of $R$}
		\item Show that
			$$
			\begin{pmatrix}
				1 & 0 \\ -4 & 6
			\end{pmatrix}
			\equiv 
			\begin{pmatrix}
				1 & 0 \\ 16 & 6 
			\end{pmatrix} \mod I
			$$
			\\
		By definition of $a \equiv b \mod I$, the following needs to be true: $a - b \in I$
			$$
			\begin{pmatrix}
			1 & 0 \\ -4 & 6
			\end{pmatrix} - 
			\begin{pmatrix}
			1 & 0 \\ 16 & 6 
			\end{pmatrix} = 
			\begin{pmatrix}
			0 & 0 \\ -20 & 0
			\end{pmatrix} \in I
			$$
		Thus the following is true:
			\boldmath
			$$
			\begin{pmatrix}
			1 & 0 \\ -4 & 6
			\end{pmatrix}
			\equiv 
			\begin{pmatrix}
			1 & 0 \\ 16 & 6 
			\end{pmatrix} \mod I
			$$
			\unboldmath
	\end{enumerate}

\newpage

\noindent \textbf{Problem 7: }Prove that the following polynomials are irreducible in $\mathbb{Q}[x]$:
	\begin{enumerate}[label = (\alph*)]
		\item $$k(x) = x^{17} + 16$$
		\\
		Notice that we know $k(x)$ is irreducible if $k(x+1)$ is irreducible. So if we were to simplify $k(x+1)$, we would get:
			$$
			k(x+1) = (x+1)^{17} + 16 = \sum_{n=0}^{17} \left[\binom{17}{n} x^{17-n}\right] + 16 = x^{17} + \sum_{n=1}^{16}\left[\binom{17}{n} x^{17-n}\right] + 17
			$$
		\\ \\
		Now by Eisenstein's Criterion, we can choose a prime number 17.  We can see for all integers $n \in [1,16]$, 17 divides $\binom{17}{n}$ because they are all multiples of 17.  We can also see that 17 does not divide the leading coefficient, 1, and $17^2$ does not divide the constant term, 17
		\\ \\
		\textbf{\boldmath Because $k(x+1)$ is irreducible, $k(x)$ is irreducible} 
		\\ 
		\item $$f(x) = \frac{(x+7)^5 - 12005x - 16807}{x^2}$$
		\\
		Notice we can simplify $f(x)$ to:
			\begin{align*}
				f(x) &= \frac{x^5+35\,x^4+490\,x^3+3430\,x^2+12005\,x+16807 - 12005x - 16807}{x^2} \\
				&= x^3+35\,x^2+490x+3430
			\end{align*}
		Notice that we can prove that $f(x) = x^3+35\,x^2+490x+3430$ is irreducible in $\mathbb{Q}[x]$ by using Theorem 4.25.  We can choose a prime number, 3, which doesn't divide 1. Now if we prove that $f(x)$ is irreducible in $\mathbb{Z}_3[x]$, then it will be irreducible in $\mathbb{Q}[x]$.
		\\ \\
		We can rewrite $f(x) \in \mathbb{Z}_3[x]$ as follows:
			$$
			f(x) = x^3 + 2x^2 + x + 1
			$$
		Because the degree of $f(x)$ is 3 and its leading coefficient is 1, its factors are polynomials of degree 2 with its roots. And notice that the only numbers in $\mathbb{Z}_3$ are 0,1,2:
		\begin{align*}
		f(0) &= [1] \not = [0] \\
		f(1) &= [2] \not = [0] \\
		f(2) &= [1] \not = [0]
		\end{align*}
		\textbf{\boldmath Because $f(x)$ is irreducible in $\mathbb{Z}_3[x]$, it is also irreducible in $\mathbb{Q}[x]$}
		\newpage
		\item $$g(x) = \frac{x^{19} - 524288}{x-2}$$
		\\ \\
		Notice that we know $g(x)$ is irreducible if $g(x+2)$ is irreducible.  So if we were to simplify $g(x+2)$, we would get: 
			$$
			g(x+2) = \frac{(x+2)^{19} - 524288}{(x+2) - 2} = \frac{\sum_{n=0}^{19}\left[  \binom{19}{n} 2^{n}x^{19-n}\right]  - 2^{19}}{x} = x^{18} + \sum_{n=1}^{17}\left[  \binom{19}{n} 2^{n}x^{18-n}\right]  + 19(2^{18})
			$$
		\\ \\
		Now by Eisenstein's Criterion, we can choose a prime number 19.  We can see for all integers $n \in [1,17]$, 19 divides $\binom{19}{n}$ because they are all multiples of 19.  We can also see that 19 does not divide the leading coefficient, 1, and $19^2$ does not divide the constant term, 4980736
		\\ \\
		\textbf{\boldmath Because $g(x+2)$ is irreducible, $g(x)$ is irreducible in $\mathbb{Q}[x]$}
	\end{enumerate}

\newpage

\noindent \textbf{Problem 8: }Find two rings of cardinality 125 of the form $\mathbb{Z}_p[x]/(q(x))$ that are not
isomorphic to each other, and prove that they are not isomorphic.
\\ \\
Notice the following rings:
	$$
	A = \mathbb{Z}_5[x]/(x^3 + 2x^2 + 2x + 2), \qquad B = \mathbb{Z}_5[x]/(x^3 - 3)
	$$
\\
Notice that $A$ and $B$'s congruence classes can be written in the form of $ax^2 + bx + c$, with the coefficients being in $\mathbb{Z}_5$.  This mean there exists 125 distinct congruence classes in each ring. This shows that both rings have cardinality 125.
\\ \\
Notice that $q(x) = x^3 + 2x^2 + 2x + 2$ is irreducible in $\mathbb{Z}_5[x]$. Because the degree of $q(x)$ is 3 and its leading coefficient is 1, its factors are polynomials of degree 2 with its roots. And notice that the only numbers in $\mathbb{Z}_5$ are 0,1,2,3,4:
	\begin{align*}
		q(0) &= [2] \not = [0] \\
		q(1) &= [2] \not = [0] \\
		q(2) &= [2] \not = [0] \\
		q(3) &= [3] \not = [0] \\
		q(4) &= [1] \not = [0] 
	\end{align*}
Because $q(x)$ is irreducible in $\mathbb{Z}_5[x]$, by Theorem 5.10, $A$ is a field.
\\ \\
Notice that $x^3 - 3$ has a root in $\mathbb{Z}_5$.  Notice if we let $p(x) = x^3 - 3$, then $p(2) = 8 - 3 = [5] = [0]$.  Thus proving that $x - 2$ is a zero divisor, showing that $B$ is not a field.
\\ \\
Propeties are preserved by isomorphisms.  This means that isormorphisms will map zero divisors of the first ring to the zero divisors of the second ring.  If one ring, $A$, doesn't have any zero divisors while the other ring, $B$ does, then they can't be isomorphic to each other. In short, both rings have to be fields or both not fields, and $A$ is a field, while $B$ is not.
\\ \\
\textbf{\boldmath Thus $A$ and $B$ are two rings of cardinality 125 of the form $\mathbb{Z}_p[x]/(q(x))$ that are not isomorphic to each other}

\newpage

\noindent \textbf{Problem EC: }Prove that the following polynomial is irreducible in $\mathbb{Q}[x]$:
	$$
	h(x) = \frac{x^7 - 109375x +468750}{x^2 - 10x + 25}
	$$
	Notice that we know $h(x)$ is irreducible if $h(x+5)$ is irreducible, so if we were to to simplify $h(x+5)$, we would get
		\begin{align*}
			h(x+5) &= \frac{(x+5)^7 - 109375(x+5) + 468750}{((x+5) - 5)^2}  \\
			&= \frac{\sum_{n=0}^{7} \left[ \binom{7}{n} 5^nx^{7-n}\right]  - 109375x - 78125}{x^2} \\
			&= \frac{\sum_{n=0}^{5}\left[ \binom{7}{n} 5^nx^{7-n}\right] }{x^2} \\
			&= \sum_{n=0}^{5} \binom{7}{n} 5^nx^{5-n} \\
			&= x^5 + \sum_{n=1}^{4} \left[ \binom{7}{n} 5^nx^{5-n}\right]  + 21(3125)
		\end{align*}
	Now by Eisenstein's Criterion, we can choose a prime number 7.  We can see for all integers $n \in [1,4]$, 7 divides $\binom{7}{n}$ because they are all multiples of 7.  We can also see that 7 does not divide the leading coefficient, 1, and $7^2$ does not divide the constant term, 65625
	\\ \\
	\textbf{\boldmath Because $h(x+5)$ is irreducible, $h(x)$ is irreducible in $\mathbb{Q}[x]$}
\newpage





















\end{document}
