\documentclass[12pt]{article}
\usepackage[margin = 1in]{geometry}
\usepackage{amsmath}
\usepackage{amssymb}
\usepackage{amsthm}
\usepackage{graphicx}
\usepackage{subfig}
\usepackage{enumitem}

\begin{document}
	
	\begin{center}
		\textbf{Classwork 7} \\
		\textbf{Abstract Algebra} \\
		\textbf{Math 320} \\
		\textbf{Stephen Giang, Austin Kovalcheck} \\
		\textbf{Ana Estrada, Soleil Leuregans} \\
	\end{center}

\noindent \textbf{Problem 1: }Determine if the following polynomials are irreducible. Justify your answers.

	\begin{enumerate}[label = (\alph*)]
		\item $x^6 + 30x^5 - 15x^3 + 6x - 120$
		\\ \\
		By Eisenstein's Criterion, $3$ is a prime that does not divide the coefficient to $x^6$, but does divide all other coefficients.  As well, $3^2 = 9$, does not divide the constant term $-120$, thus (a) is irreducible.
		\\ \\
		\item $7x^3 - 36x^2 -x +11$
		\\ \\
		We can test reducibility with the Rational Root Test, and by having no roots, proves it is irreducible by corollary 4.19.
		\\ \\
		$\frac{r}{s} = \pm 1, \pm 11, \pm 7,\pm \frac{11}{7}$.  Let $f(x) =7x^3 - 36x^2 -x +11$
		\begin{align*}
			f(1) = -19 && f(-1) = -31 \\
			f(11) = 4961 && f(-11) = -13651 \\
			f(7) = 641 && f(-7) = -4147 \\
			f\left(\frac{11}{7}\right) = \frac{-2563}{49}&& f\left(\frac{11}{7}\right) = \frac{-5071}{49}\\
		\end{align*}
		Because there does not exist $f(\frac{r}{s}) = 0$, (b) is irreducible.
		\\ \\
		\item $x^4 + 14x^3 + 9x^2 - x + 3$
		\\ \\
		We can test reducibility with the Rational Root Test, and by having roots, proves it is reducible by corollary 4.19.
		\\ \\
		Notice if we let $f(x) = x^4 + 14x^3 + 9x^2 - x + 3$, $f(-1) = 0$, thus -1 is a root. Thus we can write $f(x) = (x + 1)g(x)$, for some $g(x)$.  Thus (c) is reducible. 
	\end{enumerate}

\newpage 

\noindent \textbf{Problem 2: }Use Eisenstein’s Criterion to show that $f(x) = x^4+ 1$ is irreducible in $\mathbb{Q}[x]$ by replacing $x$ with $x+1$. You may assume the following fact: If $g(x) = f(x+1)$ is irreducible,
then so is $f(x)$.
\\ \\
Notice $g(x) = f(x + 1)$:

	$$
	(x+1)^4 + 1 = x^4 + 4x^3 + 6x^2 + 4x + 2
	$$
By Eisenstein's Criterion, $2$ is a prime that does not divide the coefficient to $x^4$, but does divide all other coefficients.  As well, $2^2 = 4$, does not divide the constant term $2$, thus $f(x)$ is irreducible.


\vspace{\baselineskip}
\vspace{\baselineskip}
\vspace{\baselineskip}
\vspace{\baselineskip}
\vspace{\baselineskip}
\vspace{\baselineskip}

\noindent \textbf{Problem 3: } Prove that $x^3+nx+2$ is irreducible over $\mathbb{Q}[x]$ for all integers $n \not= 1, -3, -5$.
\\ \\
We can test reducibility with the Rational Root Test, and by having no roots, proves it is irreducible by corollary 4.19.
\\ \\
	$\frac{r}{s} = \pm 1, \pm 2$.  Let $f(x) = x^3+nx+2$ 
	\begin{align*}
		f(1) = n + 3 && f(-1) = -(n - 1) \\
		f(2) = 2(n + 5)&& f(-2) = -2(n+3)
	\end{align*}
Because for $n \not= 1, -3, -5$, we can see that there does not exists an $x$, such that $f(x) = 0$, thus (3) is irreducible.

\end{document}
