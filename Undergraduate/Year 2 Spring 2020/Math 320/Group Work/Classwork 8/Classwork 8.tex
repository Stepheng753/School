\documentclass[12pt]{article}
\usepackage[margin = 1in]{geometry}
\usepackage{amsmath}
\usepackage{amssymb}
\usepackage{amsthm}

\begin{document}
	
	\begin{center}
		\textbf{Classwork 8} \\
		\textbf{Abstract Algebra} \\
		\textbf{Math 320} \\
		\textbf{Stephen Giang, Emily Boyd} \\
		\textbf{Mihwa Yu, Bridget Houdyshel} \\ 
	\end{center}

\vspace{\baselineskip}

\noindent \textbf{Problem 1: }Consider the polynomial $f(x) = x^4 + x^3 + x^2 + 1$ in $\mathbb{Z}_3[x]$. Prove that $f(x)$ is irreducible. If you only check for roots, you will receive 0 points, and if you use the Rational Roots Test or Eisenstein, you will receive 0 points
\\ \\
Notice: The possible roots of $f(x)$ are $0,1,2$:
	\begin{align*}
		f(0) &= 1 \\
		f(1) &= 1 \\
		f(2) &= 2
	\end{align*}
Thus, there are no roots, meaning the factors must be both degree $2$, that is:
	\begin{align*}
		x^4 + x^3 + x^2 + 1 &= (ax^2 + bx + c)(dx^2+ex+f) \\
		&= x^4 + (a+c)x^3 + (ac + b + d)x^2 + (bc + ad)x + bd
	\end{align*}
Thus we get 
	\begin{align}
		a + c &= 1 \\
		ac + b + d &= 1 \\
		bc + ad &= 1 \\
		bd &= 1
	\end{align}
Notice because of (4), we have $b=d=1$ or $b=d=2$. 
	\begin{align*}
		b = d = 1 && b = d = 2 \\
		ac + 2 = 1 && ac + 1 = 1 \\
		ac = 2 && ac = 0 \\
	\end{align*}
If $ac = 2$, then either $a=1$,$c=2$ or $a=2,c=1$.  Either way $a+c = 0 \not = 1$.
\\ \\
If $ac = 0$, then if we let $a = 0$, then $c=1$. But this contradicts (3), because this is the case when $b=2$, and $(2)(1) = 2 \not = 1$.  The same is true, when we let $c = 0$. 

\newpage 

\noindent \textbf{Problem 2: }Write out the multiplication table for $K = \mathbb{Z}_2[x]/(x^2 + x+ 1)$. Use your table to explain why K is a field 
\\ \\
	\begin{table}[h]
		\centering
		\begin{tabular}{|c|c|c|c|c|}
			\hline
			 & $0$ & $1$ & $x$ & $x+1$ \\
			 \hline
			 $0$ & $0$ & $0$ & $0$ & $0$ \\ 
			 \hline
			 $1$ & $0$ & $1$ & $x$ & $x+1$\\
			 \hline
			 $x$ & $0$ & $x$ & $x+1$ & $1$ \\
			 \hline
			 $x+1$ & $0$ & $x+1$ & $1$ & $x$ \\
			 \hline
		\end{tabular}
	\end{table}
\\ \\
A field is defined as a commutative ring with identity such that all its nonzero elements have a multiplicative inverse.
\\ \\
Well we can see that $K$ is a commutative ring as the table is symmetric meaning that $a*b = b*c \in K$.
\\ \\
Also we can see that each row, and each column contain 1, meaning each nonzero element has a multiplicative inverse.

\newpage 

\noindent \textbf{Problem 3: }Every element of $R = \mathbb{Q}[x]/(x^2 - 2)$ is a congruence class and can be written in the form $[ax + b]$. Determine the rules for multiplication of congruence classes in $R$. That if, if $[ax+b][cx+d] = [rx+s]$, solve for $r$ and $s$ in terms of $a,b,c,d$
\\ \\
Notice: $[x^2]=[2]$
	\begin{align*}
		(ax+b)(cx+d) &= acx^2 + adx + bcx + bd \\
		&= 2ac + adx + bcx + bd \\
		&= (ad + bc)x + (2ac + bd)
	\end{align*}
So we get:
	\begin{align*}
		r = ad + bc && s = 2ac + bd
	\end{align*}

\newpage 

\noindent \textbf{Problem 4: }Let $f(x), g(x) \in F[x],$ not both zero. Prove that if there exist $u(x), v(x) \in
F[x]$ such that $f(x)u(x) + g(x)v(x) = 1_F$, then $f(x)$ and $g(x)$ are relatively prime.
\\ \\
Let $d(x) = gcd(f(x),g(x))$ and $f(x)u(x) + g(x)v(x) = 1_F$. 
\\ \\
So notice now for $a(x),b(x) \in F[x]$:
	\begin{align*}
		f(x) &= a(x)d(x) \\
		g(x) &= b(x)d(x) \\
	\end{align*}
And now we can replace these values into our beginning equation:
	\begin{align*}
		f(x)u(x) + g(x)v(x) &= 1_F \\
		d(x)(a(x)u(x) + b(x)v(x)) &= 1_F
	\end{align*}
Thus we get $d(x)|1_F$. Because $d(x)|1_F$, degree of $d(x)$ must be 0, such that $d(x) = 1_F$





\end{document}
