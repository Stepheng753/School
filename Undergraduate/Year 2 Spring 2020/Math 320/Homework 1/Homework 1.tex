\documentclass[12pt]{article}
\usepackage[margin = 1in]{geometry}
\usepackage{amsmath}
\usepackage{amssymb}
\usepackage{cancel}

\begin{document}
	
	\begin{center}
		\textbf{Homework 1} \\
		\textbf{Abstract Algebra} \\
		\textbf{Math 320} \\
		\textbf{Stephen Giang} \\
	\end{center}

\noindent \textbf{Section 1.1 Problem 9: }
Prove that the cube of any integer $a$ has to be exactly one of these forms: $9k$ or $9k + 1$ or $9k + 8$ for some integer k.
\\\\
\textbf{Solution: } \\
Let $a \in \mathbb{Z}$.  By the Division Algorithm, Let $a = 3q + r, \quad 0  \leq r < 3$. \\Remark: Integers are closed under Multiplication and Addition. \\\\
Case ($r = 0$): \\
	\begin{align}
		a &= 3q \\
		a^3 &= (3q)^3 \\
		&= 9(3q^3) \\
		&= \boldsymbol{9k}, \quad \text{$k \in \mathbb{Z}$}\\
	\end{align}
Case ($r = 1$): \\
	\begin{align}
		a &= 3q + 1 \\
		a^3 &= (3q+1)^3 \\
		&= (3q)^3 + 3 (3q)^2 + 3 (3q) + 1 \\
		&= 9(3 q^3 + 3 (q^2) + q) + 1 \\
		&= \boldsymbol{9k + 1}, \quad \text{$k \in \mathbb{Z}$}\\
	\end{align}	
Case ($r = 2$): \\
	\begin{align}
		a &= 3q + 2 \\
		a^3 &= (3q+2)^3 \\
		&= (3q)^3 + 3 (3q)^2(2) + 3 (3q)(2^2) + 2^3 \\
		&= 9(3q^3 + 3 (q^2)(2) + (q)(2^2)) + 2^3 \\
		&= \boldsymbol{9k + 8}, \quad \text{$k \in \mathbb{Z}$}\\
	\end{align}	
	
Thus: $\forall a \in \mathbb{Z}, a^3$ can be written in the form: $9k$ or $9k + 1$ or $9k + 8$ for some integer k.
\newpage
\noindent \textbf{Section 1.1 Problem 10: } 
Let $n$ be a positive integer. Prove that $a$ and $c$ leave the same remainder when divided by n if and only if $a - c = nk$ for some integer k
\\\\
\textbf{Solution: ($\rightarrow$)} \\
Let $a-c = nk,$ \quad with $a,c,k,q_1,q_2,r_1,r_2 \in \mathbb{Z}$ \\
	\begin{align}
		a &= nq_1 + r_1 \\
		c &= nq_2 + r_2 \\
		a - c &= (nq_1 + r_1) - (nq_2 + r_2) \\
		&= n(q_1 - q_2) + (r_1 - r_2) \\
		&= nk
	\end{align}
Remark: 0 $\leq (r_1 - r_2) < n$.
To have $a - c = nk$, $(r_1 - r_2) = cn$ for some $c \in \mathbb{Z}$, or have $(r_1 - r_2)$ be a multiple of n.  The only multiple of n on the interval [0,$n$) is 0. So... \\
	\begin{align}
		r_1 - r_2 &= 0 \\
		\boldsymbol{r_1} &= \boldsymbol{r_2}
	\end{align} 
Thus $a$, $c$ share the same remainder when divided by n 
\\\\	
\textbf{Solution: ($\leftarrow$)} \\
Let $a,c,k,q_1,q_2,r \in \mathbb{Z}$ \\
Let $a$ and $c$ leave the same remainder when divided by $n$\\
	\begin{align}
		a &= nq_1 + r \\
		c &= nq_2 + r \\
		a - c &= nq_1 + r - (nq_2 + r) \\
		&= nq_1 + r - nq_2 - r \\
		&= nq_1 - nq_2 \\
		&= n(q_1 - q_2) \\
		&= nk \\
		\boldsymbol{a - c} &= \boldsymbol{nk} 
	\end{align}
\newpage
\noindent \textbf{Section 1.2 Problem 4: } \\
a) If $a|b$ and $a|c$, prove that $a|(b + c)$. \\
b) If $a|b$ and $a|c$, prove that $a|(br + ct)$ for any r, t $\in \mathbb{Z}$. 
\\\\
\textbf{Solution (a): } \\
Let $a|b$ and $a|c$ for some $a,b,c,x_1,x_2 \in \mathbb{Z}$ \\
	\begin{align}
		b &= ax_1 \\
		c &= ax_2 \\
		b + c &= ax_1 + ax_2 \\
		&= a(x_1 + x_2) \\
		&= ax_3 
		\\\\
		\boldsymbol{a} &| \boldsymbol{(b + c)} 
	\end{align}
\\\\
\textbf{Solution (b): } \\
Let $a|b$ and $a|c$ for some $a,b,c,x_1,x_2,r,t \in \mathbb{Z}$ \\
	\begin{align}
		b &= ax_1 \\
		c &= ax_2 \\
		br + ct &= ax_1r + ax_2t \\
		&= a(x_1r + x_2t) \\
		&= ax_3 \\
		\boldsymbol{a} &| \boldsymbol{(br + ct)} 
	\end{align}
\newpage
\noindent \textbf{Section 1.2 Problem 5: } 
If $a$ and $b$ are nonzero integers such that $a|b$ and $b|a$, prove that $a = \pm b$. 
\\\\
\textbf{Solution: } \\
Let $a, b \in \mathbb{Z}$ \textbackslash \textbraceleft 0\textbraceright \space such that $a|b$ and $b|a$, \quad  $a,b,x_1,x_2 \in \mathbb{Z}$
	\begin{align}
		b &= ax_1 \\
		a &= bx_2 \\
		b &= (bx_2)x_1 \\
		1 &= x_2 x_1 \\
		x_2 &= 1 \text{ and } x_1 = 1 \\
		&\text{ or}  \\
		x_2 &= -1 \text{ and } x_1 = -1 \\
		a &= b \text{ or} -b \\
		\boldsymbol{a} &= \boldsymbol{\pm b}
	\end{align}

\end{document}
