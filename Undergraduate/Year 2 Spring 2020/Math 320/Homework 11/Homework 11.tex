\documentclass[12pt]{article}
\usepackage[margin = 1in]{geometry}
\usepackage{amsmath}
\usepackage{amssymb}
\usepackage{amsthm}
\usepackage{graphicx}
\usepackage{subfig}
\usepackage{enumitem}

\begin{document}
	
	\begin{center}
		\textbf{Homework 11} \\
		\textbf{Abstract Algebra} \\
		\textbf{Math 320} \\
		\textbf{Stephen Giang} \\
	\end{center}

\noindent \textbf{Problem 5.1.3: }How many distinct congruence classes are there modulo $x^3 + x + 1$ in $\mathbb{Z}_2[x]$.
\\ \\
By Corollary 5.5, all congruence classes can be written in the form $ax^2 + bx + c$.
	\begin{align*}
		x^2 + x + 1 && x^2 + x && x^2 + 1 \\
		x^2			&& x + 1   && x 	  \\
		1			&& 0
	\end{align*} 
There are 8 distinct congruence classes.

\newpage 

\noindent \textbf{Problem 5.1.4: }Show that, under congruence modulo $x^3 + 2x + 1$ in $\mathbb{Z}_3[x]$, there are exactly 27 distinct congruence classes. 
\\ \\
All distinct congruence classes can be written in the form $ax^2 + bx + c$.  Because $a,b,c \in \mathbb{Z}_3$, each coefficient can only be either $0,1,$ or $2$. And because there are a total of 3 terms, that can only be one of 3 choices, the amount of combinations is $3^3 = 27$.

\newpage 

\noindent \textbf{Problem 5.1.5: }Show that there are infinitely many distinct congruence classes modulo $x^2 - 2$ in $\mathbb{Q}[x]$. Describe them. 
\\ \\
All distinct congruence classes can be written in the form $ax + b$.  Because $a,b \in \mathbb{Q}$, there are infinitely many choices that $a$ and $b$ can be, meaning there will be infinitely many distinct congruence classes

\newpage 

\noindent \textbf{Problem 5.1.10: }Prove or disprove: If $p(x)$ is irreducible in $F[x]$ and $f(x)g(x) \equiv 0_F(\text{mod } p(x))$,then $f(x) \equiv 0_F(\text{mod } p(x))$ or $g(x) \equiv 0_F(\text{mod } p(x))$.
\\ \\ 
Notice the following:
	\begin{proof}[Solution 5.1.10]
		$f(x)g(x) \equiv 0_F(\text{mod } p(x))$ $\rightarrow$ $p(x) | f(x)g(x)$
		\\ \\
		Because $p(x)$ is irreducible, the only factors are its associates and nonzero constants.
		\\ \\
		If $(p(x),f(x)) = c$, then that makes $f(x) = cq(x)$, with $p(x) \nmid q(x)$. So then we have $p(x) | cq(x)g(x)$.  Because $p(x) \nmid q(x)$, then $p(x) | cg(x)$.  Meaning that $g(x) \equiv 0_F(\text{mod } p(x))$.
		\\ \\
		If $(p(x),f(x)) = cp(x)$, then that makes $f(x) = cp(x)q(x)$. That means that $f(x) \equiv 0_F(\text{mod } p(x))$.
		\\
	\end{proof}

\newpage 

\noindent \textbf{Problem 5.1.12: }If $f(x)$ is relatively prime to $p(x)$, prove that there is a polynomial $g(x) \in F[x]$ such that $f(x)g(x) \equiv 1_F (\text{mod } p(x))$. 
\\ \\
Because $f(x)$ is relatively prime to $p(x)$, notice that for some $g(x),u(x) \in F[x]$:
	\begin{align*}
		f(x)g(x) + p(x)u(x) &= 1_F \\
		f(x)g(x) - 1_F &= p(x)(-u(x))
	\end{align*}
The result is the same as $f(x)g(x) \equiv 1_F (\text{mod } p(x))$ by definition of polynomial modulo.

\newpage 

\noindent \textbf{Problem 5.2.1: }Write out the addition and multiplication tables for the congruence class ring $F[x]/p(x)$. In each case, is $F[x]/p(x)$ a field?
\\ \\
	 $F = \mathbb{Z}_2$, $p(x) = x^3 + x + 1$
			\\ \\
			\scriptsize
			\begin{tabular}{c | c c c c c c c c}
				$+$ & $0$ & $1$ & $x$ & $x + 1$ & $x^2$ & $x^2 + x$ & $x^2 + 1$ & $x^2 + x + 1$ \\
				\hline 
				$0$ & $0$ & $1$ & $x$ & $x + 1$ & $x^2$ & $x^2 + x$ & $x^2 + 1$ & $x^2 + x + 1$ \\
				$1$ & $1$ & $0$ & $x + 1$ & $x$ & $x^2+1$ & $x^2 + x+1$ & $x^2$ & $x^2 + x $ \\
				$x$ & $x$ & $x + 1$ & $0$ & $1$ & $x^2 + x$ & $x^2$ & $x^2 +x +1$ & $x^2 + 1$ \\
				$x + 1$ & $x+1$ & $x$ & $1$ & $0$ & $x^2 + x+1$ & $x^2+1$ & $x^2 +x$ & $x^2 $ \\
				$x^2$ & $x^2$ & $x^2 + 1$ & $x^2 + x$ & $x^2 + x + 1$ & $0$ & $x$ & $1$ & $x + 1$ \\
				$x^2+x$ & $x^2 +x$ & $x^2 +x+1$ & $x^2$ & $x^2 + 1$ & $x$ & $0$ & $x+1$ & $1$ \\
				$x^2 + 1$ & $x^2 +1$ & $x^2$ & $x^2 + x + 1$ & $x^2 + x $ & $1$ & $x + 1$ & $0$ & $x $ \\
				$x^2 +x +1$ & $x^2 +x +1$ & $x^2 +x$ & $x^2+1$ & $x^2$ & $x+1$ & $1$ & $x$ & $0$ \\
			\end{tabular}
			\\ \\ \\
			\begin{tabular}{c | c c c c c c c c}
				$\times$ & $0$ & $1$ & $x$ & $x + 1$ & $x^2$ & $x^2 + x$ & $x^2 + 1$ & $x^2 + x + 1$ \\
				\hline 
				$0$ & $0$ & $0$ & $0$ & $0$ & $0$ & $0$ & $0$ & $0$ \\
				$1$ & $0$ & $1$ & $x$ & $x + 1$ & $x^2$ & $x^2 + x$ & $x^2 + 1$ & $x^2 + x + 1$ \\
				$x$ & $0$ & $x$ & $x^2$ & $x^2 + x$ & $x^3$ & $x^3 + x^2$ & $x^3 + x$ & $x^3 + x^2 + x$ \\
				$x + 1$ & $0$ & $x+1$ & $x^2 + x$ & $x^2 + 1$ & $x^3 + x^2$ & $x^3 + x$ & $x^3 + x^2 + x + 1$ & $x^3 + 1$\\
				$x^2$ & $0$ & $x^2$ & $x^3$ & $x^3 + x^2$ & $x^4$ & $x^4 + x^3$ & $x^4 + x^2$ & $x^4 + x^3 + x^2$\\
				$x^2+x$ & $0$ & $x^2 +x$ & $x^3 + x^2$ & $x^3 + x$ & $x^4 + x^3$ & $x^4 + x^2$ & $x^4 + x^3 + x^2 + x$ & $x^4 + x$\\ 
				$x^2 + 1$ & $0$ & $x^2+1$ & $x^3 + x$ & $x^3 + x^2 + x + 1$ & $x^4 + x^2$ & $x^4 + x^3 + x^2 + x$ & $x^4 + 1$ & $x^4 + x^3 + x + 1$\\
				$x^2 +x +1$ & $0$ & $x^2 +x+1$ & $x^3 + x^2 + x$ & $x^3 + 1$ & $x^4 + x^3 + x^2$ & $x^4 + 1$ & $x^4 + x^3 + x + 1$ & $x^4 + x^2 + 1$\\
			\end{tabular}
		\normalsize
		\\ \\ \\
		This is not a field because not every nonzero element has a multiplicative inverse.

\newpage 

\noindent \textbf{Problem 5.2.7: }Determine the rules for addition and multiplication of congruence classes. (In other words, if the product $[ax + b][cx + d]$ is the class $[rx + s]$, describe how to find $r$ and $s$ from $a, b, c, d,$ and similarly for addition.) 
\\ \\
$\mathbb{Q}[x]/(x^2 - 3)$.
\\ \\
Notice: $[x^2] = [3]$, for multiplication:
	\begin{align*}
		(ax+b)(cx+d) &= acx^2 + adx + bcx + bd \\
		&= 3ac + adx + bcx + bd \\
		&= (ad + bc)x + (3ac + bd)
	\end{align*}
So we get 
	$$
	r = ad + bc \qquad \qquad s = 3ac + bd
	$$
Notice for addition:
	\begin{align*}
		(ax + b) + (cx + d) = (a + c)x + (b+d)
	\end{align*}
So we get 
	$$
	r = a + c \qquad \qquad s = b + d
	$$
 
\newpage 

\noindent \textbf{Problem 5.2.8: }Determine the rules for addition and multiplication of congruence classes. (In other words, if the product $[ax + b][cx + d]$ is the class $[rx + s]$, describe how to find $r$ and $s$ from $a, b, c, d,$ and similarly for addition.) 
\\ \\
$\mathbb{Q}[x]/(x^2)$.
\\ \\
Notice: $[x^2] = [0]$, for multiplication:
\begin{align*}
(ax+b)(cx+d) &= acx^2 + adx + bcx + bd \\
&= adx + bcx + bd \\
&= (ad + bc)x + (bd)
\end{align*}
So we get 
$$
r = ad + bc \qquad \qquad s =  bd
$$
\\We can also see that because $[x^2] = [0]$, then $[x] = [0]$. So we can write the product as just $bd$, where:
$$
r = 0 \qquad \qquad s = bd
$$
\\Notice for addition:
\begin{align*}
(ax + b) + (cx + d) = (a + c)x + (b+d)
\end{align*}
So we get 
$$
r = a + c \qquad \qquad s = b + d
$$
\\We can also see that because $[x^2] = [0]$, then $[x] = [0]$. So we can write the sum as just $b+d$, where:
$$
r = 0 \qquad \qquad s = b+d
$$


























\end{document}
