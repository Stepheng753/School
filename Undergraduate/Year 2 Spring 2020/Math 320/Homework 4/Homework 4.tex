\documentclass[12pt]{article}
\usepackage[margin = 1in]{geometry}
\usepackage{amsmath}
\usepackage{amssymb}
\usepackage{amsthm}
\usepackage{graphicx}
\usepackage{subfig}
\usepackage{cancel}

\begin{document}
	
	\begin{center}
		\textbf{Homework 4} \\
		\textbf{Abstract Algebra} \\
		\textbf{Math 320} \\
		\textbf{Stephen Giang} \\
	\end{center}

\noindent \textbf{Section 2.1 Problem 14 (a): } Prove or disprove: If $ab \equiv 0$ (mod n), then $a \equiv 0$ (mod n) or $b \equiv 0$ (mod n). 
	\begin{proof}[Solution 14a] 
		\begin{align*}
			&\text{Disprove :} \\
			&\text{Let } a = 4, b = 3 \\
			&(4)(3) \equiv  0 \text{ (mod 12)} \\
			&\text{But } 4 \not \equiv 0 \text{ (mod 12) and } 3 \not \equiv 0 \text{ (mod 12)}
		\end{align*}
	\end{proof}
\vspace{\baselineskip}
\noindent \textbf{Section 2.1 Problem 14 (b): } Do part (a) when n is prime
	\begin{proof}[Solution 14b]
		\quad Let $ab \equiv 0$ (mod n) and $a,b,n,q \in \mathbb{Z}$
		\begin{align*}
			&n | (ab- 0) \\
			&n | ab		\\
			&\text{By Theorem 1.5: } \textbf{ \boldmath$a \equiv 0 $ (mod n) or $ b \equiv 0 $ (mod n) } 
		\end{align*}
	\end{proof}
\vspace{\baselineskip}
\noindent \textbf{Section 2.1 Problem 15: } If $(a, n) = 1$, prove that there is an integer b such that $ab \equiv 1$ (mod n).
	\begin{proof}[Solution 15] 
		Let $(a,n) = 1$
		\begin{align*}
			ab + n(v) &= 1 \\
			ab - 1 &= n(-v) \\
			n &| (ab - 1) \\
			\textbf{ab } &\textbf{\boldmath $ \equiv n \text{ (mod n) }$}
		\end{align*}
	\end{proof} 
\newpage
\noindent \textbf{Section 2.1 Problem 20 (a): } Prove or disprove: If $a^2 \equiv b^2$ (mod n), then $a \equiv b$ (mod n) or
$a \equiv -b$ (mod n). 
	\begin{proof}[Solution 20a]
		\begin{align*}
			&5^2 \equiv 1^2 \text{ (mod 24)} \\
			&Disprove: \\
			&5 \not \equiv 1 \text{ (mod 24)} \\
			&5 \not \equiv -1 \text{ (mod 24)} 
		\end{align*}
	\end{proof}
\noindent \textbf{Section 2.1 Problem 20 (b): } Do part (a) when n is prime.
	\begin{proof}[Solution 20b]
		Let n be prime and $a^2 \equiv b^2$ (mod n)
		\begin{align*}
			&n | (a^2 - b^2) \\
			&n | (a+b)(a-b) \\
			&\text{By Thm 1.5: } n|(a- (-b)) \text{ or } n|(a-b) \\
			&\text{Thus } a \equiv b \text{ (mod n) or } a \equiv -b \text{ (mod n)}
		\end{align*}
	\end{proof}  
\newpage
\noindent \textbf{Section 2.1 Problem 21 (a): }  Show that $10^n \equiv 1$ (mod 9) for every positive n.
	\begin{proof}[Solution 21a]
		let $n \in \mathbb{Z}^+$
		\begin{align*}
		10^n - 1 &= (10 - 1)(10^0 + 10^1 + 10^2 + ... + 10^{n-1}) \\
		&= 9(10^0 + 10^1 + 10^2 + ... + 10^{n-1}) \\
		&9 | (10^n - 1) \\
		&\textbf{\boldmath $10^n \equiv 1$ (mod 9)}
		\end{align*} 
	\end{proof}
\vspace{\baselineskip}
\noindent \textbf{Section 2.1 Problem 21 (b): } Prove that every positive integer is congruent to the sum of its digits (mod
9) [for example, 38 $\equiv$ 11 (mod 9)].
	\begin{proof}[Solution 21b]
		Notice: $\forall n \in \mathbb{Z}$, n = $10^0 a_0 + 10^1 a_1 + 10^2 a_2 + ... + 10^n a_n$, where $a_i$ are the digits of n with $ i \in \mathbb{Z}_{\geq 0}$.
		\begin{align*}
			n &= (1)a_0 + (9 + 1)a_1 + (99 + 1)a_2 +...+(999...99 + 1)a_n \\
			&= 9a_1 + 99a_2 + ... + 999...99a_n + (a_0 + a_1 + a_2 + ... + a_n) \\
			n &- (a_0 + a_1 + a_2 + ... + a_n) = 9(a_1 + 11a_2 + ... + 111...11a_n) \\
			9 &| (n- (a_0 + a_1 + a_2 + ... + a_n)) \\
			n &\equiv (a_0 + a_1 + a_2 + ... + a_n) \text{ (mod 9)} \\
			&\text{So any integer n is congruent to the sum of it digits (mod 9)}
		\end{align*}
	\end{proof}

\end{document}
