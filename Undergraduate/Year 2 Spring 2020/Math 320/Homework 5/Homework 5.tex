\documentclass[12pt]{article}
\usepackage[margin = 1in]{geometry}
\usepackage{amsmath}
\usepackage{amssymb}
\usepackage{amsthm}
\usepackage{graphicx}
\usepackage{subfig}
\usepackage{cancel}

\begin{document}
	
	\begin{center}
		\textbf{Homework 5} \\
		\textbf{Abstract Algebra} \\
		\textbf{Math 320} \\
		\textbf{Stephen Giang} \\
	\end{center}

\noindent \textbf{Section 2.2 Problem 14: } Solve the following equations: 
$$\text{a) }x^2 + x = [0] \text{ in } \mathbb{Z}_5$$
$$\text{b) }x^2 + x = [0] \text{ in } \mathbb{Z}_6$$
If p is prime, prove that solutions of the equation below are [0] and [p-1]
$$\text{c) }x^2 + x = [0] \text{ in } \mathbb{Z}_p$$

\noindent \textbf{Solution (a): x = 0, 4 }
\\
$$
\begin{tabular}{c c}
	x & $x^2$ + x \\
	\hline 
	0 & [0][0] + [0] = [0] \\
	1 & [1][1] + [1] = [3] \\
	2 & [2][2] + [2] = [1] \\
	3 & [3][3] + [3] = [2] \\
	4 & [4][4] + [4] = [0] \\
\end{tabular}
$$
\noindent \textbf{Solution (b): x = 0, 2, 3, 5 }
\\
$$
\begin{tabular}{c c}
	x & $x^2$ + x \\
	\hline 
	0 & [0][0] + [0] = [0] \\
	1 & [1][1] + [1] = [3] \\
	2 & [2][2] + [2] = [0] \\
	3 & [3][3] + [3] = [0] \\
	4 & [4][4] + [4] = [2] \\
	5 & [5][5] + [5] = [0] \\
\end{tabular}
$$
\noindent \textbf{Solution (c) } Let p be prime.
\\ 
\begin{align*}
	&x^2 + x = [0] \in \mathbb{Z}_p \\
	&[x(x+1)] = [0] \\
	&[(0)(0+1)] = [0] \\
	&[(p-1)(p-1+p)] = [p] = [0]
\end{align*}
For $x^2 + x = [0] \in \mathbb{Z}_n $, the solutions will be 0, n-1, and \{$q \in \mathbb{Z}^+ | q(q+1) = kn$  $\forall k \in \mathbb{Z}$\}.  Because prime numbers don't have any factors, except itself and 1, the only solutions would be 0 and n-1.

\newpage
\noindent \textbf{Section 2.3 Problem 1: } Find all Units in 
\begin{align*}
	\text{a) } \mathbb{Z}_7 &&& \text{b) } \mathbb{Z}_8 & \text{c) } \mathbb{Z}_9 &&& \text{d) } \mathbb{Z}_{10}
\end{align*}
\noindent \textbf{Solution }
\begin{align*}
	\text{a) } 1,2,3,4,5,6 &&& \text{b) } 1,3,5,7 & \text{c) } 1,4,5,7,8 &&& \text{d) } 1,3,7,9
\end{align*}
\\\\
\noindent \textbf{Section 2.3 Problem 2: } Find all Zero Divisors in 
\begin{align*}
\text{a) } \mathbb{Z}_7 &&& \text{b) } \mathbb{Z}_8 & \text{c) } \mathbb{Z}_9 &&& \text{d) } \mathbb{Z}_{10}
\end{align*}
\noindent \textbf{Solution }
\begin{align*}
\text{a) } \text{none} &&& \text{b) } 2,4 & \text{c) } 3 &&& \text{d) } 2,5
\end{align*}
\\\\
\noindent \textbf{Section 2.3 Problem 6: } If n is composite, prove that there is at least one \\zero divisor in $\mathbb{Z}_n$.
\\\\
\noindent \textbf{Solution }
\\
Let n be composite, so let $0 < q < n$, be a factor of n, so that $n = qr$. So 
$$[0] = [n] = [qr] = [q][r]$$
Thus their exists some $0 < q < n$, that when multiplied with an nonzero number, r, qr = 0.



	
\end{document}
