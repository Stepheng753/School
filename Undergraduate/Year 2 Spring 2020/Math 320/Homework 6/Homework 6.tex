\documentclass[12pt]{article}
\usepackage[margin = 1in]{geometry}
\usepackage{amsmath}
\usepackage{amssymb}
\usepackage{amsthm}
\usepackage{graphicx}
\usepackage{subfig}
\usepackage{cancel}

\begin{document}
	
	\begin{center}
		\textbf{Homework 6} \\
		\textbf{Abstract Algebra} \\
		\textbf{Math 320} \\
		\textbf{Stephen Giang} \\
	\end{center}

\noindent \textbf{Section 2.3 Problem 11: } If $a,b \in \mathbb{Z}_n$ and a is a unit, then the equation ax = b has a unique solution in $\mathbb{Z}_n$ [Note: You must find a solution for the equation and show that this solution is the only one.] 

\begin{proof}[Solution 2.3.11]
	Let $a,b \in \mathbb{Z}_n$ and a be a unit.
	\begin{align*}
		\text{Consider: } ax &= b \\
		a^{-1} a x &= a^{-1} b \\
		1 x &= a^{-1} b \\
		x &= a^{-1} b
	\end{align*}
	Thus there exists a solution such that ax = b.
	\\
	Let $ax_1 = b$ and $ax_2 = b$
	\begin{align*}
		ax_1 &= b \\
		ax_2 &= b \\
		ax_1 &= ax_2 \\
		a^{-1}ax_1 &= a^{-1}ax_2 \\
		x_1 &= x_2
	\end{align*}
	Thus there exists a unique solution such that ax = b. \\
\end{proof}

\noindent \textbf{Section 2.3 Problem 12: } Let $a,b,n$ be integers with $n > 1$ and let $d$ = ($a, n$). If the equation [a]x = [b] has a solution in $\mathbb{Z}_n$ prove that $d|b$. (Hint: If x = [r] is a solution, then [ar] = [b] so that ar - b = kn for some integer k.] 

\begin{proof}[Solution 2.3.12]
	Let $a,b,n \in \mathbb{Z}$ with $n > 1$ and Let $d$ = ($a, n$).Suppose $ax = b$ has a solution,$x = r$, in $\mathbb{Z}_n$.  
	\begin{align*}
		[a][r] &= [b] \\
		[ar] - [b] &= [0] \\
		ar - b &= kn \qquad \text{ for some k } \in \mathbb{Z}  \\
		b &= ar - kn \\
		b &= dq_1 + dq_2 \qquad \text{ Bc } d = (a, n) \text{, let } ar = dq_1, -kn = dq_2 \\
		b &= d(q_1 + q_2)
	\end{align*}
	Thus $d|b$ \\
\end{proof}

\newpage 

\noindent \textbf{Section 3.1 Problem 15: } Write out the addition and multiplication tables for 
	\begin{align*}
		\text{a) } \mathbb{Z}_2 \times \mathbb{Z}_3 && \text{b) } \mathbb{Z}_2 \times \mathbb{Z}_2 
	\end{align*}
	\begin{align*}
		&\text{a)} \\
		&\begin{tabular}{c | c c c c c c}
			+ & (0,0) & (0,1) & (0,2) & (1,0) & (1,1) & (1,2) \\
			\hline 
			(0,0) & (0,0) & (0,1) & (0,2) & (1,0) & (1,1) & (1,2) \\
			(0,1) & (0,1) & (0,2) & (0,0) & (1,1) & (1,2) & (1,0) \\
			(0,2) & (0,2) & (0,0) & (0,1) & (1,2) & (1,0) & (1,1) \\
			(1,0) & (1,0) & (1,1) & (1,2) & (0,0) & (0,1) & (0,2) \\	
			(1,1) & (1,1) & (1,2) & (1,0) & (0,1) & (0,2) & (0,0) \\
			(1,2) & (1,2) & (1,0) & (1,1) & (0,2) & (0,0) & (0,1) \\	
		\end{tabular}	
		\quad 	
		\begin{tabular}{c | c c c c c c}
			\text{$\times$} & (0,0) & (0,1) & (0,2) & (1,0) & (1,1) & (1,2) \\
			\hline 
			(0,0) & (0,0) & (0,0) & (0,0) & (0,0) & (0,0) & (0,0) \\
			(0,1) & (0,0) & (0,1) & (0,2) & (0,0) & (0,1) & (0,2) \\
			(0,2) & (0,0) & (0,2) & (0,1) & (0,0) & (0,2) & (0,2) \\
			(1,0) & (0,0) & (0,0) & (0,0) & (1,0) & (1,0) & (1,0) \\	
			(1,1) & (0,0) & (0,1) & (0,2) & (1,0) & (1,1) & (1,2) \\
			(1,2) & (0,0) & (0,2) & (0,1) & (1,0) & (1,2) & (1,1) \\	
		\end{tabular} \\
		&\text{b)} \\
		&\begin{tabular}{c | c c c c }
			+ & (0,0) & (0,1) & (1,0) & (1,1) \\
			\hline 
			(0,0) & (0,0) & (0,1) & (1,0) & (1,1) \\
			(0,1) & (0,1) & (0,2) & (1,1) & (1,2) \\
			(0,2) & (0,2) & (0,0) & (1,2) & (1,0) \\
			(1,0) & (1,0) & (1,1) & (0,0) & (0,1) \\	
			(1,1) & (1,1) & (1,2) & (0,1) & (0,2) \\
			(1,2) & (1,2) & (1,0) & (0,2) & (0,0) \\	
		\end{tabular}	
		\qquad	
		\begin{tabular}{c | c c c c }
			\text{$\times$} & (0,0) & (0,1) & (1,0) & (1,1) \\
			\hline 
			(0,0) & (0,0) & (0,0) & (0,0) & (0,0) \\
			(0,1) & (0,0) & (0,1) & (0,0) & (0,1) \\
			(0,2) & (0,0) & (0,2) & (0,0) & (0,2) \\
			(1,0) & (0,0) & (0,0) & (1,0) & (1,0) \\	
			(1,1) & (0,0) & (0,1) & (1,0) & (1,1) \\
			(1,2) & (0,0) & (0,2) & (1,0) & (1,2) \\	
		\end{tabular}
	\end{align*}
\\ \\
\noindent \textbf{Section 3.1 Problem 17: }	Define a new multiplication in $\mathbb{Z}$ by the rule: $ab = 0 \forall a,b \in \mathbb{Z}$. Show that with ordinary addition and this new multiplication, $\mathbb{Z}$ is a commutative ring.

	\begin{proof}[Solution 3.1.17] 
		Define multiplication in $\mathbb{Z}$ by the rule: $ab = 0 \forall a,b \in \mathbb{Z}$. Let $a,b,c \in \mathbb{Z}$
		\\
		\begin{align*}
			&\text{Axiom 6) } ab = 0 \in \mathbb{Z} \\
			&\text{Axiom 7) } a(bc) = a(0) = 0 = (0)c = (ab)c \\
			&\text{Axiom 8) } a(b+c) = 0 = 0 + 0 = ab + bc \\
			&\text{Axiom 9) } ab = 0 = ba
		\end{align*}
		Thus $\mathbb{Z}$ is a commutative ring. \\
	\end{proof}

\newpage
\noindent \textbf{Section 3.1 Problem 19: } Let S = {a, b, c} and let P(S) be the set of all subsets of S; denote the
elements of P(S) as follows:
	\begin{align*}
		S=\{a,b,c\} &&& D=\{a,b\} & E=\{a,c\} &&& F=\{b,c\} \\
		A = \{a\} &&& B = \{b\} & C = \{c\} &&& 0 = \emptyset.
	\end{align*}
Define addition and multiplication in P(S) by these rules:
M + N = (M - N) $\cup$ (N - M) and MN=M $\cap$ N.
	\begin{align*}
		&\begin{tabular}{c | c c c c c c c c}
			+ & S & D & E & F & A & B & C & 0 \\
			\hline
			S & 0 & C & B & A & F & E & D & S \\
			D & C & 0 & F & E & B & A & S & D \\
			E & B & F & 0 & D & C & S & A & E \\
			F & A & E & D & 0 & S & C & B & F \\
			A & F & B & C & S & 0 & D & E & A \\
			B & E & A & S & C & D & 0 & F & B \\
			C & D & S & A & B & E & F & 0 & C \\
			0 & S & D & E & F & A & B & C & 0 
		\end{tabular}
		&\begin{tabular}{c | c c c c c c c c}
			\text{$\times$} & S & D & E & F & A & B & C & 0 \\
			\hline
			S & S & D & E & F & A & B & C & 0 \\
			D & D & D & A & B & A & B & 0 & 0 \\
			E & E & A & E & C & A & 0 & C & 0 \\
			F & F & B & C & F & 0 & B & C & 0 \\
			A & A & A & A & 0 & A & 0 & 0 & 0 \\
			B & B & B & 0 & B & 0 & B & 0 & 0 \\
			C & C & 0 & C & C & 0 & 0 & C & 0 \\
			0 & 0 & 0 & 0 & 0 & 0 & 0 & 0 & 0 
		\end{tabular}
	\end{align*}
\\ \\
\noindent \textbf{Section 3.1 Problem 23: } Let E be the set of even integers with ordinary addition. Define a new
multiplication * on E by the rule "a * b = "ab/2" (where the product on the right is ordinary multiplication). Prove that with these operations E is a commutative ring with identity. 

	\begin{proof}[Solution 3.1.17] 
	Define multiplication in E by the rule: $a * b = ab/2 $. Let $a,b,c \in E$
	\\
	\begin{align*}
		&\text{Axiom 6) } a*b = ab/2 \in E \\
		&\text{Axiom 7) } a*(b*c) = a*\left(\frac{bc}{2}\right) = \frac{a \left(\frac{1}{2} bc\right)}{2} = \frac{\left(\frac{ab}{2}\right) c}{2} = (a*b)*c \\
		&\text{Axiom 8) } a*(b+c) = \frac{a(b+c)}{2} = \frac{ab}{2} + \frac{ac}{2} = (a*b) + (a*c)\\
		&\text{Axiom 9) } a*b = \frac{ab}{2} = \frac{ba}{2} = b*a
	\end{align*}
	Thus E is a commutative ring. \\
	\end{proof}

  

\end{document}
