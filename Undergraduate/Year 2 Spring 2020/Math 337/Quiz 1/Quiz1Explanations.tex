\documentclass[12pt]{article}
\usepackage[margin = 1in]{geometry}
\usepackage{amsmath}
\usepackage{amssymb}
\usepackage{amsthm}
\usepackage{graphicx}
\usepackage{subfig}
\usepackage{cancel}

\begin{document}
	
	\begin{center}
		\textbf{Quiz 1} \\
		\textbf{Differential Equations} \\
		\textbf{Math 337} \\
		\textbf{Stephen Giang} \\
	\end{center}

\noindent \textbf{Problem 1: }Consider the initial value problem (IVP):
$$\begin{pmatrix}
	\dot{x}_1 \\
	\dot{x}_2
\end{pmatrix} = 
\begin{pmatrix}
	0 & 1 \\
	6 & 1
\end{pmatrix}
\begin{pmatrix}
	x_1 \\
	x_2
\end{pmatrix}, \qquad 
\begin{pmatrix}
	x_1(0) \\
	x_2(0)
\end{pmatrix} = 
\begin{pmatrix}
	3 \\ 
	4
\end{pmatrix}$$
Find the general solution to this problem, create a phase portrait, and solve the initial value
problem. Describe the qualitative behavior shown in the phase portrait. 
\\ \\
\noindent \textbf{Solution 1: } Let:
$		
\begin{vmatrix}
	0 - \lambda  & 1 \\
	6 & 1 - \lambda
\end{vmatrix} = 0
$
	\begin{align*}
		(\lambda)(\lambda - 1) - 6 &= \lambda^2 - \lambda - 6 = 0 \\
		&= (\lambda - 3)(\lambda + 2) = 0 \\
		& \qquad \lambda = 3, -2
	\end{align*}
Let $\lambda_1 = -2$
$$\begin{pmatrix}
	0 - -2  & 1 \\
	6 & 1 - -2
\end{pmatrix}
\begin{pmatrix}
	x_1 \\
	x_2
\end{pmatrix} = 
\begin{pmatrix}
	2 & 1 \\
	6 & 3
\end{pmatrix}
\begin{pmatrix}
	x_1 \\
	x_2
\end{pmatrix} = 
\begin{pmatrix}
	0 \\
	0
\end{pmatrix} \qquad 
\begin{pmatrix}
	x_1 \\
	x_2
\end{pmatrix} = 
\begin{pmatrix}
	1 \\
	-2
\end{pmatrix}$$
Let $\lambda_2 = 3$
$$\begin{pmatrix}
	0 - 3  & 1 \\
	6 & 1 - 3
\end{pmatrix}
\begin{pmatrix}
	x_1 \\
	x_2
\end{pmatrix} = 
\begin{pmatrix}
	-3 & 1 \\
	6 & -2
\end{pmatrix}
\begin{pmatrix}
	x_1 \\
	x_2
\end{pmatrix} = 
\begin{pmatrix}
	0 \\
	0
\end{pmatrix} \qquad 
\begin{pmatrix}
	x_1 \\
	x_2
\end{pmatrix} = 
\begin{pmatrix}
	1 \\
	3
\end{pmatrix}$$ \\
$$
\boldmath
\begin{pmatrix}
	x_1(t) \\
	x_2(t)
\end{pmatrix} = 
c_1
\begin{pmatrix}
	1 \\
	-2
\end{pmatrix}
e^{-2t} + 
c_2
\begin{pmatrix}
	1 \\
	3
\end{pmatrix}
e^{3t}
$$
\\
$$
\begin{pmatrix}
x_1(0) \\
x_2(0)
\end{pmatrix} = 
\begin{pmatrix}
	1 & 1 \\
	-2 & 3
\end{pmatrix}
\begin{pmatrix}
	c_1 \\
	c_2	
\end{pmatrix} =
\begin{pmatrix}
	3 \\
	4
\end{pmatrix}
\quad \rightarrow \quad
\begin{pmatrix}
	1 & 1 & 3\\
	-2 & 3 & 4
\end{pmatrix} \quad \rightarrow \quad
\begin{pmatrix}
	3 & 3 & 9 \\
	2 & -3 & -4
\end{pmatrix} \quad \rightarrow \quad
\begin{pmatrix}
	1 & 1 & 3 \\
	5 & 0 & 5
\end{pmatrix}
$$ \\
So $c_1 = 1$ and $c_2 = 2$, thus the solution holds as: 
$$
\boldmath
\begin{pmatrix}
x_1(t) \\
x_2(t)
\end{pmatrix} = 
\begin{pmatrix}
1 \\
-2
\end{pmatrix}
e^{-2t} + 
2
\begin{pmatrix}
1 \\
3
\end{pmatrix}
e^{3t}
$$ \\
Because the eigenvalues have opposite signs, the phase portrait shows a saddle point.

\newpage

\noindent \textbf{Problem 2: }Consider the differential equation:
$$
\begin{pmatrix}
	\dot{x}_1 \\
	\dot{x}_2
\end{pmatrix} = 
\begin{pmatrix}
	0 & 1 \\
	0 & 0
\end{pmatrix}
\begin{pmatrix}
	x_1 \\
	x_2
\end{pmatrix}
$$
Find the general solution to this problem and create a phase portrait. Describe the qualitative
behavior shown in the phase portrait.
\\ \\
\noindent \textbf{Solution 2: } Let:
$		
\begin{vmatrix}
0 - \lambda  & 1 \\
0 & 0 - \lambda
\end{vmatrix} = 0
$ 
\begin{align*}
	\lambda^2 &= 0 \\
	\lambda &= 0 \text{ mult. 2}
\end{align*}
$$
\begin{pmatrix}
	0 - 0 & 1 \\
	0 & 0 - 0	
\end{pmatrix}
\begin{pmatrix}
	x_1 \\
	x_2
\end{pmatrix} = 
\begin{pmatrix}
	0 & 1 \\
	0 & 0 	
\end{pmatrix}
\begin{pmatrix}
	x_1 \\
	x_2
\end{pmatrix} =
\begin{pmatrix}
	0 \\
	0
\end{pmatrix}
$$
$$
\begin{pmatrix}
	x_1 \\
	x_2
\end{pmatrix} = 
\begin{pmatrix}
	1 \\
	0
\end{pmatrix}
$$ \\
Notice:
\begin{align*}
	\dot{x}_2 = 0 && \dot{x}_1 = x_2 = C_2 \\
	x_2 = C_2 && x_1 = C_2t + C_1
\end{align*}
$$
\boldmath
\begin{pmatrix}
	x_1(t) \\
	x_2(t)
\end{pmatrix} = 
c_1 
\begin{pmatrix}
	1 \\
	0
\end{pmatrix} + 
c_2 
\begin{pmatrix}
	t \\
	1
\end{pmatrix}
$$ \\ \\
Because the eigenvectors are $
\begin{pmatrix}
1 \\
0
\end{pmatrix} $ and 
$
\begin{pmatrix}
t \\
1
\end{pmatrix}$,
phase portraits are horizontal lines that are parallel to the $x_1$ axis. The $
\begin{pmatrix}
1 \\
0
\end{pmatrix} $ eigenvector shows us that the phase portraits are horizontal lines, and the $
\begin{pmatrix}
t \\
1
\end{pmatrix}$ eigenvector shows us that with forward time, the phase portrait moves right, and with backwards time, it moves left.

\newpage 

\noindent \textbf{Problem 3: }Consider the differential equation with the parameter $\alpha$ 
$$
\begin{pmatrix}
	\dot{x}_1 \\
	\dot{x}_2
\end{pmatrix} = 
\begin{pmatrix}
	\alpha & 2 \\
	-2 & 0
\end{pmatrix} 
\begin{pmatrix}
	x_1 \\
	x_2
\end{pmatrix}
$$
Find the general solutions and create phase portraits for the values $\alpha = -6$ and $\alpha = 3$. Describe the qualitative behavior shown in the phase portraits. 
\\ \\
\noindent \textbf{Solution 3: ($\alpha = -6$) } Let 
$		
\begin{vmatrix}
	-6 - \lambda  & 2 \\
	-2 & 0 - \lambda
\end{vmatrix} = 0
$ 
\\ 
\begin{align*}
	(\lambda + 6)(\lambda) + 4 &= 0 \\
	\lambda^2 + 6\lambda + 4 &= 0 \\
	\lambda &= \frac{-6 \pm \sqrt{36 - 4(4)}}{2} \\
	&= \frac{-6 \pm \sqrt{20}}{2} \\
	&= \frac{-6 \pm 2\sqrt{5}}{2} \\
	&= -3 \pm \sqrt{5}
\end{align*}
Let $\lambda_1 = -3 + \sqrt{5}$
$$
\begin{pmatrix}
	-6 - (-3 + \sqrt{5}) & 2 \\
	-2 & 0 - (-3 + \sqrt{5})
\end{pmatrix}
\begin{pmatrix}
	x_1 \\
	x_2
\end{pmatrix} = 
\begin{pmatrix}
	-3 - \sqrt{5} & 2 \\
	-2 & 3 - \sqrt{5}
\end{pmatrix}
\begin{pmatrix}
	x_1 \\
	x_2
\end{pmatrix} = 
\begin{pmatrix}
	0 \\
	0
\end{pmatrix}
$$
$$
\begin{pmatrix}
	x_1 \\
	x_2
\end{pmatrix} = 
\begin{pmatrix}
	2 \\
	3 + \sqrt{5}
\end{pmatrix}
$$
Let $\lambda_2 = -3 - \sqrt{5}$
$$
\begin{pmatrix}
	-6 - (-3 - \sqrt{5}) & 2 \\
	-2 & 0 - (-3 - \sqrt{5})
\end{pmatrix}
\begin{pmatrix}
	x_1 \\
	x_2
\end{pmatrix} = 
\begin{pmatrix}
	-3 + \sqrt{5} & 2 \\
	-2 & 3 + \sqrt{5}
\end{pmatrix}
\begin{pmatrix}
	x_1 \\
	x_2
\end{pmatrix} = 
\begin{pmatrix}
	0 \\
	0
\end{pmatrix}
$$
$$
\begin{pmatrix}
	x_1 \\
	x_2
\end{pmatrix} = 
\begin{pmatrix}
	2 \\
	3 - \sqrt{5}
\end{pmatrix}
$$ \\
$$
\boldmath
\begin{pmatrix}
	x_1(t) \\
	x_2(t)
\end{pmatrix} = 
c_1 
\begin{pmatrix}
	2 \\
	3 + \sqrt{5}
\end{pmatrix}
e^{(-3 + \sqrt{5})t} + 
c_2 
\begin{pmatrix}
	2 \\
	3 - \sqrt{5}
\end{pmatrix}
e^{(-3 - \sqrt{5})t}
$$ \\ \\
Because of the negative eigenvalues, the phase portrait has a stable node (sink).
\newpage 

\noindent \textbf{Solution 3: ($\alpha = 3$) } Let 
$		
\begin{vmatrix}
	3 - \lambda  & 2 \\
	-2 & 0 - \lambda
\end{vmatrix} = 0
$ 
\begin{align*}
	(\lambda - 3)(\lambda) + 4 &= 0 \\
	\lambda^2 - 3\lambda + 4 &= 0 \\
	\lambda &= \frac{3 \pm \sqrt{9 - 4(4)}}{2} \\
	&= \frac{3 \pm \sqrt{9 - 16}}{2} \\
	&= \frac{3 \pm \sqrt{7}i}{2} 
\end{align*}
Let $\lambda_1 = \frac{3 + \sqrt{7}i}{2}$
$$
\begin{pmatrix}
	3 - (\frac{3 + \sqrt{7}i}{2}) & 2 \\
	-2 & 0 - (\frac{3 + \sqrt{7}i}{2})
\end{pmatrix}
\begin{pmatrix}
	x_1 \\
	x_2
\end{pmatrix} = 
\begin{pmatrix}
	6 - (3 + \sqrt{7}i) & 4 \\
	-4 & 0 - (3 + \sqrt{7}i)
\end{pmatrix}
\begin{pmatrix}
	x_1 \\
	x_2
\end{pmatrix}
$$
$$
\begin{pmatrix}
	3 - \sqrt{7}i & 4 \\
	-4 & -3 - \sqrt{7}i
\end{pmatrix}
\begin{pmatrix}
	x_1 \\
	x_2
\end{pmatrix} = 
\begin{pmatrix}
	0 \\
	0
\end{pmatrix}
, \qquad 
\begin{pmatrix}
	x_1 \\
	x_2
\end{pmatrix} = 
\begin{pmatrix}
	4 \\
	-3 + \sqrt{7}i
\end{pmatrix}
$$
$$
x_1(t) = \begin{pmatrix}
	4 \\
	-3 + \sqrt{7}i
\end{pmatrix} 
e^{\frac{3}{2}t}\left( \cos\left(\frac{\sqrt{7}}{2}t\right) + i\sin\left(\frac{\sqrt{7}}{2}t\right) \right)
$$ \\
$$
u(t) + iw(t) = 
\begin{pmatrix}
	4\cos\left( \frac{\sqrt{7}}{2}t \right) \\
	-3\cos\left( \frac{\sqrt{7}}{2}t \right) - \sqrt{7}\sin\left( \frac{\sqrt{7}}{2}t \right)
\end{pmatrix} + 
i \begin{pmatrix}
	4\sin\left( \frac{\sqrt{7}}{2}t \right) \\
	\sqrt{7}\cos\left( \frac{\sqrt{7}}{2}t \right) - 3\sin\left( \frac{\sqrt{7}}{2}t \right)
\end{pmatrix}
$$ \\
$$ \boldmath
\begin{pmatrix}
	x_1(t) \\
	x_2(t)
\end{pmatrix} = 
c_1 
\begin{pmatrix}
	4\cos\left( \frac{\sqrt{7}}{2}t \right) \\
	-3\cos\left( \frac{\sqrt{7}}{2}t \right) - \sqrt{7}\sin\left( \frac{\sqrt{7}}{2}t \right)
\end{pmatrix} e^{\frac{3}{2}t} + 
c_2 
\begin{pmatrix}
	4\sin\left( \frac{\sqrt{7}}{2}t \right) \\
	\sqrt{7}\cos\left( \frac{\sqrt{7}}{2}t \right) - 3\sin\left( \frac{\sqrt{7}}{2}t \right)
\end{pmatrix} e^{\frac{3}{2}t}
$$
\\ \\
Because of the imaginary eigenvalues, with the real part being positive, the phase portrait has an unstable focus

\newpage

\noindent \textbf{Problem 4: }Consider the differential equations \text{\boldmath $\dot{x}$} = $J_i\textbf{x}$, where $J_i$ is each of the following matrices: 
\begin{align*}
	J_1 = \begin{pmatrix}
		2 & -1 \\
		4 & -3
	\end{pmatrix} &&
	J_2 = \begin{pmatrix}
		5 & 3 \\
		-2 & 2
	\end{pmatrix} &&
	J_3 = \begin{pmatrix}
		1 & -3 \\
		2 & -5
	\end{pmatrix} &&
	J_4 = \begin{pmatrix}
		3 & -2 \\
		6 & -3
	\end{pmatrix}
\end{align*}
Use the diagram on Slide 55 to classify the qualitative behavior for these differential equations
($J_i$, $i = 1, 2, 3, 4$) without solving the equations.
\\ \\
\noindent \textbf{Solution 4: } \\ 
	\begin{align*}
		\text{For $J_1$, the eigenvalues are:} && \text{For $J_1$, the Discriminant is:}  \\
		(\lambda - 2)(\lambda + 3) + 4 = 0 && D_1 = (2 - -3)^2 + 4(4)(-1) \\
		\lambda^2 + \lambda - 2 = 0 && D_1 > 0  \\
		\lambda = 2, -1 && 
	\end{align*}
	\begin{align*}
		 \text{For $J_2$, the eigenvalues are:}  &&  \text{For $J_2$, the Discriminant is:}  \\
		(\lambda - 5)(\lambda - 2) + 6 = 0  &&  D_2 = (5 - 2)^2 + 4(3)(-2)   \\
		\lambda^2 - 7\lambda + 16 = 0 && D_2 < 0   \\
		\lambda = \frac{7 \pm \sqrt{15}i}{2} && 
	\end{align*}
	\begin{align*}
		\text{For $J_3$, the eigenvalues are:} &&  \text{For $J_3$, the Discriminant is:} \\
		(\lambda - 1)(\lambda + 5) + 6 = 0 && D_3 = (1 - -5)^2 + 4(-3)(2) \\
		\lambda^2 + 4\lambda + 1 = 0 &&	  	D_3 > 0  \\
		 \lambda = \frac{-4 \pm \sqrt{12}}{2} &&  
	\end{align*} 
	\begin{align*}
		\text{For $J_4$, the eigenvalues are:} && \text{For $J_4$, the Discriminant is:}  \\
		(\lambda - 3)(\lambda + 3) + 12 = 0  && D_4 = (3 - -3)^2 + 4(-2)(6) \\
		\lambda^2 + 3 = 0 && D_4 < 0 \\
		\lambda = \pm \sqrt{3}i
	\end{align*}
By the Diagram the following is true:
	\begin{enumerate}
		\item $J_1$'s Phase Portrait is a Saddle Point 
		\item $J_2$'s Phase Portrait is an Unstable Focus 
		\item $J_3$'s Phase Portrait is a Stable Node 
		\item $J_4$'s Phase Portrait is a Center 
	\end{enumerate}































\end{document}
