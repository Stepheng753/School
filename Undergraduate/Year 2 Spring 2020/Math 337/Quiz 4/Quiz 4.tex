\documentclass[12pt]{article}
\usepackage[margin = 1in]{geometry}
\usepackage{amsmath}
\usepackage{amssymb}
\usepackage{amsthm}
\usepackage{graphicx}
\usepackage{subfig}
\usepackage{enumitem}

\begin{document}
	
	\begin{center}
		\textbf{Quiz 4} \\
		\textbf{Differential Equations} \\
		\textbf{Math 337} \\
		\textbf{Stephen Giang} \\
	\end{center}

\noindent \textbf{Problem 1: }Consider the $2^{nd}$ order linear homogeneous ODE given by:
	$$
	y'' + 6y' + 13y = 0
	$$
Find two linearly independent solutions, $y_1$ and $y_2$, for this ODE and write the general solution to this problem. Show these solutions form a Fundamental set of solutions by computing the Wronskian, $W[y_1, y_2](t)$ and showing it is nonzero for all t.

\vspace{\baselineskip}

\noindent Notice: To find the eigenvalues, we can write the characteristic equation and solve:
	$$
	\lambda^2 + 6\lambda + 13 = 0
	$$
Now we solve using the quadratic equation:
	\begin{align*}
		\lambda &= \frac{-6 \pm \sqrt{36 - 4(13)}}{2} \\
		&= \frac{-6 \pm 4i}{2} \\
		&= -3 \pm 2i
	\end{align*}
So we get the general solution to this being: 
	\begin{align*}
		c_1e^{-3t}\cos(2t) + c_2e^{-3t}\sin(2t)
	\end{align*}
Meaning: 
	\begin{align*}
		y_1 = e^{-3t}\cos(2t) && y_2 = e^{-3t}\sin(2t)
	\end{align*}
We can prove that these solutions form a Fundamental Set of Solutions by proving that the Wronskian is nonzero for all t.
	\begin{align*}
		W[y_1,y_2](t) &= 
		\begin{vmatrix}
			e^{-3t}\cos(2t) & e^{-3t}\sin(2t) \\
			-3e^{-3t}\cos(2t) - 2e^{3t}\sin(2t) & -3e^{-3t}\sin(2t) + 2e^{-3t}\cos(2t)
		\end{vmatrix} \\ \\
		&= (e^{-3t}\cos(2t))(-3e^{-3t}\sin(2t) + 2e^{-3t}\cos(2t)) - (e^{-3t}\sin(2t))(-3e^{-3t}\cos(2t) - 2e^{-3t}\sin(2t)) \\
		&=-3e^{-6t}\sin(2t)\cos(2t) + 2e^{-6t}\cos^2(2t) + 3e^{-6t}\sin(2t)\cos(2t) + 2e^{-6t}\sin^2(2t) \\
		&= 2e^{-6t}(\cos^2(2t) + \sin^2(2t)) \\
		&= 2e^{-6t} > 0 \qquad \forall t
	\end{align*}
	
\newpage 

\noindent \textbf{Problem 2: }Consider the $2^{nd}$ order linear homogeneous ODE given by:
	$$
	y'' - y' - 2y = 54te^{2t} - 20t
	$$
Find the general solution to this problem, using the Method of Undetermined Coefficients. You must show your steps for finding the coefficients of the particular solution.

\vspace{\baselineskip}

\noindent Notice: To find the eigenvalues, we can write the characteristic equation and solve:
	\begin{align*}
		\lambda^2 - \lambda - 2 &= 0 \\
		(\lambda - 2)(\lambda + 1) &= 0 \\
		\lambda &= 2, -1
	\end{align*}
So we get the homogeneous solution to this being:
	$$
	y_h = c_1e^{2t} + c_2e^{-t}
	$$
To get the particular solution, we set the following:
	\begin{align*}
		y_p &=  \left( A{t}^{2}+Bt \right) {{e}^{2\,t}}+Ct+D \\
		y_p' &=  \left( 2\,At+B \right) {{e}^{2\,t}}+2\, \left( A{t}^{2}+Bt \right) {{e}^{2\,t}}+C \\
		y_p'' &= 2\,A{{e}^{2\,t}}+4\, \left( 2\,At+B \right) {{e}^{2\,t}}+4\,\left( A{t}^{2}+Bt \right) {{e}^{2\,t}}	
	\end{align*}
By plugging this into the differential equation, we get:
	\begin{align*}
		y_p'' - y_p' - 2y_p &= 6Ate^{2t} + (2A + 3B)e^{2t} - 2Ct - (C + 2D) \\
		&= 54te^{2t} - 20t
	\end{align*}
To solve, we set the following:
	\begin{align*}
		6A = 54 && -2C = -20 \\
		2A + 3B = 0 && C + 2D = 0 \\
		A = 9, B = -6 && C = 10,  D = -5
	\end{align*}
Now we have the particular solution:
	$$
	y_p = \left( 9{t}^{2}-6t \right) {{e}^{2\,t}}+10t-5 
	$$
Now we also have the general solution:
	$$
	y(t) = c_1e^{2t} + c_2e^{-t} + \left( 9{t}^{2}-6t \right) {{e}^{2\,t}}+10t-5
	$$
	
\newpage

\noindent \textbf{Problem 3 (a): }An important $2^{nd}$ order nonlinear homogeneous ODE shown on Slide 7 describes the motion of a pendulum and satisfies:
	$$
	\theta'' + 0.2\theta' + 4.01\sin\theta = 0
	$$
where $\theta(t)$ is the angle of the pendulum from the downward vertical. Transform this $2^{nd}$ order nonlinear ODE into a system of $1^{st}$ order ODEs by letting $x_1(t) = \theta(t)$ and $x_2(t) = \dot{x}_1(t) = \theta'(t)$. Find all equilibria by letting $\dot{x}_1 = \dot{x}_2 = 0$. 

\vspace{\baselineskip}

\noindent So we can now transform this $2^{nd}$ order nonlinear ODE into a system of $1^{st}$ order ODEs
	\begin{align*}
		\dot{x}_1 &= x_2 \\
		\dot{x}_2 &= -4.01\sin(x_1) -0.2x_2
	\end{align*}
We will now set $\dot{x}_1 = \dot{x}_2 = 0$ to find all the equilibria:
	\begin{align*}
		0 &= x_2 \\
		0 &= -4.01\sin(x_1) -0.2x_2
	\end{align*}
Solving the system of $1^{st}$ order ODEs, we get:
	\begin{align*}
		0 &= x_2 \\
		0 &= -4.01\sin(x_1) -0.2(0) \\
		0 &= -4.01\sin(x_1) \\
		0 &= \sin(x_1) \\
		n\pi &= x_1 \qquad \forall n \in \mathbb{Z}
	\end{align*}
Thus the equilibria is as follows:
	$$
	(n\pi,0) \qquad \forall n \in \mathbb{Z}
	$$

\newpage

\noindent \textbf{Problem 3 (b): } Take the nonlinear system of $1^{st}$ order ODEs found in Part a and determine the Jacobian matrix, $J(x_1, x_2)$, for this system. One equilibrium is $[x_{1e}, x_{2e}]^T = [0, 0]^T$, so compute $J(0, 0)$. Find the eigenvalues for $J(0, 0)$ and use this information to determine the qualitative behavior (e.g., stable node, center, etc.) near this equilibrium, as we did in the previous section. Another equilibrium is $[x_{1e}, x_{2e}]^T = [\pi, 0]^T$, so compute $J(\pi, 0)$. Find the eigenvalues for $J(\pi, 0)$ and use this information to determine the qualitative behavior near this equilibrium.

\vspace{\baselineskip}

\noindent Using the system of $1^{st}$ order ODEs, the Jacobian is as follows:
	\begin{align*}
		J(x_1,x_2) &= 
		\begin{pmatrix}
			0 & 1 \\
			-4.01\cos(x_1) & -0.2
		\end{pmatrix}
	\end{align*}
	
\vspace{\baselineskip}

\noindent One equilibrium is $[x_{1e}, x_{2e}]^T = [0, 0]^T$, so the Jacobian at that point is:
	\begin{align*}
		J(0,0) &= 
		\begin{pmatrix}
			0 & 1 \\
			-4.01 & -0.2
		\end{pmatrix}
	\end{align*}
Eigenvalues at this point can be found by taking the determinant of Jacobian:
	\begin{align*}
		\begin{vmatrix}
			0 - \lambda & 1 \\
			-4.01 & -0.2 - \lambda			
		\end{vmatrix}
		&= \lambda(\lambda + 0.2) + 4.01 \\
		&= \lambda^2 + 0.2\lambda + 4.01 \\
		\lambda &= \frac{-0.2 \pm \sqrt{0.04 - 4(4.01)}}{2} \\
		&= \frac{-0.2 \pm \sqrt{-16}}{2} \\
		&= -0.1 \pm 2i	
	\end{align*}
These eigenvalues show us that the qualitative behavior near $(0,0)$ is a \textbf{stable focus}.

\vspace{\baselineskip}
\vspace{\baselineskip}

\noindent One equilibrium is $[x_{1e}, x_{2e}]^T = [\pi, 0]^T$, so the Jacobian at that point is:
\begin{align*}
J(\pi,0) &= 
\begin{pmatrix}
0 & 1 \\
4.01 & -0.2
\end{pmatrix}
\end{align*}
Eigenvalues at this point can be found by taking the determinant of Jacobian:
\begin{align*}
\begin{vmatrix}
0 - \lambda & 1 \\
4.01 & -0.2 - \lambda			
\end{vmatrix}
&= \lambda(\lambda + 0.2) - 4.01 \\
&= \lambda^2 + 0.2\lambda - 4.01 \\
\lambda &= \frac{-0.2 \pm \sqrt{0.04 + 4(4.01)}}{2} \\
&= \frac{-0.2 \pm \sqrt{16.08}}{2} \\
&= -0.1 \pm 2.005
\end{align*}
These eigenvalues show us that the qualitative behavior near $(\pi,0)$ is a \textbf{saddle point}.



















\end{document}
