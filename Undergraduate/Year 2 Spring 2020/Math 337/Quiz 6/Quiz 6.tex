\documentclass[12pt]{article}
\usepackage[margin = 1in]{geometry}
\usepackage{amsmath}
\usepackage{amssymb}
\usepackage{amsthm}
\usepackage{graphicx}
\usepackage{subfig}
\usepackage{enumitem}

\begin{document}
	
	\begin{center}
		\textbf{Quiz 6} \\
		\textbf{Differential Equations} \\
		\textbf{Math 337} \\
		\textbf{Stephen Giang} \\
	\end{center}

\noindent \textbf{Problem 1: }Consider the $3^{rd}$ order linear homogeneous ODE given by:
	$$
	t^2y''' - ty'' + 2y' = 0
	$$
Use similar techniques for solving the \textit{Cauchy-Euler} problem to solve this problem. Find 3 linearly independent solutions to this problem. How would one establish that these are 3 linearly
independent solutions.	
\\ \\
Let the following be true:
	\begin{align*}
		y = t^{r+1} &&&& y' = (r+1)t^r &&&& y'' = (r^2+r)t^{r-1} &&&& y''' = (r^3 - r)t^{r-2}
	\end{align*}
When we now evaluate the original problem with our $y = t^{r+1}$, we get 
	\begin{align*}
		t^2 t^{r-2} (r^3 - r) - tt^{r-1} (r^2 + r) + t^r(2r + 2) &= 0 \\
		t^r(r^3 - r^2 + 2) &= 0 \\
		t^r(r + 1)(r^2 - 2r + 2) &= 0 \\
		r = -1 \qquad r = 1 \pm i
	\end{align*}
So now we get the 3 solutions:
	\begin{align*}
		y_1 = t^{-1 + 1} = 1 && y_2 = t^2 \cos(\ln t) && y_3 = t^2\sin(\ln t)
	\end{align*}
We can see that these solutions are linearly independent by seeing that the Wronskian is nonzero:
	\begin{align*}
	W_{[y_1,y_2,y_3]}(t) &=
    \begin {vmatrix} 1&{t}^{2}\cos \left( \ln  \left( t
	\right)  \right) &{t}^{2}\sin \left( \ln  \left( t \right)  \right) 
	\\ \noalign{\medskip}0&2\,t\cos \left( \ln  \left( t \right)  \right) 
	-t\sin \left( \ln  \left( t \right)  \right) &2\,t\sin \left( \ln 
	\left( t \right)  \right) +t\cos \left( \ln  \left( t \right) 
	\right) \\ \noalign{\medskip}0&-3\,\sin \left( \ln  \left( t \right) 
	\right) +\cos \left( \ln  \left( t \right)  \right) &\sin \left( \ln 
	\left( t \right)  \right) +3\,\cos \left( \ln  \left( t \right) 
	\right) \end {vmatrix} \\
	&= 5\, \left( \sin \left( \ln  \left( t \right)  \right)  \right) ^{2}t+5
	\, \left( \cos \left( \ln  \left( t \right)  \right)  \right) ^{2}t \\
	&= 5t \qquad t > 0
	\end{align*}
So we can see that the Wronskian is nonzero for $t > 0$ thus the solutions are linearly independent. 

\newpage 

\noindent \textbf{Problem 2: }If $y_1(x)$ is known for the linear ODE:
	$$
	y'' + p(x)y' + q(x)y = 0
	$$
Then one attempts a solution of the form $y(x) = v(x)y_1(x)$. Provided $y_1(x) \not = 0$, show that	
	$$
	\frac{dv}{dx} = \frac{1}{[y_1(x)]^2}e^{-\int^x p(s)ds}
	$$
Solve for $v(x)$ to obtain the $2^{nd}$ linearly independent solution, $y_2(x)$.
\\ \\ \\
Let $y_1(x)$ be a known solution to the original equation such that $y_1'' + p(x)y_1' + q(x)y_1 = 0$. Notice the following:
	\begin{align*}
		y(x) &= v(x)y_1(x) \\
		y'(x) &= v'(x)y_1(x) + v(x)y_1'(x) \\
		y''(x) &= 2v'(x)y_1'(x) + v''(x)y_1(x) + v(x)y_1''(x)
	\end{align*}
We can also see that the second solution $y(x)$ will also satisfy the original equation.
	\begin{align*}
		2v'(x)y_1'(x) + v''(x)y_1(x) + v(x)y_1''(x) + p(x)v'(x)y_1(x) + p(x)v(x)y_1'(x) + q(x)v(x)y_1(x) &= 0 \\
		y_1(x)v''(x) + \left[ p(x)y_1(x) + 2y_1'(x)\right] v'(x) + \left[y_1''(x) + p(x)y_1'(x) + q(x)y_1(x)\right]v(x) &= 0
	\end{align*}
Notice the last term equals zero from earlier observations. Now if we let $w(x) = v'(x)$, we get:
	$$
	y_1(x)w'(x) + (p(x)y_1(x) + 2y_1'(x))w(x) = 0
	$$
Using the method of linear separation, we get:
	\begin{align*}
		\frac{dw}{w(x)} &= \frac{-p(x)y_1(x) - 2y_1'(x)}{y_1(x)}dx \\
		\ln(w(x)) &= \int -p(x)dx -2\int \frac{y_1'(x)}{y_1(x)}dx \\
		w(x) &= e^{-\int p(x)dx} e^{-2 \ln(y_1(x))} \\
		w(x) &= \frac{1}{[y_1(x)]^2}e^{-\int p(x)dx} 
	\end{align*}
Thus we get the results:
	\begin{align*}
	\frac{dv}{dx} = \frac{1}{[y_1(x)]^2}e^{-\int^x p(s)ds} && v(x) = \int \frac{1}{[y_1(x)]^2}e^{-\int^x p(s)ds}
	\end{align*}
With the second solution being:
	$$
	y_2(x) = y_1(x)\int \frac{1}{[y_1(x)]^2}e^{-\int^x p(s)ds}
	$$

\newpage 

\noindent \textbf{Problem 3: }Consider the following ODE:
	\begin{align}
	xy'' + (1 - 2x)y' + (x-1)y = 0
	\end{align}
	\begin{enumerate}[label = (\alph*)]
		\item Show that $y_1(x) = e^x$ is a solution to this differential equation. 
		\item In Part a, $y_1(x) = e^x$ was found as one solution to (1). Use the \textbf{Reduction of Order} method to find $y_2(x)$ for (1). Use the Wronskian to show this is a fundamental set of solutions. \\ \\
	\end{enumerate}
	\begin{enumerate}[label = (\alph*)]
		\item Notice that when evaluating $y_1 = e^x$ into the original equation, we get:
		$$
		xe^x + e^x - 2xe^x + xe^x - e^x = 0
		$$
		Thus $y_1(x) = e^x$ is a solution.
		\item Using the Reduction of Order, we get 
		\begin{align*}
			y_2(x) = e^x \int \frac{e^{\int \left(\frac{-1}{x} + 2\right) dx}}{e^{2x}}dx = e^x \int \frac{x^{-1}e^{2x}}{e^{2x}}dx = e^x\ln(x)
		\end{align*}
		We can show that these solutions make a fundamental set of solutions by showing that the Wronskian of the two are nonzero.
		\begin{align*}
			W_{[y_1, y_2]} = 
			\begin{vmatrix}
				e^x & e^x\ln x \\
				e^x & e^x\ln x + \frac{e^x}{x}
			\end{vmatrix} = 
			\frac{e^{2x}}{x}
		\end{align*}
		We can see that $W_{[y_1,y_2]} \not = 0$ for all x, thus making $y_1, y_2$ a fundamental set of solutions.
	\end{enumerate}


\end{document}
