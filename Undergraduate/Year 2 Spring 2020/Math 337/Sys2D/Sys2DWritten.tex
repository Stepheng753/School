\documentclass[12pt]{article}
\usepackage[margin = 1in]{geometry}
\usepackage{amsmath}
\usepackage{amssymb}
\usepackage{amsthm}
\usepackage{graphicx}
\usepackage{subfig}
\usepackage{cancel}

\begin{document}
	
	\begin{center}
		\textbf{Systems 2D} \\
		\textbf{Differential Equations} \\
		\textbf{Math 337} \\
		\textbf{Stephen Giang} \\
	\end{center}

\noindent \textbf{Problem 11 (c): } 
$$
\begin{pmatrix}
	\dot{x}_1 \\
	\dot{x}_2
\end{pmatrix} = 
\begin{pmatrix}
	0 & 2 \\
	-3 & 5
\end{pmatrix}
\begin{pmatrix}
	x_1 \\
	x_2
\end{pmatrix}
$$
\noindent \textbf{Solution 11 (c): } Let 
$
\begin{vmatrix}
	0 - \lambda & 2 \\
	-3 & 5 - \lambda
\end{vmatrix} = 0
$
	\begin{align*}
		(\lambda)(\lambda - 5) + 6 &=  \lambda^2 - 5\lambda + 6 = 0 \\
		&= (\lambda - 3)(\lambda - 2) = 0 \\
		& \qquad \lambda = 3, 2
	\end{align*}
Let $\lambda_1 = 3$
$$\begin{pmatrix}
	0 - 3  & 2 \\
	-3 & 5 - 3
\end{pmatrix}
\begin{pmatrix}
	x_1 \\
	x_2
\end{pmatrix} = 
\begin{pmatrix}
	-3 & 2 \\
	-3 & 2
\end{pmatrix}
\begin{pmatrix}
	x_1 \\
	x_2
\end{pmatrix} = 
\begin{pmatrix}
	0 \\
	0
\end{pmatrix} \qquad 
\begin{pmatrix}
	x_1 \\
	x_2
\end{pmatrix} = 
\begin{pmatrix}
	2 \\
	3
\end{pmatrix}$$
Let $\lambda_2 = 2$
$$\begin{pmatrix}
0 - 2  & 2 \\
-3 & 5 - 2
\end{pmatrix}
\begin{pmatrix}
x_1 \\
x_2
\end{pmatrix} = 
\begin{pmatrix}
-2 & 2 \\
-3 & 3
\end{pmatrix}
\begin{pmatrix}
x_1 \\
x_2
\end{pmatrix} = 
\begin{pmatrix}
0 \\
0
\end{pmatrix} \qquad 
\begin{pmatrix}
x_1 \\
x_2
\end{pmatrix} = 
\begin{pmatrix}
1 \\
1
\end{pmatrix}$$
\\
$$
\boldmath
\begin{pmatrix}
x_1(t) \\
x_2(t)
\end{pmatrix} = 
c_1
\begin{pmatrix}
	2 \\
	3
\end{pmatrix}
e^{3t} + 
c_2
\begin{pmatrix}
	1 \\
	1
\end{pmatrix}
e^{2t}
$$
\\ \\
Because the eigenvalues are both positive, we have an unstable node.  As $t \rightarrow \infty$, the phase portrait is going away from the origin, which is why it creates an unstable node. 
\newpage

\noindent \textbf{Problem 12 (c): } 
$$
\begin{pmatrix}
\dot{x}_1 \\
\dot{x}_2
\end{pmatrix} = 
\begin{pmatrix}
-12 & -10 \\
15 & 13
\end{pmatrix}
\begin{pmatrix}
x_1 \\
x_2
\end{pmatrix}
$$
\noindent \textbf{Solution 12 (c): } Let 
$
\begin{vmatrix}
-12 - \lambda & -10 \\
15 & 13 - \lambda
\end{vmatrix} = 0
$
\begin{align*}
(\lambda + 12)(\lambda - 13) + 150 &=  \lambda^2 - \lambda + 6 = 0 \\
&= (\lambda - 3)(\lambda + 2) = 0 \\
& \qquad \lambda = 3, -2
\end{align*}
Let $\lambda_1 = 3$
$$\begin{pmatrix}
-12 - 3  & -10 \\
15 & 13 - 3
\end{pmatrix}
\begin{pmatrix}
x_1 \\
x_2
\end{pmatrix} = 
\begin{pmatrix}
-15 & -10 \\
15 & 10
\end{pmatrix}
\begin{pmatrix}
x_1 \\
x_2
\end{pmatrix} = 
\begin{pmatrix}
0 \\
0
\end{pmatrix} \qquad 
\begin{pmatrix}
x_1 \\
x_2
\end{pmatrix} = 
\begin{pmatrix}
2 \\
-3
\end{pmatrix}$$
Let $\lambda_2 = -2$
$$\begin{pmatrix}
-12 - -2  & -10 \\
15 & 13 - -2
\end{pmatrix}
\begin{pmatrix}
x_1 \\
x_2
\end{pmatrix} = 
\begin{pmatrix}
-10 & -10 \\
15 & 15
\end{pmatrix}
\begin{pmatrix}
x_1 \\
x_2
\end{pmatrix} = 
\begin{pmatrix}
0 \\
0
\end{pmatrix} \qquad 
\begin{pmatrix}
x_1 \\
x_2
\end{pmatrix} = 
\begin{pmatrix}
1 \\
-1
\end{pmatrix}$$
\\
$$
\boldmath
\begin{pmatrix}
x_1(t) \\
x_2(t)
\end{pmatrix} = 
c_1
\begin{pmatrix}
2 \\
-3
\end{pmatrix}
e^{3t} + 
c_2
\begin{pmatrix}
1 \\
-1
\end{pmatrix}
e^{-2t}
$$
\\ \\
Because the eigenvalues have opposite signs, we have a saddle point.  As $t \rightarrow \infty$, the phase portrait is going away from the origin along one eigenvector and going towards the origin along the other eigenvector, which is why it creates a saddle point. 
\newpage

\noindent \textbf{Problem 13 (c): } 
$$
\begin{pmatrix}
\dot{x}_1 \\
\dot{x}_2
\end{pmatrix} = 
\begin{pmatrix}
0 & -25 \\
1 & 0
\end{pmatrix}
\begin{pmatrix}
x_1 \\
x_2
\end{pmatrix}
$$
\noindent \textbf{Solution 13 (c): } Let 
$
\begin{vmatrix}
0 - \lambda & -25 \\
1 & 0 - \lambda
\end{vmatrix} = 0
$
\begin{align*}
(\lambda)(\lambda) + 25 &=  \lambda^2 + 25 = 0 \\
&= \lambda^2 = -25 \\
& \qquad \lambda = \pm 5i
\end{align*}
Let $\lambda_1 = 5i$
$$\begin{pmatrix}
0 - 5i  & -25 \\
1 & 0 - 5i
\end{pmatrix}
\begin{pmatrix}
x_1 \\
x_2
\end{pmatrix} = 
\begin{pmatrix}
-5i & -25 \\
1 & -5i
\end{pmatrix}
\begin{pmatrix}
x_1 \\
x_2
\end{pmatrix} = 
\begin{pmatrix}
0 \\
0
\end{pmatrix} \qquad 
\begin{pmatrix}
x_1 \\
x_2
\end{pmatrix} = 
\begin{pmatrix}
5i \\
1
\end{pmatrix}$$
$$
x_1(t) = 
\begin{pmatrix}
	5i \\
	1
\end{pmatrix}
\left( \cos(5t) + i\sin(5t) \right)
$$
$$
u(t) + iw(t) = 
\begin{pmatrix}
	-5\sin(5t) \\
	\cos(5t)
\end{pmatrix} + 
i \begin{pmatrix}
	5\cos(5t) \\
	\sin(5t)
\end{pmatrix}
$$
\\ 
$$
\boldmath
\begin{pmatrix}
x_1(t) \\
x_2(t)
\end{pmatrix} = 
c_1
\begin{pmatrix}
-5\sin(5t) \\
\cos(5t)
\end{pmatrix} + 
c_2
\begin{pmatrix}
5\cos(5t) \\
\sin(5t)
\end{pmatrix}
$$
\\ \\
Because the eigenvalues' real part is 0, we have a center or ellipse.  As $t \rightarrow \infty$, the phase portrait moves in a counter clockwise rotation. 

\newpage

\noindent \textbf{Problem 14 (c): } 
$$
\begin{pmatrix}
\dot{x}_1 \\
\dot{x}_2
\end{pmatrix} = 
\begin{pmatrix}
6 & -9 \\
1 & 6
\end{pmatrix}
\begin{pmatrix}
x_1 \\
x_2
\end{pmatrix}
$$
\noindent \textbf{Solution 14 (c): } Let 
$
\begin{vmatrix}
6 - \lambda & -9 \\
1 & 6 - \lambda
\end{vmatrix} = 0
$
\begin{align*}
(\lambda - 6)(\lambda - 6) + 9 &=  \lambda^2 - 12\lambda + 45 = 0 \\
\lambda &= \frac{12 \pm \sqrt{144 - 180}}{2} \\
&= 6 \pm 3i
\end{align*}
Let $\lambda_1 = 6 + 3i$
$$\begin{pmatrix}
6 - (6 + 3i)  & -9 \\
1 & 6 - (6 + 3i)
\end{pmatrix}
\begin{pmatrix}
x_1 \\
x_2
\end{pmatrix} = 
\begin{pmatrix}
-3i & -9 \\
1 & -3i
\end{pmatrix}
\begin{pmatrix}
x_1 \\
x_2
\end{pmatrix} = 
\begin{pmatrix}
0 \\
0
\end{pmatrix} \qquad 
\begin{pmatrix}
x_1 \\
x_2
\end{pmatrix} = 
\begin{pmatrix}
3i \\
1
\end{pmatrix}$$
$$
x_1(t) = 
\begin{pmatrix}
3i \\
1
\end{pmatrix}
\left( \cos(3t) + i\sin(3t) \right) e^{6t}
$$
$$
u(t) + iw(t) = 
\begin{pmatrix}
-3\sin(3t) \\
\cos(3t)
\end{pmatrix}e^{6t} + 
i \begin{pmatrix}
3\cos(3t) \\
\sin(3t)
\end{pmatrix}e^{6t}
$$
\\ 
$$
\boldmath
\begin{pmatrix}
x_1(t) \\
x_2(t)
\end{pmatrix} = 
c_1
\begin{pmatrix}
-3\sin(3t) \\
\cos(3t)
\end{pmatrix}e^{6t} + 
c_2
\begin{pmatrix}
3\cos(3t) \\
\sin(3t)
\end{pmatrix}e^{6t}
$$
\\ \\
Because the eigenvalues' real part is positive, we have a spiral source.  As $t \rightarrow \infty$, the phase portrait is going away from the origin, which is why it creates a spiral source. 

\end{document}
