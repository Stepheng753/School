\documentclass[12pt]{article}
\usepackage[margin = 1in]{geometry}
\usepackage{amsmath}
\usepackage{amssymb}
\usepackage{cancel}

\begin{document}
	
	\begin{center}
		\textbf{Homework 2} \\
		\textbf{Linear Algebra} \\
		\textbf{Math 524} \\
		\textbf{Stephen Giang} \\
	\end{center}

\noindent \textbf{Section 2.A Problem 8: } Prove or give a counterexample: If $v_1$, $v_2$, ..., $v_m$ is a linearly independent list of vectors in $\mathbb{V}$ and $\lambda \in \mathbb{F}$ with  $\lambda \not=$  0, then $\lambda v_1$, $\lambda v_2$, ...,  $\lambda v_m$ is linearly independent.
\\\\
\noindent \textbf{Solution 2.A Problem 8: } Let $v_1$, $v_2$, ..., $v_m$ be a linearly independent list of vectors in $\mathbb{V}$ and $\lambda \in \mathbb{F}$ with  $\lambda \not=$  0. 
\\
By the Definition of Linear Independence: $$ 0= \sum_{k=1}^{m} a_k v_k $$ with  $a_1,a_2,...,a_m \in \mathbb{F}$, the following must be true: $a_k = 0$ for $\{ k = 0,1,2,...,m\}$
\\\\
To prove $\lambda v_1$, $\lambda v_2$, ...,  $\lambda v_m$ to be linearly independent, the following must be true: $$ 0= \sum_{k=1}^{m} a_k \lambda v_k $$ with  $a_1,a_2,...,a_m \in \mathbb{F}$ and $a_k = 0$ for $\{ k = 0,1,2,...,m\}$.
	\begin{align}
		\sum_{k=1}^{m} a_k \lambda v_k &= \lambda \sum_{k=1}^{m} a_k v_k 		
	\end{align}
Because the only way for $$ \sum_{k=1}^{m} a_k v_k  = 0 \text{ was for } a_k = 0 \text{ for } \{ k = 0,1,2,...,m\}.$$ \\
That means the only way for $$ \sum_{k=1}^{m} a_k \lambda v_k  = 0 \text{ is if } a_k = 0 \text{ for } \{ k = 0,1,2,...,m\} \text{ also.}$$  \\
\textbf{Thus \boldmath $\lambda v_1$, $\lambda v_2$, ...,  $\lambda v_m$ is linearly independent.}


\newpage


\noindent \textbf{Section 2.A Problem 9: } If $v_1$, $v_2$, ..., $v_m$ and $w_1$, $w_2$, ..., $w_m$ are linearly independent lists of vectors in $\mathbb{V}$, then $v_1 + w_1$, $v_2 + w_2$, ..., $v_m + w_m$ is linearly independent. 
\\\\
\noindent \textbf{Solution 2.A Problem 9: } Let $v_1$, $v_2$, ..., $v_m$ and $w_1$, $w_2$, ..., $w_m$ be linearly independent lists of vectors in $\mathbb{V}$. 
\\
By the Definition of Linear Independence, $$ 0= \sum_{k=1}^{m} a_k v_k \text{ and } 0= \sum_{k=1}^{m} b_k w_k$$ with $a_1,a_2,...,a_m, b_1,b_2,...,b_m, \in \mathbb{F}$, the following must be true: $a_k = 0$ and $b_k = 0$ for $\{ k = 0,1,2,...,m\}$
\\\\
	\begin{align}
		\sum_{k=1}^{m} c_k (v_k + w_k) &= \sum_{k=1}^{m} c_k v_k + \sum_{k=1}^{m} c_k w_k
	\end{align}
Because the only way for $$ \sum_{k=1}^{m} c_k v_k  = 0 \text{ was for } c_k = 0 \text{ for } \{ k = 0,1,2,...,m\}.$$ \\
And the only way for $$ \sum_{k=1}^{m} c_k w_k  = 0 \text{ was for } c_k = 0 \text{ for } \{ k = 0,1,2,...,m\}.$$ \\
That means the only way for $$ \sum_{k=1}^{m} c_k (v_k + w_k)  = 0 \text{ is if } c_k = 0 \text{ for } \{ k = 0,1,2,...,m\} \text{ also.}$$
\textbf{Thus \boldmath $v_1 + w_1$, $v_2 + w_2$, ..., $v_m + w_m$ is linearly independent.}


\newpage


\noindent \textbf{Section 2.A Problem 11: } Suppose $v_1$, $v_2$, ..., $v_m$ is linearly independent in $\mathbb{V}$ and $w \in \mathbb{V}$. Show that $v_1$, $v_2$, ..., $v_m$, $w$ is linearly independent if and only if $$w \not \in \text{span} (v_1, v_2, ..., v_m).$$
\noindent \textbf{Solution 2.A Problem 11: ($=>$) } \\ Let $v_1$, $v_2$, ..., $v_m$, $w$ and $v_1$, $v_2$, ..., $v_m$ be linearly independent, and suppose $w\in \text{span} (v_1, v_2, ..., v_m).$ Let $a_i \in \mathbb{R}$ with $i \in \mathbb{Z}^+$ 
\\\\
Because $w\in \text{span} (v_1, v_2, ..., v_m).$, 
	\begin{align}
		w = a_1v_1 + a_2v_2 + ... + a_mv_m \\
		0 = a_1v_1 + a_2v_2 + ... + a_mv_m + -w
	\end{align}
Because we can write 0 as a linear combination of w and vectors: $v_i$ for $i \in \mathbb{Z}^+$, the coefficients are not all 0 as the coefficient to w is -1, thus this contradicts that $v_1$, $v_2$, ..., $v_m$, $w$ is linearly independent. So $w \not \in \text{span} (v_1, v_2, ..., v_m).$  
\\\\
\noindent \textbf{Solution 2.A Problem 11: ($<=$) } Let $w \not \in \text{span} (v_1, v_2, ..., v_m)$ and $v_1$, $v_2$, ..., $v_m$, $w$ be linearly dependent, but $v_1$, $v_2$, ..., $v_m$ be linearly independent. 
\\\\
Because $v_1$, $v_2$, ..., $v_m$, $w$ is linearly dependent, $v_j$ can be written as a linear combination of $v_1$, $v_2$, ..., $v_{j-1}$.  But because $v_1$, $v_2$, ..., $v_m$ is linearly independent, there does not exist a $v_j$ such that it is a linear combination of $v_1$, $v_2$, ..., $v_{j-1}$.  Now because $w \not \in \text{span} (v_1, v_2, ..., v_m)$ and $v_j \not \in \text{span} (v_1, v_2, ..., v_{j-1})$, $v_1$, $v_2$, ..., $v_m$, $w$ has to be linearly independent.

\newpage

\noindent \textbf{Section 2.B Problem 3 (a): } Let U be the subspace of $\mathbb{R}^5$ defined by $$U = \{ (x_1,x_2,x_3,x_4,x_5) \in \mathbb{R}^5: x_1 = 3x_2 \text{ and } x_3 = 7x_4\}$$ Find a basis of U \\

\noindent \textbf{Solution 2.B Problem 3 (a): } Let $u = (3x_2, x_2, 7x_4,x_4,x_5) \in U$ Let $c_i \in \mathbb{R}$ for $i \in \mathbb{Z}^+$
\\\\
Let vectors $v_1 = (3, 1, 0, 0, 0) \in U$ and $v_2 = (0, 0, 7, 1, 1) \in U$ as they satisfy the parameters of u. 
		$$0 = c_1 v_1 + c_2 v_2$$
They are also Linearly Independent of each other as the only way for their sum to be 0, is for their coefficients to also be 0. 
$$u = c_1v_1 + c_2v_2$$
$\forall u$ can be written as a combination of $v_1$ and $v_2$, so it spans U.  \textbf{\boldmath Thus $v_1$ and $v_2$ make a basis for U}
\\\\\\
\noindent \textbf{Section 2.B Problem 3 (b): } Extend the basis in part (a) to basis in $\mathbb{R}^5$. 
\\\\
\noindent \textbf{Solution 2.B Problem 3 (b): } Let $v = (x_1,x_2,x_3,x_4,x_5) \in \mathbb{R}^5$. Let $c_i \in \mathbb{R}$ for $i \in \mathbb{Z}^+$
\\\\
Because $u = (3x_2, x_2, 7x_4,x_4,x_5)$, $x_2, x_4,x_5$ are independent of each other, but $x_1 = 3x_2$ and $x_3 = 7x_4$, are dependent of the other elements. To allow the basis to span $\mathbb{R}^5$, we have to add in $v_3 = (1,0,0,0,0)$ and $v_4 = (0,0,1,0,0)$, so we can make any vector in $\mathbb{R}^5$. \\
Any vector, $v \in \mathbb{R}^5$ can be written as 
$$v = c_1v_1 + c_2v_2 + c_3v_3 + c_4v_4$$
The vectors $v_1,v_2,v_3,v_4$ are also linearly independent of each other as the only way for
$$0 = c_1v_1 + c_2v_2 + c_3v_3 + c_4v_4$$ 
is for all $c_i = 0$ for $i = \{1,2,3,4\}$.
\textbf{\boldmath Thus this is a basis of $\mathbb{R}^5$}
\\\\\\
\noindent \textbf{Section 2.B Problem 3 (c): } Find a subspace W of $\mathbb{R}^5$ such that $\mathbb{R}^5 = U \oplus W.$
\\\\
\noindent \textbf{Solution 2.B Problem 3 (c): } Let W = span($v_3,v_4$).  Because all vectors of $\mathbb{R}^5$ can be written as a linear combination of $v_1,v_2 \in U$ and $v_3,v_4 \in W$, $\text{\boldmath $U \oplus W$}.$

\newpage

\noindent \textbf{Section 2.B Problem 5: } Prove or disprove: There exists a basis $p_0$, $p_1$, $p_2$, $p_3$ of $\mathbb{P}_3$(F) such that none of the polynomials $p_0$, $p_1$, $p_2$, $p_3$ has degree 2.
\\\\
\noindent \textbf{Solution 2.B Problem 5: } Let $p_0 = 1$, $p_1 = x$, $p_2 = x^3 + x^2$, $p_3 = x^3 - x^2$. 
\\\\
To write 0 as a linear combination of $p_0$, $p_1$, $p_2$, $p_3$, the coefficients would all have to be 0, so $p_0$, $p_1$, $p_2$, $p_3$ are linearly independent of each other.
\\\\
We can also write any polynomial as a linear combination of $p_0$, $p_1$, $p_2$, $p_3$ as $p_3 + p_2$ will get you a polynomial of degree 3.  $p_2 - p_3$ will get you a polynomial of degree 2.  $p_1$ will get you a polynomial of degree 1, and $p_0$ will get you a polynomial of degree 0. So all polynomials are in the span($p_0$, $p_1$, $p_2$, $p_3$).  
\\\\
\textbf{\boldmath Thus there exists a basis $p_0$, $p_1$, $p_2$, $p_3$ of $\mathbb{P}_3$(F) such that none of the polynomials $p_0$, $p_1$, $p_2$, $p_3$ have degree 2}
\newpage
\noindent \textbf{Section 2.C Problem 5 (a): }Let $U = \{ p \in \mathbb{P}_4(\mathbb{R}) : p''(6) = 0\}.$ \\Find the Basis of U
\\\\
\noindent \textbf{Solution 2.C Problem 5 (a): } Let $p(x) \in \mathbb{P}_4(\mathbb{R})$ and $a,b,c,d \in \mathbb{R}$
	\begin{align}
		p(x) &= a(x-6)^4 + b(x-6)^3 + c(x-6) + d \\
		p'(x) &= 4a(x-6)^3 + 3b(x-6)^2 + c \\
		p''(x) &= 12a(x-6)^2 + 6b(x-6)
	\end{align}
Because $\forall p(x)$ with $p''(6) = 0, p(x)$ can be written as a linear combination of \{$(x-6)^4, (x-6)^3, (x-6), 1$\}.  Thus all $p(x) \in$ span(\{$(x-6)^4, (x-6)^3, (x-6), 1$\}).  
\\\\
Also because \{$(x-6)^4, (x-6)^3, (x-6), 1$\} is all of different degrees, they are linearly independent of each other. \textbf{\boldmath Thus $\{(x-6)^4, (x-6)^3, (x-6), 1\}$ is a basis of U.}
\\\\
\noindent \textbf{Section 2.C Problem 5 (b): } Extend the basis in part (a) to a basis of $\mathbb{P}_4(\mathbb{R})$
\\\\
\noindent \textbf{Solution 2.C Problem 5 (b): } Because all the leading terms in each element of my previous basis are of degree, $4, 3, 1, 0$.  All I need to do is add in \{$x^2$\}.  This will allow my basis to span $\mathbb{P}_4(\mathbb{R})$ and still be linearly independent of each other
\\\\
\noindent \textbf{Section 2.C Problem 5 (c): } Find a subspace W of $\mathbb{P}_4(\mathbb{R})$ such that $\mathbb{P}_4(\mathbb{R}) = U \oplus W$ 
\\\\
\noindent \textbf{Solution 2.C Problem 5 (c): } Let W = \{$cx^2 : c \in \mathbb{R}$\}.  This will allow $U \oplus W$ to make up the entire set of $\mathbb{P}_4(\mathbb{R})$.
\newpage
\noindent \textbf{Section 2.C Problem 9: } Suppose $v_1,...,v_m$ is linearly independent in V and $w \in V$.  Prove that 
$$\text{dim span }(v_1 + w,...,v_m + w) \geq m-1$$
\noindent \textbf{Solution 2.C Problem 9: } \\
Notice the following: 
		$$v_2 - v_1 = (v_2 + w) - (v_1 + w)$$
Thus $v_i - v_1 \in$ span$(v_1 + w,...,v_m + w)$ for $2 \leq i \leq m$.
Because $v_1,...,v_m$ is linearly independent, $v_2 - v_1, ..., v_m - v_1$ is also linearly independent.  So we can now extended this to a basis in V by Thm 2.33, such that 
$$\text{dim span }(v_1 + w,...,v_m + w) \geq m-1$$



\end{document}
