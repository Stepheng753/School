\documentclass[12pt]{article}
\usepackage[margin = 1in]{geometry}
\usepackage{amsmath}
\usepackage{amssymb}
\usepackage{amsthm}
\usepackage{graphicx}
\usepackage{subfig}
\usepackage{cancel}

\begin{document}
	
	\begin{center}
		\textbf{Homework 5} \\
		\textbf{Linear Algebra} \\
		\textbf{Math 524} \\
		\textbf{Stephen Giang} \\
	\end{center}

\noindent \textbf{Section 5.A Problem 4: } Suppose that $T \in \mathcal{L}(V)$ and $U_1, ...,U_m$ are subspaces of $V$ invariant under $T$.  Prove that $U_1 + ... + U_m$ is invariant under $T$.
	\begin{proof}[Solution]
		Let $T \in \mathcal{L}(V)$ and $U_1, ...,U_m$ be subspaces of $V$ invariant under $T$. \\ Let $u \in U_1 + ... + U_m$ and $u_1 \in U_1 , ..., u_m \in U_m$ with $u = u_1 + ... + u_m $
		\begin{align*}
			&\text{Notice: } Tu_1 \in U_1, ..., Tu_m \in U_m \\
			&Tu = Tu_1 + ... + Tu_m \in U_1 + ... + U_m
		\end{align*}
		Because $Tu \in U_1 + ... + U_m$, this proves that $U_1 + ... + U_m$ is invariant under $T$. \\
	\end{proof} 
\vspace{\baselineskip}

\noindent \textbf{Section 5.A Problem 8: } Define $T \in \mathcal{L}(\mathbb{F}^2)$ by $T(w,z) = (z, w)$. Find all the eigenvalues and eigenvectors of T.
	\begin{proof}[Solution]
		Let $T \in \mathcal{L}(\mathbb{F}^2)$ with $T(w,z) = (z, w)$.
		\begin{align*}
			&\text{Notice:} \qquad Tw = z \qquad Tz = w \\
			&Tw = z = \lambda w \qquad Tz = w = \lambda z \\
			&\text{Through Subsitution: } z = \lambda ^2 z \text{ or } w = \lambda ^2 w \\
			&\lambda ^2 = 1 \\
			&\lambda = \pm 1
		\end{align*}
		For $\lambda = 1$, we need to find a $v_1$ such that $Tv_1 = v_1$. For $\lambda = -1$, we need to find a $v_2$ such that $Tv_2 = -v_2$.
		\begin{align*}
			T(w,z) = (z,w) = 1(w,z) && T(w,z) = (z,w) = -1(w,z) \\
			z = w && z = -w 
		\end{align*}
		The eigenvector corresponding to $\lambda = 1$ would be a vector whose components would be equal to each other such as $(w,w)$ with $w \in \mathbb{F}$. The eigenvector corresponding to $\lambda = -1$ would be a vector whose components would be opposite to each other such as $(w,-w)$ with $w \in \mathbb{F}$.  \\
	\end{proof}
\newpage 

\noindent \textbf{Section 5.A Problem 10 (a): }Define $T \in \mathcal{L}(\mathbb{F}^n)$ by $T(x_1,x_2,...,x_n) = (x_1,2x_2,...,nx_n)$. Find all the eigenvalues and eigenvectors of T.
	\begin{proof}[Solution]
		Let $T \in \mathcal{L}(\mathbb{F}^n)$ with $T(x_1,x_2,...,x_n) = (x_1,2x_2,...,nx_n)$.
		\begin{align*}
			&\text{Notice: } Tx_j = jx_j \qquad 1 \leq j \leq n \\
			&jx_j = \lambda x_j \\
			&j = \lambda
		\end{align*}
		For $\lambda = j$, we need to find a $(x_1,x_2,...,x_n)$ such that $T(x_1,x_2,...,x_n) = j(x_1,x_2,...,x_n)$
		\begin{align*}
			&\text{Notice: if we set j = 2: } T(x_1,x_2,...,x_n) = (x_1,2x_2,...,nx_n) = (2x_1, 2x_2,...,2x_n) \\
			&\text{So: } x_1 = 2x_1 \qquad 2x_2 = 2x_2 \qquad nx_n = 2x_n \\
			&0 = 2x_1 - x_1 = x_1 \qquad x_2 = x_2 \qquad 0 = (2-n)x_n \\
			&\text{So: } x_1 = 0 , x_3 = 0, ..., x_n = 0 \\
			&\text{Eigenvector for $\lambda = j = 2$: } (0,x_2,0, ..., 0)
		\end{align*}
		The eigenvectors that correspond to $\lambda = j$ are $(0 , 0, ..., x_j , ... , 0, 0 )$ for all $1 \leq j \leq n$. \\
	\end{proof}
\vspace{\baselineskip}

\noindent \textbf{Section 5.B Problem 1 (a): }Suppose $T \in \mathcal{L}(V)$ and there exists a positive integer $n$ such that $T^n = 0$. Prove that $I - T$ is invertible and that 
$$ (I - T)^{-1} = I + T + ... + T^{n-1} $$
	\begin{proof}[Solution]
		Let $T \in \mathcal{L}(V)$ and there exists a positive integer $n$ such that $T^n = 0$. Notice: A is invertible iff $AA^{-1} = A^{-1}A = I$
		\begin{align*}
			(I - T)(I + T + ... + T^{n-1}) &= (I + T + ... + T^{n-1}) - (T + T^2 + ... + T^{n-1} + T^n) \\
			&= I - T^n = I - 0 = I		
		\end{align*}
		This proves that $(I - T)$ is invertible and its inverse is $(I + T + ... + T^{n-1})$ \\
	\end{proof}
\vspace{\baselineskip}

\noindent \textbf{Section 5.B Problem 7: }Suppose that $T \in \mathcal{L}(V)$. Prove that 9 is an eigenvalue of $T^2$ if and only if 3 or -3 is an eigenvalue of $T$.
	\begin{proof}[Solution]
		Let $T \in \mathcal{L}(V)$. Let $\lambda$ represent the eigenvalue of $T$.
		\begin{align*}
			\text{Let }\lambda_1 = 3 && \text{Let } \lambda_2 = -3 \\
			Tv = 3v && Tv = -3v \\
			T^2v = T(Tv) = T(3v) = 3(3v) = 9v && T^2v = T(Tv) = T(-3v) = -3(-3v) = 9v
		\end{align*}
		This proves that the eigenvalue of $T^2$ is 9 when the eigenvalues of $T$ is 3 or -3. \\
	\end{proof}
\newpage

\noindent \textbf{Section 5.B Problem 14: }Give an example of an operator whose matrix with respect to some basis
contains only 0’s on the diagonal, but the operator is invertible
	\begin{proof}[Solution]
		Let $T \in \mathcal{L}(V)$ and $v_1 ,v_2, v_3$ is the basis of $V$ such that \\$Tv_j = v_1 + v_2 + v_3 - v_j$ for $1 \leq j \leq 3$ 
		$$ \mathcal{M}(T) = 
		\begin{pmatrix}
			0 & 1 & 1 \\
			1 & 0 & 1 \\
			1 & 1 & 0 
		\end{pmatrix}
		\qquad \mathcal{M}^{-1}(T) =
		\begin{pmatrix}
			-\frac{1}{2} & \frac{1}{2} & \frac{1}{2} \\
			\frac{1}{2} & -\frac{1}{2} & \frac{1}{2} \\
			\frac{1}{2} & \frac{1}{2} & -\frac{1}{2} 
		\end{pmatrix} 
		$$
		$T$ proves there exists an operator whose matrix with respect to some basis contains only 0’s on the diagonal. Because $\mathcal{M}^{-1}(T)$ exists proves that the matrix is invertible. \\
	\end{proof}
\vspace{\baselineskip}

\noindent \textbf{Section 5.B Problem 15: }Give an example of an operator whose matrix with respect to some basis
contains only nonzero numbers on the diagonal, but the operator is not invertible
	\begin{proof}[Solution]
		Let $T \in \mathcal{L}(V)$ and $v_1, v_2$ is the basis of $V$ such that for $a \not = 0,b \not = 0 \in \mathbb{F}$, $Tv_j = av_1 + bv_2$ for j = \{1, 2\}
		$$ \mathcal{M}(T) =
		\begin{pmatrix}
			a & a \\
			b & b \\
		\end{pmatrix}
		$$
		Because $\mathcal{M}^{-1}(T)$ does not exists proves that the matrix is not invertible, while having its diagonal consist on nonzero numbers.
	\end{proof}

\noindent \textbf{Section 5.C Problem 1: }Suppose $T \in \mathcal{L}(V)$ is diagonalizable. Prove that $V = $ null $T \oplus$ range $T$.
	\begin{proof}[Solution]
		Let $T \in \mathcal{L}(V)$ be diagonalizable and $v_1, ..., v_n, u_1 , ...,u_m$ be a basis of V.  
		\\ \\
		(1) Because T is diagonalizable, each element of the basis of V is an eigenvector, such that $Tv_j = \lambda v_j$ and $Tu_k = 0 u_k$.  
		\begin{align*}
			&\text{range T = span}(\lambda v_1 , ... , \lambda v_n) \\
			&\text{null T = span}(u_1 , ... , u_m) 
		\end{align*}
		(2) Let $w \in $ null $T \cap$ range $T$. 
		\begin{align*}
			w &= a_1 v_1 + ... a_n v_n \\
			&= b_1 u_1 + ... + b_m u_m \\
			&a_1 v_1 + ... a_n v_n - (b_1 u_1 + ... + b_m u_m) = 0 \\
		\end{align*}
		Because of (1), $V =$ null $T +$ range $T$. Because $v_1, ..., v_n, u_1 , ...,u_m$ is a basis of V, the constants have to be 0, which means w = 0.  This proves that null $T \cap$ range $T$ = $\{0\}$.  Which proves that $V = $ null $T \oplus$ range $T$. \\ 
	\end{proof}
\newpage

\noindent \textbf{Section 5.C Problem 2: }Prove the converse of the statement in the exercise above or give a
counterexample to the converse.
	\begin{center}
		Converse: If $V = $ null $T \oplus$ range $T$, then $T \in \mathcal{L}(V)$ is diagonalizable.
	\end{center}
	\begin{proof}[Solution]
		 Let $T \in \mathcal{L}(\mathbb{R}^2)$ such that $T(w,z) = (-z,w)$ with $w,z \in \mathbb{R}$. \\ \\
		 Notice: null $T =$ span((0,0)) and range $T = $ span(-z,w), $\mathbb{R}^2 =$ null $T \oplus$ range $T$  \\
		 The eigenvalues of T are, however, $\pm i \not \in \mathbb{R}$.  Because there does not exist any eigenvalues in $\mathbb{R}$, T is not diagonalizable. So the converse is false!
	\end{proof}






 
\end{document}
