\documentclass[12pt]{article}
\usepackage[margin = 1in]{geometry}
\usepackage{amsmath}
\usepackage{amssymb}
\usepackage{amsthm}
\usepackage{graphicx}
\usepackage{subfig}
\usepackage{enumitem}

\begin{document}
	
	\begin{center}
		\textbf{Homework 6} \\
		\textbf{Linear Algebra} \\
		\textbf{Math 524} \\
		\textbf{Stephen Giang} \\
	\end{center}

\noindent \textbf{Section 6.A Problem 1: }Show that the function that takes $\left((x_1,x_2),(y_1,y_2)\right) \in \mathbb{R}^2 \times \mathbb{R}^2$ to $|x_1y_1| + |x_2y_2|$ is not an inner product on $\mathbb{R}^2$
\\ \\
Notice: 
	\begin{align*}
		<(1,1) + (-1,-1) , (1,1)> &= <(1,1),(1,1)> + <(-1,-1),(1,1)> \\
		&=2 + 2 = 4 \not = 0 = <(0,0),(1,1)>
	\end{align*}
Because the function does not hold additivity in the first slot, it is not an inner product on $\mathbb{R}^2$

\vspace{\baselineskip}
\vspace{\baselineskip}
\vspace{\baselineskip}

\noindent \textbf{Section 6.A Problem 2: }Show that the function that takes $\left((x_1,x_2,x_3),(y_1,y_2,y_3)\right) \in \mathbb{R}^3 \times \mathbb{R}^3$ to $x_1y_1 + x_3y_3$ is not an inner product on $\mathbb{R}^3$ 
\\ \\
Notice with ($x_2 \not = 0, y_2 \not = 0$)
	\begin{align*}
		<(0,x_2,0),(0,y_2,0)> = 0
	\end{align*}
Because the function does not hold for definiteness, it is not an inner product on $\mathbb{R}^3$

\newpage 

\noindent \textbf{Section 6.A Problem 4: }Suppose $V$ is a real inner product space
	\begin{enumerate}[label = (\alph*)]
		\item Show that $<u + v, u - v> = || u ||^2 - || v ||^2$ for every $u,v \in V$
		\begin{align*}
			<u + v, u - v>  &= <u,u> + <u,-v> + <v,u> + <v,-v> \\
			&= <u,u> + - <u,v> + <u,v> - <v,v> \\
			&= || u ||^2 - || v ||^2
		\end{align*}
		\item Show that if $u,v \in V$ have the same norm, then $u + v$ is orthogonal to $u - v$ \\ \\
		Let $|| u || = || v ||$
		\begin{align*}
			|| u ||^2 &= || v ||^2 \\
			<u + v, u - v> &= || u ||^2 - || v ||^2 = 0
		\end{align*}
		Because $<u + v, u - v> = 0$, then $u + v$ is orthogonal to $u - v$ \\
		\item Use part (b) to show that the diagonals of a rhombus are perpendicular to each other
		\\ \\
		Because the diagonals of a rhombus with 2 sides being $u$ and the other 2 being $v$ can be written as $u + v$ and $u - v$, part (b) shows us that they are orthogonal, or perpendicular.   
	\end{enumerate}
 
\vspace{\baselineskip}
\vspace{\baselineskip}
\vspace{\baselineskip}

\noindent \textbf{Section 6.A Problem 5: }Suppose $T \in \mathcal{L}(V)$ is such that $|| Tv || \leq || v ||$ for every $v \in V$. Prove that $T - \sqrt{2}I$ is invertible. 
\\ \\
Notice: For T not to be invertible, then det$(T - \lambda I) = 0$. To find the eigenvalues, $\lambda$, then 
	$$
	Tv = \lambda v
	$$
	\begin{align*}
		|| Tv || &= \sqrt{<Tv,Tv>} \\
		&= \sqrt{<\lambda v, \lambda v>} \\
		&= \sqrt{\lambda ^2}\sqrt{<v,v>} \\
		&= |\lambda| || v || \\
		& \leq || v ||
	\end{align*} 
	$$
	|| Tv || = |\lambda| || v || \leq || v ||
	$$
	Thus $-1 \leq \lambda \leq 1$.  Because $\sqrt{2}$ is not within the interval, it is also not an eigenvalue.  So det$(T - \sqrt{2}I) \not = 0$, thus $T - \sqrt{2}I$ is invertible.

\newpage 

\noindent \textbf{Section 6.B Problem 1: }Suppose $\theta \in R$. Show that $(\cos \theta,\sin \theta),(-\sin \theta,\cos \theta)$ and $(\cos \theta,\sin \theta),(\sin \theta, -\cos \theta)$ are orthonormal bases of $\mathbb{R}^2$.
	\begin{align*}
		|| (\cos \theta,\sin \theta) || &= <(\cos \theta,\sin \theta),(\cos \theta,\sin \theta)> \\
		&= \sqrt{\cos ^2 \theta + \sin ^ 2 \theta} \\
		&= \sqrt{1} = 1 
		\\
		|| (-\sin \theta,\cos \theta) || &= <(-\sin \theta,\cos \theta), (-\sin \theta,\cos \theta)> \\
		&= \sqrt{\sin ^ 2 \theta + \cos ^2 \theta } \\
		&= \sqrt{1} = 1 \\
		|| (\sin \theta,-\cos \theta) || &= <(\sin \theta,-\cos \theta), (\sin \theta,-\cos \theta)> \\
		&= \sqrt{\sin ^ 2 \theta + \cos ^2 \theta } \\
		&= \sqrt{1} = 1 \\
	\end{align*}
	\begin{align*}
		<(\cos \theta,\sin \theta),(-\sin \theta,\cos \theta)> &= \sqrt{-\cos \theta\sin \theta +( \sin \theta\cos \theta } = 0 \\
		<(\cos \theta,\sin \theta),(\sin \theta,-\cos \theta)> &= \sqrt{\cos \theta\sin \theta - \sin \theta\cos \theta } = 0 \\
	\end{align*}
Because the 2 lists are orthonormal and their length, 2, is equal to $\mathbb{R}^2$'s dimension, then both lists are orthonormal bases.

\newpage 

\noindent \textbf{Section 6.B Problem 3: }Suppose $T \in \mathcal{L}(\mathbb{R}^3)$ has an upper-triangular matrix with respect to
the basis $(1,0,0),(1,1,1),(1,1,2)$. Find an orthonormal basis of $\mathbb{R}^3$ with respect to which T has an upper-triangular matrix. 
\\ \\
Because we have an upper-triangular matrix with respect to some basis, there exists an orthonormal basis $e_1, e_2, e_3$ that can be calculated from the Gram-Schmidt Procedure, to which T has an upper triangular matrix. 
\\ \\
Let $v_1 = (1,0,0), v_2 = (1,1,1), v_3 = (1,1,2)$
	\begin{align*}
		e_1 &= \frac{v_1}{|| v_1 ||} = \frac{(1,0,0)}{1} = (1,0,0) \\ \\
		e_2 &= \frac{v_2 - <v_2, e_1>e_1}{|| v_2 - <v_2, e_1>e_1 ||} = \frac{(1,1,1) - (1,0,0)}{|| (1,1,1) - (1,0,0) ||} = \frac{(0,1,1)}{\sqrt{2}} = (0,\frac{1}{\sqrt{2}},\frac{1}{\sqrt{2}}) \\ \\
		e_3 &= \frac{v_3 - <v_3, e_1>e_1 - <v_3,e_2>e_2}{|| v_3 - <v_3, e_1>e_1 - <v_3,e_2>e_2 ||}\\ \\
		&= \frac{(1,1,2) - (1,0,0) - \frac{3}{\sqrt{2}}(0,\frac{1}{\sqrt{2}},\frac{1}{\sqrt{2}})}{|| 1,1,2) - (1,0,0) - \frac{3}{\sqrt{2}}(0,\frac{1}{\sqrt{2}},\frac{1}{\sqrt{2}}) ||} 
		= 
		\frac{(0,1,2) - (0,\frac{3}{2},\frac{3}{2})}{|| (0,1,2) - (0,\frac{3}{2},\frac{3}{2}) ||} \\ \\
		&= \sqrt{2}(0,\frac{-1}{2},\frac{1}{2}) = (0, \frac{-\sqrt{2}}{2}, \frac{\sqrt{2}}{2})
	\end{align*}
Thus $e_1, e_2, e_3$ is an orthonormal basis of $\mathbb{R}^3$ with respect to which T has an upper-triangular matrix.

\newpage 

\noindent \textbf{Section 6.B Problem 9: }What happens if the Gram–Schmidt Procedure is applied to a list of vectors that is not linearly independent?
\\ \\
Let $v_1, ..., v_j$ be a linearly dependent list of vectors and $v_1, ..., v_{j-1}$ be a linearly independent list such that 
$$
v_j \in \text{span}(v_1, ..., v_{j-1})
$$
Because $v_1, ..., v_{j-1}$ is linearly independent, it can be used to turn into an orthonormal list $e_1, ..., e_{j-1}$ and by 6.30, $v_j = <v_j,e_1>e_1 + ... + <v_j,e_{j-1}>e_{j-1}$.  If we try to apply Gram-Schmidt on $v_j$, you would get a denominator of 0, Thus Gram-Schmidt doesn't work on linearly dependent lists.

\vspace{\baselineskip}
\vspace{\baselineskip}
\vspace{\baselineskip}

\noindent \textbf{Section 6.C Problem 5: }Suppose V is finite-dimensional and U is a subspace of V. Show that $P_{U^\perp} = I - P_U$, where I is the identity operator of V. 
\\ \\
By 6.47, $V = U \oplus U^\perp$.  This means that $\forall v \in V, v = u + w$, with $u \in U$ and $w \in U^\perp$.  By definition of $P_U, P_U(v) = w$. And by definition of $P_{U^\perp}, P_{U^\perp}(v) = u$.  
	$$
	P_{U^\perp}(v) = u = u + w - w = (u + w) - w = I(v) -P_{U}(v) = (I - P_{U})(v)
	$$
Thus $P_{U^\perp} = I - P_U$

\vspace{\baselineskip}
\vspace{\baselineskip}
\vspace{\baselineskip}

\noindent \textbf{Section 6.C Problem 11: }In $\mathbb{R}^4$, Let 
	$$
	U = \text{span}\left( (1,1,0,0),(1,1,1,2) \right)
	$$
Find $u \in U$ such that $|| u - (1,2,3,4) ||$ is as small as possible.
\\ \\
Using Gram-Schmidt, we can find an orthonormal basis
	\begin{align*}
		e_1 &= \frac{(1,1,0,0)}{\sqrt{2}} = (\frac{1}{\sqrt{2}},\frac{1}{\sqrt{2}}, 0, 0) \\
		e_2 &= \frac{(1,1,1,2) - <(1,1,1,2),(\frac{1}{\sqrt{2}},\frac{1}{\sqrt{2}}, 0, 0)>(\frac{1}{\sqrt{2}},\frac{1}{\sqrt{2}}, 0, 0)}{|| (1,1,1,2) - <(1,1,1,2),(\frac{1}{\sqrt{2}},\frac{1}{\sqrt{2}}, 0, 0)>(\frac{1}{\sqrt{2}},\frac{1}{\sqrt{2}}, 0, 0) ||} \\
		&= \frac{(0,0,1,2)}{\sqrt{5}} = (0,0,\frac{1}{\sqrt{5}}, \frac{2}{\sqrt{5}})
	\end{align*}
$P_U(1,2,3,4) = <(1,2,3,4),e_1>e_1 + <(1,2,3,4),e_2>e_2$ is the closest $u \in U$ to $(1,2,3,4)$ by 6.56.  Which equals 
$$
(\frac{3}{2},\frac{3}{2},\frac{11}{5},\frac{22}{5})
$$ 





\end{document}
