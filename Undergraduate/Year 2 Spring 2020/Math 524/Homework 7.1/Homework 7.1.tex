\documentclass[12pt]{article}
\usepackage[margin = 1in]{geometry}
\usepackage{amsmath}
\usepackage{amssymb}
\usepackage{amsthm}
\usepackage{graphicx}
\usepackage{subfig}
\usepackage{enumitem}

\begin{document}
	
	\begin{center}
		\textbf{Homework 7.1} \\
		\textbf{Linear Algebra} \\
		\textbf{Math 524} \\
		\textbf{Stephen Giang} \\
	\end{center}

\noindent \textbf{Problem 7.A.1: }Suppose $n$ is a positive integer. Define $T \in \mathcal{L}(\mathbb{F}^n)$ by
	$$
	T(z_1,...,z_n) = (0,z_1,...,z_{n-1})
	$$
Find a formula for $T^*(z_1,...,z_n)$.
\\ \\
Notice the following:	
	\begin{align*}
		\langle T(x_1,...,x_n) , (z_1,...,z_n)\rangle &= \langle (0,x_1,...,x_{n-1}),(z_1,...,z_n)\rangle \\
		&= x_1z_2 + ... + x_{n-1}z_n + x_n(0) \\
		&= \langle (x_1,...,x_{n-1},x_n), (z_2, ..., z_{n}, 0) \rangle \\
		&= \langle (x_1,...,x_n), T^*(z_1,...,z_n) \rangle
	\end{align*}
Thus $T^*(z_1,...,z_n) = (z_2, ..., z_{n-1}, 0)$

\newpage 

\noindent \textbf{Problem 7.A.4: }Suppose $T \in \mathcal{L}(V,W)$. Prove that 
	\begin{enumerate}[label = (\alph*)]
		\item $T$ is injective if and only if $T^*$ is surjective.
			\begin{proof}[$(=>)$]
				Let T be injective.
				\\ \\
				Because T is injective, null$\,T = $(range$T^*$)$^\perp = \{0\}$. Because (range$T^*$)$^\perp = \{0\}$, that means (range$T^*$)$ = W$, thus proving that $T^*$ is surjective.
				\\ \\
				$(<=)$ Let $T^*$ be surjective.
				\\ \\
				Because $T^*$ is surjective, range$T^* = W$, such that (range$T^*$)$^\perp =$ null$\,T = \{0\}$. Because null$T = \{ 0 \}$, T is injective. 
				\\
			\end{proof}
		\item $T$ is surjective if and only if $T^*$ is injective.
			\begin{proof}[$(=>)$]
				Let T be surjective.
				\\ \\
				Because T is surjective, range$T = W$ and (range$T$)$^\perp =$ null$T^* = \{0\}$. Because null$T^* = \{0\}$. Thus $T^*$ is injective.  
				\\ \\
				Let $T^*$ be injective.
				\\ \\
				Because $T^*$ is injective, null$\,T^* = $(range$T$)$^\perp = \{0\}$. Because (range$T$)$^\perp = \{0\}$, that means (range$T$)$ = W$, thus proving that $T$ is surjective.
			\end{proof}
	\end{enumerate}

\newpage 

\noindent \textbf{Problem 7.A.6: }Make $\mathcal{P}_2(\mathbb{R})$ into an inner product space by defining
	$$
	\langle p,q \rangle = \int_{0}^{1} p(x)q(x)
	$$
Define $T \in \mathcal{L}(\mathcal{P}_2(\mathbb{R}))$ by $T(a_0 + a_1x + a_2x^2) = a_1x$
	\begin{enumerate}[label = (\alph*)]
		\item  Show that T is not self-adjoint.
		\\ \\
		Notice that self adjoint means $\langle Tp,q \rangle = \langle p, Tq \rangle$.
		\\ \\
		Let $p = p_0 + p_1x + p_2x^2$ and $q = q_0 + q_1x + q_2x^2$
			\begin{align*}
				\langle Tp,q \rangle = \langle p_1x, q_0 + q_1x + q_2x^2 \rangle && \langle p,Tq \rangle = \langle p_0 + p_1x + p_2x^2, q_1x \rangle\\
				= \int_{0}^{1} p_1q_0x + p_1q_1x^2 + p_1q_2x^3 && =\int_{0}^{1} q_1p_0x + q_1p_1x^2 + q_1p_2x^3 \\
				= \left. p_1 \left( \frac{q_0x^2}{2} + \frac{q_1x^3}{3} + \frac{q_2x^4}{4} \right) \right|^{1}_0 && =\left. q_1 \left( \frac{p_0x^2}{2} + \frac{p_1x^3}{3} + \frac{p_2x^4}{4} \right) \right|^{1}_0\\
				= p_1 \left( \frac{q_0}{2} + \frac{q_1}{3} + \frac{q_2}{4} \right) && =q_1 \left( \frac{p_0}{2} + \frac{p_1}{3} + \frac{p_2}{4} \right)
			\end{align*}
		As long as $p_0,q_0 \not = 0$, and $q_i \not= p_i$ for $i = \{0,1,2\}$, T is not self-adjoint.
		\item The matrix of T with respect to the basis $(1, x, x^2)$ is
			$$
			\begin{pmatrix}
				0 & 0 & 0 \\
				0 & 1 & 0 \\
				0 & 0 & 0 	
			\end{pmatrix}
			$$
		This matrix equals its conjugate transpose, even though T is not
		self-adjoint. Explain why this is not a contradiction.
		\\ \\
		This is an assumption that the basis $(1, x, x^2)$ is an orthonormal basis.  Notice:
			\begin{align*}
			\langle 1 + x + x^2, 1 + x + x^2 \rangle &= \int_{0}^{1} x^4 + 2 x^3 + 3 x^2 + 2 x + 1 \\
			&= \left. \frac{x^5}{5} + \frac{x^4}{2} + x^3 + x^2 + x \right|^1_0 \\
 			&= 3.7 \not = 0		
 			\end{align*}
 		Because the inner product with itself is not 0, it is not an orthonormal basis, thus allowing the matrix to be equal with its conjugate transpose.
	\end{enumerate}	
	
\newpage 

\noindent \textbf{Problem 7.A.14: }Suppose T is a normal operator on V. Suppose also that $v, w \in V$ satisfy
the equations
	\begin{align*}
		||v|| = ||w|| = 2 && Tv = 3v && Tw = 4w
	\end{align*}
Show that $|| T(v+w) || = 10$
\\ \\
Because $||Tv|| \not= ||Tw||$, we know that $v,w$ are orthogonal, or form a right triangle, in which we can use the Pythagorean theorem:
	\begin{align*}
		\sqrt{||T(v+w)^2}|| &= \sqrt{||Tv + Tw||^2} \\
		&= \sqrt{||3v + 4w||^2} \\
		&= \sqrt{||3v||^2+ ||4w||^2} \\ 
		&= \sqrt{9||v||^2+ 16||w||^2} \\
		&= \sqrt{9(4) + 16(4)} = 10
	\end{align*}

\newpage 

\noindent \textbf{Problem 7.B.2: }Suppose that T is a self-adjoint operator on a finite-dimensional inner product space and that 2 and 3 are the only eigenvalues of T. Prove that $T^2 - 5T + 6I = 0$. 
\\ \\
Because $T$ is self-adjoint, there exists an orthonormal basis $e_1,...,e_n$ that consists of eigenvectors such that for any $i \in \{1,...,n\}$, $(T - \lambda I)e_i = 0$, for all eigenvalues, $\lambda$ , of T.
	\begin{align*}
		(T^2 - 5T + 6I)e_i = (T - 2I)(T - 3I)e_i = 0	
	\end{align*}
Because $e_i \not = 0$, then $T^2 - 5T + 6I = 0$.

\newpage 

\noindent \textbf{Problem 7.B.6: }Prove that a normal operator on a complex inner product space is self-adjoint if and only if all its eigenvalues are real.
\\ \\
	\begin{proof}[$(=>)$]
		Let T be a complex inner product space that is self-adjoint.
		\\ \\
		By 7.13, every eigenvalue of a self-adjoint operator is real.
		\\ \\
		$(<=)$ Let all the eigenvalues of some operator, T, be real.
		\\ \\
		Thus there exists a matrix in respect to its basis, $\mathcal{M}(T)$, with its eigenvalues on its diagonals. Thus $\mathcal{M}(T)$ equals its transpose, thus T is self-adjoint. 
	\end{proof}




















	
	
	
	
	
	
	
	
	
	
	
	
\end{document}
