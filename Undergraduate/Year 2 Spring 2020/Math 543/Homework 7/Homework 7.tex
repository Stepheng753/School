\documentclass[12pt]{article}
\usepackage[margin = 1in]{geometry}
\usepackage{amsmath}
\usepackage{amssymb}
\usepackage{amsthm}
\usepackage{graphicx}
\usepackage{subfig}
\usepackage{enumitem}

\begin{document}
	
	\begin{center}
		\textbf{Homework 7} \\
		\textbf{Numerical Matrix Analysis} \\
		\textbf{Math 543} \\
		\textbf{Stephen Giang} \\
	\end{center}

\noindent \textbf{Problem 24.3: }
\\ \\
Results I am able to see is that the two functions, $||e^{tA}||_2, e^{t\alpha(A)}$, are almost identical on a log scale. However, we see that $||e^{tA}||_2$ either looks like an exponential or an oscillating solution. I am also able to see that for matrices, A, with lesser real eigenvalues, the more likely it was to be oscillating as $t \rightarrow \infty$.  For matrices with all complex eigenvalues, it would look more and more like the straight line $e^{t\alpha A}$. On page (4-5), you can see an example of an oscillating function with only 2/10 real eigenvalues.

\end{document}
