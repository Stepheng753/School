\documentclass[11pt]{article}
\usepackage[margin = 1in]{geometry}
\usepackage{amsmath}
\usepackage{amssymb}
\usepackage{amsthm}
\usepackage{graphicx}
\usepackage{enumitem}
\usepackage{url}
\usepackage[parfill]{parskip}
\usepackage{listings}
\newcommand{\skipline}{\vspace{\baselineskip}}
\newcommand{\spacer}{\noalign{\medskip}}
\newenvironment{problem}[1]{\textbf{Problem #1: }}{\newpage}
\usepackage{caption}
\usepackage{subcaption}
\usepackage[utf8]{inputenc}
\usepackage{xcolor}
\definecolor{codegreen}{rgb}{0,0.6,0}
\definecolor{codegray}{rgb}{0.5,0.5,0.5}
\definecolor{codepurple}{rgb}{0.58,0,0.82}
\definecolor{backcolour}{rgb}{0.95,0.95,0.92}
\lstdefinestyle{mystyle}{
	backgroundcolor=\color{backcolour},   
	commentstyle=\color{codegreen},
	keywordstyle=\color{magenta},
	numberstyle=\tiny\color{codegray},
	stringstyle=\color{codepurple},
	basicstyle=\ttfamily\footnotesize,
	breakatwhitespace=false,         
	breaklines=true,                 
	captionpos=b,                    
	keepspaces=true,                 
	numbers=left,                    
	numbersep=5pt,                  
	showspaces=false,                
	showstringspaces=false,
	showtabs=false,                  
	tabsize=2
}
\lstset{style=mystyle}
\newcommand{\codefont}[1]{{\small \fontfamily{qcr}\selectfont #1}}
\newcommand{\wspace}{\text{ }}

\begin{document}
	
	\begin{center}
		\textbf{Homework 4} \\
		\textbf{Programming Languages} \\
		\textbf{CS 320} \\
		\textbf{Stephen Giang RedID: 823184070} \\
		\skipline \skipline
	\end{center}

	\begin{enumerate}[label = (\arabic*)]
		\item Give a Scheme expression for a list containing the values such that the number 7 is the first element,	the symbol \codefont{a} the second, the number -12 the third, and the symbol ??? the fourth.
		\\ \\
		\codefont{\textbf{'(7 a -12 ???)}}
		\item  Scheme has a function \codefont{symbol?} that takes one argument and returns \codefont{\#t} or \codefont{\#f} depending on whether
		the argument is a symbol or not. Give a Scheme expression that determines whether the first element of
		a three-element list is a symbol. (The result of the evaluation should be \codefont{\#t}.)
		\\ \\
		\codefont{\textbf{(symbol? (car '(? a b c) ) )}}
		\item Suppose that into the Scheme evaluator we enter the definition: \codefont{(define dozen 12)}
		Give the value of the Scheme expression: \codefont{(number? dozen)}
		\\ \\
		\codefont{\textbf{\#t}}
		\item Give two ways to write a Scheme expression of which the value is a one-element list with the empty	list as its element.
		\\ \\
		\codefont{\textbf{'( '() )} \quad \& \quad \textbf{'( () )}} 
		\item Give a Scheme definition that will determine the sum of 30, 31, 30, 15 and bind the identifier
		\codefont{days-in-semester} to this sum. Be as concise as possible.
		\\ \\
		\codefont{\textbf{(define days-in-semester (+ 30 31 30 15))}}
		\item Evaluate this Scheme statement. (Do this by hand first; then check your answer by running it.) \\
		\codefont{((lambda (x) (+ x x)) 4)}
		\\ \\
		\codefont{\textbf{4}} 
		\item Evaluate these Scheme statements. (Do this by hand first; then check your answer by running it.) \\
		\codefont{(define reverse-subtract} \\ \codefont{\wspace (lambda (x y)} \\ \codefont{\wspace\wspace\wspace(- y x)))} \\ \codefont{(reverse-subtract 7 10)}
		\\ \\
		\codefont{\textbf{3}}
		\newpage 
		\item  These expressions form repetitions. Trace by hand the following loop and show the set of two lists it
		produces (in other words, the result will be lists within a list). \\
		\codefont{(let loop \\
		\wspace\wspace\wspace\wspace\wspace((numbers '(3 -2 1 6 -5)) \\
		\wspace\wspace\wspace\wspace\wspace\wspace(nonneg '()) \\
		\wspace\wspace\wspace\wspace\wspace\wspace(neg '())) \\
		\wspace\wspace(cond ((null? numbers) \\
		\wspace\wspace\wspace\wspace\wspace\wspace\wspace\wspace\wspace(list nonneg neg)) \\
		\wspace\wspace\wspace\wspace\wspace\wspace\wspace\wspace((>= (car numbers) 0) \\
		\wspace\wspace\wspace\wspace\wspace\wspace\wspace\wspace\wspace(loop (cdr numbers) \\
		\wspace\wspace\wspace\wspace\wspace\wspace\wspace\wspace\wspace\wspace\wspace\wspace\wspace\wspace\wspace(cons (car numbers) nonneg) \\
		\wspace\wspace\wspace\wspace\wspace\wspace\wspace\wspace\wspace\wspace\wspace\wspace\wspace\wspace\wspace neg)) \\
		\wspace\wspace\wspace\wspace\wspace\wspace\wspace\wspace (else \\
		\wspace\wspace\wspace\wspace\wspace\wspace\wspace\wspace\wspace\wspace loop (cdr numbers) \\
		\wspace\wspace\wspace\wspace\wspace\wspace\wspace\wspace\wspace\wspace\wspace\wspace\wspace\wspace\wspace nonneg \\
		\wspace\wspace\wspace\wspace\wspace\wspace\wspace\wspace\wspace\wspace\wspace\wspace\wspace\wspace\wspace\wspace (cons (car numbers) neg)))))}
		\\ \\
		\codefont{\textbf{((6 1 3) (-5 -2))}}
		\item Here is a function called "\codefont{deleteall}" that strips out all occurrences of a value in a given list. The call to the function: (deleteall 0 '(4 0 3 0 2 0 1)) results in the list (4 3 2 1). \\
		\codefont{
			(define (deleteall atm list) \\
			\wspace\wspace\wspace (cond \\
			\wspace\wspace\wspace\wspace\wspace\wspace( (NULL? list) '() ) \\
			\wspace\wspace\wspace\wspace\wspace\wspace( (EQ? atm (CAR list)) (deleteall atm (CDR list))) \\
			\wspace\wspace\wspace\wspace\wspace\wspace( ELSE (CONS (CAR list) (deleteall atm (CDR list)))) \\
			))}
		\\ \\
		Write a recursive Scheme procedure called "tally" that counts the number of occurrences of a value in a
		given list. Do not use a counter variable, but let the function result in the desired count. For example,	the value of \codefont{(tally 0 '(4 0 3 0 2 0 1))} will be 3. The procedure will be very similar to \codefont{deleteall}.
		\\ \\
		\codefont{\textbf{
		(define (tally atm list) \\
		\wspace\wspace\wspace(cond \\
		\wspace\wspace\wspace( (null? list) 0 ) \\
		\wspace\wspace\wspace( (eq? atm (car list)) (+ 1 (tally atm (cdr list))) ) \\
		\wspace\wspace\wspace( else (tally atm (cdr list))) \\
		))
		}}
		\newpage
		\item The binding construct \codefont{let} defines Scheme block structure. In a \codefont{let} expression, the initial values are computed before any of the variables become bound. Evaluate these expressions by hand showing the effects of lexical binding in Scheme. \\ \\
		\codefont{
		A. (let ((x 2) (y 3)) \\
		\wspace\wspace\wspace\wspace\wspace\wspace\wspace\wspace\wspace\wspace(* x y))}
		\\ \\
		\codefont{\textbf{6}}
		\\ \\
		\codefont{
		B. (let ((x 2) (y 3)) \\
		\wspace\wspace\wspace\wspace\wspace\wspace(let ((foo (lambda (z) (+ x y z))) \\ 
		\wspace\wspace\wspace\wspace\wspace\wspace\wspace\wspace\wspace\wspace\wspace(x 7)) \\
		\wspace\wspace; call foo \\
		\wspace\wspace\wspace(foo 4)))}
		\\ \\
		\codefont{\textbf{9}}
	\end{enumerate}


\end{document}
