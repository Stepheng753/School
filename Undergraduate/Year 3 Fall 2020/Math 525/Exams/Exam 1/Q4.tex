\documentclass[11pt]{article}
\usepackage[margin = 1in]{geometry}
\usepackage{amsmath}
\usepackage{amssymb}
\usepackage{amsthm}
\usepackage{graphicx}
\usepackage{subfig}
\usepackage{enumitem}
\usepackage{url}
\usepackage[parfill]{parskip}
\newcommand{\skipline}{\vspace{\baselineskip}}
\newenvironment{problem}[1]{\textbf{Problem #1: }}{\newpage}


\begin{document}
	
	\begin{center}
		\textbf{Exam 1} \\
		\textbf{Algebraic Coding Theory} \\
		\textbf{Math 525} \\
		\textbf{Stephen Giang RedID: 823184070} \\
	\end{center}

	\begin{problem}{4}
	Let $C$ be the code: 
	\[C = \{ 000100, 101010, 001001 \}\]
	\begin{enumerate}
		\item Determine the error patterns that Theorem 1.12.9 guarantees that C corrects.
		\\ \\
		Notice the minimum distance, $d = 3$.  By Theorem 1.12.9, "A code of distance $d$ will correct all error patterns of weight $\leq \lfloor \frac{d-1}{2}\rfloor$".  So $C$ will correct all error patterns of weights $\leq 1$. That is:
		\[e = \{000000, 100000, 010000, 001000, 000100, 000010, 000001\}\]
		\item Use the technique described in Example 1.12.11 to decide whether or not $C$ corrects the error pattern $110000$.
		\\ \\
		Notice the IMLD table: \\ \\
		\begin{tabular}{| c | c | c | c | c |}
			\hline
			$w$ & 000100 + $w$ & 101010 + $w$ & 001001 + $w$ & $v$ \\
			\hline
			110100 &  $11000^*$ & 011110 & 111101 & 000100 \\
			\hline
			011010 &  011110 & $110000^*$ & 010011 & 101010 \\
			\hline
			111001 & 111101 & 010011 & $11000^*$ & 001001 \\
			\hline
		\end{tabular}
		\\ \\ \\
		So $C$ does in fact correct $u = 110000$
	\end{enumerate}
\end{problem}




\end{document}
