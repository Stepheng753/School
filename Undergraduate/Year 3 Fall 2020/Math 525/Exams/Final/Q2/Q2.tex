\documentclass[11pt]{article}
\usepackage[margin = 1in]{geometry}
\usepackage{amsmath}
\usepackage{amssymb}
\usepackage{amsthm}
\usepackage{graphicx}
\usepackage{enumitem}
\usepackage{url}
\usepackage[parfill]{parskip}
\usepackage{listings}
\newcommand{\skipline}{\vspace{\baselineskip}}
\newcommand{\spacer}{\noalign{\medskip}}
\newcommand{~}{\sim}
\newenvironment{problem}[1]{\textbf{Problem #1: }}{\newpage}
\usepackage{caption}
\usepackage{subcaption}
\usepackage[utf8]{inputenc}
\usepackage{xcolor}
\definecolor{codegreen}{rgb}{0,0.6,0}
\definecolor{codegray}{rgb}{0.5,0.5,0.5}
\definecolor{codepurple}{rgb}{0.58,0,0.82}
\definecolor{backcolour}{rgb}{0.95,0.95,0.92}
\lstdefinestyle{mystyle}{
	backgroundcolor=\color{backcolour},   
	commentstyle=\color{codegreen},
	keywordstyle=\color{magenta},
	numberstyle=\tiny\color{codegray},
	stringstyle=\color{codepurple},
	basicstyle=\ttfamily\footnotesize,
	breakatwhitespace=false,         
	breaklines=true,                 
	captionpos=b,                    
	keepspaces=true,                 
	numbers=left,                    
	numbersep=5pt,                  
	showspaces=false,                
	showstringspaces=false,
	showtabs=false,                  
	tabsize=2
}
\lstset{style=mystyle}

\begin{document}
	
	\begin{center}
		\textbf{Final} \\
		\textbf{Algebraic Coding Theory} \\
		\textbf{Math 525} \\
		\textbf{Stephen Giang RedID: 823184070} \\
		\skipline \skipline
	\end{center}

	\begin{problem}{2}
		Before starting this problem, you will need to obtain two words of length four, namely,$s_1$ and $s_3$, as follows.  If the last three letters in your last name are one of:
		\[s_1 = 0011 \qquad s_3 = 1011\]
		Consider the field GF($2^4$) constructed from $1 +x+x^4$, see Table 5.1, p.  114.  Let $C_{15}$ be the BCH code of length 15 with generator polynomial $g(x) =m_1(x)\cdot m_3(x)$ where $m_1(x)$ and $m_3(x)$ are the minimal polynomials of $\beta$ and $\beta^3$, respectively, with $\beta$ a primitive element of GF($2^4$),exactly as in Table 5.1.  Suppose messages are encoded using $C_{15}$ and a certain received vector $r$ has syndrome equal to $[s_1,s_3]$.  Determine the location of the errors (if any) in $r$.  Note:  Each location is an integer in [0..14].
		\begin{enumerate}[label = (\arabic*)]
			\item $s = [s_1, s_3] = [0011, 1011] = [\beta^6, \beta^{13}]$
			\item $s_1 \not= 0$ and $s_3 \not = s_1^3$
			\begin{align*}
				x^2 + s_1x + \bigg(\frac{s^3}{s_1} + s_1^2\bigg) &= 0 \\
				x^2 + \beta^6 x + \bigg(\beta^{7} + \beta^{12}\bigg) &= 0 \\
				x^2 + \beta^6x + \beta^{2} &= 0
			\end{align*}
			Notice the following:
			\[\beta^6 = \beta^{7} + \beta^{10} \text{ and } \beta^7 \cdot \beta^{10} = \beta^{17} = \beta^2 \]
			So $e(x) = x^2 + x^3$
			\item So locations are 3 and 4
		\end{enumerate}
	\end{problem}


\end{document}
