\documentclass[11pt]{article}
\usepackage[margin = 1in]{geometry}
\usepackage{amsmath}
\usepackage{amssymb}
\usepackage{amsthm}
\usepackage{graphicx}
\usepackage{subfig}
\usepackage{enumitem}
\usepackage{url}
\usepackage[parfill]{parskip}
\newcommand{\cvec}[2]{\begin{pmatrix} #1 \\ #2 \end{pmatrix}}
\newcommand{\smat}[4]{\begin{pmatrix} #1 & #2 \\ #3 & #4 \end{pmatrix}}
\newcommand{\bmath}[1]{\text{\boldmath#1\unboldmath}}
\newcommand{\skipline}{\vspace{\baselineskip}}
\newenvironment{problem}[1]{\textbf{Problem #1: }}{\newpage}


\begin{document}
	
	\begin{center}
		\textbf{Homework 3} \\
		\textbf{Ordinary Differential Equations} \\
		\textbf{Math 537} \\
		\textbf{Stephen Giang RedID: 823184070} \\
		\skipline \skipline
	\end{center}

	\begin{problem}{1}
		Consider the following system:
		\[X' = AX, \tag{1.1}\]
		where 
		\[A = \smat{-5}{2}{2}{-2} \text{ and } X = \cvec{x}{y}\]
		\skipline 
		\begin{enumerate}[label = (\alph*)]
			\item  Solve for eigenvalue(s) and eigenvector(s).
			\\ \\
			Notice we can get the characteristic equation from $A - \lambda I$:
			\begin{align*}
				(\lambda + 5)(\lambda  + 2) - 4 &= 0 \\
				\lambda^2 + 7\lambda + 10 - 4 &= 0 \\
				\lambda + 7\lambda + 6 &= 0 \\
				(\lambda + 6)(\lambda + 1) &= 0 \\
				\lambda &= -6, -1
			\end{align*}
			Notice the eigenvectors found from $A - \lambda I$ with \bmath{$\lambda_1 = -6$} and \bmath{$\lambda_2 = -1$}: \\
			\begin{align*}
				\smat{-5 - \lambda_1}{2}{2}{-2 - \lambda_1}\cvec{x}{y} &= \smat{1}{2}{2}{4}\cvec{x}{y} = \cvec{0}{0}, & v_1 = \cvec{x}{y} = \bmath{$\cvec{2}{-1}$} \\
				\smat{-5 - \lambda_2}{2}{2}{-2 - \lambda_2}\cvec{x}{y} &= \smat{-4}{2}{2}{-1}\cvec{x}{y} = \cvec{0}{0}, & v_2 = \cvec{x}{y} = \bmath{$\cvec{1}{2}$}
			\end{align*}
			\item  Construct $T$ using the results from problem (1a) and calculate $T^{-1}AT$
			\\ \\
			Notice that the eigenvalues were real and different.  So we can construct T from the eigenvectors, such that:
			\[\bmath{$T = \smat{2}{1}{-1}{2}$}\]
			Notice $T^{-1}AT$:
			\begin{align*}
				T^{-1}AT &= \frac{1}{5}\smat{2}{-1}{1}{2}\smat{-5}{2}{2}{-2}\smat{2}{1}{-1}{2} \\
				&= \bmath{$\smat{-6}{0}{0}{-1}$}
			\end{align*}
			\newpage
			\item Let $X = T Y$ . Show
			\[Y' = (T^{-1}AT)Y,\tag{1.2}\]
			Here $Y$ is a column vector and its transpose is defined as $Y^T = (u, w)$. 
			\\ \\
			Notice the following:
			\[\cvec{u'}{w'} = \smat{-6}{0}{0}{-1} \cvec{u}{w} = \cvec{-6u}{-w}\]
			Because we have that \bmath{$u' = \lambda_1 u $} and \bmath{$w' =\lambda_2 w$}, we have shown the above statement to be true.
			\skipline
			\item Solve Eq. (1.2) for $Y$.
			\\ \\
			We can see the eigenvalues because $T^{-1}AT$ is an upper triangular matrix. So we get that $\lambda_1 = -6$ and $\lambda_2 = -1$.  We can also easily see the eigenvectors being $v_1 = \cvec{1}{0}, v_2 = \cvec{0}{1}$.
			\\ \\
			So we get
			\[\bmath{$Y = Ae^{-6t}\cvec{1}{0} + Be^{-t}\cvec{0}{1}$}\]
			\item Find the solution $X$ to Eq. (1.1).
			\begin{align*} 
				X = TY &= \smat{2}{1}{-1}{2}\smat{Ae^{-6t}}{0}{0}{Be^{-t}} = \smat{2Ae^{-6t}}{Be^{-t}}{-Ae^{-6t}}{2Be^{-t}} \\
				&= \bmath{$Ae^{-6t}\cvec{2}{-1} + Be^{-t}\cvec{1}{2}$}
			\end{align*}
		\end{enumerate}
	\end{problem}

	\begin{problem}{2}
		Consider the following set of differential equations:
		\begin{align*}
			\frac{dx}{dt} &= y \\
			\frac{dy}{dt} &= -\omega^2x - by
		\end{align*}
		here both $b$ and $\omega$ are real.
		\begin{enumerate}[label = (\alph*)]
			\item Find the conditions under which the system is hyperbolic.
			\\ \\
			Notice we can rewrite the system as the following:
			\[\cvec{x'}{y'} = \smat{0}{1}{-\omega^2}{-b}\cvec{x}{y}\]
			Notice we can get the characteristic equation from $A - \lambda I$:
			\begin{align*}
				\lambda(\lambda + b) + \omega^2 &= 0 \\
				\lambda^2 + b\lambda + \omega^2 &= 0
			\end{align*}
			Notice the eigenvalues from the quadratic formula:
			\begin{align*}
				\lambda_1 &= \frac{-b + \sqrt{b^2 - 4\omega^2}}{2} \\
				\lambda_2 &= \frac{-b - \sqrt{b^2 - 4\omega^2}}{2}
			\end{align*}
			A system is hyperbolic if its matrix $A$ does not have any eigenvalues with real parts 0.  In this case, we get eigenvalues with real parts 0 if \bmath{$b = 0$} or \bmath{$\omega = 0$}, where the 'or' is an inclusive 'or'. So as long as the system does not have these conditions, the system is hyperbolic
			\item Discuss whether the system has a saddle point.
			\\ \\
			A saddle point occurs when the eigenvalues are real and have opposite signs.  To meet the real parameter, we have that $b^2 - 4\omega^2 \geq 0$.  From that, we also know that $0 \leq \sqrt{b^2 - 4w^2} \leq b$.  From this we have the following:
			\[ \frac{-b}{2} \leq \lambda_1 \leq 0,\qquad -b \leq \lambda_2 \leq \frac{-b}{2} \]
			Because we see that $\lambda_1$ and $\lambda_2$ never have opposite signs, \textbf{the system does not have a saddle point}.
		\end{enumerate}
	\end{problem}

	\begin{problem}{3}
		Consider the following two differential equations
		\begin{align*}
			x'' + ax' + bx &= 0 \\
			x'' + cx' + dx &= 0
		\end{align*}
		Show that the two systems are topologically conjugate when $a, b, c$ and $d$ are positive.
		\begin{proof}
			Notice we can rewrite the systems as the following when we let $y = x'$ with $a, b, c$ and $d$ being positive:
			\begin{align*}
				X' = \cvec{y}{y'} &= \smat{0}{1}{-b}{-a}\cvec{x}{x'} = Ax \tag{3.1} \\
				X' = \cvec{y}{y'} &= \smat{0}{1}{-d}{-c}\cvec{x}{x'} = Bx \tag{3.2}
			\end{align*}
			Notice we can get the characteristic equation of Eq (3.1) from $A - \lambda I$:
			\begin{align*}
				\lambda(\lambda + a) + b &= 0 \\
				\lambda^2 + a\lambda + b &= 0
			\end{align*} 
			Notice the eigenvalues from the quadratic formula:
			\begin{align*}
				\lambda_1 &= \frac{-a + \sqrt{a^2 - 4b}}{2}, &
				\lambda_2 &= \frac{-a - \sqrt{a^2 - 4b}}{2}
			\end{align*}
			Notice the three cases:
			\begin{enumerate}[label = (\arabic*)]
				\item $a^2 - 4b > 0$, We get that $0 < \sqrt{a^2 - 4b} < a$
				\begin{align*}
					\frac{-a}{2} < \lambda_1 &= \frac{-a + \sqrt{a^2 - 4b}}{2} < 0 \\
					-a < \lambda_2 &= \frac{-a - \sqrt{a^2 - 4b}}{2} < \frac{-a}{2}
				\end{align*}
				\item $a^2 - 4b = 0$, We get that $\sqrt{a^2 - 4b} = 0$
				\begin{align*}
				\lambda_1 &= \frac{-a + \sqrt{a^2 - 4b}}{2} = \frac{-a}{2} \\
				\lambda_2 &= \frac{-a - \sqrt{a^2 - 4b}}{2} = \frac{-a}{2}
				\end{align*}
				\item $a^2 - 4b < 0$, We get that $\sqrt{a^2 - 4b} < 0$
				\begin{align*}
					\lambda_1 &= \frac{-a + \sqrt{a^2 - 4b}}{2} = \frac{-a}{2} + i\sqrt{|a^2 - 4b|} \\
					\lambda_2 &= \frac{-a - \sqrt{a^2 - 4b}}{2} = \frac{-a}{2} - i\sqrt{|a^2 - 4b|}
				\end{align*}
			\end{enumerate}
			\skipline 
			Notice that in all three cases, we get that both eigenvalues do not have real parts 0 and have all negative real parts.  Without loss of generality, we can say the same for Eq (3.2).  So we get that $A$ and $B$ are hyperbolic. \textbf{Finally by the Theorem in Lecture 15, the two systems are conjugate as they both have the same number of eigenvalues (2) with negative real parts. }
		\end{proof}
	\end{problem}


\end{document}
