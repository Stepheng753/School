\documentclass[11pt]{article}
\usepackage[margin = 1in]{geometry}
\usepackage{amsmath}
\usepackage{amssymb}
\usepackage{amsthm}
\usepackage{graphicx}
\usepackage{subfig}
\usepackage{enumitem}
\usepackage{url}
\usepackage[parfill]{parskip}
\usepackage{listings}
\newcommand{\cvector}[2]{\begin{pmatrix} #1 \\ #2 \end{pmatrix}}
\newcommand{\smatrix}[4]{\begin{pmatrix} #1 & #2 \\ #3 & #4 \end{pmatrix}}
\newcommand{\bmath}[1]{\text{\boldmath#1\unboldmath}}
\newcommand{\skipline}{\vspace{\baselineskip}}
\newenvironment{problem}[1]{\textbf{Problem #1: }}{\newpage}


\begin{document}
	
	\begin{center}
		\textbf{Quiz 4} \\
		\textbf{Ordinary Differential Equations} \\
		\textbf{Math 537} \\
		\textbf{Stephen Giang RedID: 823184070} \\
		\skipline \skipline
	\end{center}

	\begin{problem}{1}
		Consider the Logistic equation:
		\[\frac{dX}{dt} = rX(1 - X). \tag{1.1}\]
		\begin{enumerate}[label = (\alph*)]
			\item Assume a time step $\Delta t$ and apply the Euler method to derive a discrete equation where $X_{n+1}$ can be computed from $X_n$.
			\[\boldsymbol{X_{n+1} = X_n + \Delta t \, \left(rX_n(1-X_n)\right)}\]
			\item Introduce a new variable $Y$ and transform the above discrete equation into the following equation:
			\[Y_{n+1} = \rho Y_n(1 - Y_n). \tag{1.2}\]
			Express $Y_n$ in terms of $X_n$ and find $\rho$.
			\\ \\
			Notice the following:
			\begin{align*}
				X_{n+1} &= (1 + \Delta t \,r) X_n - \Delta t \,r X_n^2 \\
				&= (1 + \Delta t \,r) X_n \left(1 - \frac{\Delta t\,r}{1 + \Delta t \,r}X_n\right) \\
				&= \frac{(1 + \Delta t \,r)^2}{\Delta t \,r} \frac{\Delta t\,r}{1 + \Delta t \,r}X_n \left(1 - \frac{\Delta t\,r}{1 + \Delta t \,r}X_n\right) 
			\end{align*}
			Let the following be true:
			\begin{align*}
				Y_n = \frac{\Delta t\,r}{1 + \Delta t \,r}X_n, \qquad \text{ such that } \qquad X_{n+1} = \frac{1 + \Delta t \,r}{\Delta t\,r}Y_{n+1}
			\end{align*}
			So we get:
			\begin{align*}
				\frac{1 + \Delta t \,r}{\Delta t\,r}Y_{n+1} &= \frac{(1 + \Delta t \,r)^2}{\Delta t \,r}Y_n \left(1 - Y_n\right) \\
				Y_{n+1} &= (1 + \Delta t\,r)Y_n(1 - Y_n)
			\end{align*}
			Thus we get the final solutions being:
			\[\boldsymbol{Y_n = \frac{\Delta t\,r}{1 + \Delta t \,r}X_n \qquad \rho = 1 + \Delta t\,r}\]
		\end{enumerate}
		Eq. (1.2) is called the Logistic map that possesses chaotic solutions for large values of $\rho$
	\end{problem}

	\begin{problem}{2}
		 Consider the general first-order ODE:
		 \[x' = f(x). \tag{2}\]
		 When both $f$ and $f'$ are zero at the critical point, the stability is determined by the sign of the first non-vanishing higher derivatives. Apply Taylor series expansions and provide simple functions $f(x)$ to illustrate the following:
		 \begin{enumerate}[label = (\alph*)]
		 	\item If the first non-vanishing higher derivative is even (e.g., $f''$), the point is a saddle point, attracting on one side but repelling on the other.
		 	\item If that derivative is odd, it follows the same sign rules as $f'$.
		 \end{enumerate}
	 	Notice the following:
	 	\begin{align*}
	 		x' = f(x) = f(x_c) + f'(x_c)(x-x_c) + f''(x_c)\frac{(x-x_c)^2}{2} + f'''(x_c)\frac{(x-x_c)^3}{6}+ ...
	 	\end{align*}
 		Notice the following examples:
 		\begin{enumerate}[label = (\alph*)]
 			\item $f(x) = x^2$, $x_c = 0$
 			\begin{align*}
 				x' &= f(0) + f'(0)x + f''(0)\frac{x^2}{2} + f'''(0)\frac{x^3}{6}+ \cdots + = x^2
 			\end{align*}
 			Notice for $x_c < 0$, $x' > 0$ and $x_c > 0$, $x' > 0$.  Notice for $x_c < 0$, the phase portrait is attracted to the critical point. Notice for $x_c > 0$, the phase portrait is repelled to the critical point.  Thus we get a saddle point for $x' = x^2$, where the first non-vanishing derivative is even ($2^{nd}$). 
 			\\ \\
 			Notice for all functions with the first non-vanishing higher derivative being even, we get:
 			\[x' = f^{(2n)}(x_c) \frac{(x-x_c)^{2n}}{(2n)!} + \cdots, \text{ where } f^{(2n)}(x_c) = C, \text{ and } n \in \mathbb{Z}\backslash\{0,1\} \]
 			Now we can see that the exponent on $x-x_c$ being an even number, means that for all values of $x \not= x_c$, we get that $\frac{(x-x_c)^{2n}}{(2n)!} > 0$.  Thus we get both sides of the critical point being the same sign, meaning attracting one way and repelling the other, which is known as a saddle point.
 			 \item $f(x) = x^3$, $x_c = 0$
 			\begin{align*}
 				x' &= f(0) + f'(0)x + f''(0)\frac{x^2}{2} + f'''(0)\frac{x^3}{6}+ \cdots + = x^3
 			\end{align*}
 			Notice for $x_c < 0$, $x' < 0$ and $x_c > 0$, $x' > 0$, we get a source.  For $f'''(x_c) > 0$, we get an unstable source, and for $f'''(x_c) < 0$, we get a stable sink.  This follows the same sign rules as f'.
 			\\ \\
 			Notice for all functions with the first non-vanishing higher derivative being odd, we get:
 			\[x' = f^{(2n + 1)}(x_c) \frac{(x-x_c)^{2n + 1}}{(2n + 1)!} + \cdots, \text{ where } f^{(2n + 1)}(x_c) = C, \text{ and } n \in \mathbb{Z}\backslash\{0,1\} \]
 			Now we can see that the exponent on $x-x_c$ being an odd number, means that for values of $x < x_c$, we get that $\frac{(x-x_c)^{2n + 1}}{(2n + 1)!} < 0$ and for values of $x > x_c$, we get that $\frac{(x-x_c)^{2n + 1}}{(2n + 1)!} > 0$. Thus we get each side being opposite signs of each other.  If $f^{(2n + 1)}(x_c) < 0$, we get for $x < x_c$, $x' > 0$ and for $x > x_c$, $x' < 0$. We get the reverse for $f^{(2n + 1)}(x_c) > 0$.  Thus this follows the same sign rules as $f'$
 		\end{enumerate}
	\end{problem}


\end{document}
