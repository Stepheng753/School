\documentclass[11pt]{article}
\usepackage[margin = 1in]{geometry}
\usepackage{amsmath}
\usepackage{amssymb}
\usepackage{amsthm}
\usepackage{graphicx}
\usepackage{subfig}
\usepackage{enumitem}
\usepackage{url}
\usepackage[parfill]{parskip}
\usepackage{listings}
\newcommand{\cvector}[2]{\begin{pmatrix} #1 \\ #2 \end{pmatrix}}
\newcommand{\smatrix}[4]{\begin{pmatrix} #1 & #2 \\ #3 & #4 \end{pmatrix}}
\newcommand{\skipline}{\vspace{\baselineskip}}
\newenvironment{problem}[1]{\textbf{Problem #1: }}{\newpage}


\begin{document}
	
	\begin{center}
		\textbf{Quiz 5} \\
		\textbf{Ordinary Differential Equations} \\
		\textbf{Math 537} \\
		\textbf{Stephen Giang RedID: 823184070} \\
		\skipline \skipline
	\end{center}

	\begin{problem}{1}
		 Consider the following a $3 \times 3$ matrices with repeated eigenvalue:
		 \[\begin{pmatrix}
		 	\lambda & 0 & 0 \\
		 	0 & \lambda & 0 \\
		 	0 & 0 & \lambda
		 \end{pmatrix}, \begin{pmatrix}
			 \lambda & 1 & 0 \\
			 0 & \lambda & 0 \\
			 0 & 0 & \lambda
		 \end{pmatrix}, \begin{pmatrix}
			 \lambda & 1 & 0 \\
			 0 & \lambda & 1 \\
			 0 & 0 & \lambda
		 \end{pmatrix}\]
	 	\begin{enumerate}[label = (\alph*)]
	 		\item  Determine the dimensions of the kernel and range for each
	 		of the above matrices.
	 		\\ \\
	 		Notice for the case 1 matrix, we have 
	 		\[(A - \lambda I) = 0\]
	 		So we get that $\boldsymbol{dim(Ker(T)) = 3, dim(Range(T)) = 0}$
	 		\\ \\
	 		Notice for the case 2 matrix, we have 
	 		\[(A - \lambda I)V_1 = 0, \qquad (A - \lambda I)V_2 = V_1\]
	 		So we get that $\boldsymbol{dim(Ker(T)) = 2, dim(Range(T)) = 1}$
	 		\\ \\
	 		Notice for the case 3 matrix, we have 
	 		\[(A - \lambda I)V_1 = 0, \qquad (A - \lambda I)V_2 = V_1, \qquad (A - \lambda I)^2 V_3 = V_1\]
	 		So we get that $\boldsymbol{dim(Ker(T)) = 1, dim(Range(T)) = 2}$
	 		\newpage
	 		\item Provide examples for each of the above cases.
	 		\\ \\
	 		Notice $A = \begin{pmatrix}
	 			2 & 0 & 0 \\
	 			0 & 2 & 0 \\
	 			0 & 0 & 2
	 		\end{pmatrix}$.  When we multiply $T^{-1}AT$, where $T$ is the matrix with its columns being the eigenvectors of $A$.  We get an uncoupled matrix, with $\lambda = 2$, such that $T^{-1}AT = \begin{pmatrix}
	 			2 & 0 & 0 \\ 0 & 2 & 0 \\ 0 & 0 & 2 
	 		\end{pmatrix}$, which is in the form of the case 1 matrix.
	 		\\ \\ \\ \\
	 		Notice $A = \begin{pmatrix}
	 			1 & 1 & 0 \\
	 			-1 & 3 & 0 \\
	 			-1 & 1 & 2
	 		\end{pmatrix}$.  When we multiply $T^{-1}AT$, where $T$ is the matrix with its columns being the eigenvectors of $A$.  We get a matrix, with $\lambda = 2$, such that $T^{-1}AT = \begin{pmatrix}
	 			2 & 1 & 0 \\ 0 & 2 & 0 \\ 0 & 0 & 2 
	 		\end{pmatrix}$, which is in the form of the case 2 matrix.
 			\\ \\ \\ \\
	 		Notice $A = \begin{pmatrix}
	 			2 & 0 & -1 \\
	 			0 & 2 & 1 \\
	 			-1 & -1 & 2
	 		\end{pmatrix}$.  When we multiply $T^{-1}AT$, where $T$ is the matrix with its columns being the eigenvectors of $A$.  We get a matrix, with $\lambda = 2$, such that $T^{-1}AT = \begin{pmatrix}
	 			2 & 1 & 0 \\ 0 & 2 & 1 \\ 0 & 0 & 2 
 		\end{pmatrix}$, which is in the form of the case 3 matrix.
	 	\end{enumerate}
	\end{problem}


\end{document}
