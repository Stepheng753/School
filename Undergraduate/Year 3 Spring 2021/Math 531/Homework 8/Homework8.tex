\documentclass[11pt]{article}
\usepackage[margin = 1in]{geometry}
\usepackage{amsmath}
\usepackage{amssymb}
\usepackage{amsthm}
\usepackage{graphicx}
\usepackage{enumitem}
\usepackage{url}
\usepackage[parfill]{parskip}
\usepackage{listings}
\usepackage{caption}
\usepackage{subcaption}
\usepackage[utf8]{inputenc}
\usepackage{xcolor}
\definecolor{codegreen}{rgb}{0,0.6,0}
\definecolor{codegray}{rgb}{0.5,0.5,0.5}
\definecolor{codepurple}{rgb}{0.58,0,0.82}
\definecolor{backcolour}{rgb}{0.95,0.95,0.92}
\lstdefinestyle{mystyle}{
	backgroundcolor=\color{backcolour},   
	commentstyle=\color{codegreen},
	keywordstyle=\color{magenta},
	numberstyle=\tiny\color{codegray},
	stringstyle=\color{codepurple},
	basicstyle=\ttfamily\footnotesize,
	breakatwhitespace=false,         
	breaklines=true,                 
	captionpos=b,                    
	keepspaces=true,                 
	numbers=left,                    
	numbersep=5pt,                  
	showspaces=false,                
	showstringspaces=false,
	showtabs=false,                  
	tabsize=2
}
\lstset{style=mystyle}
\newcommand{\skipline}{\vspace{\baselineskip}}
\newcommand{\spacer}{\noalign{\medskip}}
\newcommand{~}{\sim}
\newcommand{\approches}{\rightarrow}
\newcommand{\qarrow}{\quad \rightarrow \quad}
\newcommand{\qqarrow}{\qquad \rightarrow \qquad}
\newcommand{\qtext}[1]{\quad \text{ #1 } \quad}
\newcommand{\qqtext}[1]{\qquad \text{ #1 } \qquad}
\newcommand{\pard}[2]{\frac{\partial #1}{\partial #2}}
\newcommand{\answer}[1]{\textbf{\boldmath #1}}
\newenvironment{problem}[1]{\textbf{Excersise #1: }}{\newpage}

\begin{document}
	
	\begin{center}
		\textbf{Homework 8} \\
		\textbf{Partial Differential Equations} \\
		\textbf{Math 531} \\
		\textbf{Stephen Giang RedID: 823184070} \\
		\skipline \skipline
	\end{center}

	\begin{problem}{7.7.1}
		Solve as simply as possible:
		\[\pard{^2u}{t^2} = c^2\nabla^2u\]
		with $u(a,\theta,t) = 0, u(r,\theta,0) = 0$ and $\pard{u}{t}(r,\theta,0) = \alpha(r)\sin3\theta$
		\\ \\
		Let the following be true:
		\[u(r,\theta,t) = f(r)g(\theta)h(t)\]
		From this, and our original equation, we get the following:
		\[\frac{d^2h}{dt^2} = -\lambda c^2 h \qquad \frac{d^2g}{d\theta^2} = -\mu g \qquad r\frac{d}{dr}\left(r\frac{df}{dr}\right) + \left(\lambda r^2 - \mu\right)f = 0\]
		Solving the second ODE, we get the following eigenvalues and eigenfunctions:
		\[\mu_m = m^2 \qquad g(\theta) = a_m\cos m\theta + b_m\sin m\theta\]
		Now we consider the third ODE, we use the product rule, substitute our value of $\mu_m$ and then using a simple scaling transformation ($z = \sqrt{\lambda}r$):
		\[z^2\frac{d^2f}{dz^2} + z\frac{df}{dz} + \left(z^2 - m^2\right)f = 0\]
		This is the Bessel's Differential Equation with solution:
		\[f(z) = c_1J_m(z) + c_2Y_m(z) \quad \Longrightarrow \quad f(r) = c_1J_m(\sqrt{\lambda}r) + c_2Y_m(\sqrt{\lambda}r)\] 
		We know that $|f(0)| < \infty$ and that $\lim\limits_{z \approches 0}Y_m(z) = \pm \infty$, we have that $c_2 = 0$. Now we use the fact that $f(a) = 0$:
		\[f(a) = c_1J_m(\sqrt{\lambda}a) = 0 \qarrow \lambda_{mn} = \left( \frac{z_{mn}}{a}\right)^2 \]
		where $z_{mn}$ represents the $n^{th}$ zero of $J_m(z)$.
		\\ \\
		Now we consider the first ODE:
		\begin{enumerate}[label = (\alph*)]
			\item If we let $\lambda = 0$, we get the following:
			\[h(t) = c_1t + c_2 \qarrow h(0) = c_2 = 0 \qqarrow h(t) = c_1t\]
			\item If we let $\lambda > 0$, we get the following:
			\[h(t) = c_1\cos c\sqrt{\lambda_{mn}}t + c_2\sin c\sqrt{\lambda_{mn}}t \qarrow h(0) = c_1 = 0 \qarrow h(t) = c_2\sin c\sqrt{\lambda_{mn}}t\]
		\end{enumerate}
		\newpage
		Notice our solution for $u(r,\theta,t)$:
		\begin{align*}
			u(r,\theta,t) &= c_1t + \sum_{m=0}^{\infty}\sum_{n=1}^{\infty}A_{mn}J_m\left(\sqrt{\lambda_{mn}}r\right)\cos m\theta\sin\left( c\sqrt{\lambda_{mn}}t\right) \\
			&+ \sum_{m=0}^{\infty}\sum_{n=1}^{\infty}B_{mn}J_m\left(\sqrt{\lambda_{mn}}r\right)\sin m\theta\sin\left( c\sqrt{\lambda_{mn}}t\right) 
		\end{align*}
		Now notice the derivative in respect to $t$:
		\begin{align*}
			\pard{u}{t}(r,\theta,t) &= c_1 + \sum_{m=0}^{\infty}\sum_{n=1}^{\infty}c\sqrt{\lambda_{mn}}A_{mn}J_m\left(\sqrt{\lambda_{mn}}r\right)\cos m\theta\cos\left( c\sqrt{\lambda_{mn}}t\right) \\
			&+ \sum_{m=0}^{\infty}\sum_{n=1}^{\infty}c\sqrt{\lambda_{mn}}B_{mn}J_m\left(\sqrt{\lambda_{mn}}r\right)\sin m\theta\cos\left( c\sqrt{\lambda_{mn}}t\right) 
		\end{align*}
		Now we include our last boundary condition:
		\begin{align*}
			\pard{u}{t}(r,\theta,0) = \alpha(r)\sin3\theta &= c_1 + \sum_{m=0}^{\infty}\sum_{n=1}^{\infty}c\sqrt{\lambda_{mn}}A_{mn}J_m\left(\sqrt{\lambda_{mn}}r\right)\cos m\theta \\
			&+ \sum_{m=0}^{\infty}\sum_{n=1}^{\infty}c\sqrt{\lambda_{mn}}B_{mn}J_m\left(\sqrt{\lambda_{mn}}r\right)\sin m\theta
		\end{align*}
		From this, we get that $c_1 = 0$ for all $m$ with $A_{mn} = 0$ and $B_{mn} = 0$ for all $m \not= 3$.  From here, we solve for $\alpha(r)$:
		\[\alpha(r) = \sum_{n=1}^\infty c\sqrt{\lambda_{3n}}B_{3n}J_3\left(\sqrt{\lambda_{3n}}r\right)\]
		Using the orthogonality of the Bessel functions with weight $r$, we get the following coefficient:
		\[B_{3n} = \frac{\int_{0}^{a} \alpha(r)J_3\left(\sqrt{\lambda_{3n}}r\right)r\,dr}{c\sqrt{\lambda_{3n}}\int_{0}^a J_3^2\left(\sqrt{\lambda_{3n}}r\right)r\,dr}\]
		Thus, we get the final solution:
		\[\boldsymbol{u(r,\theta,t) = \sum_{n=1}^{\infty}B_{3n}J_3\left(\sqrt{\lambda_{3n}}r\right)\sin 3\theta\sin\left( c\sqrt{\lambda_{3n}}t\right)} \]
	\end{problem}

	\begin{problem}{7.7.2a}
		Solve as simply as possible:
		\[\pard{^2u}{t^2} = c^2\nabla^2u \qtext{subject to} \pard{u}{r}(a,\theta,t) = 0\]
		with initial conditions:
		\[u(r,\theta,0) = 0, \qquad \pard{u}{t}(r,\theta,0) = \beta(r)\cos5\theta\]
		Let the following be true:
		\[u(r,\theta,t) = f(r)g(\theta)h(t)\]
		From this, and our original equation, we get the following:
		\[\frac{d^2h}{dt^2} = -\lambda c^2 h \qquad \frac{d^2g}{d\theta^2} = -\mu g \qquad r\frac{d}{dr}\left(r\frac{df}{dr}\right) + \left(\lambda r^2 - \mu\right)f = 0\]
		Solving the second ODE, we get the following eigenvalues and eigenfunctions:
		\[\mu_m = m^2 \qquad g(\theta) = a_m\cos m\theta + b_m\sin m\theta\]
		Now we consider the third ODE, we use the product rule, substitute our value of $\mu_m$ and then using a simple scaling transformation ($z = \sqrt{\lambda}r$):
		\[z^2\frac{d^2f}{dz^2} + z\frac{df}{dz} + \left(z^2 - m^2\right)f = 0\]
		This is the Bessel's Differential Equation with solution:
		\[f(z) = c_1J_m(z) + c_2Y_m(z) \quad \Longrightarrow \quad f(r) = c_1J_m(\sqrt{\lambda}r) + c_2Y_m(\sqrt{\lambda}r)\]
		We know that $|f(0)| < \infty$ and that $\lim\limits_{z \approches 0}Y_m(z) = \pm \infty$, we have that $c_2 = 0$. Now we use the fact that $f'(a) = 0$:
		\[f'(a) = c_1\sqrt{\lambda_{mn}}J_m'(\sqrt{\lambda_{mn}}a) = 0 \qarrow \lambda_{mn} = \left( \frac{z_{mn}}{a}\right)^2 \]
		where $z_{mn}$ represents the $n^{th}$ zero of $J_m'(z)$.
		\begin{enumerate}[label = (\alph*)]
			\item If we let $\lambda = 0$, we get the following:
			\[h(t) = c_1t + c_2 \qarrow h(0) = c_2 = 0 \qqarrow h(t) = c_1t\]
			\item If we let $\lambda > 0$, we get the following:
			\[h(t) = c_1\cos c\sqrt{\lambda_{mn}}t + c_2\sin c\sqrt{\lambda_{mn}}t \qarrow h(0) = c_1 = 0 \qarrow h(t) = c_2\sin c\sqrt{\lambda_{mn}}t\]
		\end{enumerate}
		\newpage
		Thus we get the following:
		\begin{align*}
			u(r,\theta,t) &= c_1t + \sum_{m=0}^\infty\sum_{n=1}^\infty J_m(\sqrt{\lambda_{mn}}r)\cos(m\theta)\left(B_{mn}\sin c\sqrt{\lambda_{mn}}t\right) \\
			&+ \sum_{m=0}^\infty\sum_{n=1}^\infty J_m(\sqrt{\lambda_{mn}}r)\sin(m\theta)\left(D_{mn}\sin c\sqrt{\lambda_{mn}}t\right) \\
			\pard{u}{t}(r,\theta,t) &= c_1 + \sum_{m=0}^\infty\sum_{n=1}^\infty J_m(\sqrt{\lambda_{mn}}r)\cos(m\theta)\left(B_{mn}c\sqrt{\lambda_{mn}}\cos c\sqrt{\lambda_{mn}}t\right) \\
			&+ \sum_{m=0}^\infty\sum_{n=1}^\infty J_m(\sqrt{\lambda_{mn}}r)\sin(m\theta)\left(D_{mn}c\sqrt{\lambda_{mn}}\cos c\sqrt{\lambda_{mn}}t\right) \\	
		\end{align*}
		Notice the boundary condition:
		\begin{align*}
			\pard{u}{t}(r,\theta,0) = \beta(r)\cos5\theta &= c_1 + \sum_{m=0}^\infty\sum_{n=1}^\infty J_m(\sqrt{\lambda_{mn}}r)\cos(m\theta)\left(B_{mn}c\sqrt{\lambda_{mn}}\right) \\
			&+ \sum_{m=0}^\infty\sum_{n=1}^\infty J_m(\sqrt{\lambda_{mn}}r)\sin(m\theta)\left(D_{mn}c\sqrt{\lambda_{mn}}\right)
		\end{align*}
		From here, we can see that:
		\[\beta(r) = \sum_{n=1}^\infty J_5(\sqrt{\lambda_{5n}}r)\left(B_{5n}c\sqrt{\lambda_{5n}}\right)\]
		With this, we can solve for the following coefficient:
		\[B_{5n} = \frac{\int_{0}^{a} \beta(r)J_5\left(\sqrt{\lambda_{5n}}r\right)r\,dr}{c\int_{0}^a J_5^2\left(\sqrt{\lambda_{5n}}r\right)r\,dr}\]
		Thus we get:
		\[\boldsymbol{u(r,\theta,t) = \sum_{n=1}^\infty J_5(\sqrt{\lambda_{5n}}r)\cos(5\theta)\left(B_{5n}\sin c\sqrt{\lambda_{5n}}t\right)}\]
	\end{problem}

	\begin{problem}{7.9.1c}
		Solve Laplace's equation inside a circular cylinder subject to the boundary conditions
		\[u(r,\theta,0) = 0, \qquad u(r,\theta,H) = \beta(r)\cos3\theta, \qquad \pard{u}{r}(a,\theta,z) = 0\]
		Let the following be true:
		\[u(r,\theta,z) = f(r)g(\theta)h(z)\]
		From this, and our original equation, we get the following:
		\[\frac{d^2h}{dt^2} = \lambda h \qquad \frac{d^2g}{d\theta^2} = -\mu g \qquad r\frac{d}{dr}\left(r\frac{df}{dr}\right) + \left(\lambda r^2 - \mu\right)f = 0\]
		Solving the second ODE, we get the following eigenvalues and eigenfunctions:
		\[\mu_m = m^2 \qquad g(\theta) = a_m\cos m\theta + b_m\sin m\theta\]
		Now we consider the third ODE, we use the product rule, substitute our value of $\mu_m$ and then using a simple scaling transformation ($z = \sqrt{\lambda}r$):
		\[z^2\frac{d^2f}{dz^2} + z\frac{df}{dz} + \left(z^2 - m^2\right)f = 0\]
		This is the Bessel's Differential Equation with solution:
		\[f(z) = c_1J_m(z) + c_2Y_m(z) \quad \Longrightarrow \quad f(r) = c_1J_m(\sqrt{\lambda}r) + c_2Y_m(\sqrt{\lambda}r)\]
		We know that $|f(0)| < \infty$ and that $\lim\limits_{z \approches 0}Y_m(z) = \pm \infty$, we have that $c_2 = 0$. Now we use the fact that $f'(a) = 0$:
		\[f'(a) = c_1\sqrt{\lambda_{0n}}J_m'(\sqrt{\lambda_{0n}}a) = 0 \qarrow \lambda_{mn} = \left( \frac{z_{mn}}{a}\right)^2 \]
		where $z_{mn}$ represents the $n^{th}$ zero of $J_m'(z)$.
		\begin{enumerate}[label = (\alph*)]
			\item If we let $\lambda = 0$, we get the following:
			\[h(z) = c_1z + c_2 \qarrow h(0) = c_2 = 0 \qarrow h(z) = c_1z\]
			\item If we let $\lambda > 0$, we get the following:
			\[h(z) = c_1\cosh(\sqrt{\lambda_{mn}}z) + c_2\sinh(\sqrt{\lambda_{mn}}z) \qarrow h(0) = c_1 = 0 \qarrow h(z) = c_2\sinh(\sqrt{\lambda_{mn}}z)\]
		\end{enumerate}
		\newpage
		Thus we get the following:
		\begin{align*}
			u(r,\theta,z) &= c_1z + \sum_{m=0}^\infty \sum_{n=1}^\infty A_{mn}J_m(\sqrt{\lambda_{mn}}r)\cos m\theta\sinh(\sqrt{\lambda_{mn}}z) \\
			&+ \sum_{m=0}^\infty \sum_{n=1}^\infty B_{mn}J_m(\sqrt{\lambda_{mn}}r)\sin m\theta\sinh(\sqrt{\lambda_{mn}}z)
		\end{align*}
		Notice the boundary condition:
		\begin{align*}
			u(r,\theta,H) = \beta(r)\cos3\theta &= c_1H + \sum_{m=0}^\infty \sum_{n=1}^\infty A_{mn}J_m(\sqrt{\lambda_{mn}}r)\cos m\theta\sinh(\sqrt{\lambda_{mn}}H) \\
			&+ \sum_{m=0}^\infty \sum_{n=1}^\infty B_{mn}J_m(\sqrt{\lambda_{mn}}r)\sin m\theta\sinh(\sqrt{\lambda_{mn}}H)
		\end{align*}
		From here we can see that:
		\[\beta(r) = \sum_{n=1}^\infty A_{3n}J_3(\sqrt{\lambda_{3n}}r)\sinh(\sqrt{\lambda_{3n}}H)\]
		With this, we can solve for the following coefficient:
		\[A_{3n} = \frac{\int_{0}^{a} \beta(r)J_3\left(\sqrt{\lambda_{3n}}r\right)r\,dr}{\int_{0}^a J_3^2\left(\sqrt{\lambda_{3n}}r\right)r\,dr}\]
		Thus we get:
		\[\boldsymbol{u(r,\theta,z) = \sum_{n=1}^\infty A_{3n}J_3(\sqrt{\lambda_{3n}}r)\cos 3\theta\sinh(\sqrt{\lambda_{3n}}z)}\]
	\end{problem}

	\begin{problem}{7.9.2b}
		Solve Laplace’s equation inside a semicircular cylinder, subject to the boundary conditions
		\begin{align*}
			u(r,\theta,0) = 0, &&&\pard{u}{z}(r,\theta,H) = 0, &&u(r,0,z) = 0, \\
			u(r,\pi,z) = 0, &&&u(a,\theta,z) = \beta(\theta,z) &&
		\end{align*}
		Let the following be true:
		\[u(r,\theta,z) = f(r)g(\theta)h(z)\]
		From this, and our original equation, we get the following:
		\[\frac{d^2h}{dt^2} = -\lambda h \qquad \frac{d^2g}{d\theta^2} = -\mu g \qquad r\frac{d}{dr}\left(r\frac{df}{dr}\right) - \left(\lambda r^2 + \mu\right)f = 0\]
		Solving the second ODE, we get the following eigenvalues and eigenfunctions with the boundary condition of $g(0) = g(\pi) = 0$:
		\[\mu_m = m^2 \qquad g(\theta) = b_{mn}\sin m\theta\]
		Notice the solution for $h(z)$:
		\begin{enumerate}[label = (\alph*)]
			\item If we let $\lambda = 0$, we get the following:
			\[h(t) = c_1z + c_2 \qarrow h(0) = c_2 = 0 \qquad h'(z) = c_1 \qarrow h'(H) = c_1 = 0 \qqarrow h(z) = 0 \]
			\item If we let $\lambda > 0$, we get the following: 
			\[h(z) = c_1\cos\sqrt{\lambda}z + c_2\sin\sqrt{\lambda}z \qquad h'(z) = -c_1\sqrt{\lambda}\sin\sqrt{\lambda}z + c_2\sqrt{\lambda}\cos\sqrt{\lambda}z\]
			\[h(0) = c_1 = 0 \qarrow h'(H) = c_2\sqrt{\lambda}\cos\sqrt{\lambda}H = 0 \qarrow \lambda_{n} = \left( \frac{(2n + 1)\pi}{2H}\right)^2\]
		\end{enumerate}
		Notice the Modified Bessel's Differential Equation with solution:
		\[f(r) = c_1K_m\left( \frac{(2n + 1)\pi \,r}{2H}\right)  + c_2I_m\left( \frac{(2n + 1)\pi \,r}{2H}\right)\]
		We know that $K_m$ is singular at $r = 0$ and that $I_m$ is not, we have that $c_1 = 0$.
		\[f(r) =  c_2I_m\left( \frac{(2n + 1)\pi \,r}{2H}\right)  \]
		\newpage
		Thus, we get the following:
		\[u(r,\theta,z) = \sum_{m=1}^\infty \sum_{n=0}^\infty B_{mn}I_m\left( \frac{(2n + 1)\pi \,r}{2H}\right)\sin m\theta\sin \left( \frac{(2n + 1)\pi\,z}{2H}\right)\]
		Now we can use our other boundary condition:
		\[u(a,\theta, z) = \beta(\theta, z) = \sum_{m=1}^\infty \sum_{n=0}^\infty B_{mn}I_m\left( \frac{(2n + 1)\pi \,a}{2H}\right)\sin m\theta\sin \left( \frac{(2n + 1)\pi\,z}{2H}\right)\]
		Using orthogonality of sines we get the following coefficient:
		\begin{align*}
			\int_{0}^H \int_{0}^\pi \beta(\theta, z)\sin m\theta&\sin \left( \frac{(2n + 1)\pi\,z}{2H}\right)\,d\theta dz \\
			&= \sum_{m=1}^\infty \sum_{n=0}^\infty B_{mn}I_m\left( \frac{(2n + 1)\pi \,a}{2H}\right)\int_{0}^\pi \sin^2 m\theta\,d\theta \int_{0}^H \sin^2 \left( \frac{(2n + 1)\pi\,z}{2H}\right)\,dz \\
			&= \sum_{m=1}^\infty \sum_{n=0}^\infty B_{mn}I_m\left( \frac{(2n + 1)\pi \,a}{2H}\right)\left(\frac{\pi}{2}\right)  \left(\frac{H}{2}\right) \\
			B_{mn} &= \frac{4}{\pi H I_m\left( \frac{(2n + 1)\pi \,a}{2H}\right)}\int_{0}^H \int_{0}^\pi \beta(\theta, z)\sin m\theta\sin \left( \frac{(2n + 1)\pi\,z}{2H}\right)\,d\theta dz
		\end{align*}
		
	\end{problem}


\end{document}
