\documentclass[11pt]{article}
\usepackage[margin = 1in]{geometry}
\usepackage{amsmath}
\usepackage{amssymb}
\usepackage{amsthm}
\usepackage{graphicx}
\usepackage{enumitem}
\usepackage{url}
\usepackage[parfill]{parskip}
\usepackage{listings}
\usepackage{caption}
\usepackage{subcaption}
\usepackage{mdframed}
\usepackage[utf8]{inputenc}
\usepackage{xcolor}
\definecolor{codegreen}{rgb}{0,0.6,0}
\definecolor{codegray}{rgb}{0.5,0.5,0.5}
\definecolor{codepurple}{rgb}{0.58,0,0.82}
\definecolor{backcolour}{rgb}{0.95,0.95,0.92}
\lstdefinestyle{mystyle}{
	backgroundcolor=\color{backcolour},   
	commentstyle=\color{codegreen},
	keywordstyle=\color{magenta},
	numberstyle=\tiny\color{codegray},
	stringstyle=\color{codepurple},
	basicstyle=\ttfamily\footnotesize,
	breakatwhitespace=false,         
	breaklines=true,                 
	captionpos=b,                    
	keepspaces=true,                 
	numbers=left,                    
	numbersep=5pt,                  
	showspaces=false,                
	showstringspaces=false,
	showtabs=false,                  
	tabsize=2
}
\lstset{style=mystyle}
\newcommand{\skipline}{\vspace{\baselineskip}}
\newcommand{\spacer}{\noalign{\medskip}}
\newcommand{~}{\sim}
\newcommand{\approches}{\rightarrow}
\newcommand{\qarrow}{\quad \rightarrow \quad}
\newcommand{\qqarrow}{\qquad \rightarrow \qquad}
\newcommand{\qtext}[1]{\quad \text{ #1 } \quad}
\newcommand{\qqtext}[1]{\qquad \text{ #1 } \qquad}
\newcommand{\pard}[2]{\frac{\partial #1}{\partial #2}}
\newcommand{\answer}[1]{\textbf{\boldmath #1}}
\newenvironment{problem}[1]{\textbf{Excersise #1: }}{\newpage}

\begin{document}
	
	\begin{center}
		\textbf{Homework 9} \\
		\textbf{Partial Differential Equations} \\
		\textbf{Math 531} \\
		\textbf{Stephen Giang RedID: 823184070} \\
		\skipline \skipline
	\end{center}

	% Page 367
	\begin{problem}{8.2.2}
		Consider the heat equation with time-dependent sources and boundary conditions
		\[\pard{u}{t} = k\pard{^2u}{x^2} + Q(x,t) \qquad u(x,0) = f(x)\]
		Reduce the problem to one with homogeneous boundary conditions if
		\begin{itemize}
			\item[\textbf{(b)}] \[u(0,t) = A(t) \qqtext{and} \pard{u}{x}(L,t) = B(t)\]
		\end{itemize}
		Let $u_E(x,t)$ be the linear PDE that satisfies the steady state problem:
		\[u_E(0,t) = A(t) \qquad \pard{u_E}{x}(L,t) = B(t) \qqarrow u_E(x,t) = A(t) + xB(t)\]
		We can now define $v(x,t) = u(x,t) - u_E(x,t)$ with the following PDE's:
		\[\pard{v}{t} = \pard{u}{t} - \pard{u_E}{t} \qarrow \pard{u}{t} = \pard{v}{t} + \pard{u_E}{t} \qquad\qquad \pard{^2v}{x^2} = \pard{^2u}{x^2} - \pard{^2u_E}{x^2} = \pard{^2u}{x^2}\]
		Now we can convert our original PDE:
		\[\pard{v}{t} + \pard{u_E}{t} = k\pard{^2v}{x^2} + Q(x,t)\]
		Now notice this is a PDE with the following homogeneous boundary conditions:
		\[v(0,t) = u(0,t) - u_E(0,t) = 0 \qquad \pard{v}{x}(L,t) = \pard{u}{x}(L,t) - \pard{u_E}{x}(L,t) = 0\]
		and the following initial condition:
		\[v(x,0) = u(x,0) - u_E(x,0) = f(x) - \left( A(0) + xB(0) \right) \]
	\end{problem}

	\begin{problem}{8.2.5}
		Solve the initial value problem for a two-dimensional heat equation inside a circle (of
		radius a) with time-independent boundary conditions
		\[\pard{u}{t} = k\nabla^2u \qquad u(a,\theta,t) = g(\theta) \qquad u(r,\theta,0) = f(r,\theta)\]
		Let $u_E(r,\theta)$ be the equilibrium temperature distribution: 
		\[\nabla^2u_E = 0 \qqarrow \frac{1}{r}\pard{}{r}\left(r\pard{u_E}{r}\right) + \frac{1}{r^2}\pard{^2u_E}{\theta^2} = 0 \]
		Using separation of variables, we get: $u_E(r,\theta) = F(r)G(\theta)$:
		\[\frac{r}{F} \frac{d}{dr}\left(r \frac{dF}{r}\right) + \frac{1}{G}\frac{d^2G}{d\theta^2} = 0 \qqarrow \frac{r}{F} \frac{d}{dr}\left(r \frac{dF}{r}\right) = -\frac{1}{G}\frac{d^2G}{d\theta^2} = \lambda \]
		Thus we get the following ODE's:
		\[r^2F'' + rF' - \lambda F = 0 \qquad G'' + \lambda G = 0\]
		Using the fact of the geometry of the circle, and that $F(r)$ is finite at $r = 0$, we get the following eigenvalues and eigenfunction:
		\[\lambda  = 0 \qqarrow G(\theta) = 1 \qquad F(r) = 1\]
		\[\lambda  = m^2 \qqarrow G(\theta) = a\cos m\theta + b\sin m\theta \qquad F(r) = c_1r^m\]
		From this, we get the following:
		\[u_E(r,\theta) = \sum_{m=0}^{\infty} \left(A_m\cos m\theta + B_m\sin m\theta\right)r^m \]
		Using the boundary conditions, we get:
		\[u_E(a,\theta) = A_0 + \sum_{m=1}^{\infty} \left(A_m\cos m\theta + B_m\sin m\theta\right)a^m\]
		From here, we get the following coefficients:
		\[A_0 = \frac{1}{2\pi}\int_{-\pi}^{\pi} g(\theta)\,d\theta \qquad A_m = \frac{1}{a^m\pi}\int_{-\pi}^{\pi} g(\theta)\cos m\theta\,d\theta \qquad B_m = \frac{1}{a^m\pi}\int_{-\pi}^{\pi} g(\theta)\sin m\theta\,d\theta\]
		\newpage
		We let the following be true now:
		\[v(r,\theta,t) = u(r,\theta,t) - u_E(r,\theta) \qtext{with} v(a,\theta,t) = 0 \qquad v(r,\theta,0) = f(r,\theta) - u_E(r,\theta)\]
		Similar to previous homeworks, we get the following for $v(r,\theta,t)$:
		\[v(r,\theta,t) = \sum_{n=1}^{\infty} A_{0n}J_0(\sqrt{\lambda_{0n}}r)e^{-\lambda_{0n}kt} + \sum_{m=1}^\infty \sum_{n=1}^\infty A_{mn}J_m(\sqrt{\lambda_{mn}}r)\left(A_{mn}\cos m\theta + B_{mn}\sin m\theta\right)e^{-\lambda_{mn}kt} \]
		From here, we use our initial conditions:
		\begin{align*}
			v(r,\theta,0) &= f(r,\theta) - u_E(r,\theta) \\
			&= \sum_{n=1}^{\infty} A_{0n}J_0(\sqrt{\lambda_{0n}}r) + \sum_{m=1}^\infty \sum_{n=1}^\infty A_{mn}J_m(\sqrt{\lambda_{mn}}r)\left(A_{mn}\cos m\theta + B_{mn}\sin m\theta\right)
		\end{align*}
		From here, we get the following coefficients:
		\begin{align*}
			A_{0n} &= \frac{\int_{0}^a \int_{-\pi}^\pi \left(f(r,\theta) - u_E(r,\theta)\right)J_0\left(\sqrt{\lambda_{0n}}r\right)r\,dr\,d\theta}{2\pi \int_{0}^a J_0^2\left(\sqrt{\lambda_{0n}}r\right)\,dr} \\
			A_{mn} &= \frac{\int_{0}^a \int_{-\pi}^\pi \left(f(r,\theta) - u_E(r,\theta)\right)J_m\left(\sqrt{\lambda_{mn}}r\right)\cos m\theta \,r\,dr\,d\theta}{2\pi \int_{0}^a J_m^2\left(\sqrt{\lambda_{mn}}r\right)\,dr} \\
			B_{mn} &= \frac{\int_{0}^a \int_{-\pi}^\pi \left(f(r,\theta) - u_E(r,\theta)\right)J_m\left(\sqrt{\lambda_{mn}}r\right)\sin m\theta \,r\,dr\,d\theta}{2\pi \int_{0}^a J_m^2\left(\sqrt{\lambda_{mn}}r\right)\,dr} \\
		\end{align*}
	\end{problem}

	\begin{problem}{8.4.2}
		Use the method of eigenfunction expansions to solve, without reducing to homogeneous
		boundary conditions,
		\[\pard{u}{t} = k\pard{^2u}{x^2} \qquad u(0,t) = A \qquad u(L,t) = B \qquad u(x,0) = f(x) \]
		Using an eigenfunction expansion, we choose get the Strum-Liouville Problem in the spatial domain:
		\[\frac{d^2\phi}{dx^2} + \lambda\phi = 0 \qtext{with} \phi(0) = 0 \quad \phi(L) = 0\]
		Notice the eigenvalues and eigenfunctions:
		\[\lambda_n = \left(\frac{n\pi}{L}\right)^2 \qquad \phi_n(x) = \sin \frac{n\pi x}{L}\]
		Thus we can let the following be true:
		\[u(x,t) = \sum_{n=1}^{\infty} B_n(t)\phi(x) \qtext{with} u(x,0) = f(x) = B_n(0)\phi(x) \qarrow B_n(0) = \frac{2}{L}\int_{0}^L f(x)\phi(x)\,dx\]
		Notice the following from that and orthogonality of sines:
		\[\pard{u}{t} = \sum_{n=1}^{\infty} \frac{dB_n(t)}{dt}\phi(x) = k\pard{^2u}{x^2} \qquad \frac{dB_n(t)}{dt} = \frac{2k}{L}\int_{0}^L \pard{^2u}{x^2}\phi(x)\]
		Notice the following from Green's formula:
		\begin{align*}
			\int_{0}^L \left(u \frac{d^2\phi_n}{dx^2} - \phi_n\pard{^2u}{x^2}\right)\,dx &= u\frac{d\phi_n}{dx} - \phi_n\pard{u}{x} \bigg|_0^L \\
			&= \left( u(L,t)\frac{d\phi_n(L)}{dx} - \phi_n(L)\pard{u}{x}(L,t) \right) - \left( u(0,t)\frac{d\phi_n(0)}{dx} - \phi_n(0)\pard{u}{x}(0,t) \right) \\
			&=  B\frac{d\phi_n(L)}{dx} - A\frac{d\phi_n(0)}{dx} \\
			&= \frac{n\pi}{L}\bigg((-1)^nB - A\bigg)  
		\end{align*}
		Notice the following equality:
		\begin{align*}
			\int_{0}^L u \frac{d^2\phi_n}{dx^2}\,dx - \int_{0}^L \phi_n\pard{^2u}{x^2}\,dx &= \int_{0}^L \left(u \frac{d^2\phi_n}{dx^2} - \phi_n\pard{^2u}{x^2}\right)\,dx \\
			\int_{0}^L u \frac{d^2\phi_n}{dx^2}\,dx - \int_{0}^L \left(u \frac{d^2\phi_n}{dx^2} - \phi_n\pard{^2u}{x^2}\right)\,dx &= \int_{0}^L \phi_n\pard{^2u}{x^2}\,dx 
		\end{align*}
		Thus we get the following:
		\begin{align*}
			\int_{0}^L \phi_n\pard{^2u}{x^2}\,dx &= \int_{0}^L u \frac{d^2\phi_n}{dx^2}\,dx - \int_{0}^L \left(u \frac{d^2\phi_n}{dx^2} - \phi_n\pard{^2u}{x^2}\right)\,dx \\
			&= \int_{0}^L u \frac{d^2\phi_n}{dx^2}\,dx - \frac{n\pi}{L}\bigg((-1)^nB - A\bigg) \\
			&= \int_{0}^L u \left(-\lambda_n\phi_n\right) \,dx - \frac{n\pi}{L}\bigg((-1)^nB - A\bigg) \\
			&= -\lambda_n\int_{0}^L \left(\sum_{m=1}^{\infty} B_m(t)\phi_m(x)\right)\phi_n \,dx - \frac{n\pi}{L}\bigg((-1)^nB - A\bigg) \\
			&= -\lambda_n\left(\frac{L}{2}\right)B_n(t) - \frac{n\pi}{L}\bigg((-1)^nB - A\bigg)
		\end{align*}
		From here, we get the following:
		\begin{align*}
			\frac{dB_n(t)}{dt} &= \frac{2k}{L}\int_{0}^L \pard{^2u}{x^2}\phi(x) \\
			&= \frac{2k}{L}\left(-\lambda_n\left(\frac{L}{2}\right)B_n(t) - \frac{n\pi}{L}\bigg((-1)^nB - A\bigg)\right) \\
			&= -\lambda_n k B_n(t) - \frac{2kn\pi}{L^2}\bigg((-1)^nB - A\bigg)
		\end{align*}
		Now we multiply by $e^{-\lambda_n k t}$ and solve for $B_n$, we get:
		\[B_n(t) = B_n(0)e^{-\lambda_n kt} - \frac{2}{n\pi}\bigg((-1)^nB - A\bigg)\left(1 - e^{-\lambda_n k t}\right) \]
	\end{problem}

	\begin{problem}{8.4.3}
		Consider
		\[c(x)\rho(x)\pard{u}{t} = \pard{}{x} \left[K_0(x)\pard{u}{x}\right] + q(x)u + f(x,t)\]
		\[u(x,0) = g(x) \qquad u(0,t) = \alpha(t) \qquad u(L,t) = \beta(t)\]
		Assume that the eigenfunction $\phi_n(x)$ of the related homogeneous problem are known.
		\begin{enumerate}[label = \textbf{(\alph*)}]
			\item Solve without reducing to a problem with homogeneous boundary conditions. 
			\\ \\
			Let the following be true:
			\[\sigma = c\rho \qquad u(x,t) = \sum_{n=0}^{\infty} B_n(t)\phi_n(t) \]
			Substituting, we get:
			\[\sigma \sum_{n=0}^{\infty} B_n'(t)\phi_n(t) = \pard{}{x} \left[K_0(x)\pard{u}{x}\right] + q(x)u + f(x,t) \]
			Similar to our last problem, we can use Green's formula to get:
			\[\sigma \frac{dB_n(t)}{dt} = -\lambda_n kB_n(t) + f_n(t) - \frac{k\sqrt{\lambda_n}\left( (-1)^n \beta(t) - \alpha(t) \right) }{\int_{0}^L \phi_n^2(x)\sigma\,dx}\]
			with the following:
			\[f_n(t) = \frac{\int_{0}^L f(x,t)\phi_n(x)\sigma \,dx}{\int_{0}^L \phi_n^2(x)\sigma\,dx}\]
			Multiplying by $e^{\frac{\lambda_n}{\sigma}t}$ and solving for $B_n$, we get:
			\[B_n(t) = e^{-\frac{\lambda_n}{\sigma}t} \left(\frac{1}{\sigma}\int e^{\frac{\lambda_n}{k}t}f_n(t)\,dt + \frac{\frac{k}{n}\sqrt{\lambda_n}}{\int_0^L \phi_n^2(x)\sigma,dx}\int -e^{\frac{\lambda_n}{\sigma}}\left( (-1)^n \beta(t) - \alpha(t) \right)\,dt\right) + B_n(0)e^{-\frac{\lambda_n}{\sigma}t} \]
			with the following coefficients found from the initial condition:
			\[B_n(0) = \frac{\int_0^L g(x)\phi_n(x)\sigma\,dx}{\int_{0}^L \phi_n^2(x)\sigma,dx}\]
			\newpage
			\item Solve by first reducing to a problem with homogeneous boundary conditions
			\\ \\
			Let the following be true:
			\[u_E(x,t) = \alpha(t) + \frac{x}{L}\left(\beta(t) - \alpha(t)\right)\]
			Now we set the following and substitute:
			\[v(x,t) = u(x,t) - u_E(x,t) \qquad c\rho\pard{v}{t} = \pard{}{x}\left( K_0\pard{v}{x}\right) + vq(x) + f(x,t) \]
			with the following conditions:
			\[v(0,t) = v(\pi,t) = 0 \qquad v(x,0) = g(x) - \alpha(0) - \frac{x}{L}\left(\beta(0) - \alpha(0)\right)\]
			We now let the following:
			\[v(x,t) = \sum_{n=1}^\infty B_n(t)\phi_n(x) \qquad \sigma = c\rho\]
			Using Green's formula, we get:
			\[\sigma \frac{dB_n(t)}{dt} = -\lambda_nB_n(t) + f_n(t)\]
			where
			\[f_n(t) = \frac{\int_{0}^L f(x,t)\phi_n(x)\sigma \,dx}{\int_{0}^L \phi_n^2(x)\sigma\,dx}\]
			Multiplying by $e^{\frac{\lambda_n}{\sigma}t}$ and solving for $B_n$, we get:
			\[B_n(t) = e^{-\frac{\lambda_n}{\sigma}t}\frac{1}{\sigma}\int e^{\frac{\lambda_n}{\sigma}t}f_n(t)\,dt + B_n(0)e^{-\frac{\lambda_n}{\sigma}}\]
			with the following coefficients found from the initial condition:
			\[B_n(0) = \frac{\int_0^L g(x) - \alpha(0) - \frac{x}{L}\left(\beta(0) - \alpha(0)\right)\phi_n(x)\sigma \,dx}{\int_0^L \phi_n^2(x)\sigma\,dx}\]
		\end{enumerate}
	\end{problem}

	\begin{problem}{9.2.1}
		Consider
		\[\pard{u}{t} = k\pard{^2u}{x^2} + Q(x,t) \qquad u(x,0) = g(x)\]
		In all cases, obtain formulas similar to (2.20) by introducing a Green’s function.
		\\
		\begin{mdframed}
			Notice equation (2.20):
			\[u(x,t) = \int_{0}^{L} g(x_0)G(x,t;x_0,0)\,dx_0 + \int_{0}^L \int_{0}^t Q(x_0,t_0)G(x,t;x_0,0)\,dt_0\,dx_0 \tag{2.20} \]
		\end{mdframed}
		\begin{itemize}
			\item[\textbf{(c)}] Solve using any method if
			\[\pard{u}{t}(0,t) = 0 \qqtext{and} \pard{u}{t}(L,t) = 0\]
			
			\item[\textbf{(d)}] Use Green’s formula instead of term-by-term differentiation if
			\[\pard{u}{x}(0,t) = A(t) \qqtext{and} \pard{u}{x}(L,t) = \beta(t)\]
		\end{itemize}
	\end{problem}


\end{document}
