\documentclass[12pt]{article}
\usepackage[margin = 1in]{geometry}
\usepackage{amsmath}
\usepackage{amssymb}
\usepackage{amsthm}
\usepackage{graphicx}
\usepackage{subfig}
\usepackage{enumitem}

\begin{document}
	
	\begin{center}
		\textbf{Classwork 6} \\
		\textbf{Abstract Algebra} \\
		\textbf{Math 320} \\
		\textbf{Stephen Giang, Brooke Tyler} \\
		\textbf{Jakob Lepur, Sammi Zimmerie}
	\end{center}

\noindent \textbf{Problem 1: }Let $f(x),g(x),h(x) \in F[x]$ with $f(x)$ and $g(x)$ relatively prime. If $h(x)|f(x),$ prove that $h(x)$ and $g(x)$ relatively prime. 
\\ \\

\noindent Let $h(x)|f(x)$, so the following is true:
	\begin{align*}
		f(x) &= h(x)q(x)
	\end{align*}
Let $d(x) = (h(x),g(x))$, so $d(x)|h(x)$ and $d(x)|g(x)$
	\begin{align*}
		h(x) &= d(x)a_1(x) \\
		g(x) &= d(x)a_2(x)
	\end{align*}
Putting it all together gets us:
	\begin{align*}
		f(x) &= h(x)q(x) = d(x)a_1(x)q(x) = d(x)b_1(x) \\
		g(x) &= d(x)a_2(x)
	\end{align*}
Because $f(x)$ and $g(x)$ are relatively prime, they do not share any factors. And because they both share $d(x)$, then $d(x) = 1$.  Because $d(x)$ is also $(h(x),g(x))$, $h(x)$ and $g(x)$ are relatively prime.

\newpage 

\noindent \textbf{Problem 2: }Express $x^4 - 4$ as a product of irreducibles in $\mathbb{Q}[x], \mathbb{R}[x], \mathbb{C}[x]$.
	\begin{align*}
		x^4 - 4 &= (x^2 - 2)(x^2 + 2) \in \mathbb{Q}[x] \\
		&= (x - \sqrt{2})(x + \sqrt{2})(x^2 + 2) \in \mathbb{R}[x] \\
		&= (x - \sqrt{2})(x + \sqrt{2})(x - \sqrt{2}i)(x + \sqrt{2}i) \in \mathbb{C}[x]
	\end{align*} 

\newpage 

\noindent \textbf{Problem 3: }Show that $x^7 - x$ factors in $\mathbb{Z}_7[x]$ as $x(x - 1)(x - 2)(x - 3)(x - 4)(x - 5)(x - 6)$ without doing any polynomial multiplication/division.
\\ \\

\noindent We can show this by using our knowledge of roots.  If we plug in $0,1,2,3,4,5,6$ into $f(x) = x^7 - x$, then the result should be 0.
	\begin{align*}
		f(0) &= 0 \\
		f(1) &= 0 \\
		f(2) &= 7(18) = 0 \\
		f(3) &= 7(312) = 0 \\
		f(4) &= 7(2340) = 0 \\
		f(5) &= 7(11160) = 0 \\
		f(6) &= 7(39990) = 0
	\end{align*}
Because we have proved that $0,1,2,3,4,5,6$ are roots of $f(x) = x^7 - x$, then $f(x)$ can be factored into $x(x - 1)(x - 2)(x - 3)(x - 4)(x - 5)(x - 6)$.
\\ \\

\noindent \textbf{Bonus (a): }Find all polynomials of degree 2 in $\mathbb{Z}_2[x]$
\\ \\

\noindent Notice in $\mathbb{Z}_2[x]$, all numbers are equal to either 0 or 1. All polynomials can be written as $ax^2 + bx + c$.  So by looking at the different $a,b,c$ values, we get:
	\begin{align*}
		&x^2 \\
		&x^2 + x + 1 \\
		&x^2 + 1 \\
		&x^2 + x \\
	\end{align*}

\noindent \textbf{Bonus (b): }Find all \textit{irreducible} polynomials of degree 2 in $\mathbb{Z}_2[x]$
\\ \\

\noindent To find the following, all we need to do is reduce all the reducible polynomials from part (a) to find the irreducible ones.
	\begin{align*}
		x^2 &= x(x) \\
		x^2 + x + 1 & \\
		x^2 + 1 &= (x + 1)^2  \\
		x^2 + x &= x(x+1) 
	\end{align*}
Because we were able to reduce all polynomials except $x^2 + x + 1$, $x^2 + x + 1$ is irreducible	

\end{document}
