\documentclass[12pt]{article}
\usepackage[margin = 1in]{geometry}
\usepackage{amsmath}
\usepackage{amssymb}
\usepackage{amsthm}
\usepackage{graphicx}
\usepackage{subfig}
\usepackage{enumitem}

\begin{document}
	
	\begin{center}
		\textbf{Class Work 9} \\
		\textbf{Abstract Algebra} \\
		\textbf{Math 320} \\
		\textbf{Stephen Giang, William Diebolt} \\
		\textbf{Sobhan Ahmadi Pishkouhi}
	\end{center}

\noindent \textbf{Problem 1: }For parts (a) and (b), determine if the given ring is a field. If it is, explain
why. If not, explain why and provide one zero divisor.

	\begin{enumerate}[label = (\alph*)]
		\item $\mathbb{Q}[x]/(x^6 - 144)$ 
		\\ \\
		Notice that $x^3 - 12$ and $x^3 + 12$ are both in $\mathbb{Q}/(x^6 - 144)$ as they both have degree less than $6$.  When multiplied, they equal $(x^6 - 144)$.  So the following is a zero divisor:
			$$
			[x^3 - 12][x^3 + 12] = [x^6 - 144] = [0]
			$$
		\item $\mathbb{Z}_3[x]/(2x^3 + x + 1)$
		\\ \\
		Notice that all constants of $\mathbb{Z}_3$ are $0,1,2$. Notice that all zero divisors will be factors of the given polynomial. Also notice because of the congruence class, the only factors will be of degree 2 and degree 1 at the same time.  Because degree one, then its factor is a root.  Notice:	
			\begin{align*}
				f(x) &= 2x^3 + x + 1 \\
				f(0) &= 1 \not = 0 \\
				f(1) &= 1 \not = 0 \\
				f(2) &= 1 \not = 0
			\end{align*}
	\end{enumerate}

\newpage 

\noindent \textbf{Problem 2: }Find the multiplicative inverse of $[x - 1]$ in $\mathbb{Q}[x]/(x^2 - 3)$ using the following method:
	\\ \\
	Let $(x-1, x^2 - 3) = 1$, such that the following is true:
		$$
		(x-1)u(x) + (x^2 - 3)v(x) = 1
		$$
	We assume that $u(x)$ and $v(x)$ are both degree 1. So we can write the following:
		\begin{align*}
				u(x) = ax + b && v(x) = cx + d
		\end{align*}
	We now have 
		\begin{align*}
			(x-1)(ax+b) + (x^2 - 3)(cx + d) &= 1 \\
			ax^2 + (b-a)x - b + cx^3 + dx^2 - 3cx - 3d &= 1 \\
			cx^3 + (a+d)x^2 + (b-a-3c)x + (-b - 3d) &= 1
		\end{align*}
	So we get the systems of equation:	
		\begin{align*}
			c &= 0 \\
			a + d &= 0 \\
			b - a - 3c &= 0 \\
			-b - 3d &= 1
		\end{align*}
	So after reducing the system, we get $a = \frac{1}{2}, b = \frac{1}{2}, c = 0, d = \frac{-1}{2}$. So now we get the following:
		\begin{align*}
			\frac{1}{2}(x-1)(x + 1) - \frac{-1}{2}(x^2 - 3) &= 1 \\
			\frac{1}{2}(x-1)(x + 1) &= 1
		\end{align*}
	Thus $\frac{1}{2}(x + 1) = [x-1]^{-1}$
\newpage 

\noindent \textbf{Problem 3: }
	\begin{enumerate}[label = (\alph*)]
		\item Prove that the set $\{(2a,0) : a \in \mathbb{Z}\}$ is an ideal of $\mathbb{Z} \times \mathbb{Z}$
		\\ \\
		Notice the following: 
			\\ \\
			Let $(2a, 0) \in \mathbb{Z} \times \mathbb{Z}$.  If we let $a = 0$, then $(0,0) = \mathbb{Z} \times \mathbb{Z}$.  Thus $\mathbb{Z} \times \mathbb{Z}$ contains the zero element. 
			\\ \\
			Let $x = (2a,0)$ and $y = (2b, 0)$.
				$$
				x - y = (2a,0) - (2b,0) = (2(a-b),0) \in \mathbb{Z} \times \mathbb{Z}
				$$
			Thus closed under subtraction.
			\\ \\
			Let $x = (2a , 0)$ and $y = (b, c)$
				\begin{align*}
					xy = (2a, 0)(b,c) = (2ab,0) = (2ba,0) = (b,c)(2a,0) = yx
				\end{align*}
			Thus absorption property proven.
			\\ \\
		\item Let $F$ be a field, $c \in F$, and consider the set $K_c$ consisting of polynomials that have $c$
		as a root:
		\\ \\
		Notice the following:
			\\ \\
			Let $f(x) = 0$.  Notice that $f(c) = 0_F \in K_c$.
			\\ \\
			Let $f(x), g(x) \in K_c$. 
				$$
				f(x) - g(x) = (f-g)(x), \qquad f(c) - g(c) = (f-g)(c) = 0_F \in K_c
				$$
			Let $f(x) \in K_c$ and $g(x) \in F[x]$.
				$$
				f(c)g(c) = 0_F = g(c)f(c)
				$$
			Thus the absorption property is proven
	\end{enumerate}

\end{document}
