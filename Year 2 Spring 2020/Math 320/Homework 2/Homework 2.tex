\documentclass[12pt]{article}
\usepackage[margin = 1in]{geometry}
\usepackage{amsmath}
\usepackage{amssymb}
\usepackage{cancel}

\begin{document}
	
	\begin{center}
		\textbf{Homework 2} \\
		\textbf{Abstract Algebra} \\
		\textbf{Math 320} \\
		\textbf{Stephen Giang} \\
	\end{center}

\noindent \textbf{Section 1.2 Problem 11a:} 
If $n \in \mathbb{Z}$, what are the possible values of $(n, n + 2)$.
\\\\
\textbf{Solution:} \\
Let $n \in \mathbb{Z}$
	\begin{align}
		&n+2 = n(1)+2\\
		&\text{By the Euclidean Algorithm:} (n+2,n) = (n,2) \\
		&\text{By divisibility rules, the only divisors of } 2 \text{ is } \pm 1,\pm 2 \\
		&\text{Also by divisibility rules, the only divisors of n have to be } \leq |n| \\
		&\text{Thus the common divisors would be } \pm 1,\pm 2 \\
		&\text{The greatest would be their positive counterparts: } \boldmath{\text{$1,2$}} \\
	\end{align}

\vspace{\baselineskip}
\vspace{\baselineskip}
\noindent \textbf{Section 1.2 Problem 15c:} 
Use the Euclidean Algorithm to find (1003,456)
\\\\
\textbf{Solution:} 
	\begin{align}
		1003 &= 456(2) + 91 \\
		456  &= 91(5) + 1 \\
		91   &= 1(91) 
	\end{align}
GCD = $(1003,456)$ = \textbf{1}

\vspace{\baselineskip}
\vspace{\baselineskip}
\noindent \textbf{Section 1.2 Problem 15j:} 
Use the method described in parts (f)-(i) to express the GCD in part (c) as a linear combination of 1003 and 456.
\\\\
\textbf{Solution:} \\
Let $u,v \in \mathbb{Z}$ \\
Show : $1 = 1003u + 456v$
	\begin{align}
		1 
		&= 456 - 91(5) \\
		&= 456 - (1003 - 456(2))(5) \\
		&= 1003(-5) + (456)(11) 
	\end{align}
(1003,456) can be written as a \textbf{Linear Combination when \boldmath$u = -5, v = 11$} 

\newpage

\noindent \textbf{Section 1.2 Problem 17:} 
Suppose $(a, b) = 1$. If $a|c$ and $b|c$, prove that $ab|c$. [Hint: $c = bt$ (why?), so $a|bt$. Use Theorem 1.4.] \\
Theorem 1.4 : If $a|bc$ and $(a, b)= 1$, then $a|c$.  
\\\\
\noindent \textbf{Solution:} \\
Suppose $(a, b) = 1$. Let $a|c$ and $b|c$ for some $a,b,c,r,t \in \mathbb{Z}$
	\begin{center}
		$c = br$ \\
		so $a|br$ \\
		by Theorem 1.4, $a|r$ \\
		so $r = at$ \\
		so $c = b(at)$ \\
		$c = ab(t)$ \\
		Thus \boldmath{$ab|c$}
	\end{center}

\vspace{\baselineskip}
\noindent \textbf{Section 1.2 Problem 19:} 
If $a|(b + c)$ and $(b, c) = 1$, prove that $(a, b) = 1 = (a, c)$. 
\\\\
\noindent \textbf{Solution:} \\
Let $a|(b+c)$ and $(b,c) = 1$ for some $a,b,c,r \in \mathbb{Z}$
	\begin{align}
		b + c &= ar \\
		c &= ar - b \\
		bu + cv &= 1 \\
		bu + (ar-b)v &= 1 \\
		bu + arv - bv &= 1 \\
		b(u-v) + a(rv) &= 1 
	\end{align}
	\begin{align}
		b + c &= ar \\
		b &= ar - c \\
		bu + cv &= 1 \\
		(ar-c)u + cv &= 1 \\
		aru - cu + cv &= 1 \\
		a(ru) + c(v-u) &= 1
	\end{align}
Thus \boldmath $(a,b) = 1 = (a,c)$	

\end{document}
