\documentclass[12pt]{article}
\usepackage[margin = 1in]{geometry}
\usepackage{amsmath}
\usepackage{amssymb}
\usepackage{cancel}

\begin{document}
	
	\begin{center}
		\textbf{Homework 3} \\
		\textbf{Abstract Algebra} \\
		\textbf{Math 320} \\
		\textbf{Stephen Giang} \\
	\end{center}
\vspace{\baselineskip}
\noindent \textbf{Section 1.2 Problem 34a: } Prove that $(a,b)|(a+b,a-b)$ 
\\\\
\noindent \textbf{Solution Problem 34a: } Let $d_1 = (a,b)$ and $d_2=(a+b,a-b)$ \qquad $a$,$b$,$d_1$,$d_2$,$q_1$,$q_2 \in \mathbb{Z}$ 
\\\\
By Corollary 1.3, if $c|(a+b)$ and $c|(a-b)$, then $c|d_2$ \quad $c \in \mathbb{Z}$
\\\\
	\begin{align}
		a &= d_1q_1 \\
		b &= d_1q_2 \\
		a+b &= d_1(q_1 + q_2) \\
		a-b &= d_1(q_1 - q_2) \\
		d_1 | (a+b) &\text{ and } d_1 | (a-b) \\
		d_1 | d_2 &= (a,b) | (a+b,a-b)
	\end{align}
By proving that $d_1 | a$ and $d_1|b$, we can prove it divides its sum and difference,\\ thus \text{\boldmath $(a,b) | (a+b,a-b)$}
\newpage
\noindent \textbf{Section 1.2 Problem 34b: } Prove that if $a$ is odd and $b$ is even, then $(a, b) = (a + b, a - b)$
\\\\
\noindent \textbf{Solution Problem 34b: } Let $a=2q+1$ and $b = 2k$, $d_1 = (a,b)$ and $d_2=(a+b,a-b)$ \qquad $a,b,d_1,d_2,q,k,r,r_1,s,s_1,c_1,c_2 \in \mathbb{Z} $.
	\begin{align}
		a + b &= 2(q+k) + 1 \\
		&= 2c_1 + 1 \\
		a - b &= 2(q-k) + 1 \\
		&= 2c_2 + 1
	\end{align}
Because $a+b$ and $a-b$ is odd, then $d_2$ is odd. 
	\begin{align}
		a+b &= d_2 r \\
		a-b &= d_2 s \\
		(a + b) + (a - b) &= 2a \\
		&= d_2 r + d_2 s \\
		&= d_2(r+s) \\
		(a + b) - (a - b) &= 2b \\
		&= d_2 r - d_2 s \\
		&= d_2(r-s)
	\end{align}
Because $d_2 | 2a$ and $d_2 | 2b$, and because $(d_2, 2) = 1$, $d_2|a$ and $d_2|b$, so $d_2|d_1$.  
	\begin{align}
		a &= d_1 r_1 \\
		b &= d_1 s_1 \\
		a + b &= d_1(r_1 + s_1) \\
		a - b &= d_1(r_1 - s_1) 
	\end{align} 
So $d_1 | d_2$.  Because $d_1 | d_2$ and $d_2|d_1$, \text{ \boldmath $d_1 = d_2$}
\\\\\\
\noindent \textbf{Section 1.3 Problem 7: } If $a$, $b$, $c$ are integers and $p$ is a prime that divides both $a$ and $a + bc$ , prove that $p|b$ or $p|c$. 
\\\\
\noindent \textbf{Solution Problem 7: } Let $p|a$ and $p|a+bc$ \qquad $a,b,c,p,q_1,q_2 \in \mathbb{Z}$ 
\\\\
	\begin{align}
		a &= pq_1 \\
		a + bc &= pq_1 + bc \\
		&= pq_2
	\end{align}
To have $a+bc = pq_2$, $bc$ needs to be divisible by $p$.  And by prime factorization, \textbf{ \boldmath only $b$ or $c$ needs to be divisible by $p$, for their product to be divisible by $p$}.
\newpage
\noindent \textbf{Section 1.3 Problem 16: } Prove that $(a, b) = 1$ if and only if there is no prime $p$ such that $p|a$ and $p|b$.
\\\\
\noindent \textbf{Solution Problem 16: ($=>$)} Let $(a, b) = 1$ \qquad $a$,$b$ $\in \mathbb{Z}$

	\begin{align}
		&\text{By Prime Factorization: } a = 1*\prod_{i=0}^{m} p_i \qquad m = \text{min\{Amount of Primes for a\}} \\
		&\text{By Prime Factorization: } b = 1*\prod_{j=0}^{n} p_j \qquad n = \text{min\{Amount of Primes for b\}} \\
	\end{align}
If $p_i$ for any $i \in \mathbb{Z}$ equals $p_j$ for any $j \in \mathbb{Z}$, then $p_i = p_j$ would be the GCF, thus contradicting $(a, b) = 1$. \textbf{ \boldmath So there are no primes $p$ that would divide both $a$ and $b$}. 
\\\\
\noindent \textbf{Solution Problem 16: ($<=$)} Let there be no prime $p$ such that $p|a$ and $p|b$. 
\\\\
Because of prime factorization, all integers can be written as product of primes.  If $a$ and $b$, do not share any divisor $p$ that is prime, then they have no common divisors greater than 1,\textbf{ \boldmath thus (a,b) = 1}.
\\\\\\
\noindent \textbf{Section 1.3 Problem 27: }  If $p > 3$ is prime, prove that $p^2 + 2$ is composite. [Hint: Consider the possible remainders when p is divided by 3.] 
\\\\
\noindent \textbf{Solution Problem 27: } Let $p$ be prime, and $p>3$ \qquad $p$,$q$,$k \in \mathbb{Z}$ 
Case 1: (p = 3k+1)\begin{align}
					p^2 + 2 &= (3k+1)^2 + 2 \\
					&= 9k^2 + 6k + 1 + 2 \\
					&= 3q
				  \end{align}
Case 2: (p = 3k+ 2)\begin{align}
					p^2 + 2 &= (3k + 2)^2 + 2 \\
					&= 9k^2 + 12k + 6 \\
					&= 3q
				   \end{align}
Because $p\not = 3, p^2 + 2 \not = 3$.  Because $p^2 + 2 \not = 3$, and it is divisible by 3, then \text{\boldmath $p^2 + 2$ is composite.} (Note: $p \not = 3k$, because then $p$ would not be prime)
		
\end{document}
