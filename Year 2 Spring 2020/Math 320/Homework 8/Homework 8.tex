\documentclass[12pt]{article}
\usepackage[margin = 1in]{geometry}
\usepackage{amsmath}
\usepackage{amssymb}
\usepackage{amsthm}
\usepackage{graphicx}
\usepackage{subfig}
\usepackage{enumitem}


\begin{document}
	\begin{center}
		\textbf{Homework 8} \\
		\textbf{Abstract Algebra} \\
		\textbf{Math 320} \\
		\textbf{Stephen Giang} \\
	\end{center}

\noindent \textbf{Section 3.3 Problem 19: }$S = \{0, 4, 8, 12, 16, 20, 24\}$ is a subring of $\mathbb{Z}_{28}$. Prove that the map $f: \mathbb{Z}_7 \rightarrow S$ given by $f([x]_7) = [8x]_{28}$ is an isomorphism.
\\ \\
	Let $S = \{0, 4, 8, 12, 16, 20, 24\}$ be a subring of $\mathbb{Z}_{28}$. Notice:
	\begin{align*}
		f([0]_7) &= [0]_{28} \\ 
		f([1]_7) &= [8]_{28} \\ 
		f([2]_7) &= [16]_{28} \\ 
		f([3]_7) &= [24]_{28} \\ 
		f([4]_7) &= [32]_{28} = [4]_{28} \\ 
		f([5]_7) &= [40]_{28} = [12]_{28} \\ 
		f([6]_7) &= [48]_{28} = [20]_{28}
	\end{align*}
	Because for all $x$ in $\mathbb{Z}_7$, $f$ maps $x$ to a unique $y$ in $S$, $f$ is injective. Because for all $y$ in $S$, there exists an $x$ in $\mathbb{Z}_7$ such that $f$ maps $x$ to $y$, $f$ is also surjective, thus being bijective. 
	\\ \\
	Let $x_1,x_2 \in \mathbb{Z}_7$
	\begin{align*}
		f(x_1) + f(x_2) &= [8x_1]_{28} + [8x_2]_{28} = [8(x_1 + x_2)]_{28} = f(x_1 + x_2) \\
		f(x_1) f(x_2) &= [8x_1]_{28} \times [8x_2]_{28} = [64]_{28}[x_1 x_2]_{28} = [8]_{28}[x_1 x_2]_{28} \\
		&= [8(x_1 x_2)]_{28} = f(x_1 x_2)
	\end{align*}
	This shows that $f$ is a homomorphism. 
	\\ \\
	\textbf{\boldmath Thus $f$ is an isomorphism.}

\newpage 

\noindent \textbf{Section 3.3 Problem 21: }Let $\mathbb{Z}^*$ denote the ring of integers with the $\oplus$ and $\odot$ operations defined as: 
	\begin{align*}
		a \oplus b &= a + b - 1 \\
		a \odot b &= a + b - ab
	\end{align*}
Prove that $\mathbb{Z}$ is isomorphic to $\mathbb{Z}^*$.
\\ \\
	Let $f: \mathbb{Z} \rightarrow \mathbb{Z}^*$ such that $f(x) = 1 - x$, with $x \in \mathbb{Z}$ 
	\\ \\ \\
	Let $\exists x_1, x_2 \in \mathbb{Z}$, such that $f(x_1) = f(x_2)$
	\begin{align*}
		f(x_1) = 1 - x_1 = 1 - x_2 = f(x_2).
	\end{align*}
	Thus $x_1 = x_2$, proving that $f$ maps $x$ to a unique $y$ in $\mathbb{Z}^*$, which proves injectivity.
	\\ \\ \\
	Let $y \in \mathbb{Z}^*$
	\begin{align*}
		y = 1 - x = f(x) \qquad \text{ f.s } x \in \mathbb{Z}
	\end{align*}
	 Because for all $y \in \mathbb{Z}^*$, $y$ can be written as a function of $f$, this proves surjectivity.
	 \\ \\ \\
	 Let $a,b \in \mathbb{Z}$
	 \begin{align*}
	 	f(a) \oplus f(b) &= (1 - a) + (1 - b) - 1 \\
	 	&= 2 - a - b - 1 = 1 - (a+b) = f(a+b) \\
	 	f(a) \odot f(b) &= (1 - a) + (1 - b) - (1 - a)(1 - b) \\
	 	&= 2 - (a + b) - 1 + (a + b) - ab = 1 - ab = f(ab)
	 \end{align*}
		This shows that $f$ is a homomorphism. 
	\\ \\
	\textbf{\boldmath Thus $f$ is an isomorphism, with $\mathbb{Z}$ being isomorphic to $\mathbb{Z}^*$.}
	
\newpage 

\noindent \textbf{Section 4.1 Problem 5 (d): }Find Polynomials $q(x)$ and $r(x)$ such that $f(x) = g(x)q(x) + r(x)$, and $r(x) = 0$ or deg $r(x) <$ deg $g(x)$:
	\begin{align*}
		f(x) &= 4x^4 + 2x^3 + 6x^2 + 4x + 5 \\
		g(x) &= 3x^2 + 2
	\end{align*}
	with $f(x), g(x) \in \mathbb{Z}_7[x]$
	\\ 
	\begin{center} 
	\text{} \qquad \qquad \qquad \qquad $\frac{4}{3}x^2 + \frac{2}{3}x + \frac{10}{9}$ \\ 
	$3x^2 + 2 \overline{|4x^4 + 2x^3 + 6x^2 + 4x + 5}$ \\ 
	\text{} \qquad $-(4x^4 + \qquad + \frac{8}{3}x^2)$ \\ 
	\text{}\qquad \qquad \qquad $\overline{2x^3 + \frac{10}{3}x^2 + 4x + 5}$ \\ 
	\text{}\qquad \qquad \quad $-(2x^3 + \qquad + \frac{4}{3}x)$ \\ 
	\text{}\qquad \qquad \qquad \qquad \quad $\overline{\frac{10}{3}x^2 + \frac{8}{3}x + 5}$  \\ 
	\text{}\qquad \qquad \qquad \qquad $-(\frac{10}{3}x^2 + \quad + \frac{20}{9})$ \\ 
	\text{}\qquad \qquad \qquad \qquad \qquad \qquad $\overline{\frac{8}{3}x + \frac{25}{9}}$ 
	\end{center}
	\begin{align*}
		q(x) &= 4(3^{-1})x^2 + 2(3^{-1})x + 10(9^{-1}) \\
		r(x) &= 8(3^{-1})x + 25(9^{-1})
	\end{align*}
	Notice: Because the polynomials are in $\mathbb{Z}_7[x]$, $3^{-1} = 5$ and $9^{-1} = 4$
	\begin{align*}
	q(x) &= 4(5)x^2 + 2(5)x + 10(4) \\
	&= 20x^2 + 10x + 40 \\
	&= 6x^2 + 3x + 5 \\
	r(x) &= 8(5)x + 25(4) \\
	&= 40x + 100 \\
	&= 5x + 2
	\end{align*}

\newpage 

\noindent \textbf{Section 4.1 Problem 18: }Let $\phi:R[x] \rightarrow R$ be the function that maps each polynomial in $R[x]$ onto its constant term (an element of $R$). Show that $\phi$ is a surjective homomorphism of rings. 
\\ \\
Let $\phi(f(x)) = \phi(ax^n + ... + C) = C $
\\ \\
Let $c \in R$
	$$
	c = \phi(ax^n + ... + c)
	$$
Because for all $c \in R$, there exists a  polynomial in which $c = \phi(ax^n + ... + c)$, thus $\phi(f(x)) = c$ is surjective.
\\ \\
Let $a,b \in R[x]$, with $ax^n + ... + c_1$ and $bx^m + ... + c_2$
	\begin{align*}
		\phi(a + b) &= \phi(ax^n + bx^m + ... + (c_1 + c_2)) = c_1 + c_2 = \phi(a) + \phi(b) \\
		\phi(ab) &= \phi(abx^{n+m} + ... + c_1c_2) = c_1c_2 = \phi(a)\phi(b)
	\end{align*}
\textbf{\boldmath This proves that $\phi$ is a surjective homomorphism of rings.} 

\newpage 

\noindent \textbf{Section 4.1 Problem 20: }Let $D:R[x] \rightarrow R[x]$ be the derivative map defined by 
	$$
	D(a_0 + a_1x + ... + a_nx^n) = a_1 + ... + na_nx^{n-1}
	$$
Is D a homomorphism of rings? An isomorphism? 
\\ \\
Notice: 
	\begin{align*}
		D(x)D(x^2 + 1) = 1(2x) = 2x \not = 3x^2 + 1 = D(x(x^2 + 1)) \\
	\end{align*}
Because D is not a homomorphism, as it does not hold for multiplication of polynomials, D is also not an isomorphism.

\newpage 

\noindent \textbf{Section 4.2 Problem 14: }Let $f(x)g(x)h(x) \in F[x]$, with $f(x)$ and $g(x)$ relatively prime. If $f(x)|h(x)$ and $g(x)|h(x)$, prove that $f(x)g(x)|h(x)$
\\ \\
Let $f(x),g(x),h(x),u(x),v(x),w(x) \in F[x]$. Assume $f(x)|h(x)$ and $g(x)|h(x)$, and $f(x)$ and $g(x)$ relatively prime.
	\begin{align*}
		h(x) &= f(x)u(x) = g(x)v(x) \\
	\end{align*}
Now we can see that $f(x)|g(x)v(x)$. Because $f(x)$ and $g(x)$ relatively prime, we know that $f(x)|v(x)$
	\begin{align*}
		v(x) &= f(x)w(x) \\
		h(x) &= g(x)v(x) = g(x)f(x)w(x) 
	\end{align*}
Thus $f(x)g(x)|h(x)$.
	
\end{document}