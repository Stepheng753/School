\documentclass[12pt]{article}
\usepackage[margin = 1in]{geometry}
\usepackage{amsmath}
\usepackage{amssymb}
\usepackage{amsthm}
\usepackage{graphicx}
\usepackage{subfig}
\usepackage{enumitem}

\begin{document}
	
	\begin{center}
		\textbf{Homework 9} \\
		\textbf{Abstract Algebra} \\
		\textbf{Math 320} \\
		\textbf{Stephen Giang} \\
	\end{center}

\noindent \textbf{Problem 1: }Let $f(x), g(x) \in F[x],$ not both zero. Prove that if there exist $u(x), v(x) \in
F[x]$ such that $f(x)u(x) + g(x)v(x) = 1_F$, then $f(x)$ and $g(x)$ are relatively prime.
\\ \\ \\
By Theorem 4.8, $f(x)u(x) + g(x)v(x) = d(x) = 1_F$, such that $d(x) = gcd(f(x),g(x))$. Because $d(x) = gcd(f(x),g(x)) = 1_F$, by definition of relatively prime, $f(x)$ and $g(x)$ are relatively prime.

\newpage 

\noindent \textbf{Problem 2: }List all associates of $x^2 + x + 1$ in $\mathbb{Z}_5[x]$.
\\ \\
All associates of $f(x) = x^2 + x + 1$ can be written as $cf(x)$ for $c \in \mathbb{Z}_5$.
	\begin{align*}
		1(x^2 + x + 1) = x^2 + x + 1 \\
		2(x^2 + x + 1) = 2x^2 + 2x + 2 \\
		3(x^2 + x + 1) = 3x^2 + 3x + 3 \\
		4(x^2 + x + 1) = 4x^2 + 4x + 4
	\end{align*}
	
\newpage 

\noindent \textbf{Problem 3: }Show that $x - 1_F$ divides $a_nx^n + ... + a_2x^2 + a_1x + a_0 \in F[x]$ if and only if $a_n + a_{n-1} + ... + a_2 + a_1 + a_0 = 0_F$ .
\\
	\begin{proof}[($=>$)]
		Let $f(x) = a_nx^n + ... + a_2x^2 + a_1x + a_0 \in F[x]$ such that $x - 1_F$ divides $f(x)$
		\\ \\
		Because $x - 1_F$ divides $f(x)$, $1_F$ is a root of $f(x)$, such that 
		$$
		f(1_F) = a_n + a_{n-1} + ... + a_2 + a_1 + a_0 = 0_F
		$$ 
		\\ \\
		($<=$). Let $a_n + a_{n-1} + ... + a_2 + a_1 + a_0 = 0_F$. 
		\\ \\
		Thus there exists $f(x) \in F[x]$, with $f(x) = a_nx^n + ... + a_2x^2 + a_1x + a_0$, such $f(1_F) = 0_F$. Because $f(1_F) = 0_F$, $1_F$ is a root, meaning that $x - 1_F$ divides $f(x)$ 
		\\
	\end{proof}

\newpage 

\noindent \textbf{Problem 4: }We say that $a \in F$ is a multiple root of $f(x) \in F[x]$ if $(x - a)^k$ is a factor of $f(x)$ for some $k \geq 2$
	\\
	\begin{enumerate}[label = (\alph*)]
		\item Prove that $a \in \mathbb{R}$ is a multiple root of $f(x) \in \mathbb{R}[x]$ if and only if $a$ is a root of both $f(x)$ and $f'(x)$, where $f'(x)$ is the derivative of $f(x)$. You may assume that the Product Rule is true
		\begin{proof}[($=>$)]
			Let $a \in \mathbb{R}$ be a multiple root of $f(x) \in \mathbb{R}[x]$.
			\\ \\
			So $\exists u(x) \in \mathbb{R}[x]$, such that $f(x) = u(x)(x - a)^k$, with $k \geq 2$. Notice that: 
			\begin{align*}
			f'(x) &= u(x)(k(x - a)^{k-1}) + u'(x)(x - a)^k, \text{ with } (k-1) \geq 1 \\
			&= (x-a)\left[ u(x)(k(x - a)^{k-2}) + u'(x)(x - a)^{k-1} \right]
			\end{align*}
			Because $(x-a)$ is a factor of both $f(x)$ and $f'(x)$, $a$ is a root of both $f(x)$ and $f'(x)$ 
			\\ \\ \\
			($<=$) Let $a$ be a root of both $f(x)$ and $f'(x)$. 
			\\ \\
			Thus $\exists u(x) \in \mathbb{R}[x]$, such that $f(x) = u(x)(x - a)^k$. 
			\\ \\
			If ($k < 1$), then $a$ would not be a root of $f(x)$. \\
			If ($k = 1$), then $f'(x) = u(x) + u'(x)(x - a)^k$, meaning $a$ would not be a root of $f'(x)$. \\
			If ($k > 1$), then $f'(x) = (x-a)\left[ u(x)(k(x - a)^{k-2}) + u'(x)(x - a)^{k-1} \right]$. \\ 
			\\
			Thus $k > 1$, or $k \geq 2$, to have $a$ be a root of both $f(x)$ and $f'(x)$. And because $k \geq 2$, $a$ is a  multiple root of $f(x)$
			\\
		\end{proof}
		\item If $f(x) \in \mathbb{R}[x]$ and $f(x)$ is relatively prime to $f'(x)$, prove that f(x) has no multiple roots in $\mathbb{R}$.
		\\ \\
		So we can prove this by proving the contraposition. 
		$$
		\textit{If f(x) has multiple roots in $\mathbb{R}$, then $f(x) \in \mathbb{R}[x]$ and $f(x)$ is not relatively prime to $f'(x)$}
		$$
		\begin{proof}[Solution 4b]
			By part (a), if $f(x)$ has multiple roots in $\mathbb{R}$, then $f(x)$ and $f'(x)$ share a root $a$, thus sharing a factor $x-a$. Thus $f(x)$ is not relatively prime to $f'(x)$
			\\ \\
		\end{proof}
	\end{enumerate}

\newpage 

\noindent \textbf{Problem 5: }Determine if the following polynomials are irreducible:
	\begin{enumerate}[label = (\alph*)]
		\item $x^3 - 9$ in $\mathbb{Z}_{11}[x]$
		\\ \\
		We can use the Rational Roots Theorem, and see if it contains any roots, $\pm 1, \pm 9$. Let $f(x) = x^3 - 9 \in \mathbb{Z}_{11}[x]$
			\begin{align*}	
				f(1) = -8 && f(-1) = -10 \\
				f(9) = 720 = 5 && f(-9) = -738 = 1 	
			\end{align*}
		Because the degree of $f(x)$ is 3, then its factors must be of degree 1 and 2, meaning that its factors will contain its root, but because there does not exist a root in $\mathbb{Z}_{11}[x]$, (a) is irreducible.
		\\
		\item $x^4 + x^2 + 2$ in $\mathbb{Z}_{3}[x]$
		\\ \\
		We can use the Rational Roots Theorem, and see if it contains any roots, $\pm 1, \pm 2$. Let $f(x) = x^4 + x^2 + 2 \in \mathbb{Z}_{3}[x]$
			\begin{align*}
				f(1) = f(-1) = 4 = 1 \\
				f(2) = f(-2) = 22 = 1
			\end{align*}
		Because there does not exist a root, the only factors of $f(x)$ have to be of degree 2, such that for $a,b,c,d \in \mathbb{Z}_{3}[x]$ 
		\begin{align*}
			f(x) &= x^4 + x^2 + 2 = (x^2 + ax + b)(x^2 + cx + d) \\
			&= x^4 + (a+c)x^3 + (ac + b + d)x^2 + (bc + ad)x + bd
		\end{align*}
		Now we just need to solve for $a,b,c,d$
		\begin{align}
			a + c &= 0 \\
			ac + b + d &= 1 \\
			bc + ad &= 0 \\
			bd &= 2	
		\end{align}
		Now we can see that $c = -a$ from $(1)$, and $b = 2, d = 1$ or $b = 1, d = 2$, such that $b+d = 3 = 0$ from $(4)$. Now by evaluating, we can see in $(2)$, $a^2 = -1 = 2$.  Because there does not exist an $a \in \mathbb{Z}_{3}$, such that $a^2 = 2$, there does not exist any factors of $f(x)$. Proving that $f(x)$ is irreducible.
	\end{enumerate}










\end{document}
