\documentclass[12pt]{article}
\usepackage[margin = 1in]{geometry}
\usepackage{amsmath}
\usepackage{amssymb}
\usepackage{amsthm}
\usepackage{enumitem}

\begin{document}
	
	\begin{center}
		\textbf{Midterm 1} \\
		\textbf{Abstract Algebra} \\
		\textbf{Math 320} \\
		\textbf{Stephen Giang, Emily Boyd} \\
		\textbf{Bennett Guillaume, Dacia Bond}
	\end{center}

\noindent \textbf{Problem 1: }
	\begin{enumerate}[label = (\alph*)]
		\item Let $q > 0$ be prime.  Prove that for $1 \leq s \leq q - 1, q$ divides $\binom{q}{s}$, where $\binom{q}{s} = \frac{q!}{s!(q-s)!}$. You may assume $\binom{q}{s}$ is an integer. 
		\begin{proof}[Solution]
			Let $q > 0$ and  be prime and $1 \leq s \leq q - 1$
			\begin{align*}
				&\binom{q}{s} = \frac{q!}{s!(q-s)!} \\
				&q! = \binom{q}{s}(s!(q-s)!) \\
				&q\left |\prod_{k = 0}^{q} q-k \right. \text{ so } q|q! \\
				&q|q! = q\left|\binom{q}{s}(s!(q-s)!)\right. \\
				&\text{By Corrollary 1.6: } q\left|\binom{q}{s}\right. \text{ or } q|(s!(q-s)!) 
			\end{align*}
			Because $s$ and $(q-s) < q$, their prime factorization doesn't contain q, so $q \not | (s!(q-s)!)$. Thus $q$ has to divide $\binom{q}{s}$ \\
		\end{proof} 
		\item Let $q > 0$ be prime. Prove that for any $\beta, \gamma \in \mathbb{Z}_q, (\beta + \gamma)^q = \beta^q + \gamma^q \text{ in } \mathbb{Z}_q$
		\begin{proof}[Solution]
			Let $q > 0$ and  be prime and $\beta, \gamma \in \mathbb{Z}_q$
			\begin{align*}
				&\text{By Binomial Therorem: } \\
				&(\beta + \gamma)^q = \beta^q + \left (\sum_{k = 1}^{q-1} \binom{q}{k} \beta^{q - k} \gamma^{k} \right )+ \gamma^q \\
				&\text{Because } q\left|\binom{q}{s}\right. \text{ with } 1 \leq s \leq q - 1 \\
				&q\left|\binom{q}{k}\right. \text{, which means that } \left [ \binom{q}{k} \right] = [0]_q \qquad 1 \leq k \leq q-1 \\
				&\text{So } (\beta + \gamma)^q = \beta^q + \left ( \sum_{k = 1}^{q-1} (0) \beta^{q - k} \gamma^{k} \right ) + \gamma^q 
			\end{align*}
			Thus for any $\beta, \gamma \in \mathbb{Z}_q, (\beta + \gamma)^q = \beta^q + \gamma^q \text{ in } \mathbb{Z}_q$  \\
		\end{proof} 
	\end{enumerate}


\newpage 
\noindent \textbf{Problem 2: }
	\begin{enumerate}[label = (\alph*)]
		\item Let $x,y,z \in \mathbb{Z}$. If $x|z$ and $y|z$ and $(x,y) = w$, prove that $xy|wz$
		\begin{proof}[Solution]
			Let $x,y,z \in \mathbb{Z}$ with $x|z$ and $y|z$ and $(x,y) = w$
			\begin{align*}
				z &= xa = yb \\
				w &= xu + yv \\
				wz = w(xa) &= xa(xu + yv) \\
				&= xaxu + xayv \\
				&= ybxu + xayv \\
				&= xy(bu + av)
			\end{align*}
			Because $wz$ can be written as a product of an integer and $xy$, $xy|wz$ \\  
		\end{proof}
		\item Suppose $\chi$ and $\rho$ are primes, and $\chi, \rho \geq 5$. Prove that $24|(\chi^2 - \rho^2)$ 
		\begin{proof}[Solution]
			Let $\chi$ and $\rho$ are primes, and $\chi, \rho \geq 5$. Let $q_1,q_2,k_1,k_2 \in \mathbb{Z}$
			\begin{align*}
				&\text{Notice: } \chi^2 \text{ prime factorization: } 1, \chi \\
				&\text{Notice: } \rho^2 \text{ prime factorization: } 1, \rho \\
				&\text{Because } \chi \geq 5 \text{ and prime }, \quad (\chi^2,3) = 1  \quad \& \quad (\chi^2,(2)(2)(2)) = 1 \\
				&\text{Because } \rho \geq 5 \text{ and prime }, \quad (\rho^2,3) = 1  \quad \& \quad (\rho^2,(2)(2)(2)) = 1 
			\end{align*}
			If $\rho$ or $\chi = \{3k + 1, \quad 3k + 2,\quad 8k + 1,\quad 8k + 3,\quad 8k + 5, \quad 8k + 7\}$, then \\ $\chi^2 = \{3q_1 + 1,\quad 8k_1 + 1\}$ and $\rho^2 = \{3q_2 + 1,\quad 8k_2+2\}$
			\begin{align*}
				\chi^2 = 3q_1 + 1 \quad \rho^2 = 3q_2 + 1  &&  \chi^2 = 8k_1 + 1 \quad \rho^2 = 8k_2 + 1  \\
				(\chi^2 - \rho^2) = 3(q_1 - q_2) &&  (\chi^2 - \rho^2) = 8(k_1 - k_2) \\
				3 | (\chi^2 - \rho^2)	&&  8 | (\chi^2 - \rho^2)
			\end{align*}
			From part (A), because $3 | (\chi^2 - \rho^2)$ and $8 | (\chi^2 - \rho^2)$ with $(8,3) = 1$,  $24 | (\chi^2 - \rho^2)$ \\
		\end{proof}
	\end{enumerate}

\newpage 
\noindent \textbf{Problem 3: }
	\begin{enumerate}[label = (\alph*)]
		\item Prove that if $\mu,\nu \in \mathbb{Z}$ and $(\mu,\nu) = 1$, then $(\mu + \nu,\nu) = 1$
		\begin{proof}[Solution]
			Let $\mu,\nu,q \in \mathbb{Z}$ and $(\mu,\nu) = 1$, Let $d|(\mu + \nu)$ and $d|\nu$
			\begin{align*}
				&\nu = dk \qquad \qquad \text{for some k} \in \mathbb{Z} \\
				&\mu + \nu = dq \qquad \text{ for some q} \in \mathbb{Z} \\
				&\mu = dq - dk = d(q - k) \\
				&d|\mu
			\end{align*}
			Because $d|\mu$ and $d|\nu,$ the only d to divide $\mu, \nu, \mu + \nu$ is 1, thus $(\mu + \nu,\nu) = 1$
		\end{proof}
		\item Prove that if $\mu,\nu \in \mathbb{Z}$ and $(\mu,\nu) = 1$, then $(\mu + \nu,\nu^n) = 1$
		\begin{proof}[Solution]
			Let $\mu,\nu,q \in \mathbb{Z}$ and $(\mu,\nu) = 1$ for $n \geq 1$ \\ \\
			Because $\mu$ and $\nu$ dont share any prime factors, $\mu$ and $\nu^n$ wont share any prime factors as $\nu^n$ will consist of multiple $\nu$'s that again dont share any prime factors with $\mu$. 
			\begin{align*}
				(\mu,\nu^n) = 1
			\end{align*} 
			We know from part (a), that because $(\mu,\nu) = 1, (\mu + \nu,\nu) = 1$ If we let d divide $\mu + \nu$ and $\nu$, then d divides $\mu$. And because d divides $\nu$, it divides $\nu^n$.  So because d divides $\mu$ and $\nu ^ n$, d has to be one, meaning $(\mu + \nu,\nu^n) = 1$ 
		\end{proof}
		\item Let $q$ be prime, and $\mu,\nu \in \mathbb{Z}_{>0}$ such that $(\mu,\nu) = 1$. Prove that
		$$ \left( \mu + \nu, \frac{\mu ^ q + \nu^q}{\mu + \nu}\right) = 1 \text{ or } q $$ 
		\begin{enumerate}[label = (\roman*)]
			\item Notice: $\mu^q = ((\mu + \nu) - \nu)^q$
			\begin{align*}
				\mu^q +  \nu^q &= ((\mu + \nu) - \nu)^q + \nu^q \\
				&= \left(\sum_{k = 0}^{q} \binom{q}{k} (\mu + \nu)^{q-k} (-\nu)^k\right) + \nu^q \\
				&= \left(\sum_{k = 0}^{q - 1} \binom{q}{k}  (\mu + \nu)^{q-k} (-\nu)^k\right) + (-\nu)^q + \nu^q \\
				&\text{Notice: $q$ cannot be even, because it is prime, so $(-\nu)^q = -\nu^q$} \\
				&= \sum_{k = 0}^{q - 1} \binom{q}{k} (\mu + \nu)^{q-k} (-\nu)^k \\
				\frac{\mu ^ q + \nu^q}{\mu + \nu} &= \frac{1}{\mu + \nu}\sum_{k = 0}^{q - 1} \binom{q}{k} (\mu + \nu)^{q-k} (-\nu)^k \\
				&= \sum_{k = 0}^{q - 1} \binom{q}{k} (\mu + \nu)^{q-1-k} (-\nu)^k
			\end{align*}
			Because integers are closed under (+) and ($\times$), $\frac{\mu ^ q + \nu^q}{\mu + \nu} \in \mathbb{Z}$ 
			\item Let $d = \left( \mu + \nu, \frac{\mu ^ q + \nu^q}{\mu + \nu}\right)$, so $d|\frac{\mu ^ q + \nu^q}{\mu + \nu}$, $d|(\mu + \nu)$ and Let $a,b,c \in \mathbb{Z}$
			\begin{align*}
				da &= \frac{\mu ^ q + \nu^q}{\mu + \nu} \\
				&= \sum_{k = 0}^{q - 1} \binom{q}{k} (\mu + \nu)^{q-1-k} (-\nu)^k \\
				&= \left(\sum_{k = 0}^{q - 2} \binom{q}{k} (\mu + \nu)^{q-1-k} (-\nu)^k\right) + q(-\nu)^{q-1}
			\end{align*}
			Because $d|(\mu + \nu)$, $d|(\mu + \nu)^z$ for $z \in \mathbb{Z}^+$, so 
			$$d \left | \sum_{k = 0}^{q - 2} \binom{q}{k} (\mu + \nu)^{q-1-k} (-\nu)^k ,\qquad  \sum_{k = 0}^{q - 2} \binom{q}{k} (\mu + \nu)^{q-1-k} (-\nu)^k \right. = db$$
			\begin{align*}
				da - db &= q(-\nu)^{q-1} \\
				dc &= q(-\nu)^{q-1}
			\end{align*}
			Because q is prime, q is odd, such that (q-1) is even, so 
			$$ dc = q\nu^{q-1} $$
			Thus, $d| q\nu^{q-1} $
			\item Notice from part (b): $(\mu + \nu, \nu^{q-1}) = 1$.  This means $\mu + \nu$ and $\nu^{q-1}$ don't share any factors except 1.  Because d is a factor of $\mu + \nu$, d is not a factor of $\nu^{q-1}$, unless $d = 1$.  Thus $(d, \nu^{q-1}) = 1$ 
			\item Because $(d,\nu^{q-1}) = 1$ and $d|q\nu^{q-1}$, $d|q$ by Theorem 1.4, if d divides a prime, q, \\ d = 1 or q, so: $$ \left( \mu + \nu, \frac{\mu ^ q + \nu^q}{\mu + \nu}\right) = 1 \text{ or } q $$ 
		\end{enumerate}
	\end{enumerate}

\newpage 
\noindent \textbf{Problem 4: } Let $L$ be the set of positive real numbers. Define alternate addition and multiplication operations on L by 
	\begin{align*}
		a \oplus b = ab && a \otimes b = a^{\ln b}
	\end{align*}
	\begin{enumerate}[label = (\alph*)]
		\item Prove or disprove: $L$ is commutative.
		\begin{proof}[Solution]
			Let $a \otimes b = y = a^{\ln b}$, and $b \otimes a = x = b^{\ln a}$
			\begin{align*}
				y = a^{\ln b} && x = b^{\ln a}\\
				\ln y = \ln a^{\ln b} && \ln x = \ln b ^{\ln a}\\
				= \ln b \ln a && = \ln a \ln b
			\end{align*}
			Because multiplication of real numbers is commutative, $\ln b \ln a = \ln a \ln b$ , so $\ln y = \ln x$, which means that $y = a \otimes b = x = b \otimes a$, thus $L$ is commutative.
		\end{proof}
		\item Find the multiplicative identity of L.
		\begin{proof}[Solution]
			Let $x$ be the multiplicative identity, $1_L$, such that $a \otimes x = a$
			\begin{align*}
				&a \otimes x = a^{\ln x} \\
				&\text{Because $a \otimes x = a$} \qquad 1 = \ln x 
			\end{align*}
			Thus the multiplicative identity, $x = 1_L = e$  \\
		\end{proof}
		\item Prove that $L$ is a field
		\begin{proof}[Solution]
			From part (b): $1_L = e$, let $x$ be denoted as the multiplicative inverse of $a$ such that $a \otimes x = 1_L$ 
			\begin{align*}
				&a \otimes x = 1_L \\
				&a ^ {\ln x} = e \\
				&\ln x \ln a = 1 \\
				&x = e^{1 / \ln a} \text{ or } e^{\log_a (e)}
			\end{align*}
			Because $\forall a \in L, \exists x$ such that $a \otimes x = 1_L$, L is a field.
		\end{proof}
	\end{enumerate}

\newpage
\noindent \textbf{Problem 5: } Let S be a set, and let $2^S$ denote the power set of S, i.e. the set of all subsets of S. Define addition and multiplication in $2^S$ by the rules: 
	\begin{align*}
		&M + N = (M - N) \cup (N - M), &MN = M \cap N
	\end{align*}
where 
	$$ M - N = M \backslash N = \{ x \in S: x \in M, x \not \in N\} $$
Under these operations, we may assume that $2^S$ is a ring. \\
	\begin{enumerate}[label = (\alph*)]
		\item Show that S is the multiplicative identity of this ring.
		\begin{proof}[Solution]
			Let $M \in 2^S$ and represent any arbitrary element of $2^S$
			\begin{align*}
				MS = M \cap S = M
			\end{align*}
		Thus by the definition of multiplicative identity, S is the multiplicative identity \\
		\end{proof}
		\item Show that the empty set $\emptyset$ is the additive inverse of $2^S$
		\begin{proof}[Solution]
			Let $M \in 2^S$ and represent any arbitrary element of $2^S$
			\begin{align*}
				M + \emptyset = (M - \emptyset) \cup (\emptyset - M) = M \cup \emptyset = M
			\end{align*}
			Thus by the definition of the additive inverse, $\emptyset$ is the additive inverse of $2^S$ \\
		\end{proof}
		\item Prove that if $T \in 2^S$ and $T \subsetneq S$, then T is not a unit in $2^S$.
			\begin{proof}[Solution]
				Let $T,R \in 2^S$ and $T,R \subsetneq S$
				\begin{align*}
					&TR = T \cap R. 
				\end{align*}
				Because $[T,R \subsetneq S], [T \cap R \subsetneq S]$, which means $TR \not = S$ thus T is not a unit \\
			\end{proof}
		\newpage
		\item Prove that under these operations, $2^S$ is an integral domain iff $|S| = 1$
			\begin{proof}[$\rightarrow$]
				Contrapositive: "If $|S| \not = 1$, then $2^S$ is not an integral domain" \\
				Let $|S| \not = 1$ \\
				Case 1: $|S| < 1$
					\begin{align*}
						&\text{So } |S| = 0, S = \emptyset
					\end{align*} 
					For $2^S$ to be an integral domain, there has to be 2 nonzero elements that multiply to equal 0.  But because S doesn't have any nonzero elements, $2^S$ is not an integral domain \\ \\ 
				Case 2: $|S| > 1$, Let $(M-N),(N-M) \in 2^S$ with (M-N),(N-M) both being nonzero elements.
					\begin{align}
						&(M-N)(N-M) = (M-N) \cap (N-M) = \emptyset
					\end{align}
				Because both of them are not the zero element, the ring is not an integral domain. 
				\\ \\
				Because for $|S| \not = 1$, the result of $2^S$ not being an integral domain holds true. So by contraposition, if $2^S$ is an integral domain, then $|S| = 1$ \\
			\end{proof}
			\begin{proof}[$\leftarrow$]
				Let $|S| = 1$, so let $A,B \in 2^S$
				$$\text{Because } |S| = 1, A = \emptyset \text{ and } B = S \text{ or } A = S \text{ and } B = \emptyset$$
				So $A = \emptyset$ or S. In any case, $AB = \emptyset$ with A or B $ = \emptyset$. \\ 
			\end{proof}
	\end{enumerate}

\newpage 
\noindent \textbf{Problem 6: } An element $a$ of a ring is called nilpotent if $a^n = 0_R$ for some postive integer $n$.
	\begin{enumerate}[label = (\alph*)]
		\item Let $a$ and $b$ be nilpotent elements in a commutative ring R. Prove that $a + b$ and $ab$ are also nilpotent.
		\begin{proof}[Solution]
			Let $a$ and $b$ be nilpotent elements in a commutative ring R
			\begin{align*}
				&(a + b)^n = \sum_{k = 0}^{n} a^{n-k}b^k = \sum_{k = 0}^{n} 0_R 0_R = 0_R \\
				&(ab)^n = a^nb^n = 0_R0_R = 0_R
			\end{align*}
			Thus $a + b$ and $ab$ are also nilpotent \\
		\end{proof}
		\item Prove that if $a$ is a nilpotent element of ring R, then $-a$ is also nilpotent.
			\begin{proof}[Solution]
				Let $a$ be nilpotent in R \\
				Case 1) n is even 
				\begin{align*}
					&(-a)^n = a^n = 0_R
				\end{align*}
				Case 2) n is odd
				\begin{align*}
					&(-a)^n = -a^n = -0_R = 0_R
				\end{align*}
				Thus $-a$ is also nilpotent
			\end{proof} 
		\item Let N be the set of all nilpotent elements of a commutative ring R. Show that N is a subring of R.
			\begin{proof}[Solution]
				Let N be the set of all nilpotent elements of a commutative ring R.
				\begin{align*}
					&(a + b)^n = \sum_{k = 0}^{n} a^{n-k}b^k = \sum_{k = 0}^{n} 0_R 0_R = 0_R \in N \\
					&(a - b)^n = \sum_{k = 0}^{n} a^{n-k}(-b)^k = \sum_{k = 0}^{n} 0_R 0_R = 0_R \in N\\
					&(ab)^n = a^nb^n = 0_R0_R = 0_R \in N \\
					&0^n = 0_R \in N
				\end{align*} 
				Because of closure under addition,subtraction,multiplication and containing $0_N$, N is a subring of R
			\end{proof}
	\end{enumerate}


\newpage
\noindent \textbf{Problem 7: (a)} If $R,S$ are rings such that $R \cong S$, then $S \cong R$
	\begin{proof}[Solution]
		Let $R,S$ be rings such that $R \cong S$.  So $f: R \rightarrow S$ with f being bijective and holding the homomorphism properties. Let $g:S \rightarrow R$
		\begin{align*}
			&\text{Let $x \in R$ and $y \in S$} \qquad f(x) = y \qquad g(y) = g(f(x)) = x \\ \\
			&\text{Let: } g(f(x_1)) = g(f(x_2)) \\
			&x_1 = x_2 \\
			&\text{Thus $f(x_1) = f(x_2)$, and $g$ is injective} \\\\
			&\text{Notice: } g(f(x)) = x \\
			&\text{Because f is surjective, for all } x \in R, \text{ there exists an } f(x) \text{ such that $g(f(x)) = x$,} \\
			&\text{Thus g is surjective.} \\ \\
			&\text{Notice: } f(x_1 + x_2) = f(x_1) + f(x_2)\\
			&g(f(x_1) + f(x_2)) = g(f(x_1 + x_2)) = x_1 + x_2 \\
			&g(f(x_1)) + g(f(x_2)) = x_1 + x_2 \\ \\
			&\text{Notice: } f(x_1x_2) = f(x_1)f(x_2) \\
			&g(f(x_1)f(x_2)) = g(f(x_1x_2)) = x_1x_2 \\
			&g(f(x_1))g(f(x_2)) = x_1x_2			
		\end{align*}
		Because g is bijective and holds the homomorphism properties, $S \cong R$ 
	\end{proof}

\newpage 
\noindent \textbf{Problem 8: } Let C be the set $\mathbb{R} \times \mathbb{R}$ with the usual coordinate addition and a new multiplication given by 
$$ (a,b)(c,d) = (ac - bd, ad + bc) $$
Under these operations, $\mathbb{R} \times \mathbb{R}$ is a field.
	\begin{enumerate}[label = (\alph*)]
		\item Find the multiplicative identity of C and show that every nonzero element $(a,b)$ has a multiplicative inverse in C.
		\begin{proof}[Solution]
			Find $(x,y)$ such that $(a,b)(x,y) = (ax - by, ay + bx) = (a,b)$ (Read Left to Write for Elimination Steps)
			\begin{align*}
				1) ax - by = a   && 2) ax - by = a && 3) ax - by = a\\ 
				bx + ay = b   && \frac{-a}{b}(bx + ay = b) && -ax - \frac{a^2}{b}y = -a \\ \\
				-y(b + \frac{a^2}{b}) = 0 && -y(b^2 + a^2) = 0 && y = 0
			\end{align*}
			By repluggin in $y = 0$, we get $x = 1$, so then the multiplicative identity is (1,0)
		\end{proof}
		\begin{proof}[Solution]
			Find $(x,y)$ such that $(a,b)(x,y) = (ax - by, ay + bx) = (1,0)$ (Read Left to Write for Elimination Steps)
			\begin{align*}
				1) ax - by = 1   && 2) ax - by = 1 && 3) ax - by = 1\\ 
				bx + ay = 0   && \frac{-a}{b}(bx + ay = 0) && -ax - \frac{a^2}{b}y = 0 \\ \\
				-y(b + \frac{a^2}{b}) = 1 && -y(b^2 + a^2) = b && y = \frac{-b}{b^2 + a^2}
			\end{align*}
			By replugging in $y = \frac{-b}{b^2 + a^2}$, we get $x = \frac{a}{b^2 + a^2}$, so then the multiplicative inverse is $$ \left(\frac{a}{b^2 + a^2}, \frac{-b}{b^2 + a^2}\right) $$
		\end{proof}
		\newpage
		\item Prove that C $\cong \mathbb{C}$
		\begin{proof}[Solution]
			Define f as f: C $\rightarrow \mathbb{C}$ 
			\begin{align}
				&\text{Let } f((a,b)) = f((c,d)) \\
				&f((a,b)) = a + bi = f((c,d)) = c + di
			\end{align}
			Because $f((a,b)) = f((c,d)), a = c, b =d $ such that $(a,b) = (c,d)$, so f is injective \\
			\begin{align*}
				&\text{Let } y = a + bi
			\end{align*}
			If we set $x = (a,b)$.  Thus there exists an $x$, that satisfies $f(x) = y$ for all $y \in \mathbb{C}$.  So f is surjective.
			\begin{align*}
				&f((a,b)+(c,d)) = f((a+c,b+d)) = (a+c) + (b+d)i \\
				&f((a,b)) + f((c,d)) = (a+c) + (b+d)i \\ \\
				&f((a,b)(c,d)) = f((ac - bd, ad + bc)) = (ac-bd) + (ad + bc)i \\
				&f((a,b))f((c,d)) = (a+bi)(c+di) = (ac-bd) + (ad + bc)i
			\end{align*}
			Because the homomorphism properties hold, along with the function f is bijective, $C \cong \mathbb{C}$ \\
		\end{proof}
	\end{enumerate}

\newpage 
\noindent \textbf{Problem 9: } 
	\begin{enumerate}[label = (\alph*)]
		\item Show that $\mathbb{Z}$ and $\mathbb{Q}$ both have characteristic zero, and that $\mathbb{Z}_n$ has a characteristic $n$
		\begin{proof}[Solution]
			Notice: $1_\mathbb{Z} = 1_\mathbb{Q} = 1$ and $0_\mathbb{Z} = 0_\mathbb{Q} = 0$ and $1_{\mathbb{Z}_n} = [1]_n$ and $0_{\mathbb{Z}_n} = [0]_n$. Let x denote the solution of $x1_R = 0_R$
			\begin{align*}
				x(1) = 0 	&&	[x]_n[1]_n = [0]_n\\
				\frac{x(1)}{1} = \frac{0}{1} && [x]_n = [0]_n \\
				x = 0 && x = n
			\end{align*}
			Thus $\mathbb{Z}$ and $\mathbb{Q}$ both have characteristic zero and $\mathbb{Z}_n$ has a characteristic $n$ \\
		\end{proof} 
		\item What is the characteristic of $A = M_2(\mathbb{Z}_2) \times \mathbb{Z}_3$
		\begin{proof}[Solution]
			Notice $1_A = \left( 
			\begin{pmatrix}
			[1]_2 & [0]_2 \\
			[0]_2 & [1]_2
			\end{pmatrix},[1]_3
			\right)$ and $0_A = \left( 
			\begin{pmatrix}
			[0]_2 & [0]_2 \\
			[0]_2 & [0]_2
			\end{pmatrix},[0]_3
			\right)$.\\ Let x denote the solution to $x1_A = 0_A$
			\begin{align*}
				x\left( 
				\begin{pmatrix}
				[1]_2 & [0]_2 \\
				[0]_2 & [1]_2
				\end{pmatrix},[1]_3
				\right) = \left( 
				\begin{pmatrix}
				[0]_2 & [0]_2 \\
				[0]_2 & [0]_2
				\end{pmatrix},[0]_3
				\right)
			\end{align*}
			Because the characteristic of  $M_2(\mathbb{Z}_2)$ is 
			$\begin{pmatrix}
				[2]_2 & [2]_2 \\
				[2]_2 & [2]_2
			\end{pmatrix}$, and the characteristic of $\mathbb{Z}_3$ is 3, and by properties of multiplication under Cartesian Product of Rings, the characteristic of $A =
			\left( 
			\begin{pmatrix}
				[2]_2 & [2]_2 \\
				[2]_2 & [2]_2
			\end{pmatrix}, 3\right)$
		\end{proof}
		\item Prove that the characteristic of an integral domain D must either be 0 or a prime p.
		\begin{proof}[Solution]
			Let the characteristic of D be a composite number, n.
			\begin{align*}
				&n = mk \\
				&n1_D = 0_D \\
				&(m1_D)(k1_D) = 0_D \\
				&\text{Because D is an integral domain, either $m1_D = 0_D$ or $k1_D = 0$} 
			\end{align*}
			If either is true, then n would not be smallest positive integer that satisfies $n1_D = 0_D$, thus n cannot be composite by contradiction. \\ \\
			Let the characteristic of D be $1_D$ \\
			This would be impossible because this would contradict the definition of the multiplicative identity. 
			\\ \\
			Because the characteristic is the smallest positive number, n, and we have proved it can't be 1 or composite for all integral domains, D, the characteristic is either 0 or prime p.
		\end{proof}
	\end{enumerate}

\newpage
\noindent \textbf{Problem 10: }
	\begin{enumerate}[label = (\alph*)]
		\item Prove that $\mathbb{Z}$ and $M_3(\mathbb{Z}_2)$ are not isomorphic.
		\begin{proof}[Solution]
			Define f: $\mathbb{Z} \rightarrow M_3(\mathbb{Z}_2)$ such that for some $a \in \mathbb{Z}, f(a) = 
			\begin{pmatrix}
				[a]_2 & [a]_2 & [a]_2 \\
				[a]_2 & [a]_2 & [a]_2 \\
				[a]_2 & [a]_2 & [a]_2 
			\end{pmatrix}$.
			\begin{align*}
				\text{Notice: } f(0) = f(2) = f(4) = 
				\begin{pmatrix}
					[0]_2 & [0]_2 & [0]_2 \\
					[0]_2 & [0]_2 & [0]_2 \\
					[0]_2 & [0]_2 & [0]_2 
				\end{pmatrix}
			\end{align*}
			This means that f is not injective, thus proving that $\mathbb{Z} \not \cong M_3(\mathbb{Z}_2)$ \\
		\end{proof}
		\item Prove that $\mathbb{Z}_4 \times \mathbb{Z}_2$ and $\mathbb{Z}_8$ are not isomorphic
		\begin{proof}[Solution]
			Define f: $\mathbb{Z}_4 \times \mathbb{Z}_2 \rightarrow \mathbb{Z}_8$ such that for some $a,b \in \mathbb{Z}, f(([a]_4,[b]_2)) = [ab]_8$ 
			\begin{align*}
				&f([1]_4,[2]_2) = f([1]_4,[0]_2) \\ 
				&f([1]_4,[2]_2) = [2]_8 \\
				&f([1]_4,[0]_2) = [0]_8 
			\end{align*}
			Because $[2]_8 \not = [0]_8$, f is not injective, thus $\mathbb{Z}_4 \times \mathbb{Z}_2 \not \cong \mathbb{Z}_8$
		\end{proof}
	\end{enumerate}

\end{document}
