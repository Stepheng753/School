\documentclass[12pt]{article}
\usepackage[margin = 1in]{geometry}
\usepackage{amsmath}
\usepackage{amssymb}
\usepackage{amsthm}
\usepackage{graphicx}
\usepackage{subfig}
\usepackage{enumitem}

\begin{document}
	
	\begin{center}
		\textbf{Exam 3} \\
		\textbf{Differential Equations} \\
		\textbf{Math 337} \\
		\textbf{Stephen Giang} \\
	\end{center}

\noindent \textbf{Problem 1: }For this problem use the Method of Undetermined Coefficients to solve this problem. In your written answer be sure to show all of the steps for obtaining your answer
\\ 
	\begin{enumerate}[label = (\alph*)]
		\item Find a particular solution to the nonhomogeneous differential equation
			$$
			y'' - 5y' = 40e^{5x} - 100x
			$$
		Using the Method of Undetermined Coefficients, the particular solution and its derivatives can be written in the form:
			\begin{align*}
				y_p &= Ax{{e}^{5\,x}}+B{x}^{2}+Cx \\
				y_p' &= A{{e}^{5\,x}}+5\,Ax{{e}^{5\,x}}+2\,Bx+C \\
				y_p'' &= 10\,A{{e}^{5\,x}}+25\,Ax{{e}^{5\,x}}+2\,B
			\end{align*}
		By substituting these equations into the left hand side, we can find $A,B$ 
			\begin{align*}
				y_p'' - 5y_p' &= 10\,A{{e}^{5\,x}}+25\,Ax{{e}^{5\,x}}+2\,B -5\,A{{e}^{5\,x}}-25\,Ax{{ e}^{5\,x}}-10\,Bx-5\,C \\
				&= 5\,A{{e}^{5\,x}}-10\,Bx + (2\,B-5\,C) \\
				&= 40e^{5x} - 100x
			\end{align*}
		So now we can see that $A = 8, B = 10, C = 4$, thus giving us:
			\boldmath
			$$
			y_p = 8x{{e}^{5\,x}}+10{x}^{2}+4x
			$$
			\unboldmath
		\item Find the most general solution to the associated homogeneous differential equation
		\\ \\
		We can find the homogeneous solution can be find by finding the eigenvalues through the following characteristic equation:
			\begin{align*}
				\lambda^2 - 5\lambda &= 0 \\
				\lambda(\lambda - 5) &= 0 
			\end{align*} 
		So we can now see that we get the following eigenvalues: $\lambda_1 = 0, \lambda_2 = 5$, giving us the following general solution to the associated homogeneous differential equation:
			\boldmath
			$$
			y_h = c_1 + c_2e^{5x}
			$$
			\unboldmath
		\item Find the most general solution to the original nonhomogeneous differential equation
		\\ \\
		We can simply just add the particular and homogenous solution to get the most general solution to the original nonhomogeneous differential equation:	
			\boldmath
			$$
			y(x) = c_1 + c_2e^{5x} + 8x{{e}^{5\,x}}+10{x}^{2}+4x
			$$
			\unboldmath
	\end{enumerate}

\newpage 

\noindent \textbf{Problem 2: }For this problem use the Variation of Parameters method to solve this problem. In your written answer be sure to show all of the steps for obtaining your answer.
\\ 
	\begin{enumerate}[label = (\alph*)]
		\item Find a particular solution to the nonhomogeneous differential equation 
			$$
			y'' + 16y = 12\sec^2(4x)
			$$
		We first need to find the fundamental set of solutions to the homogeneous problem by solving for the eigenvalues of the following characteristic equation:
			$$
			\lambda^2 + 16 = 0 \\
			$$
		So we can see that we get the eigenvalues $\lambda = \pm 4i$. This gives us the fundamental set of solutions:	
			$$
			y_1 = \cos(4x), \qquad y_2 = \sin(4x)
			$$
		For the Variation of Parameters, we will need the Wronskian of the two solutions:
			$$
			W_{[y_1,y_2]} = 
			\begin{vmatrix}
				\cos(4x) & \sin(4x) \\
				-4\sin(4x) & 4\cos(4x)
			\end{vmatrix}
			= 4(\cos^2(4x) + \sin^2(4x)) = 4
			$$
		Now we can find the particular solution:
			\begin{align*}
				y_p &= -\cos(4x) \int^{x} \frac{12\sin(4s)\sec^2(4s)}{4}\,ds + \sin(4x)\int^{x} \frac{12\cos(4s)\sec^2(4s)}{4}\,ds \\
				&= -3\cos(4x) \int^x \tan(4s)\sec(4s)\,ds + 3\sin(4x) \int^x \sec(4s)\,ds \\
				&= \frac{-3}{4}\cos(4x)\sec(4x) + \frac{3}{4}\sin(4x)\ln|\tan(4x) + \sec(4x) | \\
				\text{\boldmath $y_p $} &= \text{\boldmath$\frac{3}{4}\left( -1 + \sin(4x)\ln|\tan(4x) + \sec(4x) |\right)$}
			\end{align*}
		\item Find the most general solution to the associated homogeneous differential equation.
		\\ \\
		The most general solution to the associated homogeneous differential equation is just a linear combination of the fundamental set of solutions from part (a):
			\boldmath
			$$
			y_h = c_1\cos(4x) + c_2\sin(4x)
			$$
			\unboldmath
		\item Find the most general solution to the original nonhomogeneous differential equation
		\\ \\
		We can simply just add the particular and homogenous solution to get the most general solution to the original nonhomogeneous differential equation:
			\boldmath	
			$$
			y(x) = c_1\cos(4x) + c_2\sin(4x) + \frac{3}{4}\left( -1 + \sin(4x)\ln|\tan(4x) + \sec(4x) |\right)
			$$
			\unboldmath
	\end{enumerate}

\newpage 

\noindent \textbf{Problem 3: }For this problem use the Variation of Parameters method to solve this problem. In your written answer be sure to show all of the steps for obtaining your answer.
\\ 
	\begin{enumerate}[label = (\alph*)]
		\item Find a particular solution to the nonhomogeneous differential equation 
		$$
		x^2y'' + xy' - 4y = 12x^{-2}
		$$
		We first need to find the fundamental set of solutions to the homogeneous problem.  If we let our homogeneous solution be of the following form by Cauchy Euler, we will get the fundamental set of solutions:
			$$
			y = x^r, \qquad y' = rx^{r-1}, \qquad y'' = (r^2 - r)x^{r-2}
			$$
		Now we substitute this into the homogeneous problem, and solve for $r$:
			\begin{align*}
				x^2(r^2 - r)x^{r-2} + xrx^{r-1} - 4x^r &= 0 \\
				x^r(r^2 - 4) &= 0
			\end{align*}
		Thus we get the two solutions to our fundamental set of solutions to the homogeneous problem:
			$$
			y_1 = x^2, \qquad y_2 = x^{-2}
			$$
		For the Variation of Parameters, we will need the Wronskian of the two solutions:
		$$
		W_{[y_1,y_2]} = 
		\begin{vmatrix}
			x^2 & x^{-2} \\
			2x & -2x^{-3}
		\end{vmatrix}
		= -2x^{-1} - 2x^{-1} = -4x^{-1}
		$$
		To use the Variation of Parameters, we need to change the original equation to the form:
			$$
			y'' + x^{-1}y' - 4x^{-2}y = 12x^{-4}
			$$
		Now we can find the particular solution:
			\begin{align*}
				y_p &= -x^2 \int^x \frac{12s^{-2}s^{-4}}{-4s^{-1}}\,ds + x^{-2} \int^x \frac{12s^{2}s^{-4}}{-4s^{-1}}\,ds \\
				&= 3x^2 \int^x s^{-5}\,ds - 3x^{-2}\int^x s^{-1}\,ds \\
				&= 3x^2 \left(\frac{x^{-4}}{-4}\right) - 3x^{-2}\ln(x) \\
				\text{\boldmath $y_p$} &= \text{\boldmath $\frac{-3}{4x^2} + \frac{-3\ln(x)}{x^2}$}
 			\end{align*}
 			
 		\newpage 
 		
 		\item Find the most general solution to the associated homogeneous differential equation.
 		\\ \\
 		The most general solution to the associated homogeneous differential equation is just a linear combination of the fundamental set of solutions from part (a):
	 		\boldmath
	 		$$
	 		y_h = c_1x^2 + c_2x^{-2}
	 		$$
	 		\unboldmath
 		\item Find the most general solution to the original nonhomogeneous differential equation
 		\\ \\
 		We can simply just add the particular and homogenous solution to get the most general solution to the original nonhomogeneous differential equation:
	 		\boldmath	
	 		$$
	 		y(x) = c_1x^2 + c_2x^{-2} + \frac{-3}{4x^2} + \frac{-3\ln(x)}{x^2}
	 		$$
	 		\unboldmath
	\end{enumerate}

\newpage 

\noindent \textbf{Problem 4: }This problem uses the Method of Undetermined Coefficients to analyze and find the solution. You are given the form of the particular solution that must be used to solve this problem, which allows one to solve for unknown constants in the differential equation. Subsequently, you are asked to find the general solution to this problem. In your written answer be sure to show all of the steps for obtaining your answer.
\\ 
	\begin{enumerate}[label = (\alph*)]
		\item Consider the following nonhomogeneous differential equation, which contains unknown constants $\alpha$ and $\beta$
			$$
			y'' + \alpha y' + \beta y = 324x^2  -42\sin(3x)
			$$
		Suppose the form of the particular solution to this differential equation as prescribed by the method of undetermined coefficients satisfies
			$$
			y_p(x) = A_2x^2 + A_1x + A_0 + B_1x\cos(3x) + C_1x\sin(3x)
			$$
		Determine the constants $\alpha$ and $\beta$:
		\\ \\
		Notice the following derivatives:
			\begin{align*}
				y_p &= A_2x^2 + A_1x + A_0 + B_1x\cos(3x) + C_1x\sin(3x) \\
				y_p' &= 2A_2x + A_1 + B_1\cos(3x) - 3 B_1x\sin(3x) + C_1\sin(3x) + 3C_1x\cos(3x) \\
				y_p'' &= 2A_2 - 6B_1\sin(3x) -9B_1x\cos(3x) + 6C_1\cos(3x) - 9C_1x\sin(3x)
			\end{align*}
		When substituting this back into the original equation, we get the following system of equations:
			\begin{align*}
				A_2\beta &= 324 & B_1\alpha &= -6C_1 \\
				2A_2\alpha + A_1\beta &= 0 & C_1\alpha &= -42 + 6B_1 \\
				A_1\alpha + A_0\beta &= -2A_2 & 3C_1\alpha + B_1\beta &= 9B_1 \\
				& &	-3B_1\alpha + C_1\beta &= 9C_1
			\end{align*}
		Notice the form of a second order differential equation:	
			$$
			y'' + p(x)y' + q(x)y = g(x)
			$$
		Notice when using the Method of Undetermined Coefficients, our particular solution will always satisfy any differential equation as long as it is in the form needed for $g(x)$, and that the coefficients, $p(x),q(x),$ and $g(x)$ are continuous on the same interval.  We can see in our problem that we have constant coefficients that are continuous on the same interval, all real numbers, as the right hand side of our equation. Thus allowing us to let \boldmath$\alpha = 0,$ \unboldmath because it will allow us to solve for the other unknown constant, $\beta$, and still have the particular solution satisfy the original nonhomogeneous equation. From there, we can see that \boldmath$\beta = 9$ 
		\unboldmath
		
		\newpage
		
		\item With these constants $\alpha$ and $\beta$, find the most general solution to the associated homogeneous differential equation.
		\\ \\
		We just need to solve the characteristic equation of the homogeneous differential equation:
			\begin{align*}
				\lambda^2 + 9 = 0
			\end{align*}
		Thus we get $\lambda = \pm 3i$.  Which will give us the most general solution to the associated homogeneous differential equation:	
			\boldmath
			$$
			y_h = c_1\cos(3x) + c_2\sin(3x)
			$$
			\unboldmath
		\item With the form of the particular solution above use the Method of Undetermined Coefficients to find the unknown coefficients, $A_2, A_1, A_0, B_1,$ and $C_1$, thus, finding the general solution to the original nonhomogeneous differential equation.
		\\ \\
		For this, all we need to do is substitute $\alpha = 0$ and $\beta = 9$ into our previous systems of equartions and get:
			\begin{align*}
				A_2 = 36 &&&&& A_1 = 0 &&&&& A_0 = -8 &&&&& B_1 = 7 &&&&& C_1 = 0
			\end{align*} 
		Which gives us our general solution to the original nonhomogeneous differential equation:
			\boldmath
			$$
			y_p(x) = 36x^2 - 8 + 7x\cos(3x) 
			$$
			\unboldmath
	\end{enumerate}

\newpage

\noindent \textbf{Problem 5: }This problem examines the Method of Undetermined Coefficients. Consider the following nonhomogeneous differential equation:
	$$
	y'' + 6y' + 9y  = 4xe^{-3x}\sin(3x) - 7x^2e^{-3x}
	$$
	\begin{enumerate}[label = (\alph*)]
		\item Find the most general solution to the associated homogeneous differential equation
		\\ \\
		We can find the homogeneous solutions by finding its eigenvalues found by solving its characteristic equation:
			\begin{align*}
				\lambda^2 + 6\lambda + 9 &= 0 \\
				(\lambda + 3)^2 &= 0
			\end{align*}
		So we get $\lambda = -3$.  Because of the repeated eigenvalue, we get:
			\boldmath
			$$
			y_h = c_1e^{-3x} + c_2xe^{-3x}
			$$
			\unboldmath
		\item In your written answer, give the form of the particular solution that you would guess when using the Method of Undetermined Coefficients:
			\boldmath
			$$
			y_p = (A_1x + A_0)e^{-3x}\sin(3x) + (B_1x + B_0)e^{-3x}\cos(3x) + (C_2x^2 + C_1x + C_0)e^{-3x}
			$$
			\unboldmath
	\end{enumerate}

\newpage 

\noindent \textbf{Problem 6: }Consider the initial value problem
	$$
	y'' - 4y' + 4y = 32e^{-2t}, \qquad y(0) = 5, \qquad y'(0) = 5
	$$
	\begin{enumerate}[label = (\alph*)]
		\item Take the Laplace transform of both sides of the given differential equation to create the corresponding algebraic equation
			\begin{align*}
				s^2Y(s) - sy(0) - y'(0) - 4sY(s) + 4y(0) + 4Y(s) &= \frac{32}{s + 2} \\
				\text{\boldmath $(s^2 - 4s + 4)Y(s) - (5s - 15)$ }&= \text{\boldmath $\frac{32}{s + 2}$}
			\end{align*}
		\item Solve your equation for $Y(s)$
			\boldmath
			$$
			Y(s) = \frac{5s - 15}{(s-2)^2} + \frac{32}{(s+2)(s-2)^2}
			$$		
			\unboldmath
		\item Take the inverse Laplace transform of both sides of the previous equation to solve for $y(t)$
		\\ \\
		Notice the Partial Fractions Decomposition:
			\begin{align*}
				\frac{32}{(s+2)(s-2)^2} &= \frac{A}{s+2} + \frac{B}{s-2} + \frac{C}{(s-2)^2} \\
				32 &= As^2 - 4As + 4A + Bs^2 - 4B + Cs + 2C \\
				&= (A + B)s^2 + (-4A + C)s + (4A - 4B + 2C)  
			\end{align*}
		So we get the following system of equations:
			\begin{align*}
				rref
				\begin{bmatrix}
					1 & 1 & 0 & 0 \\
					-4 & 0 & 1 & 0 \\
					4 & -4 & 2 & 32
				\end{bmatrix}
				=
				\begin{bmatrix}
					1 & 0 & 0 & 2 \\
					0 & 1 & 0 & -2 \\
					0 & 0 & 0 & 8
				\end{bmatrix}
			\end{align*}
		Thus we get the following:
			$$
			\frac{32}{(s+2)(s-2)^2} = \frac{2}{s+2} + \frac{-2}{s-2} + \frac{8}{(s-2)^2}
			$$
			\begin{align*}
				Y(s) &= \frac{5(s-2)}{(s-2)^2} - \frac{5}{(s-2)^2} + \frac{2}{s+2} + \frac{-2}{s-2} + \frac{8}{(s-2)^2} \\ 
				&= \frac{5}{s-2} +  \frac{3}{(s-2)^2} + \frac{2}{s+2} + \frac{-2}{s-2} \\
				&= \frac{3}{s-2} +  \frac{3}{(s-2)^2} + \frac{2}{s+2} 
			\end{align*}
		Now after taking the Laplace Inverse of each term, we get:
			\boldmath
			$$
			y(t) = 3e^{2t} + 3e^{2t}t + 2e^{-2t}
			$$
			\unboldmath
	\end{enumerate}

\newpage

\noindent \textbf{Problem 7: }Consider the initial value problem
	$$
	y''  + 6y' + 25y = g(t), \qquad y(0) = 4, \qquad y'(0) = 0
	$$
where 
	$
	g(t) = 
	\begin{cases}
		0 & 0 \leq t < 5 \\
		80e^{-3(t-5)} & 5 \leq t < \infty
	\end{cases}
	$
	\begin{enumerate}[label = (\alph*)]
		\item Take the Laplace transform of both sides of the given differential equation to create the corresponding algebraic equation.
		\\ \\
		Notice we can write $g(t)$ as below:
			$$
			g(t) = 80e^{-3(t-5)}h(t-5)
			$$
		So we get the following after taking the Laplace transform of each side:
			\begin{align*}
				s^2Y(s) - sy(0) - y'(0) + 6sY(s) - 6y(0) + 25Y(s) &= \frac{80}{s+3} \\
				\text{\boldmath $(s^2 + 6s + 25)Y(s) - (4s + 24)$} &= \text{\boldmath $\frac{80e^{-5s}}{s+3}$} 
			\end{align*}
		\item Solve your equation for $Y(s)$
			\boldmath
			$$
			Y(s) = \frac{4s + 24}{(s+3)^2 + 16} + \frac{80e^{-5s}}{(s+3)(s^2 + 6s + 25)}
			$$
			\unboldmath
		\item Take the inverse Laplace transform of both sides of the previous equation to solve for $y(t)$
		\\ \\
		Notice the Partial Fractions Decomposition:
		\begin{align*}
			\frac{1}{(s+3)(s^2 + 6s + 25)} &= \frac{A}{s+3} + \frac{Bs + C}{s^2 + 6s + 25} \\
			1 & = As^2 + 6As + 25A + Bs^2 + 3Bs + Cs + 3C \\
			&= (A+B)s^2 + (6A + 3B + C)s + (25A + 3C)
		\end{align*}
		So we get the following system of equations:
		\begin{align*}
		rref
		\begin{bmatrix}
		1 & 1 & 0 & 0 \\
		6 & 3 & 1 & 0 \\
		25 & 0 & 3 & 1
		\end{bmatrix}
		=
		\begin{bmatrix}
		1 & 0 & 0 & \frac{1}{16} \\
		0 & 1 & 0 & \frac{-1}{16} \\
		0 & 0 & 0 & \frac{-3}{16}
		\end{bmatrix}
		\end{align*}
		Thus we get the following:
		$$
		\frac{1}{(s+3)(s^2 + 6s + 25)} = \frac{1}{16(s+3)} + \frac{-s - 3}{16(s^2 + 6s + 25)}
		$$
		\newpage
		Now we can simplify $Y(s)$
		\begin{align*}
			Y(s) &= \frac{4(s+3)}{(s+3)^2 + 16} + \frac{12}{(s+3)^2 + 16} + \frac{5e^{-5s}}{s+3} + \frac{-5e^{-5s}(s+3)}{(s+3)^2 + 16}
		\end{align*}
		Now we can take the Laplace inverse of each term:
			\boldmath
			$$
			y(t) = 4e^{-3t}\cos(4t) + 3e^{-3t}\sin(4t) + 5e^{-3(t-5)}h(t-5) -5e^{-3(t-5)}\cos(4(t-5))h(t-5)
			$$
			\unboldmath
	\end{enumerate}

\newpage

\noindent \textbf{Problem 8: }
	\begin{enumerate}[label = (\alph*)]
		\item The first part of this problem has you use the definition of the Laplace transform to find of $te^{iat}$ and $te^{-iat}$, so you evaluate the improper integrals
		\begin{align*}
			\mathcal{L}\{te^{iat}\} &= \int_0^{\infty} e^{-st}te^{iat}\,dt & \mathcal{L}\{te^{-iat}\} &= \int_0^{\infty} e^{-st}te^{-iat}\,dt \\
			&= \int_0^{\infty} te^{-(s-ia)t}\,dt & &= \int_0^{\infty} te^{-(s+ia)t}\,dt \\
			&= \frac{-te^{-(s-ia)t}}{s-ia} + \frac{1}{s-ia}\int_0^\infty e^{-(s-ia)t} & &= \frac{-te^{-(s+ia)t}}{s+ia} + \frac{1}{s+ia}\int_0^\infty e^{-(s+ia)t}  \\
			&= \lim\limits_{A \rightarrow \infty} \left. \frac{-te^{-(s-ia)t}}{s-ia} - \frac{e^{-(s-ia)t}}{(s-ia)^2} \right|^{A}_0 & &= \lim\limits_{A \rightarrow \infty} \left. \frac{-te^{-(s+ia)t}}{s+ia} - \frac{e^{-(s+ia)t}}{(s+ia)^2} \right|^{A}_0 \\
			&=\text{\boldmath $\frac{1}{(s-ia)^2}$} & 	&= \text{\boldmath $\frac{1}{(s+ia)^2}$}
		\end{align*}
		\item You are reminded that the trig functions are readily defined by complex exponentials. In particular,
			$$
			\sin(at) = \frac{e^{iat} - e^{-iat}}{2i} \text{ and } \cos(at) = \frac{e^{iat} + e^{-iat}}{2}
			$$
		\\In your written work use the linearity of the integral and the Laplace transform with your results from Part a to find the following Laplace transforms.
			\begin{align*}
				&\mathcal{L}\{t\sin(at)\} = \frac{1}{2i}\mathcal{L}\{te^{iat}\} - \frac{1}{2i}\mathcal{L}\{te^{-iat}\} = \frac{1}{2i(s-ia)^2} - \frac{1}{2i(s+ia)^2} = \text{\boldmath $\frac{2as}{(s^2 + a^2)^2}$}\\  \\
				&\mathcal{L}\{t\cos(at)\} = \frac{1}{2}\mathcal{L}\{te^{iat}\} + \frac{1}{2}\mathcal{L}\{te^{-iat}\} = \frac{1}{2(s-ia)^2} + \frac{1}{2(s+ia)^2} = \text{\boldmath $\frac{s^2 - a^2}{(s^2 + a^2)^2}$}
			\end{align*}
			
		\newpage
		
		\item Consider the initial value problem:
			$$
			y''  + 4y'  + 20y = 32e^{-2t}\cos(4t), \qquad y(0) = 20, \qquad y'(0) = 0
			$$
		Take the Laplace transform of both sides of the given differential equation to create the corresponding algebraic equation:
			\begin{align*}
				s^2Y(s) - sy(0) - y'(0) + 4sY(s) - 4y(0) + 20Y(s) = \frac{32(s+2)}{(s+2)^2 + 16} \\
				(s^2 + 4s + 20)Y(s) - (20s + 80) = \frac{32(s+2)}{(s+2)^2 + 16} 
			\end{align*}
		Solve your equation for $Y(s)$
			$$
			Y(s) = \frac{20(s+2)}{(s+2)^2 + 16} + \frac{40}{(s+2)^2 + 16} + \frac{32(s+2)}{((s+2)^2 + 16)^2}
			$$
		Now we can take the Laplace inverse of each term and get:
			\boldmath
			$$
			y(t) = 20e^{-2t}\cos(4t) + 10e^{-2t}\sin(4t) + 4te^{-2t}\sin(4t)
			$$
			\unboldmath
	\end{enumerate}













\end{document}
