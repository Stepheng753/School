\documentclass[12pt]{article}
\usepackage[margin = 1in]{geometry}
\usepackage{amsmath}
\usepackage{amssymb}
\usepackage{cancel}

\begin{document}
	
	\begin{center}
		\textbf{Intro Dirf} \\
		\textbf{Differential Equations} \\
		\textbf{Math 337} \\
		\textbf{Stephen Giang} \\
	\end{center}

\textbf{Problem 18 Part D} Briefly discuss how you found your vertical and horizontal asymptotes. Describe how you found your points of intersection.

	\begin{align}
		&f(x) = x - 4.1
		&g(x) = \frac{2.1x}{x^2 + x - 6}
	\end{align}
\textbf{Solution:} \\
To find the Vertical Asymptotes, we set the denominator of our rational function to 0, and solve for the x.  At this x-value, the denominator equals 0, which is undefined.  \\

To find the Horizontal Asymptotes, we divide the term with the highest degree from the numerator with the term with the lowest degree from the denominator.  Thus we get $\frac{2.1}{x}$.  Because the only way for a rational function to equal 0 is for the numerator to equal 0, our $\frac{2.1}{x}$ will never be 0. \\

To find the Intersection Points, we set $f(x) = g(x)$ and solve for x.  This means we will find the same y-value, per the same x-value. \\\\

\textbf{Problem 19 Part D} \\ a) Compare and contrast these graphs. Discuss how each of these environmental temperature graphs coincide with your understanding of daily temperature during this period of time. \\
b) Briefly, discuss the differences between the predictions and how accurate you believe these predictions to be. Do the more complicated environmental temperature approximations predict a significantly different time of death? \\

\textbf{Solution:} \\
a) During the earlier time period, they thought that temperature was constant, then soon learned it changed over time.  However, their new understanding came to thinking temperature was linear and decreased throughout the day.  Our understanding of temperature is that temperature fluctuates through time, and stays within a certain interval of temperature. \\

b) Based off our own knowledge of temperature being a fluctuating function, I believe the prediction based off this would be the most accurate.  And the linear temperature, I think would be the least accurate because temperature is not always decreasing as time continues.  The more complicated temperature approximations predicted only minute differences between each other, so I would consider it significant. 

\end{document}