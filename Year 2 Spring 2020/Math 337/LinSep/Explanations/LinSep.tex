\documentclass[12pt]{article}
\usepackage[margin = 1in]{geometry}
\usepackage{amsmath}
\usepackage{amsthm}
\usepackage{graphicx}
\usepackage{subfig}
\usepackage{cancel}

\begin{document}
	
	\begin{center}
		\textbf{Linear Separation} \\
		\textbf{Differential Equations} \\
		\textbf{Math 337} \\
		\textbf{Stephen Giang} \\
	\end{center}

\textbf{Problem 18 (b): } In your written HW, create a graph with the data and the Von Bertalanffy Model for t $\in$ [0,15]. Create a short paragraph that briefly describes the rate of growth of this fish from the graph and what the maximum size of this fish can be. Include how well the model simulates the data. \\
\begin{align*}
	&\begin{tabular}{ |c|c| } 
		\hline
		Age (yr) & Length (m) \\
		\hline 
		1 & 0.4 \\ 
		2 & 0.6 \\ 
		3 & 0.84 \\ 
		4 & 0.97 \\ 
		5 & 1.02 \\ 
		6 & 1.09 \\ 
		7 & 1.13 \\ 
		8 & 1.15 \\ 
		9 & 1.16 \\ 
		10 & 1.17 \\ 
		\hline
	\end{tabular}
	& L(t) = 1.204415785 (1 - e^{-0.384651137t})
\end{align*}
\\
\noindent The Von Bertalanffy Model accurately predicts the length of albacore throughout time.  As the fish reach their older years, they're growth slows down and reach their maximum length of 1.204415785 meters.  The fish's growth starts to slow down around the age of about 5 and slowly grows from then on.  Guessing from the data, the average lifespan of albacore is about 10 - 12 years, thus showing a peak length at that age.
\newpage
\textbf{Problem 18 (d): } In your written HW, create a graph with the data and the allometric model found above. Create a short paragraph that briefly describes this graph and describe how well the model simulates the data. \\
\begin{align*}
	&\begin{tabular}{ |c|c| } 
		\hline
		Length (m) & Weight (kg) \\
		\hline 
		0.55 & 3 \\ 
		0.62 & 6 \\ 
		0.71 & 7 \\ 
		0.77 & 10 \\ 
		0.86 & 14 \\ 
		0.92 & 15 \\ 
		1.01 & 21 \\ 
		1.09 & 26 \\ 
		1.14 & 34 \\ 
		1.21 & 39 \\ 
		\hline
	\end{tabular}
	& W(L) = 21.2319 * L^{3.0589}
\end{align*}
\\
\noindent The Allometric Model is a exponential growth model that does not have a horizontal asymptote.  So the model is only accurate up until the max length of the albacore. As shown, after the max length of 1.204415785 meters, the fish with the length of 1.21 meters, has a weight (39 kg) greater than the max weight(37.50385478 kg).  The graph is very accurate as well because it models that the longer fish will have more weight which makes sense due to the fact, the fish has more mass.
\\\\\\
\indent \textbf{Problem 18 (f): } In your written HW, create a graph the weight of a Albacore as it ages, W(t). Also, create a graph of the derivative, W '(t). Write a short paragraph describing these graphs. Include a discussion explaining the significance of the point of inflection in the first graph and how it is reflected in the second graph. Summarize your modeling efforts in this lab and briefly discuss the strengths and weaknesses of these models.

\begin{align*}
	&W(t) =  21.2319 (1.204415785(1 - e^{-0.384651137t}))^{3.0589} \\
	&W'(t) = 30.08829639(1.204415785(1-e^{-0.384651137t}))^{2.0589}*e^{-0.384651137t}
\end{align*}
\\
\noindent As the albacore age, their weight increases but reach a max weight of 37.50385478 kg.  When the albacore's age reaches the point of inflection, which is $t_p = 2.906673779$ yrs, the fish grows the fastest. At $t_p$, W'(t) reaches a max, which means that the fish is growing at the fastest rate of 6.384930855 kg / yr.  Each of these models are very accurate as it greatly models the data given.  The only weakness would be that it doesn't model their weight after their death.
\newpage
\textbf{Problem 19 (b): } In your written HW, create a graph for the weighted activity time of exposure, A(t), for t $\in$ [0,12]. Briefly discuss if this graph reasonably models lead exposure for young children based on your understanding of child behavior and where the lead persists.
\\\\
The graph shows that the max of weighted activity at the age of 2.299322301 years.  From there, the weighted activity begins to decrease substantially as the child learns to walk and put their hands in their mouth.  This seems to accurately model the weighted activity of children as during the ages of 0-3, children are just learning how to walk, and once learned, their hands are not as dirty from crawling, leading to weighted activity.
\\\\\\
\indent \textbf{Problem 19 (f): } In your written HW, graph this solution for accumulation of lead in the boy for t $\in$ [0,12]. Briefly discuss this graph and explain how well this differential equation models the accumulation of lead in a child.
\\\\
The graph shows a steep slope of accumulation of lead in children ages 0-4.  This shows due to their high weighted activity.  As a child ages around 9-12, their lead accumulation begins to level off due to the almost full non-exposure of lead.  So through how young boys age, this differential equation does accurately model the accumulation of lead in young boys.
\\\\\\
\indent \textbf{Problem 19 (e): } In your written HW, graph this solution for the weight of the boy for t $\in$ [0,12]. Briefly discuss this graph and explain how well this differential equation models the weight of a child.
\\\\
The graph shows that the boy's weight is growing as he ages.  The weight of about 45kg for a 12 year old does make sense. And a weight of about 2 kg for a new born also makes sense.  So this differential equation accurately models the weight of a child.
\newpage
\indent \textbf{Problem 19 (g): } In your written HW, graph this solution for the concentration of lead in the boy for t $\in$ [0,12]. Briefly discuss this graph and explain how well this differential equation models the concentration of lead in a child. Check the website given above or any other sources and write a brief paragraph describing the health risks that the boy modeled above might encounter. 
\\\\
The graph shows a max concentration of lead at the age of 5.783690455 years.  This makes sense as the ratio of amount of lead in the child's body with the weight of the child is the highest at that age.  The concentration for before that point is less because the child hasn't had enough accumulation of lead for its weight. And the concentration also decreases after that point because the accumulation of lead stays the same while, the weight increases over time.
\\\\
Lead exposure in children leads to abnormalities in the brain and nervous system.  This could lead to stunted growth and behavioral problems.  Other cases have seen hearing loss as a result of lead exposure. The early lead exposure could also lead to anemia and other issues regarding the growth or development of young children.  



\end{document}
