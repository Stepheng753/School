\documentclass[12pt]{article}
\usepackage[margin = 1in]{geometry}
\usepackage{amsmath}
\usepackage{amssymb}
\usepackage{amsthm}
\usepackage{graphicx}
\usepackage{subfig}
\usepackage{enumitem}

\begin{document}
	
	\begin{center}
		\textbf{Quiz 5} \\
		\textbf{Differential Equations} \\
		\textbf{Math 337} \\
		\textbf{Stephen Giang} \\
	\end{center}

\noindent \textbf{Problem 1: }Consider the 2nd order linear homogeneous ODE given by:
	$$
	y'' + 4y' + 4y = 24te^{-2t} + 40\cos(2t)
	$$ 
Use the \textit{Method of Undetermined Coefficients} to solve this problem
\\ \\

\noindent We can write the particular solution in the form: 
	\begin{align*}
		y_p &=  \left( A{t}^{3}+B{t}^{2} \right) {{e}^{-2\,t}}+C\cos \left( 2\,t \right) +D\sin \left( 2\,t \right) 
		\\
		y_p' &=  \left( 3\,A{t}^{2}+2\,Bt \right) {{e}^{-2\,t}}-2\, \left( A{t}^{3}+B{t}^{2} \right) {{e}^{-2\,t}}-2\,C\sin \left( 2\,t \right) +2\,D\cos \left( 2\,t \right)
		\\
		y_p'' &=  \left( 6\,At+2\,B \right) {{\rm e}^{-2\,t}}-4\, \left( 3\,A{t}^{2}+2 \,Bt \right) {{\rm e}^{-2\,t}}+4\, \left( A{t}^{3}+B{t}^{2} \right) { {\rm e}^{-2\,t}}-4\,C\cos \left( 2\,t \right) -4\,D\sin \left( 2\,t \right)
	\end{align*}
We can now plug $y_p$ into our original equation and solve for the Undetermined Coefficients.
	\begin{align*}
		y_p'' + 4y_p' + 4y_p &= 6\,At{{e}^{-2\,t}}+2\,B{{e}^{-2\,t}}-8\,C\sin \left( 2\,t \right) +8\,D\cos \left( 2\,t \right) 
		\\ 
		&= 24te^{-2t} + 40\cos(2t)
	\end{align*}
Thus we can see that $A = 4, B = 0, C = 0, D = 5$, so the particular solution is:
	$$
	y_p = 4{t}^{3} {{e}^{-2\,t}}+ 5\sin \left( 2\,t \right)
	$$
To find the homogeneous solution, we set the original equation to 0 and solve for it eigenvalues:
	\begin{align*}
		y'' + 4y' + 4y &= 0 \\
		\lambda ^2 + 4\lambda + 4 &= 0 \\
		(\lambda + 2)^2 &= 0 \\
		\lambda = -2
	\end{align*}
Thus our homogeneous solution is as follows:
	\begin{align*}
		y_h = c_1e^{-2t} + c_2te^{-2t}
	\end{align*}
Thus our complete solution is as follows:
	$$
	y(t) = c_1e^{-2t} + c_2te^{-2t} + 4{t}^{3} {{e}^{-2\,t}}+ 5\sin \left( 2\,t \right)
	$$

\newpage 

\noindent \textbf{Problem 2: }For the following non-homogeneous differential equation give the form of the particular solution that you would guess in using the \textbf{method of undetermined coefficients}. Include your solution to the homogeneous problem. (\textbf{DO NOT} solve for the undetermined coefficients.)
	$$
	y'' - 2y' + y = 5te^{t}\sin(2t) + 20t^2e^{t}
	$$
\\ \\	

\noindent I would guess that the form of the particular solution would be:
	$$
	y_p = (At + B)e^{t}\sin(2t) + (Ct + D)e^{t}\cos(2t) + (Et^2 + Ft + G)e^t
	$$
To find the homogeneous solution, we set the original equation to 0 and solve for it eigenvalues:
	\begin{align*}
		y'' - 2y' + y &= 0 \\
		\lambda ^2 - 2\lambda + 1 &= 0 \\
		(\lambda - 1)^2 &= 0 \\
		\lambda = 1
	\end{align*}
Thus our homogeneous solution is as follows:
	\begin{align*}
		y_h = c_1e^{t} + c_2te^{t}
	\end{align*}

\newpage 

\noindent \textbf{Problem 3: }A crude tuning device can be created by an LRC-circuit forced by an external signal. An LRC-circuit is equipped with a variable capacitor, which can be dialed to different values to obtain the maximum response from an incoming radio signal. If I(t) is the current in the tuning device, which contains an inductor, $L$, a resistor, $R$, and a tunable capacitor, C, and receives an external signal, $V_0\omega \cos(\omega t)$, the ODE describing this system satisfies:
	\begin{align}
		L\ddot{I} + R\dot{I} + \frac{1}{C}I = V_0\omega \cos(\omega t)
	\end{align}
where $V_0$ is the strength of the signal and $\omega = 2 \pi f$ and $f$ is the frequency of the signal. 
\\ 

	\begin{enumerate}[label = (\alph*)]
		\item  Suppose the inductor is $L = 30$ mH, the resistor is $R = 10 \Omega$, the maximum signal is $V_0 = 50$, and the frequency $f = 60$ Hz. Solve Eqn. (1) for any C. Give the solution for $t \rightarrow \infty$. Find the amplitude of this oscillatory solution.
		\\ \\
		We can see that this function is the same as a spring mass problem.  We can correlate this equation to (Lecture 2ndODE, Slide 32).
			\begin{align*}
				L\ddot{I} + R\dot{I} + \frac{1}{C}I = V_0\omega \cos(\omega t) = \ddot{I} + \frac{R}{L}\dot{I} + \frac{1}{LC}I = \frac{V_0\omega}{L} \cos(\omega t)
			\end{align*}
		By correlation, we can say
			\begin{align*}
				2\delta = \frac{R}{L} = \frac{10}{30} = \frac{1}{3} && \omega_0^2 = \frac{1}{LC} = \frac{1}{30C} && K = \frac{V_0\omega}{L} = \frac{50\omega}{30} = \frac{5\omega}{3}
			\end{align*}
		Thus also by correlation, the particular solution is:
			$$
			y_p = \frac{\frac{5\omega}{3}\left[ (\frac{1}{30C} - \omega^2)\cos(\omega t) + \frac{\omega}{3}\sin(\omega t)\right]}{(\frac{1}{30C} - \omega^2)^2 + \frac{\omega}{3}^2}
			$$
		To find the homogeneous solution, we need the eigenvalues:
			\begin{align*}
				30\lambda^2 + 10\lambda + \frac{1}{C} = 0 && \lambda = \frac{-10 \pm \sqrt{100 - \frac{120}{C}}}{60}
			\end{align*}
		Thus we get the homogeneous solution:
			$$
			y_h = c_1e^{\frac{-10 - \sqrt{100 - \frac{120}{C}}}{60}t} + c_2e^{\frac{-10 + \sqrt{100 - \frac{120}{C}}}{60}t}
			$$
		Now we get our complete solution (with $\omega = 120\pi$):
			$$
			y(t) = c_1e^{\frac{-10 - \sqrt{100 - \frac{120}{C}}}{60}t} + c_2e^{\frac{-10 + \sqrt{100 - \frac{120}{C}}}{60}t} + \frac{\frac{5\omega}{3}\left[ (\frac{1}{30C} - \omega^2)\cos(\omega t) + \frac{\omega}{3}\sin(\omega t)\right]}{(\frac{1}{30C} - \omega^2)^2 + \frac{\omega}{3}^2}
			$$
		
		\newpage 
		
		\noindent As $t \rightarrow \infty$, the homogeneous solution decays and we get:
			$$
			y(t) = \frac{\frac{5\omega}{3}\left[ (\frac{1}{30C} - \omega^2)\cos(\omega t) + \frac{\omega}{3}\sin(\omega t)\right]}{(\frac{1}{30C} - \omega^2)^2 + \frac{\omega}{3}^2}
			$$
		Now we can get the amplitude of $A\cos t + B\sin t$ as $\sqrt{A^2 + B^2}$
			\begin{align*}
				Amp &= \frac{\frac{5\omega}{3}\sqrt{(\frac{1}{30C} - \omega^2)^2 + \frac{\omega}{3}^2}}{(\frac{1}{30C} - \omega^2)^2 + \frac{\omega}{3}^2} \\
				&= \frac{\frac{5\omega}{3}}{\sqrt{(\frac{1}{30C} - \omega^2)^2 + \frac{\omega}{3}^2}}
			\end{align*}
		After plugging in $\omega = 120\pi$:
			$$
			Amp = \frac{200\pi}{\sqrt{(\frac{1}{30C} - (120\pi)^2)^2 + (40\pi)^2}}
			$$
		\\ \\
		\item Find the value of C that gives the optimal response of this circuit to the external signal above, $C_{max}$, and determine the amplitude of that response. With this value, $C_{max}$, of tuning, what is the magnitude of the response to a $f = 50$ Hz signal
		\\ \\
		To get $C_{max}$, all we need to do is set $\omega_0^2 = \omega ^2$,
			\begin{align*}
				\frac{1}{30C} &= (120\pi)^2 \\
				C_{max} &= \frac{1}{30(120\pi)^2}	
			\end{align*}
		When plugging in $C_{max}$, we get the amplitude to be:
			\begin{align*}
				Amp = \frac{200\pi}{40\pi} = 5
			\end{align*}
		At $C_{max}$, we get the amplitude being:
			\begin{align*}
				Amp = \frac{\frac{5\omega}{3}}{\sqrt{((120\pi)^2 - \omega^2)^2 + \frac{\omega}{3}^2}}	
			\end{align*}
		Now we can plug in $f = 50$, or $\omega = 100\pi$
			\begin{align*}
				Amp &= \frac{\frac{5(100\pi)}{3}}{\sqrt{((120\pi)^2 - (100\pi)^2)^2 + \frac{100\pi}{3}^2}}	
				= {\frac {5\pi}{\sqrt {17424\,{\pi}^{4}+{\pi}^{2}}}} \approx .01206
			\end{align*}
	\end{enumerate}	

	
	
	
	
	
	
	
	
	
	
	
	
	
	
	
	
	
	
	
	
	
	
	
	
	
	
	
	
	
	
	
	
\end{document}
