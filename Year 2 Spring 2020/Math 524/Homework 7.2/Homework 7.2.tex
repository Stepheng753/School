\documentclass[12pt]{article}
\usepackage[margin = 1in]{geometry}
\usepackage{amsmath}
\usepackage{amssymb}
\usepackage{amsthm}
\usepackage{graphicx}
\usepackage{subfig}
\usepackage{enumitem}

\begin{document}
	
	\begin{center}
		\textbf{Homework 7.2} \\
		\textbf{Linear Algebra} \\
		\textbf{Math 524} \\
		\textbf{Stephen Giang} \\
	\end{center}

\noindent \textbf{Problem 7.C.2: }Suppose $T$ is a positive operator on $V$. Suppose $v,w \in V$ are such that
	\begin{align*}
		Tv = w \qquad \text{and} \qquad Tw = v
	\end{align*}
Prove that $v = w$.
\\ \\
Because $T$ is a positive operator, $\langle Tv, v\rangle \geq 0$. Notice the following:
	\begin{align*}
		\langle T(v -w), (v-w) \rangle &\geq 0 \\
		\langle Tv - Tw, v - w \rangle &= - \langle v - w, v - w \rangle \geq 0 \\
		&= - || v - w ||^2 \geq 0
	\end{align*}
Because of the square, it means that $- || v - w ||^2 = 0 $.  Thus $|| v - w || = 0$, so $v = w$

\newpage 

\noindent \textbf{Problem 7.C.4: }Suppose $T \in \mathcal{L}(V,W)$. Prove that $T^*T$ is a positive operator on $V$ and $TT^*$ is a positive operator on W
\\ \\
Notice the following:
	\begin{align*}
		(T^*T)^* = T^*T^{**} = T^*T	&&	(TT^*)^* = T^{**}T^* = TT^*
	\end{align*}
Because of the following, we can see that both are self-adjoint. So now we can notice the following:
	\begin{align*}
		\langle T^*Tv, v \rangle &= \langle T^2v,v \rangle & 
		\langle TT^*v, v \rangle &= \langle (T^*)^2v,v \rangle \\
		&= \langle Tv,Tv \rangle & &= \langle T^*v,T^*v \rangle \\
		&= || Tv ||^2 \geq 0 & 	&=  || T^*v ||^2 \geq 0
	\end{align*}
Thus $T^*T$ and $TT^*$ are positive operators on $V$ and $W$ respectively

\newpage 

\noindent \textbf{Problem 7.C.7: }Suppose $T$ is a positive operator on $V$. Prove that $T$ is invertible if and
only if
	$$
	\langle Tv,v \rangle > 0  
	$$
for every $v \in V$, with $v \not = 0$
\\ \\
	\begin{proof}[$(=>)$]
		Let $\langle Tv,v \rangle > 0$. 
		\\ \\
		If T is not invertible then there exists $v \not = 0 \in V$, such that $Tv = 0$. So we can see that $\langle Tv, v \rangle = 0$. This contradicts that $\langle Tv,v \rangle > 0$, so T has to be invertible
		\\ \\
		$(<=)$ Let T be invertible.
		\\ \\
		We can define an operator, $S^2 = T$, where S is the square root operator of T because T is a positive operator. So now we can notice the following
			$$
			\langle Tv,v \rangle = \langle S^2v,v \rangle = \langle Sv,Sv \rangle = ||Sv||^2 > 0
			$$
	\end{proof}

\newpage 

\noindent \textbf{Problem 7.D.1: }Fix $u,v \in V$ with $u \not= 0$. Define $T \in \mathcal{L}(V)$ by
	$$
	Tv = \langle v,u \rangle x
	$$
for every $v \in V$. Prove that 
	$$
	\sqrt{T^*T}v = \frac{||x||}{||u||}\langle v,u \rangle u
	$$
By definition of T, we can see the following:
	\begin{align*}
		T^*Tv = T^*\langle v,u \rangle x = \langle v,u \rangle T^*(x) = \langle v,u \rangle \langle x,x \rangle u = ||x||^2 \langle v,u \rangle u
	\end{align*}
We can see that the map $R \in \mathcal{L}(V)$, with:
	$$
	Rv = \frac{||x||}{||u||}\langle v,u \rangle u
	$$
is a square root of $T^*T$. We can also see that $\langle Rv,v \rangle \geq 0$.  If we let $e_1, ...,e_n$ be an orthonormal basis of V, then we can write the following:
	$$
	u = a_1e_1 + ... + a_ne_n
	$$
for some $a_1, ..., a_n$. So now we have: 
	\begin{align*}
		R(e_j) = \frac{||x||}{||u||}\langle e_j,u \rangle u = \frac{||x||}{||u||}\left(a_j\bar{a_1}e_1 + ... + a_j\bar{a_n}e_n \right)
	\end{align*}
Now we can see that $\mathcal{M}(R) = \frac{||x||}{||u||}a_j\bar{a_k}$. Now we can see that $\mathcal{M}(R) = \mathcal{M}(R^*)$.  Thus R is the square root of $T^*T$ and self-adjoint.  So the result below is true:
	$$
	\sqrt{T^*T}v = \frac{||x||}{||u||}\langle v,u \rangle u
	$$ 












\newpage

\noindent \textbf{Problem 7.D.2: }Give an example of $T \in\mathcal{L}(\mathbb{C}^2)$ such that 0 is the only eigenvalue of T and the singular values of T are 5, 0.
\\ \\
Notice the following:
	\begin{align*}
		\mathcal{M}(T) = 
		\begin{pmatrix}
			0 & 0 \\
			5 & 0
		\end{pmatrix}
	\end{align*}
Because $\mathcal{M}(T)$ is a triangular matrix, the eigenvalues are its entries in the diagonal, being 0.
\\ \\
Also Notice that to find the singular values:
	\begin{align*}
		\mathcal{M}(T)\mathcal{M}(T^*) = 
		\begin{pmatrix}
		0 & 0 \\
		5 & 0
		\end{pmatrix}
		\begin{pmatrix}
		0 & 5 \\
		0 & 0
		\end{pmatrix} = 
		\begin{pmatrix}
		0 &  0\\
		0 & 25
		\end{pmatrix}
	\end{align*}
Because $\mathcal{M}(T^*)\mathcal{M}(T)$ is a triangular matrix, the eigenvalues are its entries in the diagonal, being 0 and 25.  By 7.52, the singular values are the non-negative square root values of $\mathcal{M}(T^*)\mathcal{M}(T)$'s eigenvalues, which are 0 and 5

\newpage 

\noindent \textbf{Problem 7.D.5: }Suppose $T \in \mathcal{L}(\mathbb{C}^2)$ is defined by $T(x,y) = (-4y,x)$. Find the singular values of T.
\\ \\
Notice we can write $\mathcal{M}(T)$ as a matrix in respect to a basis $(x,y)$:
	$$
	\mathcal{M}(T) = 
	\begin{pmatrix}
		0 & -4 \\
		1 & 0
	\end{pmatrix}
	$$ 
Thus $\mathcal{M}(T^*) = \begin{pmatrix}
	0 & 1 \\
	-4 & 0
\end{pmatrix}$.
Now we can see the following:
	$$
	\mathcal{M}(T)\mathcal{M}(T^*) = 
		\begin{pmatrix}
	0 & -4 \\
	1 & 0
	\end{pmatrix}
	\begin{pmatrix}
	0 & 1 \\
	-4 & 0
	\end{pmatrix}
	= 
	\begin{pmatrix}
	16 & 0 \\
	0 & 1
	\end{pmatrix}
	$$
The singular values are going to be the square root of the eigenvalues of $\mathcal{M}(T)\mathcal{M}(T^*)$, so they are 4 and 1.


























\end{document}
