\documentclass[12pt]{article}
\usepackage[margin = 1in]{geometry}
\usepackage{amsmath}
\usepackage{amssymb}
\usepackage{amsthm}
\usepackage{graphicx}
\usepackage{subfig}
\usepackage{enumitem}
\everymath{\displaystyle}

\begin{document}
	
	\begin{center}
		\textbf{Midterm 2} \\
		\textbf{Linear Algebra} \\
		\textbf{Math 524} \\
		\textbf{Stephen Giang} \\
	\end{center}

\noindent \textbf{Problem 1: }Let $T \in \mathcal{L}(\mathbb{F}^3)$ be defined by $T(z_1,z_2,z_3) = (2z_2,0,5z_3)$
	\begin{enumerate}[label = (\alph*)]
		\item Find all eigenvalues and eigenspaces of T
		\\ \\
		Notice: 
		We can define the matrix of T with respect to the basis $z_1, z_2, z_3$
		$$ 
		M(T) = 
			\begin{pmatrix}
			0 & 2 & 0 \\
			0 & 0 & 0 \\
			0 & 0 & 5
			\end{pmatrix}
		$$
		Because $M(T)$ is an upper triangular matrix, we can see that $\lambda_1 = 0, \lambda_2 = 5$ \\ \\
		When $\lambda_1 = 0$
		$$
		M(T) - 0I = 
			\begin{pmatrix}
				0 & 2 & 0 \\
				0 & 0 & 0 \\
				0 & 0 & 5
			\end{pmatrix}
		$$
		$$
		\begin{pmatrix}
		0 & 2 & 0 \\
		0 & 0 & 0 \\
		0 & 0 & 5
		\end{pmatrix}
		\begin{pmatrix}
			z_1 \\
			z_2 \\
			z_3
		\end{pmatrix}
		= 
		\begin{pmatrix}
		0 \\
		0 \\
		0
		\end{pmatrix}
		$$
		So $\vec{x} = \begin{pmatrix}
			1 \\
			0 \\
			0
		\end{pmatrix}$, for $(M(T) - \lambda I)\vec{x} = \vec{0}$ and $\lambda = 0$, so the eigenspace for this eigenvalue is $\vec{x}$ and the zero vector. \\ \\
		When $\lambda_2 = 5$
		$$
		M(T) - 5I = 
		\begin{pmatrix}
			-5 & 2 & 0 \\
			0 & -5 & 0 \\
			0 & 0 & 0
		\end{pmatrix}
		$$
		$$
		\begin{pmatrix}
		-5 & 2 & 0 \\
		0 & -5 & 0 \\
		0 & 0 & 0
		\end{pmatrix}
		\begin{pmatrix}
		z_1 \\
		z_2 \\
		z_3
		\end{pmatrix}
		= 
		\begin{pmatrix}
		0 \\
		0 \\
		0
		\end{pmatrix}
		$$
		So $\vec{x} = \begin{pmatrix}
		0 \\
		0 \\
		1
		\end{pmatrix}$, for $(M(T) - \lambda I)\vec{x} = \vec{0}$ and $\lambda = 5$, so the eigenspace for this eigenvalue is $\vec{x}$ and the zero vector. \\ \\
		\item Find a Basis for range(T) \\ \\
		range(T) = span\{ 
		$
		\begin{pmatrix}
			1 \\
			0 \\
			0
		\end{pmatrix},
		\begin{pmatrix}
			0 \\
			0 \\
			1
		\end{pmatrix}
		$
		\}
		\item Find a Basis for the null(T) \\ \\
		null(T) = span\{
		$
		\begin{pmatrix}
			1 \\
			0 \\
			0
		\end{pmatrix}
		$
		\}
		\item Is $\mathbb{F}^3 =$ null(T) $\oplus$ range(T)? \\
		Because null(T) $\cap$ range(T) = 
		$
		\begin{pmatrix}
			1 \\
			0 \\
			0
		\end{pmatrix} \not = \vec{0}
		$: \qquad 
		$\mathbb{F}^3 \not =$ null(T) $\oplus$ range(T)
		\item Is $\mathbb{F}^3 = E(\lambda_1,T) \oplus E(\lambda_2,T)$? \\
		Because $E(\lambda_1,T) = 
		span\{
		\begin{pmatrix}
			1 \\
			0 \\
			0
		\end{pmatrix} 
		\}
		$ and $E(\lambda_2,T) = span \{
		\begin{pmatrix}
		0 \\
		0 \\
		1
		\end{pmatrix} 
		\}
		$, such that $E(\lambda_1,T) + E(\lambda_2,T) = \text{span} \{ 
		\begin{pmatrix}
		1 \\
		0 \\
		0
		\end{pmatrix},
		\begin{pmatrix}
		0 \\
		0 \\
		1
		\end{pmatrix} \}$,
		 $\mathbb{F}^3 \not = E(\lambda_1,T) \oplus E(\lambda_2,T)$
		 \item Is T diagonalizable? Why/Why not? \\ \\
		We can define the matrix of T with respect to the basis $z_2, z_1, z_3$
		$$
		M(T) = 
		\begin{pmatrix}
		2 & 0 & 0 \\
		0 & 0 & 0 \\
		0 & 0 & 5
		\end{pmatrix}
		$$ 
		Because the only non zero entries lie in the diagonal, T is diagonalizable
		\newpage 
		\item let $S = T^2$ (T from above)
			\begin{enumerate}[label = (\roman*)]
				\item Find all eigenvalues and eigenspaces of S
				\\
				Notice:
				$$
				S(z_1,z_2,z_3) = T(T(z_1,z_2,z_3)) = T(2z_2,0,5z_3) = (0,0,25z_3)
				$$
				We can define the matrix of S with respect to the basis $z_1, z_2, z_3$
				$$ 
				M(S) = 
				\begin{pmatrix}
				0 & 0 & 0 \\
				0 & 0 & 0 \\
				0 & 0 & 25
				\end{pmatrix}
				$$
				Because $M(S)$ is an upper triangular matrix, we can see that $\lambda = 25$ \\ \\
				$$
				M(S) - 25I = 
				\begin{pmatrix}
					-25 & 0 & 0 \\
					0 & -25 & 0 \\
					0 & 0 & 0
				\end{pmatrix}
				$$
				$$
				\begin{pmatrix}
					-25 & 0 & 0 \\
					0 & -25 & 0 \\
					0 & 0 & 0
				\end{pmatrix}
				\begin{pmatrix}
					z_1 \\
					z_2 \\
					z_3
				\end{pmatrix}
				= 
				\begin{pmatrix}
				0 \\
				0 \\
				0
				\end{pmatrix}
				$$
				So $\vec{x} = \begin{pmatrix}
				0 \\
				0 \\
				1
				\end{pmatrix}$, for $(M(S) - \lambda I)\vec{x} = \vec{0}$ and $\lambda = 25$, so the eigenspace for this eigenvalue is $\vec{x}$ and the zero vector. 
				\newpage
				\item Find a Basis for range(S) \\ \\
				range(S) = span\{ 
				$
				\begin{pmatrix}
				0 \\
				0 \\
				1
				\end{pmatrix}
				$
				\}
				\item Find a Basis for the null(S) \\ \\
				null(S) = span\{
				$
				\begin{pmatrix}
				1 \\
				0 \\
				0
				\end{pmatrix},
				\begin{pmatrix}
				0 \\
				1 \\
				0
				\end{pmatrix}
				$
				\}
				\item Is $\mathbb{F}^3 =$ null(S) $\oplus$ range(S)? 
				\\ \\
				null(S) + range(S) = span\{
				$
				\begin{pmatrix}
					1 \\
					0 \\
					0 
				\end{pmatrix},
				\begin{pmatrix}
					0 \\
					1 \\
					0
				\end{pmatrix},
				\begin{pmatrix}
					0 \\
					0 \\
					1
				\end{pmatrix}
				$
				\}, and null(S) $\cap$ range(S) = {0}, then $\mathbb{F}^3 =$ null(S) $\oplus$ range(S)?
				\item Is $\mathbb{F}^3 = E(\lambda, S)$
				\\ \\
				$E(\lambda, S) = 
				span \{
				\begin{pmatrix}
					0 \\
					0 \\
					1
				\end{pmatrix} \} \not = span\{
				\begin{pmatrix}
				1 \\
				0 \\
				0 
				\end{pmatrix},
				\begin{pmatrix}
				0 \\
				1 \\
				0
				\end{pmatrix},
				\begin{pmatrix}
				0 \\
				0 \\
				1
				\end{pmatrix}
				\}$, so  $\mathbb{F}^3 \not = E(\lambda, S)$
				\item Is S diagonalizable? Why/Why not? 
				\\ \\
				Because M(S) can be written as a matrix with its only non zero entries being on its diagonal, M(S) is diagonalizable.
			\end{enumerate}
		
	\end{enumerate}
\newpage
\noindent \textbf{Problem 2: }Consider $(x_1,...,x_n) \in \mathbb{R}^n$, where $x_\ell > 0, \forall \ell \in \{1, ..., n\}$; find a lower bound for 
$$
\left( \sum_{k = 1}^{n} x_k	\right)\left( \sum_{k = 1}^{n} \frac{1}{x_k}\right)
$$ 
Notice the following:  \\
(n = 1)
$$
\left( \sum_{k = 1}^{n} x_k	\right)\left( \sum_{k = 1}^{n} \frac{1}{x_k}	\right) = \left( \sum_{k = 1}^{1} x_k	\right)\left( \sum_{k = 1}^{1} \frac{1}{x_k}	\right) = \frac{x_1}{x_1} = 1
$$
(n = 2)
$$
\left( \sum_{k = 1}^{n} x_k	\right)\left( \sum_{k = 1}^{n} \frac{1}{x_k}	\right) = \left( \sum_{k = 1}^{2} x_k	\right)\left( \sum_{k = 1}^{2} \frac{1}{x_k}	\right) = (x_1 + x_2)\left(\frac{1}{x_1} + \frac{1}{x_2}\right) = 1 + 1 + \frac{x_1}{x_2} + \frac{x_2}{x_1}
$$
(n = 3)
\begin{align*}
\left( \sum_{k = 1}^{n} x_k	\right)\left( \sum_{k = 1}^{n} \frac{1}{x_k}	\right) = \left( \sum_{k = 1}^{3} x_k	\right)\left( \sum_{k = 1}^{3} \frac{1}{x_k}	\right) &= (x_1 + x_2 + x_3)\left(\frac{1}{x_1} + \frac{1}{x_2} + \frac{1}{x_3} \right) \\
&= 1 + 1 + 1 + \frac{x_1}{x_2} + \frac{x_1}{x_3} + \frac{x_2}{x_1} + \frac{x_2}{x_3} + \frac{x_3}{x_1} + \frac{x_3}{x_2} 
\end{align*}
Thus the following is true:
$$
\left( \sum_{k = 1}^{n} x_k	\right)\left( \sum_{k = 1}^{n} \frac{1}{x_k}\right) = n + \sum_{k=1}^{n} \left(x_k \sum_{j = 1, j \not = k}^{n} \frac{1}{x_j}\right)
$$ 
Thus the lower bound is $n$ and also the infinum, no matter the value of n, 
$$
n \leq \left( \sum_{k = 1}^{n} x_k	\right)\left( \sum_{k = 1}^{n} \frac{1}{x_k}\right) = n + \sum_{k=1}^{n} \left(x_k \sum_{j = 1, j \not = k}^{n} \frac{1}{x_j}\right)  
$$ 
with 
$$
\sum_{k=1}^{n} \left(x_k \sum_{j = 1, j \not = k}^{n} \frac{1}{x_j}\right)   > 0, \text{ as }x_\ell > 0
$$
Note: Messy Notation: 
$$
\sum_{k=1}^{n} \left(x_k \sum_{j = 1, j \not = k}^{n} \frac{1}{x_j}\right)  = \left( \frac{x_1}{x_2} + ... + \frac{x_1}{x_n} \right)+ \left( \frac{x_2}{x_1} + \frac{x_2}{x_3} + ... + \frac{x_2}{x_n} \right)+ ... + \left(\frac{x_n}{x_1} + ... \frac{x_n}{x_{n-1}}\right)
$$
\newpage 

\noindent \textbf{Problem 3: }Consider the inner product $<p,q> = \int_{0}^{1} p(x)q(x) dx$ for $p,q \in \mathcal{P}(\mathbb{R})$. On $\mathcal{P}_2(\mathbb{R})$ our friends Gram \& Schmidt kindly provide an orthonormal basis:
\begin{align*}
	\{ \qquad u_1(x) = 1 &&& u_2(x) = \sqrt{3}(-1 + 2x) &&& u_3(x) = \sqrt{5}(1 - 6x + 6x^2) \qquad \}
\end{align*}
Find a polynomial $q \in \mathcal{P}_2(\mathbb{R})$ so that $\forall p \in \mathcal{P}_2(\mathbb{R})$:
$$
p\left(\frac{1}{2}\right) = \int_{0}^{1} p(x)q(x) dx
$$
Let $\phi(p(x)) = <p,q> = \int_{0}^{1} p(x)q(x) dx = p\left(\frac{1}{2}\right)$ \\ \\
By (6.43) in the textbook, 
	\begin{align*}
		q(x) &= \phi(u_1(x))u_1(x) + \phi(u_2(x))u_2(x) + \phi(u_3(x))u_3(x) \\
		&= u_1\left(\frac{1}{2}\right)u_1(x) + u_2\left(\frac{1}{2}\right)u_2(x) +u_3\left(\frac{1}{2}\right)u_3(x) \\
		&= 1 + 0(\sqrt{3}(-1 + 2x)) + \frac{-\sqrt{5}}{2}(\sqrt{5}(1 - 6x + 6x^2)) \\
		&= 1 + \frac{-5}{2}(1 - 6x + 6x^2)
	\end{align*}








\end{document}
