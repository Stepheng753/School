\documentclass[12pt]{article}
\usepackage[margin = 1in]{geometry}
\usepackage{amsmath}
\usepackage{amssymb}
\usepackage{amsthm}
%\usepackage{graphicx}
%\usepackage{subfig}
%\usepackage{cancel}

\begin{document}
	
	\begin{center}
		\textbf{Homework 4} \\
		\textbf{Numerical Matrix Analysis} \\
		\textbf{Math 543} \\
		\textbf{Stephen Giang} \\
	\end{center}

\noindent \textbf{Problem 1: } \\
	$$\text{Let } 
	A = 
	\begin{bmatrix}
		1 & 2 & 3 \\
		4 & 5 & 6 \\
		7 & 8 & 9
	\end{bmatrix} 
	\text{ $\rightarrow$ } 
	Q = 
	\begin{bmatrix}
		0.1231  &  0.9045  & -0.1111 \\
		0.4924  &  0.3015  & -0.4444 \\
		0.8616  & -0.3015  & -0.8889
	\end{bmatrix}
	\text{ , }
	R = 
	\begin{bmatrix}
		8.1240  &  9.6011  & 11.0782 \\
		0  &  0.9045  &  1.8091 \\
		0    &     0  &  0.0000
	\end{bmatrix}$$
	
\vspace{\baselineskip}
\vspace{\baselineskip}
\vspace{\baselineskip}

\noindent \textbf{Problem 9.1 (a): }
Run the six-line MATLAB program of Experiment 1 to produce a
plot of approximate Legendre polynomials.  
\\ \\
\noindent \textbf{Problem 9.1 (b): } For k = 0,1,2,3, plot the difference on the 257-point grid between these
approximations and the exact polynomials (7.11). How big are the errors, and how are they distributed? \\

\noindent \textbf{Solution 9.1 (b): } The errors when k = 0, and k = 1, are 0.  The errors for the other k values are in between $\pm$ 0.015. The errors get larger as the degree of each polynomial gets bigger, or for greater k values. \\ \\
\\ \\
\noindent \textbf{Problem 9.2: } In Experiment 2, the singular values of A match the diagonal elements of a QR factor R approximately. Consider now a very different example. Suppose Q = I and A = R, the m x m matrix (a Toeplitz matrix) with 1 on the main diagonal, 2 on the first superdiagonal, and 0 everywhere else \\

\noindent \textbf{Solution 9.2 (a): } What are the eigenvalues, determinant, and rank of A? \\
	\begin{align*}
		&\text{All }eig(A) = 1 \\
		&det(A) = 1 \\
		&rank(A) = m 
	\end{align*}
\noindent \textbf{Solution 9.2 (b): } What is $A^{-1}$? \\
	\begin{align*}
		A^{-1} = m \times m \text{ matrix with diagonal entries being 1, and its superdiagonal entries being -2}
	\end{align*}


\end{document}
