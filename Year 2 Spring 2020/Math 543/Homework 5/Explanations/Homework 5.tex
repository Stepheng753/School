\documentclass[12pt]{article}
\usepackage[margin = 1in]{geometry}
\usepackage{amsmath}
\usepackage{amssymb}
\usepackage{amsthm}
\usepackage{enumitem}
\usepackage{graphicx}
\usepackage{subfig}
\usepackage{cancel}

\begin{document}
	
	\begin{center}
		\textbf{Homework 5} \\
		\textbf{Numerical Matrix Analysis} \\
		\textbf{Math 543} \\
		\textbf{Stephen Giang} \\
	\end{center}

\noindent \textbf{Section 12 Problem 3: }The goal of this problem is to explore some properties of random matrices. Your job is to be a laboratory scientist, performing experiments that lead to conjectures and more refined experiments. Do not try to prove anything. Do produce well-designed plots, which are worth a thousand numbers. Define a random matrix to be an $m x m$ matrix whose entries are independent samples from the real normal distribution with mean zero and standard deviation $m^{-1/2}$. (In MATLAB, A = randn(m,m)/sqrt (m).) The factor $\sqrt{m}$ is introduced to make the limiting behavior clean as m $\rightarrow \infty$.

	\begin{enumerate}[label = (\alph*)]
		\item What do the eigenvalues of a random matrix look like? What happens, say, if you take 100 random matrices and superimpose all their eigenvalues in a single plot? If you do this for m = 8, 16, 32, 64, ... , what pattern is suggested? How does the spectral radius $\rho(A)$ (Exercise 3.2) behave as m $\rightarrow \infty$ ?
		\item  What about norms? How does the 2-norm of a random matrix behave as m $\rightarrow \infty$ ? Of course, we must have $\rho(A) < || A ||$ (Exercise 3.2). Does this inequality appear to approach an equality as m $\rightarrow \infty$ ?
		\item  What about condition numbers—or more simply, the smallest singular value $\sigma_{min}$ ? Even for fixed m this question is interesting. What proportions of random matrices in $\mathbb{R}^{m \times m}$ seem to have $\sigma_{min} < 2^{-1}, 4^{-1}, 8^{-1},$ ...? In other words, what does the tail of the probability distribution of smallest singular values look like? How does the scale of all this change with m ? 
	\end{enumerate} 

\vspace{\baselineskip}
\vspace{\baselineskip}

	\begin{enumerate}[label = (\alph*)]
		\item The eigenvalues of random matrices take the shape of circles. As m $\rightarrow \infty$, the spectral radius, $\rho(A)$ approaches 1.
		\item As  m $\rightarrow \infty$, the 2-norm of the random matrices approach 2. The inequality $\rho(A) < || A ||$, remains true as m $\rightarrow \infty$.  The inequality does not approach an equality as m approaches infinity.
		\item As m $\rightarrow \infty$, the proportion between $\delta_{min} < 2^{-1}, 4^{-1}, 8^{-1}$ and the number of random matrices approach 100 \%.  The proportion between $\delta_{min} < 2^{-1}$ and the number of iterations approaches 100\% the fastest as m increases.
	\end{enumerate}


\end{document}
