\documentclass[12pt]{article}
\usepackage[margin = 1in]{geometry}
\usepackage{amsmath}
\usepackage{amssymb}
\usepackage{amsthm}
\usepackage{graphicx}
\usepackage{subfig}
\usepackage{enumitem}
\usepackage{array}

\begin{document}
	
	\begin{center}
		\textbf{Homework 6} \\
		\textbf{Numerical Matrix Analysis} \\
		\textbf{Math 543} \\
		\textbf{Stephen Giang} \\
	\end{center}

\noindent \textbf{Problem TB-18.1: }
	\begin{align*}
		A = 
		\begin{bmatrix}
			1 & 1 \\
			1 & 1.0001 \\
			1 & 1.0001
		\end{bmatrix},
		\qquad 
		b = 
		\begin{bmatrix}
			2 \\
			0.0001 \\
			4.0001
		\end{bmatrix}
	\end{align*}
	\begin{enumerate}[label = (\alph*)]
		\item What are the matrices $A^+$ and $P$ for this example? Give exact answers.
			\begin{align*}
			A^+ = (A^*A)^{-1}A^* &=
			\left(
			\begin{bmatrix}
				1  &  1  &  1 \\
				1  &  1.0001  &  1.0001
			\end{bmatrix}
			\begin{bmatrix}
				1 & 1 \\
				1 & 1.0001 \\
				1 & 1.0001
			\end{bmatrix}
			\right)^{-1}
			\begin{bmatrix}
				1  &  1  &  1 \\
				1  &  1.0001  &  1.0001
			\end{bmatrix} \\
			&= 
			\begin{bmatrix}
				3 & 3.0002 \\
				3.0002 &  3.00040002
			\end{bmatrix}^{-1}
			\begin{bmatrix}
				1  &  1  &  1 \\
				1  &  1.0001  &  1.0001
			\end{bmatrix} \\
			&= \frac{1}{2 * 10^{-8}}
			\begin{bmatrix}
				3.00040002 & -3.0002 \\
				-3.0002 &  3
			\end{bmatrix}
			\begin{bmatrix}
				1  &  1  &  1 \\
				1  &  1.0001  &  1.0001
			\end{bmatrix} \\
			&= 50,000,000
			\begin{bmatrix}
				.00020002 & -.0001 & -.0001 \\
				-.0002 & .0001 & .0001
			\end{bmatrix} \\
			&= 
			\begin{bmatrix}
				10,001 & -5,000 & -5,000 \\
				-10,000 & 5,000 & 5,000
			\end{bmatrix} \\
			P = AA^+ &= 
			\begin{bmatrix}
				1 & 1 \\
				1 & 1.0001 \\
				1 & 1.0001
			\end{bmatrix}
			\begin{bmatrix}
				10,001 & -5,000 & -5,000 \\
				-10,000 & 5,000 & 5,000
			\end{bmatrix} \\
			&=
			\begin{bmatrix}
				1 & 0 & 0 \\
				0 & 0.5 & 0.5  \\
				0 & 0.5 & 0.5
			\end{bmatrix}
			\end{align*}
		\item Find the exact solutions $x$ and $y = Ax$ to the least squares problem $Ax \approx b$
			\begin{align*}
				A^{+}Ax = x = A^+b &= 
				\begin{bmatrix}
					10,001 & -5,000 & -5,000 \\
					-10,000 & 5,000 & 5,000
				\end{bmatrix}
				\begin{bmatrix}
					2 \\
					0.0001 \\
					4.0001
				\end{bmatrix}
				=
				\begin{bmatrix}
					1 \\ 1
				\end{bmatrix} \\
				y = Ax = AA^+b = Pb &=
				\begin{bmatrix}
				 	1 & 0 & 0 \\
					0 & 0.5 & 0.5  \\
					0 & 0.5 & 0.5
				\end{bmatrix}
				\begin{bmatrix}
					2 \\
					0.0001 \\
					4.0001
				\end{bmatrix}
				= 
				\begin{bmatrix}
					2 \\ 2.0001 \\ 2.0001
				\end{bmatrix}
			\end{align*}
		
		\newpage 
		
		\item What are $\kappa(A), \theta$, and $\eta$? From here on, numerical answers are acceptable.
			\begin{align*}
				\kappa(A) = ||A|| ||A^+|| &\approx 42429.2354161703 \\
				\theta = \cos^{-1}\frac{||y||}{||b||} &\approx 0.684702873261185^R \\
				\eta = \frac{||A|| ||x||}{||y||} &\approx 1.000000000555537
			\end{align*}
		\item What are the four condition numbers of Theorem 18.1?
			\begin{center}
				\renewcommand{\arraystretch}{2}
				\begin{tabular}{ l| m{1cm}|m{3cm}|}
					& $y$ & $x$ \\
					\hline 
					$b$ & $\frac{1}{\cos \theta}$ & $\frac{\kappa(A)}{\eta \cos \theta}$ \\
					\hline
					$A$ & $\frac{\kappa(A)}{\cos \theta}$ & $\kappa(A) + \frac{\kappa(A)^2\tan \theta}{\eta}$
				\end{tabular}
				 = 
				\begin{tabular}{c|c|c|}
					& $y$ & $x$ \\
					\hline 
					$b$ & $1.290977236078942$ & $54775.17703608065$ \\
					\hline
					$A$ & $54775.17706651028$ & $1469883252.863362$
				\end{tabular}
			\end{center}
		\item  Give examples of perturbations $\delta b$ and $\delta A$ that approximately attain these
		four condition numbers. 
		\\ \\
		Let $\delta b$ =
		$
		\begin{bmatrix}
			a \\ 0 \\ 0	
		\end{bmatrix}
		$, with $a \in \mathbb{R}$, Now notice that $P\delta b = \delta b$, and $y = Pb$
			\begin{align*}
				\kappa _{b \rightarrowtail y} &= \frac{|| P(b + \delta b) - Pb ||}{|| y ||} \left/ \frac{||\delta b ||}{||b||} \right. \\
				&= \frac{|| P \delta b = \delta b ||}{|| y||} * \frac{||b||}{||\delta b ||} \\
				&= \frac{||b||}{||y||} \\
				&= 1.290977236078942
			\end{align*}
		\\ \\
		Let $\delta b$ =
		$
		\begin{bmatrix}
		1 \\ -0.5 \\ -0.5	
		\end{bmatrix}
		$, so $||A^+ \delta b || = 21213.91056005508 = ||A^+||||\delta b ||$, and $x = A^+b$.
			\begin{align*}
				\kappa_{b \rightarrowtail x} &= \frac{||A^+\delta b||}{||A^+ b||} \left/ \frac{||\delta b||}{||b||}\right. = \frac{21213.91056005508}{1.414213562373095} * \frac{4.472225399060293}{1.224744871391589} = 54775.17703608065
			\end{align*}
		\newpage 
		
		Let $\delta A$ = 
		$
		\begin{bmatrix}
			10^{-12} & -10^{-12} \\
			-10^{-12} & 10^{-12} \\
			10^{-12} & -10^{-12}
		\end{bmatrix}
		$, so $\tilde{A} = A + \delta A$, and $\tilde{y} = \tilde{A}(\tilde{A}^*\tilde{A})^{-1}\tilde{A}^*b = 
		\begin{bmatrix}
			2.000000080004251 \\
			2.000099959999875 \\
			2.000099960003150
		\end{bmatrix}$
		\begin{align*}
			\kappa_{A \rightarrowtail y} = \frac{||\tilde{y} - y||}{||y||}\left/ \frac{||\delta A||}{||A||}\right. &= \frac{9.798024764246804 * 10^{-8}}{3.464217086160112} * \frac{2.449571394482489}{2.449489742783178 * 10^{-12}} \\
			&= 28284.46119148417
		\end{align*}
		\\ \\ \\ \\
		Let the above be true, so $\tilde{x} = \tilde{A}^+b =
		\begin{bmatrix}
			1.001200092248810 \\
			0.998799987752254
		\end{bmatrix}$.
		\begin{align*}
			\kappa_{A \rightarrowtail x} = \frac{||\tilde{x} - x ||}{||x||} \left/ \frac{||\delta A||}{||A||}\right. &= \frac{0.001697102831166}{1.414213562373095} * \frac{2.449571394482489}{2.449489742783178 * 10^{-12}} \\
			&= 1200072922.383391
		\end{align*}
	\end{enumerate}

\newpage 

\noindent \textbf{Problem PB-14.1: }We could use these matrices $(A_k)$ to least-squares-fit polynomials (of matching degree $k$) to some data-set with 101 measurements. Is it necessarily better to have more model parameters (i.e. fitting a higher degree polynomial)? — Discuss.

\vspace{\baselineskip}
\noindent Based on my observations, it is not necessarily better to have more model parameters as the Vandermonde Matrix is very ill-conditioned.  So as k increases, (the degree of the polynomial gets higher), we reach larger and larger condition numbers showing us that the matrix is ill conditioned, meaning we lose a lot of accuracy when trying to use it.
As you can see from the plot, you get insanely high condition numbers such as $10^{27}$
\end{document}
