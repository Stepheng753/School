\documentclass[11pt]{article}
\usepackage[margin = 1in]{geometry}
\usepackage{amsmath}
\usepackage{amssymb}
\usepackage{amsthm}
\usepackage{graphicx}
\usepackage{subfig}
\usepackage{enumitem}
\usepackage{url}
\usepackage[parfill]{parskip}
\newcommand{\skipline}{\vspace{\baselineskip}}
\newenvironment{problem}[1]{\textbf{Problem #1: }}{\newpage}


\begin{document}
	
	\begin{center}
		\textbf{Notebook} \\
		\textbf{Computer Vision} \\
		\textbf{CS 559} \\
		\textbf{Stephen Giang} \\
	\end{center}
	
	\textbf{08/24/20 - Chapter 1.docx}
	\begin{enumerate}
		\item Computer Vision - Natural Images; Ex: photos, GPS, medical scanning
		\item Computer Graphics - Man-made Images; Ex: cartoons, animations
		\item Wavelength($\lambda$) - [$10^{-16}, 10^{-6}$] 
		\item Frequency($F$) - [$10^{24}, 10^{2}$]
		\item $\lambda f = c = 10^{8}$
	\end{enumerate}
	\skipline
	\textbf{08/26/20 - Chapter 1.docx}
	\begin{enumerate}
		\item Passive imaging: Uses energy sources that are already present in the scene, Ex: Light from Sun
		\item Active imaging: Uses artificial energy source to probe environment, Ex: Radiation in medical field 
		\item Sampling: Digitizing the arguments $x$ and $y$ in the imaging function $f(x,y)$.  Sampling is kind of like the resolution and amount of pixels.
		\item Quantization: Digitizing the value of the imaging function
			\begin{enumerate}
				\item This determines the amount of gray levels ranging from $g = 2^0 - 2^8$. 
				\item To calculate gray bit levels: $g = 2^b$.  If $g = 2^8$, then we call it an 8-bit image.  For $2^b$, we call it a b-bit image.  
				\item Ex: A 1-bit image only has $2^1$ different gray levels: black and white
				\item Ex: A 2-bit image has 4 colors: white, black, and 2 different grays.
			\end{enumerate}
			\skipline
			\begin{enumerate}
				\item This also determines RGB data.  Each spectrum (red, green, blue) all range with $2^8$ levels of red, green, blue.  This means there are $2^{24}$ levels of RGB combinations or essentially $2^{24}$ colors  
			\end{enumerate}
	\end{enumerate}
	\skipline
	\textbf{08/31/20 - Chapter 2.docx}
	\begin{enumerate}
		\item RGB can be converted into CMYK (Cyan, Magenta, Yellow, Black)
	\end{enumerate}
	\skipline
	\textbf{09/14/20 - }
	\begin{enumerate}
		\item 
	\end{enumerate}
\end{document}
