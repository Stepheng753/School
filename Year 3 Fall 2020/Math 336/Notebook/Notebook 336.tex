\documentclass[11pt]{article}
\usepackage[margin = 1in]{geometry}
\usepackage{amsmath}
\usepackage{amssymb}
\usepackage{amsthm}
\usepackage{graphicx}
\usepackage{subfig}
\usepackage{enumitem}
\usepackage{url}
\usepackage[parfill]{parskip}
\newcommand{\skipline}{\vspace{\baselineskip}}
\newenvironment{problem}[1]{\textbf{Problem #1: }}{\newpage}


\begin{document}
	
	\begin{center}
		\textbf{Notebook} \\
		\textbf{Intro Math Modeling} \\
		\textbf{Math 336} \\
		\textbf{Stephen Giang} \\
	\end{center}

	\textbf{08/26/20}
	\begin{enumerate}
		\item Numbers and radians are dimensionless 
		\item $e^{i\phi} = cos\phi + i\sin\phi$ - Euler's equation
		\item This ties together the dimensionless of trig, exponential, and logarithmic. 
		\item radians and degrees(degree for angles) is the only thing that has units but no dimension.
		\begin{enumerate}
			\item Length of arc length ($s$):
			\item $s = \theta r  \rightarrow [L] = [\theta][L]$ 
			\item So $[\theta] = 1$, thus proving it dimensionless 
		\end{enumerate}
		\item Pressure is F/A but can be converted to E/V
		\item $MLT^{-2}/L^2 = ML^{-1}T^{-2} = M(LT^{-1})^2L^{-3} = [E]/[V]$
	\end{enumerate}
	\skipline
	\textbf{08/28/20}
	\begin{enumerate}
		\item A universal math model in Nature - In general, a physical quantity $X$ can be written as the product of powers of all the $k$
		relevant quantities $X_1, X_2, ......, X_k$ as follows:
		$$X = \alpha X_1^{n_1}X_2^{n_2}X_3^{n_3}...X_k^{n_k},$$
		where $\alpha$ is a constant
	\end{enumerate}
	\skipline
	\textbf{09/02/20}
	\begin{enumerate}
		\item 
	\end{enumerate}

	


\end{document}
