\documentclass[11pt]{article}
\usepackage[margin = 1in]{geometry}
\usepackage{amsmath}
\usepackage{amssymb}
\usepackage{amsthm}
\usepackage{graphicx}
\usepackage{enumitem}
\usepackage{url}
\usepackage[parfill]{parskip}
\usepackage{listings}
\newcommand{\skipline}{\vspace{\baselineskip}}
\newcommand{\spacer}{\noalign{\medskip}}
\newcommand{~}{\sim}
\newenvironment{problem}[1]{\textbf{Problem #1: }}{\newpage}
\usepackage{caption}
\usepackage{subcaption}
\usepackage[utf8]{inputenc}
\usepackage{xcolor}
\definecolor{codegreen}{rgb}{0,0.6,0}
\definecolor{codegray}{rgb}{0.5,0.5,0.5}
\definecolor{codepurple}{rgb}{0.58,0,0.82}
\definecolor{backcolour}{rgb}{0.95,0.95,0.92}
\lstdefinestyle{mystyle}{
	backgroundcolor=\color{backcolour},   
	commentstyle=\color{codegreen},
	keywordstyle=\color{magenta},
	numberstyle=\tiny\color{codegray},
	stringstyle=\color{codepurple},
	basicstyle=\ttfamily\footnotesize,
	breakatwhitespace=false,         
	breaklines=true,                 
	captionpos=b,                    
	keepspaces=true,                 
	numbers=left,                    
	numbersep=5pt,                  
	showspaces=false,                
	showstringspaces=false,
	showtabs=false,                  
	tabsize=2
}
\lstset{style=mystyle}

\begin{document}
	
	\begin{center}
		\textbf{Final} \\
		\textbf{Algebraic Coding Theory} \\
		\textbf{Math 525} \\
		\textbf{Stephen Giang RedID: 823184070} \\
		\skipline \skipline
	\end{center}

	\begin{problem}{1}
		\begin{enumerate}[label = (\alph*)]
			\item  Find a parity-check matrix $H$ for a cyclic Hamming code of length 15 using GF($2^4$)constructed from $1 + x + x^4$(see Table 5.1, p.  114), where the generator polynomial is $m_7(x)$.   Each  entry  of $H$ must be expressed as a  power of $\beta$,  where $\beta$ is  the  primitive element of the field, exactly as in Table 5.1.
			\\ \\
			Notice the generator polynomial: $m_7(x)$
			\[[\beta^7] \Rightarrow [\beta^{14}] \Rightarrow [\beta^{28} = \beta^{13}] \Rightarrow [\beta^{26} = \beta^{11}] \Rightarrow [\beta^{22} = \beta^{7}]\]
			\begin{align*}
				m_7(x) &= (x + \beta^7)(x + \beta^{14})(x + \beta^{13})(x + \beta^{11}) \\
				&= \bigg(x^2 + (\beta^7 + \beta^{14})x + \beta^{21}\bigg)\bigg(x^2 + (\beta^{13} + \beta^{11})x + \beta^{24}\bigg) \\
				&= \bigg(x^2 + \beta x + \beta^6\bigg)\bigg(x^2 + \beta^4x + \beta^9\bigg) \\
				&= x^4 + \bigg(\beta + \beta^4\bigg)x^3 + \bigg(\beta^9 + \beta^5 + \beta^6\bigg)x^2 + \bigg(\beta^{10} + \beta^{10}\bigg)x + \beta^{15} \\
				&= x^4 + x^3 + 1
			\end{align*}
			We get the Parity Check Matrix below:
			\[H = \begin{bmatrix}
				1 \\ \beta^7 \\ \beta^{14}
			\end{bmatrix}\]
			\item Now suppose $r$ is received, where $r$ is the word of length 15 that you obtained from your last name.  Find the most likely codeword transmitted.
			\\ \\
			Notice my last name is GIANG, so we get
			\[r(x) = \{00100\,10010\,01010\} (ANG) = x^2 + x^5 + x^8 + x^{11} + x^{13}\]
			Notice the following:
			\begin{align*}
				r(\beta) &= \beta^2 + \beta^5 + \beta^8 + \beta^{11} + \beta^{13} = \beta^2
			\end{align*}
			Thus we get the most likely codeword is \{01100\,10010\,01010\} (ING).
		\end{enumerate}
	\end{problem}


\end{document}
