\documentclass[11pt]{article}
\usepackage[margin = 1in]{geometry}
\usepackage{amsmath}
\usepackage{amssymb}
\usepackage{amsthm}
\usepackage{graphicx}
\usepackage{subfig}
\usepackage{enumitem}
\usepackage{url}
\usepackage[parfill]{parskip}
\newcommand{\skipline}{\vspace{\baselineskip}}
\newenvironment{problem}[1]{\textbf{Problem #1: }}{\newpage}


\begin{document}
	
	\begin{center}
		\textbf{Notebook} \\
		\textbf{Algebraic Coding Theory} \\
		\textbf{Math 525} \\
		\textbf{Stephen Giang} \\
	\end{center}
	
	\textbf{08/24/20 - Introduction}
	\begin{enumerate}
		\item This class covers the science of "error-correcting codes." 
		\item Binary Symmetric Channel
	\end{enumerate}
	\skipline
	\textbf{08/26/20 - Introduction.pdf + Section 1.1 - 1.6.pdf}
	\begin{enumerate}
		\item Example of Coding:
			\begin{enumerate}
				\item 0 is encoded as 000, and 1 is 111 - Encoder
				\item After corruption, the decoder will use the majority vote:
				$000 \rightarrow \{000,100,010,001,110,011,101,111\}$ 
				\item If the decoder receives a three digit number, it will decode it as which ever number is the majority - ex: $001 \rightarrow 0, 110 \rightarrow 1$
				\item Probability of $000 \rightarrow 110$ is $qqp = q^2p$, where $q = 1 - p$ is the probability of error, and $p$ is probability of correct conversion.
				\item The probability of $000 \rightarrow 111, 110, 101, 011$ is $Pr(E) = q^3 + 3q^2p$.  When evaluating $p = .9, q = .1$, we get $Pr(E) = .028$. 
				\item For probability of this or that, we add the probabilities. 
				\item In other examples, encoding will always convert the 0 or 1 into a string of odd number 0s or 1s. Ex: $0 \rightarrow 000, 00000, 0000000$
			\end{enumerate}
		\item Definitions:
			\begin{enumerate}
				\item Digits or Bits: 0,1 
				\item Word: Sequence of digits
				\item Length of word: \# of digits a word has
				\item Channel: Physical Link that connects data source to data sink.  In this course, we will model these channels with error characteristics.  Refer to the Binary Symmetric Channel
				\item Binary Channel: Only 0's or 1s are transmitted or received over it
				\item Binary Code: Set of words. Ex: \{00,110,01,11\}
				\item Block Code - Binary Code, but all words have the same length
				\item Repetition and parity-check codes: 
					\begin{enumerate}
						\item Repetition Codes: \{000...0,111...1,...\} with $n$ copies.
						\item Rate is $\frac{1}{n}$
						\item Rate is for every n bits, the receiver receives 1 bit of information, 
						\item Parity-Check Code: $C = \{(x_1 x_2 ... x_n)|x_1 + ... + x_n \text{ is even}, x_i \text{ are 0s and 1s}\}$
						\item $n = 3: C = \{000,110,011,101\}$
						\item $n = 4: C = \{0000, 1100 , 1010, 0011, 0110, 1001, 0101, 1111\}$
						\item For $n = 4$, the rate is $\frac{3}{4}$ 
					\end{enumerate}
				\item $C$ is Code, and $|C|$ is the number of code words, or words held in the code. Its also known as the size or cardinality of the code.
			\end{enumerate}
	\end{enumerate}
	\skipline
	\textbf{08/28/20 - Section 1.1 - 1.6}
	\begin{enumerate}
		\item For $n=3$:
			\begin{itemize}
				\item Given 00 - 001
				\item Given 10 - 101
			\end{itemize}
		\item Notice that we give it 2, and then it adds an extra number to make sure there are an even number of 1s.
		\item So the rate is $\frac{2 \text{ bits info}}{3 \text{ bit word}} = \frac{2}{3}$
		\item We will assume that errors occur independently, that is, the occurrence of error during a time slot does not imply anything about the next time slot.
		\item Special Case: $p = 1$ and $p = 0$
		\item We can always assume that $\frac{1}{2} \leq p < 1$
		\item The information rate of a code C is the proportion of digits that convey information.
		$$R = \frac{\log_2|C|}{n} \text{ bits per block},$$
		where n is the length of C (length of codewords within C)
		\item Example:
			\begin{align*}
				C &= \{000,010,100,001,110,101,011,111\} \\
				&\rightarrow \{00011,01001,10000,00111,11001,10110,01101,11111\}
			\end{align*}
		Notice $|C| = 8$, so $\log_2|C| = \log_2 (8) = 3$.  Length of C $ = 5$.  Thus $R = \frac{3}{5}$.
		\item Example of error-correcting:
		\begin{align*}
			C_1 &= \{00,01,10,11\} \text{ cannot detect any errors, let alone correct any errors.} \\
			C_2 &= \{000,011,101,110\} \text{ (Parity Check Code of Length 3) can detect one error (affecting any codeword).} \\
			C_3 &= \{000000, 010101, 101010, 111111\}
		\end{align*}
		$C_3$ can detect up to 2 errors (affecting any codeword).  Suppose 110101 is received. The most likely code transmitted is 010101, So we make the correction
	\end{enumerate}
	\newpage
	\textbf{08/31/20 - Section 1.7,1.8}
	\begin{enumerate}
		\item Let $\phi_p(v,w) =$ probability of receiving $w$ given that $v$ was sent. We have:
			$$\phi_p(v,w) = p^{n-d}q^{d}$$,
		where n is the length of the codewords, and d is the number of disagreements or areas of corruption.  
		\item Ex: 
		\begin{align*}
			v &= 1110101 \\
			w &= 1010010 \\
			\phi_p(v,w) &= pqppqqq = p^{7-4}q^{4} = p^3q^4			
		\end{align*}
		\item Suppose we have a BSC with $\frac{1}{2} \leq p < 1$. Suppose $v_1$ and $w$ disagree in $d_1$ positions and $v_2$ and $w$ disagree in $d_2$ positions. Then
		$$\phi_p(v_1,w) \leq \phi_p(v_2,w) \Longleftrightarrow d_2 \leq d_1$$
		\item Proof:
			\begin{align*}
				\phi_p(v_1,w) &\leq \phi_p(v_2,w) \\
				p^{n-d_1}q^{d_1} &\leq  p^{n-d_2}q^{d_2} \\
				\left(\frac{p}{q}\right)^{d_2 - d_1} &\leq 1
			\end{align*}
		Notice that$\frac{1}{2} \leq p < 1$ with $q = 1-p$, so that makes $\frac{p}{q} \geq 1$, thus making $d_2 \leq d_1$ 
		\item Let $K = \{0,1\}$ and define two operations on it, + and $\cdot$ , as addition and multiplication modulo 2.  Endowed with these two operations, $K$ becomes a field.
		\item Let n be a positive integer, then:
			$$K^n = K \times K \times ... \times K = \{(v_1,...v_n) | v_i \in K, i = 1, ..., n\}$$
		\item In $K^n$, define addition componentwise, that is:
		$$(v_1, v_2, ..., v_n) + (w_1, w_2, ..., w_n) = (v_1+w_1, v_2+w_2, ..., v_n+w_n)$$
		for all $(v_1, v_2, ..., v_n), (w_1, w_2, ..., w_n) \in K^n$ with $+$ as addition in modulo 2. 
		\item We define multiplication by scalar as 
		$$a(v_1, v_2, ..., v_n) = (av_1, av_2, ..., av_n)$$ 
		for all $a \in K$ and for all $(v_1, v_2, ..., v_n) \in K^n$
		\item Thus $K^n$ becomes a vector space over $K$.
		\item If $v$ is sent and $w$ is received, then $e = v + w$ is called the error pattern or error vector. The nonzero components of $e$ indicate errors. Ex: $v = 010100$ and $w = 011101$, then $e = 001001$ is the error pattern, where the 1's are the errors.
		\newpage
		\item Let $v \in K^n$. The Hamming weight (or just weight) of $v$, denoted by $wt(v)$ is the number of nonzero components. Ex: $wt(0111001) = 4$ 
		\item Let $v,w \in K^n$. The Hamming distance (or just distance) between $v$ and $w$, denoted by $d(v,w)$, is the number of postions in which they disagree. Ex: $d(010101,101001) = 4$. 
		\item Note that $d(v,w) = wt(v + w)$
		\item The Hamming distance is a metric, meaning it has the reflexive, symmetric, and triangle inequality properties.
		\begin{align*}
			d(v,w) &= 0 \Longleftrightarrow v = w \tag{Reflexive} \\
			d(v,w) &= d(w,v) \tag{Symmetric} \\
			d(v,w) &\leq d(v,u) + d(u,w) \tag{Triangle Inequality} 
		\end{align*}
	\end{enumerate}
	\skipline
	\textbf{09/02/20 - Section 1.9, 1.11}
	\begin{enumerate}
		\item Complete Maximum Likelihood Decoding (CMLD) -  Let
		$v \in C.$ If $d(v,w) < d(v_1,w)$  $\forall v_1 \in C, v_1 \not= v$, then decode $w$ as $v$. If there is more than one codeword closest to $w$, select one of them arbitrarily and conclude that it was the
		sent codeword.
		\item Incomplete Maximum Likelihood Decoding (IMLD) -  Let
		$v \in C.$ If $d(v,w) < d(v_1,w)$  $\forall v_1 \in C, v_1 \not= v$, then decode $w$ as $v$. If there is more than one codeword closest to w, request retransmission.
		\item Recall that $w = v + e$, where $w$ is the recieved word, $v$ is the sent codeword, and $e$ is the error pattern. Thus,
			$$d(v,w) = wt(v+w),wt(e)$$
		\item In conclusion, the decoder’s strategy is to decode $w$ into the codeword $v$ which yields the error pattern of smallest weight.
		\item Ex: Let $C = \{000,001,010,011\}$. Length of codewords $(n = 3)$, $K^3 =$ all binary triples. Construct an IMLD table for it: 
		\\ \\
		\begin{tabular}{|c|c|c|c|c|c|}
			\hline
			Recieved $w$ & $w + 000$ & $w + 001$ & $w + 010$ & $w + 011$ & Decode $v$ \\
			\hline
			000 & 000 & 001 & 010 & 011 & 000 \\
			\hline
			100 & 100 & 101 & 110 & 111 & 000 \\
			\hline
			010 & 010 & 011 & 000 & 001 & 010 \\
			\hline
			001 & 001 & 000 & 011 & 010 & 001 \\
			\hline
			110 & 110 & 111 & 100 & 101 & 010 \\
			\hline
			101 & 101 & 100 & 111 & 110 & 001 \\
			\hline
			011 & 011 & 010 & 001 & 000 & 011 \\
			\hline
			111 & 111 & 110 & 101 & 100 & 011 \\
			\hline
		\end{tabular}
		\newpage 
		\item We say that code $C$ detects the error pattern $e$ iff $v + e \not \in C, \forall v \in C$
		\item Ex: $C = \{00000, 10101, 00111, 11100\}$.  Determine whether $C$ detects each of the error patterns:
		$e = 10101, e = 01010, e = 11011$ 
		\begin{itemize}[label  = -]
			\item Notice that if $e = 10101$, We get $00000 \rightarrow 10101$.  Thus the code $C$ does not detect the error pattern
			\item Notice that if $e = 01010$, $v + e \not \in C, \forall v \in C$, so $C$ does detect the error pattern
			\item Notice that if $e = 11011$, We get $00111 \rightarrow 11100 \in C$.  Thus the code $C$ does not detect the error pattern
		\end{itemize}
	\end{enumerate}

	\skipline
	\textbf{09/04/20 - Section 1.11}
	\begin{enumerate}
		\item Minimum distance - $d(C) = \text{min}\{d(u,v)| u,v \in C, u \not= v\}$
		\item If $d(C) = d$, then $C$ detects all non-zero error patterns of weight $d - 1$ or less. Moreover, there is at least one error pattern of weight $d$ which $C$ will not detect.
  
		\item A code $C$ is said to be a t-error-detecting code if it detects all error patterns of weight t or less and it does not detect at least one error pattern of weight $t + 1$.
		\item Ex: $C = \{000, 111\}$ detects all error patterns of weight two or less
	\end{enumerate}

	\skipline
	\textbf{09/09/20 - Section 1.12}
	\begin{enumerate}
		\item A code $C$ corrects the error pattern $e$ if $\forall v \in C$, 
		$$d(v+e,v) < d(v+e,u), \forall u \in C, u \not = v$$
		\item A code of distance $d$ will correct all error patterns of weight $\leq \lfloor \frac{d-1}{2}\rfloor$.  Moreover, there exists at least one error pattern of weight $1 + \lfloor\frac{d-1}{2} \rfloor$ which $C$ will not correct.
	\end{enumerate}

	\skipline
	\textbf{09/11/20 - Section 1.10}
	\begin{enumerate}
		\item $\theta_p(C,v) =$ probability that if $v$ is sent over a BSC of reliablity $p$, then IMLD will correctly conclude that $v$ was sent.
		\\ \\
		To evaluate $\theta_p(C,v)$, we construct the set $L(v)$ which consists of all words in $K^n$ that are closer to $v$ than to any other word in $C$. It follows that 
		$$\theta_p(C,v) = \sum_{w \in L(v)} \phi_p(v,w)$$
	\end{enumerate}

	\skipline
	\textbf{09/14/20 - Section 1.10}
	\begin{enumerate}
		\item 
	\end{enumerate}








\end{document}
