\documentclass[11pt]{article}
\usepackage[margin = 1in]{geometry}
\usepackage{amsmath}
\usepackage{amssymb}
\usepackage{amsthm}
\usepackage{graphicx}
\usepackage{enumitem}
\usepackage{url}
\usepackage[parfill]{parskip}
\usepackage{listings}
\newcommand{\skipline}{\vspace{\baselineskip}}
\newcommand{\spacer}{\noalign{\medskip}}
\newcommand{~}{\sim}
\newcommand{\modn}{\text{ mod }}
\newenvironment{problem}[1]{\textbf{Problem #1: }}{\newpage}
\usepackage{caption}
\usepackage{subcaption}
\usepackage[utf8]{inputenc}
\usepackage{xcolor}
\definecolor{codegreen}{rgb}{0,0.6,0}
\definecolor{codegray}{rgb}{0.5,0.5,0.5}
\definecolor{codepurple}{rgb}{0.58,0,0.82}
\definecolor{backcolour}{rgb}{0.95,0.95,0.92}
\lstdefinestyle{mystyle}{
	backgroundcolor=\color{backcolour},   
	commentstyle=\color{codegreen},
	keywordstyle=\color{magenta},
	numberstyle=\tiny\color{codegray},
	stringstyle=\color{codepurple},
	basicstyle=\ttfamily\footnotesize,
	breakatwhitespace=false,         
	breaklines=true,                 
	captionpos=b,                    
	keepspaces=true,                 
	numbers=left,                    
	numbersep=5pt,                  
	showspaces=false,                
	showstringspaces=false,
	showtabs=false,                  
	tabsize=2
}
\lstset{style=mystyle}

\begin{document}
	
	\begin{center}
		\textbf{Quiz 8} \\
		\textbf{Algebraic Coding Theory} \\
		\textbf{Math 525} \\
		\textbf{Stephen Giang RedID: 823184070} \\
		\skipline \skipline
	\end{center}

	\begin{problem}{1}
		Consider  the  Galois  field  $GF(2^4)$  constructed  from  the  primitive  polynomial $p(x)  =x^4+x^3+ 1$.   Let $\beta$ be  a  root  of $p(x)$,  that  is, $p(\beta) = 0$.   The  table below displays the word and respective power of $\beta$ representations of each field element except for four of them.  The entries in red, namely, (a), (b), (c), and (d) are missing and you are asked to determine them.
		\begin{figure}[h!]
			\centering
			\begin{tabular}{c|c}
				word & power of $\beta$ \\
				\hline
				0000 & \\
				\hline
				1000 & $\beta^0$ \\
				\hline
				0100 & $\beta^{1}$ \\
				\hline
				0010 & $\beta^{2}$  \\
				\hline
				0001 & $\beta^{3}$  \\
				\hline
				(a)  & $\beta^{4}$ \\
				\hline
				1101 & $\beta^{5}$ \\
				\hline
				1111 & $\beta^{6}$  \\
				\hline
				1110 & $\beta^{7}$  \\
				\hline
				0111 & $\beta^{8}$  \\
				\hline
				(b)  & $\beta^{9}$ \\
				\hline
				0101 & $\beta^{10}$  \\
				\hline
				1011 & $\beta^{11}$  \\ 
				\hline
				(c)  & $\beta^{12}$ \\
				\hline
				0110 & $\beta^{13}$  \\
				\hline
				(d)  & $\beta^{14}$ 
			\end{tabular}
		\end{figure}
		\newpage
		\begin{enumerate}[label = (\arabic*)]
			\item Determine the missing entries of the table, that is, the word rep-resentations of $\beta^4,\beta^9,\beta^{12}$, and $\beta^{14}$.
			\begin{enumerate}[label = (\alph*)]
				\item Notice the following for $\beta^4$:
				\[(1 + x^3) \modn p(x) = x^4 \modn p(x)\]
				Thus we get $\boldsymbol{(a) = 1001}$
				\item Notice the following for $\beta^9$:
				\[(1 + x^2) \modn p(x) = x^9 \modn p(x)\]
				Thus we get $\boldsymbol{(b) = 1010}$
				\item Notice the following for $\beta^{12}$:
				\[(1 + x) \modn p(x) = x^{12} \modn p(x)\]
				Thus we get $\boldsymbol{(c) = 1100}$
				\item Notice the following for $\beta^{14}$:
				\[(x^2 + x^3) \modn p(x) = x^{14} \modn p(x)\]
				Thus we get $\boldsymbol{(d) = 0011}$
			\end{enumerate}
			\skipline
			\item Calculate the minimal polynomials of $\beta^7$ and $\beta^{13}$.  Express each answer as a polynomial with binary coefficients.
			\\ \\
			Notice the following:
			\[\boldsymbol{\beta^7} \quad \Rightarrow \quad (\beta^7)^2 = \boldsymbol{\beta^{14}} \quad \Rightarrow \quad (\beta^14)^2 = \beta^{28} = \boldsymbol{\beta^{13}} \quad \Rightarrow \quad (\beta^{13})^2 = \beta^{26} = \boldsymbol{\beta^{11}} \]
			\[\Rightarrow \quad (\beta^{11})^2 = \beta^{22} = \beta^7\]
			Now we can calculate the minimal polynomial:
			\begin{align*}
				m_\alpha(x) &= (x + \beta^7)(x + \beta^{14})(x + \beta^{13})(x + \beta^{11}) \\
				&= \bigg(x^2 + (\beta^7 + \beta^{14})x + \beta^{21}\bigg)\bigg(x^2 + (\beta^{13} + \beta^{11})x + \beta^{24}\bigg) \\
				&= \bigg(x^2 + \beta^5x + \beta^{6}\bigg)\bigg(x^2 + \beta^5x + \beta^{9}\bigg) \\
				&= x^4 + (\beta^5 + \beta^5)x^3 + (\beta^9 + \beta^{10} + \beta^6)x^2 + (\beta^{14} + \beta^{11})x + \beta^{15} \\
				&= \boldsymbol{x^4 + x + 1}
			\end{align*}
			\item Calculate $r(\beta^{12})$ where $r(x) =x^5+x^4+x^3+ 1$.  Your answer must be represented as either 0 or a power of $\beta$(e.g.,$\beta^5,\beta^9$, etc.).
			\\ \\
			Notice the following:
			\begin{align*}
				r(\beta^{12}) &= (\beta^{12})^5 + (\beta^{12})^4 + (\beta^{12})^3 + 1 \\
				&= \beta^{60} + \beta^{48} + \beta^{36} + 1 \\
				&= \beta^{3} + \beta^{6} + 1 \\
				&= \boldsymbol{\beta^{13}}				
			\end{align*}
		\end{enumerate}
	\end{problem}


\end{document}
