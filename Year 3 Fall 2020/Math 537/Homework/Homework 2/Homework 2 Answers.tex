\documentclass[11pt]{article}
\usepackage[margin = 1in]{geometry}
\usepackage{amsmath}
\usepackage{amssymb}
\usepackage{amsthm}
\usepackage{graphicx}
\usepackage{subfig}
\usepackage{enumitem}
\usepackage{url}
\usepackage[parfill]{parskip}
\newcommand{\skipline}{\vspace{\baselineskip}}
\newenvironment{problem}[1]{\textbf{Problem #1: }}{\newpage}


\begin{document}
	
	\begin{center}
		\textbf{Homework 2} \\
		\textbf{Ordinary Differential Equations} \\
		\textbf{Math 537} \\
		\textbf{Stephen Giang RedID: 823184070} \\
	\end{center}

	\begin{problem}{1}
		Consider the following second-order ordinary differential equations (ODEs) for linear pendulum oscillations:
		\[\frac{d^2x}{dt^2} + c\frac{dx}{dt} + Kx = 0,\tag{1}\]
		which is a linearized version of the nonlinear system:
		\[\frac{d^2x}{dt^2} + c\frac{dx}{dt} + K\sin(x) = 0\]
		Assume $c = 5$ and $K = 4$.
		\begin{enumerate}[label = (\alph*)]
			\item Solve Eq. (1) for solutions. \\
			Notice we can substitute the given values into the constants:
			\[\frac{d^2x}{dt^2} + 5\frac{dx}{dt} + 4x = 0\]
			Notice the characteristic equation and its corresponding eigenvalues, $\lambda$:
			\begin{align*}
			-\lambda (-5 - \lambda) + 4 &= 0 \\
			\lambda^2 + 5\lambda + 4 &= 0 \\
			\lambda = -4&,-1
			\end{align*}
			Thus we get the following solutions:
			\[ x(t) = c_1e^{-4t} + c_2e^{-t}\]
			\item Convert Eq. (1) into a system of first-order ODEs by introducing $y =
			dx/dt$. Solve the system of the first-order ODEs.
\\
			Notice the following system:
			\[ \begin{pmatrix}
				y \\
				y'
			\end{pmatrix} = Ax = 
			\begin{pmatrix}
				0 & 1 \\
				-4 & -5\\
			\end{pmatrix}
			\begin{pmatrix}
				x \\
				x'
			\end{pmatrix} \]
			Notice the characteristic equation and its corresponding eigenvalues, $\lambda$:
			\begin{align*}
				-\lambda (-5 - \lambda) + 4 &= 0 \\
				\lambda^2 + 5\lambda + 4 &= 0 \\
				\lambda = -4&,-1
			\end{align*}
			For $\lambda = -4$, we get $A - \lambda I = \begin{pmatrix}
				4 & 1 \\
				-4 -1 
			\end{pmatrix}$, so our eigenvector is $\begin{pmatrix}
				1 \\
				-4
			\end{pmatrix}$
			\\ \\
			For $\lambda = -1$, we get $A - \lambda I = \begin{pmatrix}
			1 & 1 \\
			-4 -4 
			\end{pmatrix}$, so our eigenvector is $\begin{pmatrix}
			1 \\
			-1
			\end{pmatrix}$.
			\\ \\
			So our final solution is as below:
			\[\begin{pmatrix}
				x \\
				x'
			\end{pmatrix} = 
			c_1 e^{-4t}\begin{pmatrix}
				1 \\ -4
			\end{pmatrix} + c_2e^{-t}\begin{pmatrix}
				1 \\ -1
			\end{pmatrix}\]
		\end{enumerate}
	\end{problem}
	
	\begin{problem}{2}
		Consider the following system of linear ODEs:
		\begin{align*}
			\frac{dx}{dt} &= \alpha y, \tag{2a} \\
			\frac{dy}{dt} &= - \beta x, \tag{2b}
		\end{align*}
		Discuss the region in the $\alpha\beta$-plane where this system has different types of eigenvalues.
		\\ \\
		Notice we can rewrite the equations as the following:
		\[
		\begin{pmatrix}
			x' \\
			y'
		\end{pmatrix} = 
		\begin{pmatrix}
			0 & \alpha \\
			-\beta & 0 
		\end{pmatrix}
		\begin{pmatrix}
			x \\
			y
		\end{pmatrix}
		\]
		Notice we can get the following characteristic equation and following eigenvalues:
		\begin{align*}
			\lambda^2 + \alpha\beta &= 0 \\
			\lambda &= \pm \sqrt{-\alpha\beta}
		\end{align*}
		Notice the following regions and let the following be true: 
		\[X'= \begin{pmatrix}
			x' \\ y'
		\end{pmatrix}, A = \begin{pmatrix}
		0 & \alpha \\
		-\beta & 0 
		\end{pmatrix}, X = \begin{pmatrix}
			x \\ y
		\end{pmatrix}\]
		\newpage
		\begin{enumerate}[label = (\roman*)]
			\item \boldmath $\alpha > 0, \beta > 0$ \unboldmath
			\\ \\
			We get that $\lambda = \pm i \sqrt{\alpha\beta}$. 
			We can obviously see because of the eigenvalues, we get a center.  We can determine its direction by testing a point like the following:
			\[\begin{pmatrix}
				x' \\ y' 
			\end{pmatrix} = \begin{pmatrix}
				0 & \alpha \\
				-\beta & 0 
			\end{pmatrix}\begin{pmatrix}
				1 \\ 0
			\end{pmatrix} = \begin{pmatrix}
				0 \\ -\beta 
			\end{pmatrix}\]
			So now we can see that we get a center that has a clockwise direction.
			\item \boldmath $\alpha < 0, \beta > 0$ \unboldmath 
			\\ \\
			We get that $\lambda = \pm \sqrt{- \alpha \beta} = \pm \sqrt{|\alpha\beta|}$.  We can see that we get a saddle point. To see its behavior, we need to find it eigenvectors.
			\\ \\
			Let $\lambda = \sqrt{|\alpha\beta|}$,
			\[
			(A - \lambda I)X = 
			\begin{pmatrix}
				-\sqrt{|\alpha\beta|} & \alpha \\
				-\beta & -\sqrt{|\alpha\beta|} 
			\end{pmatrix} 
			\begin{pmatrix}
				x \\ y
			\end{pmatrix} = 
			\begin{pmatrix}
				0 \\ 0
			\end{pmatrix}
			\]
			We get the following eigenvalue: $\begin{pmatrix}
				\alpha \\ \sqrt{|\alpha\beta|}
			\end{pmatrix}$
			\\ \\
			Let $\lambda = -\sqrt{|\alpha\beta|}$,
			\[
			(A - \lambda I)X = 
			\begin{pmatrix}
			\sqrt{|\alpha\beta|} & \alpha \\
			-\beta & \sqrt{|\alpha\beta|} 
			\end{pmatrix} 
			\begin{pmatrix}
			x \\ y
			\end{pmatrix} = 
			\begin{pmatrix}
			0 \\ 0
			\end{pmatrix}
			\]
			We get the following eigenvalue: $\begin{pmatrix}
			\alpha \\ -\sqrt{|\alpha\beta|}
			\end{pmatrix}$
			\\ \\
			So we can actually get the solution:
			\[X(t) = \begin{pmatrix}
				x \\ y
			\end{pmatrix} = Ae^{\sqrt{|\alpha\beta|} t}\begin{pmatrix}
			\alpha \\ \sqrt{|\alpha\beta|}
			\end{pmatrix} + Be^{-\sqrt{|\alpha\beta|} t}\begin{pmatrix}
			\alpha \\ -\sqrt{|\alpha\beta|}
			\end{pmatrix} \]
			Finally we can see we get a saddle point with the following characteristics with $A \not= 0, B \not=0$:	
			\begin{align*}
				\lim\limits_{t \rightarrow -\infty} X(t) \sim X_2(t) =  Be^{-\sqrt{|\alpha\beta|} t}\begin{pmatrix}
				\alpha \\ -\sqrt{|\alpha\beta|} \end{pmatrix}, && \lim\limits_{t \rightarrow \infty} X(t) \sim X_1(t) =  Ae^{\sqrt{|\alpha\beta|} t}\begin{pmatrix}
				\alpha \\ \sqrt{|\alpha\beta|}
				\end{pmatrix}
			\end{align*}
			\item \boldmath $\alpha < 0, \beta < 0$ \unboldmath
			\\ \\
			We get that $\lambda = \pm i \sqrt{\alpha\beta}$. 
			We can obviously see because of the eigenvalues, we get a center.  We can determine its direction by testing a point like the following:
			\[\begin{pmatrix}
			x' \\ y' 
			\end{pmatrix} = \begin{pmatrix}
			0 & \alpha \\
			-\beta & 0 
			\end{pmatrix}\begin{pmatrix}
			1 \\ 0
			\end{pmatrix} = \begin{pmatrix}
			0 \\ -\beta 
			\end{pmatrix}\]
			So now we can see that we get a center that has a counter-clockwise direction.
			\item \boldmath $\alpha > 0, \beta < 0$ \unboldmath
			\\ \\
			We get the same result as part (ii) however, the phase portrait is a reflection across the y-axis from the phase portrait in (ii).
			\newpage
			\item \boldmath $\alpha = 0$ \unboldmath
			\\ \\
			We get that $\lambda = 0$. Notice the eigenvector for the eigenvalue, $\begin{pmatrix}
				0 \\ 1
			\end{pmatrix}$.
			So we know that the phase portrait will consist of vertical lines.  We can see through evaluation, we get that for $\beta < 0$, we get downward lines for $x < 0$ and upward lines for $x > 0$.  We also get that for $\beta > 0$, we get upward lines for $x < 0$ and downward lines for $x > 0$.
			\item \boldmath $\beta = 0$ \unboldmath
			\\ \\
			We get that $\lambda = 0$. Notice the eigenvector for the eigenvalue, $\begin{pmatrix}
			1 \\ 0
			\end{pmatrix}$.
			So we know that the phase portrait will consist of horizontal lines.  We can see through evaluation, we get that for $\alpha < 0$, we get rightward lines for $y < 0$ and leftward lines for $y > 0$.  We also get that for $\alpha > 0$, we get leftward lines for $y < 0$ and rightward lines for $y > 0$.
			\item \boldmath $\beta = \alpha = 0$ \unboldmath
			\\ \\
			We get that $\lambda = 0$.  However we get infinite solutions so we do not get a phase portrait . 
		\end{enumerate}
	\end{problem}

	\begin{problem}{3}
		Consider the following linearized Lorenz model (Lorenz, 1963):
		\begin{align*}
			\frac{dX}{dt} &= -\sigma X + \sigma Y, \tag{3a} \\
			\frac{dY}{dt} &= rX - Y. \tag{3b} 			
		\end{align*}
		Perform a stability analysis for $\sigma > 0$ (i.e., discuss the cases with $r > 1$, $r = 1$, and $r < 1$, respectively.)
		\\ \\
		Notice we can rewrite the equations as the following:
		\[\begin{pmatrix}
			X' \\ Y'	
		\end{pmatrix} = 
		\begin{pmatrix}
			-\sigma & \sigma \\
			r & -1
		\end{pmatrix}\begin{pmatrix}
			X \\ Y 
		\end{pmatrix}
		\]
		Notice we can get the following characteristic equation and following eigenvalues:
		\begin{align*}
			 (-\sigma - \lambda )(-1 - \lambda ) - r\sigma &= 0 \\
			\lambda^2+ (\sigma + 1)\lambda + (1 - r)\sigma &= 0 
		\end{align*}
		Now we can use the quadratic formula to get the eigenvalues:
		\[\lambda = \frac{-(\sigma + 1)\pm \sqrt{(\sigma + 1)^2 - 4\sigma(1-r)} }{2} \]
		Notice the following cases:
		\begin{enumerate}[label = (\alph*)]
			\item ($r > 1$):
			We get two real distinct eigenvalues. Notice that we can use the triangle inequality to see the following:
			\[(\sigma + 1) < \sqrt{(\sigma + 1)^2 - 4\sigma(1-r)} \]
			because with $r  >1$, $-4\sigma(1-r) > 0$. Now notice the following:
			\begin{align*}
				\lambda_1 &= \frac{-(\sigma + 1) + \sqrt{(\sigma + 1)^2 - 4\sigma(1-r)} }{2} > 0 \\
				\lambda_2 &= \frac{-(\sigma + 1) - \sqrt{(\sigma + 1)^2 - 4\sigma(1-r)} }{2} < 0
			\end{align*}
			Because of we have $\lambda_2 < 0 < \lambda_1$, we have a saddle point.  We can see that as $t \rightarrow -\infty$, we have that $\begin{pmatrix}
			X \\ Y
			\end{pmatrix}$ approaches the solution with $\lambda_2 < 0$, and as $t \rightarrow \infty$, $\begin{pmatrix}
			X \\ Y
			\end{pmatrix}$ approaches the solution with $\lambda_1 > 0$
			\item ($r = 1$):
			We get two real distinct eigenvalues.  Notice the following equality:
			\[(\sigma + 1) = \sqrt{(\sigma + 1)^2 - 4\sigma(1-r)} \]
			because with $r = 1$, $-4\sigma(1-r) = 0$.  Now notice the following:
			\begin{align*}
				\lambda_1 &= \frac{-(\sigma + 1) + \sqrt{(\sigma + 1)^2 - 4\sigma(1-r)} }{2} = 0 \\
				\lambda_2 &= \frac{-(\sigma + 1) - \sqrt{(\sigma + 1)^2 - 4\sigma(1-r)} }{2} = -(\sigma + 1) < 0
			\end{align*}
			\newpage
			Notice we can get the eigenvectors from these eigenvalues:
			\\ \\
			First, let the following be true:
			$\boldsymbol{\dot{X}'} = \begin{pmatrix}
				X' \\ Y'
			\end{pmatrix}$, $A = \begin{pmatrix}
				-\sigma & \sigma \\
				r & -1
			\end{pmatrix}$, and $\boldsymbol{\dot{X}} = \begin{pmatrix}
				X \\ Y
			\end{pmatrix}$
			\\ \\
			Let $\lambda_1 = 0$:
			\[(A - \lambda I)\boldsymbol{\dot{X}} = \begin{pmatrix}
			-\sigma & \sigma \\
			1 & -1
			\end{pmatrix}\begin{pmatrix}
			X \\ Y
			\end{pmatrix}\]
			We get the eigenvector: $V_1 = \begin{pmatrix}
				1 \\ 1
			\end{pmatrix}.$
			\\ \\
			Let $\lambda = -(\sigma + 1)$:
			\[(A - \lambda I)\boldsymbol{\dot{X}} = \begin{pmatrix}
			1 & \sigma \\
			1 & \sigma
			\end{pmatrix}\begin{pmatrix}
			X \\ Y
			\end{pmatrix}\]
			We get the eigenvector: $V_2 = \begin{pmatrix}
			\sigma \\ -1
			\end{pmatrix}.$
			Thus we get the solution:
			\[\boldsymbol{\dot{X}} = A\begin{pmatrix}
			1 \\ 1
			\end{pmatrix} + Be^{-(\sigma + 1)t}\begin{pmatrix}
			\sigma \\ -1
			\end{pmatrix}\]
			So we have a phase portrait that as $t \rightarrow \infty$, $\boldsymbol{\dot{X}}$ goes along $V_2 = \begin{pmatrix}
			\sigma \\ -1
			\end{pmatrix}$ towards the equilibrium line, $V_1 = \begin{pmatrix}
			1 \\ 1
			\end{pmatrix}$
			\item ($r < 1$):
			Notice the following inequality:
			\begin{align*}
				(\sigma + 1)^2 - 4\sigma(1-r) &> 0 \\
				(\sigma + 1)^2 &> 4\sigma(1-r) \\
				1 - \frac{(\sigma + 1)^2}{4\sigma} &< r
			\end{align*}
			\begin{enumerate}[label = (\alph*)]
				\item We get two distinct eigenvalues such that $1 - \frac{(\sigma + 1)^2}{4\sigma} < r < 1$
				For this inequality, we get another inequality:
				\[0 < \sqrt{(\sigma + 1)^2 - 4\sigma(1-r)} < \sigma + 1\]	
				Using this inequality we can see that for both eigenvalues:
				\[\lambda_{1,2} = \frac{-(\sigma + 1) \pm \sqrt{(\sigma + 1)^2 - 4\sigma(1-r)} }{2} < 0\]
				Because we have $\lambda_1 < \lambda_2 < 0$, we have a sink.	
				\item We get a single real eigenvalue such that $1 - \frac{(\sigma + 1)^2}{4\sigma} = r$ 
				\\ \\
				We get the eigenvalue to be $\lambda = \frac{-(\sigma + 1)}{2}$.  Because we have a single, negative eigenvalue, we can determine that the phase portrait is a sink.
				\item 	We get complex eigenvalues such that $1 - \frac{(\sigma + 1)^2}{4\sigma} > r$.
				\\ \\
				We have the eigenvalues in the form of $-\alpha + i\beta$.  Because the real part is negative, we have a spiral sink.
			\end{enumerate}
				
			
			
		
		\end{enumerate}
	\end{problem}

	\begin{problem}{4}
		Consider the following epidemic model (Kermack and McKendrick, 1927), which is called the "SIR" model:
		\begin{align*}
			\frac{dS}{dt} &= - \frac{\beta}{N}SI, \tag{4.1} \\
			\frac{dI}{dt} &= \frac{\beta}{N}SI - \nu I, \tag{4.2} \\
			\frac{dR}{dt} &= \nu I \tag{4.3}
		\end{align*}
		Here, $S, I,$ and $R$ denote susceptible, infected, and recovered individuals,
		respectively. Three parameters, $\beta > 0, \nu > 0$, and $N > 0$, represent a
		transmission rate, a recovery rate, and a fixed population $(N = S + I + R)$,
		respectively. Complete the following derivations to convert Eqs. (4.1)-(4.3)
		into the following equations:
		\begin{align*}
			S &= S(0)e^{-\frac{\beta}{N \nu}(R(t) - R(0))}, \tag{4.4} \\
			I &= N  - S(0)e^{-\frac{\beta}{N \nu}(R(t) - R(0))} - R, \tag{4.5} \\
			\frac{dR}{dt} &= \nu \left(N  - R - S(0)e^{-\frac{\beta}{N \nu}(R(t) - R(0))}\right) \tag{4.6}
		\end{align*}
		where $S(0)$ and $R(0)$ represent the initial values of $S$ and $R$, respectively.
		\newpage
		\begin{enumerate}[label = (\alph*)]
			\item Show
			\[S + I + R = constant = N, \tag{4.7}\]
			$\left(\text{i.e., } \frac{d(S + I + R)}{dt} = 0\right)$
			\\ \\ \\
			Notice the following:
			\[\frac{d(S + I + R)}{dt} = \frac{d}{dt}\left(S+I+R\right) = \frac{dS}{dt} +\frac{dI}{dt} + \frac{dR}{dt} =  - \frac{\beta}{N}SI + \frac{\beta}{N}SI - \nu I + \nu I = 0 \]
			Because $\left(\text{i.e., } \frac{d(S + I + R)}{dt} = 0\right)$, we can determine that $N = S + I + R$ is a constant
			\\ \\
			\item Apply Eqs (4.1) and (4.3) to obtain the following:
			\[\frac{S'}{S} = -\frac{\beta}{N \nu}R'.\]
			Integrate the above Eq. to obtain Eq. (4.4), yielding $S = S(R)$
			\\ \\ \\
			Notice the integration:
			\begin{align*}
				\int_{S(0)}^{S(R)}\frac{S'}{S} &= \int_{R(0)}^{R(t)}-\frac{\beta}{N \nu}R'\\
				\ln S\bigg|_{S(0)}^{S(R)} &= \left.-\frac{\beta}{N \nu}R\right|_{R(0)}^{R(t)} \\
				\ln \frac{S(R)}{S(0)} &= -\frac{\beta}{N \nu}\left(R(t) - R(0)\right) \\
				S(R) &= S(0)e^{-\frac{\beta}{N \nu}\left(R(t) - R(0)\right)}
			\end{align*}
			\item Apply Eqs. (4.4) and (4.7) to find Eq. (4.5) for $I$, which is a function of $R$
			\\ \\
			Notice that we get the following from (4.7):
			\[I = N - S - R\]
			Notice that we get the following from substituting the result of (4.4)
			\[I = N - S(0)e^{-\frac{\beta}{N \nu}\left(R(t) - R(0)\right)} - R\]
			\item Apply the above to obtain Eq. (4.6).
			\\ \\
			Notice we can get the following from (4.5) and (4.3):
			\[\frac{dR}{dt} = \nu \left(N  - R - S(0)e^{-\frac{\beta}{N \nu}(R(t) - R(0))}\right)\]
			\\ \\
			\item Briefly discuss how to analyze Eq. (4.6) to reveal the characteristics of
			the solution.
			\\ \\
			We can analyze Eq (4.6) by seeing the recovery rate over time.  By taking the integral of it, we can see the total number of people recovered. Evaluating all the constants, we can see that is differential equation is a first order differential equation.  We can solve this using many of our techniques.
			\\ \\ \\
		\end{enumerate}
		Note that based on Eqs. (4.4)-(4.6), we can obtain the solutions by solving
		a single first order ODE in Eq. (4.6) for $R(t)$, and then compute $S(t)$ and
		$I(t)$ using Eqs. (4.4) and (4.5), respectively.
		
		
	\end{problem}

\end{document}
