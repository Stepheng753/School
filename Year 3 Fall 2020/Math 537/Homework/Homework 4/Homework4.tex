\documentclass[11pt]{article}
\usepackage[margin = 1in]{geometry}
\usepackage{amsmath}
\usepackage{amssymb}
\usepackage{amsthm}
\usepackage{graphicx}
\usepackage{subfig}
\usepackage{enumitem}
\usepackage{url}
\usepackage[parfill]{parskip}
\usepackage{listings}
\newcommand{\skipline}{\vspace{\baselineskip}}
\newcommand{\spacer}{\noalign{\medskip}}
\newenvironment{problem}[1]{\textbf{Problem #1: }}{\newpage}


\begin{document}
	
	\begin{center}
		\textbf{Homework 4} \\
		\textbf{Ordinary Differential Equations} \\
		\textbf{Math 537} \\
		\textbf{Stephen Giang RedID: 823184070} \\
		\skipline \skipline
	\end{center}

	\begin{problem}{1}
		 Consider the Lorenz model:
		 \begin{align*}
		 	\frac{dX}{dt} &= -\sigma X + \sigma Y \\
		 	\frac{dY}{dt} &= -XZ + rX - Y \\
		 	\frac{dZ}{dt} &= XY -\beta Z
		 \end{align*}
	 	\begin{enumerate}[label = (\alph*)]
	 		\item Find the Jacobian matrix at the trivial critical point $(X,Y,Z) = (0, 0, 0)$. 
	 		\\ \\
	 		Let $f_1 = \frac{dX}{dt}$, $f_2 = \frac{dY}{dt}$, $f_3 = \frac{dZ}{dt}$.  Notice the following Jacobian:
	 		\[J(X,Y,Z) = \left[\begin{array}{ccc}
	 			\frac{\partial f_1}{\partial X} & \frac{\partial f_1}{\partial Y} & \frac{\partial f_1}{\partial Z} \\
	 			\spacer\frac{\partial f_2}{\partial X} & \frac{\partial f_2}{\partial Y} & \frac{\partial f_2}{\partial Z} \\
	 			\spacer\frac{\partial f_3}{\partial X} & \frac{\partial f_3}{\partial Y} & \frac{\partial f_3}{\partial Z} \\
	 		\end{array}\right] = \left[ \begin{array}{ccc}
	 			-\sigma & \sigma & 0 \\
	 			\spacer -Z + r & -1 & -X \\
	 			\spacer Y & X & -\beta
 			\end{array}\right]\]
 			Now we get the following Jacobian at the trivial critical point:
 			\[J(0,0,0) = \left[ \begin{array}{ccc}
 				-\sigma & \sigma & 0 \\
 				\spacer r & -1 & 0 \\
 				\spacer 0 & 0 & -\beta
 			\end{array}\right]\] 
	 		\item Choose $\sigma = 10$.  Perform a (linear) stability analysis in $r,\beta$-space using the matrix in (a). \\
	 		{[Hint:  Describe the regions where the Jacobian matrix has real and/or
	 		complex eigenvalues.]} 
 			\\ \\
 			Notice the following Jacobian at the trivial critical point with $\sigma = 10$:
 			\[J(0,0,0) = \left[ \begin{array}{ccc}
 				-10 & 10 & 0 \\
 				\spacer r & -1 & 0 \\
 				\spacer 0 & 0 & -\beta
 			\end{array}\right]\]
 			Notice we can find the eigenvalues by solving $det(J - \lambda I) = 0$:
 			\begin{align*}
 				(-10 - \lambda)(-1 - \lambda)(-\beta - \lambda) - 10r(-\beta - \lambda) &= 0 \\
 				-(\beta + \lambda)\bigg((10 + \lambda)(1 + \lambda) - 10r\bigg) &= 0 \\
 				-(\beta + \lambda)(\lambda^2 + 11\lambda + 10(1 - r)) &= 0 
 			\end{align*}
 			Thus we get the following eigenvalues:
 			\[\lambda_1 = -\beta, \qquad \lambda_{2,3} = \frac{-11 \pm \sqrt{121 - 4(10)(1-r)}}{2}\]
 			\newpage
 			Notice we can the following:
 			\begin{align*}
 				121 - 4(10)(1 - r) &= 0 & 121 - 4(10)(1 - r) &= 121 \\
 				121 - 40 + 40r &= 0 & (1 - r) &= 0 \\
 				r &= \frac{-81}{40} & r &= 1
 			\end{align*}
 			Notice the linear stability analysis at the specified $r,\beta$ spaces.
 			\begin{enumerate}[label = (\alph*)]
 				\item Let $r < \frac{-81}{40}$
 				\\ \\
 				We get the following:
 				\[\lambda_{2,3} = \frac{-11 \pm i\sqrt{|121 - 4(10)(1-r)|}}{2}\]
 				If we let $\beta > 0$, we get $\lambda_1 < 0$, thus giving us 1 negative real eigenvalue and 2 complex eigenvalues with negative real parts.  This results in a \textbf{Spiral Sink}.
 				\\ \\
 				If we let $\beta < 0$, we get $\lambda_1 > 0$, thus giving us 1 positive real eigenvalue and 2 complex eigenvalues with negative real parts.  This results in a \textbf{Saddle Focus}.
 				\\ 
 				\item Let $r = \frac{-81}{40}$
 				\\ \\
 				We get the following:
 				\[\lambda_{2,3} = \frac{-11}{2}\]
 				If we let $\beta > 0$, we get $\lambda_1 < 0$, thus giving us 3 negative eigenvalues.  This results in a \textbf{Sink}.
 				\\ \\
 				If we let $\beta < 0$, we get $\lambda_1 > 0$, thus giving us 1 positive eigenvalue and 2 negative eigenvalues.  This results in a \textbf{Saddle}.
 				\\
 				\item Let $\frac{-81}{40} < r < 1$
 				\\ \\
 				We get the following:
 				\[\lambda_{2} = \frac{-11 + \sqrt{121 - 4(10)(1-r)}}{2} < 0, \qquad \lambda_{3} = \frac{-11 - \sqrt{121 - 4(10)(1-r)}}{2} < 0\]
 				If we let $\beta > 0$, we get $\lambda_1 < 0$, thus giving us 3 negative eigenvalues.  This results in a \textbf{Sink}.
 				\\ \\
 				If we let $\beta < 0$, we get $\lambda_1 > 0$, thus giving us 1 positive eigenvalue and 2 negative eigenvalues.  This results in a \textbf{Saddle}.
 				\newpage
 				\item Let $r = 1$
 				\\ \\
 				We get the following:
 				\[\lambda_2 = \frac{-11 + 11}{2} = 0, \qquad \lambda_3 = \frac{-11 - 11}{2} = -11\]
 				If we let $\beta > 0$, we get $\lambda_1 < 0$, thus giving us a zero eigenvalue, and 2 negative eigenvalues.  This results in a \textbf{Attractive Line of Equilibrium}
 				\\ \\
 				If we let $\beta < 0$, we get $\lambda_1 > 0$, thus giving us a zero eigenvalue, 1 positive eigenvalue, and 1 negative eigenvalue.  This results in a \textbf{Saddle Around a Line of Equilibrium}.
 				\\
 				\item Let $r > 1$
 				\\ \\
 				We get the following:
 				\[\lambda_{2} = \frac{-11 + \sqrt{121 - 4(10)(1-r)}}{2} > 0, \qquad \lambda_{3} = \frac{-11 - \sqrt{121 - 4(10)(1-r)}}{2} < 0\]
 				If we let $\beta > 0$, we get $\lambda_1 < 0$, thus giving us 1 positive eigenvalues and 2 negative eigenvalue.  This results in a \textbf{Saddle}
 				\\ \\
 				If we let $\beta < 0$, we get $\lambda_1 > 0$, thus giving us 2 positive eigenvalues and 1 negative eigenvalue.  This results in a \textbf{Saddle}
 			\end{enumerate}
	 	\end{enumerate}
	\end{problem}

	\begin{problem}{2}
		 Consider the non-dissipative Lorenz model:
		 \begin{align*}
		 	\frac{dX}{dt} &= \sigma Y \\
		 	\frac{dY}{dt} &= -XZ + rX \\
		 	\frac{dZ}{dt} &= XY
		 \end{align*}
	 	\begin{enumerate}[label = (\alph*)]
	 		\item Find critical points.
	 		\\ \\
	 		Notice we can find the critical points by solving:
	 		\[\left[\begin{array}{c}
	 			\frac{dX}{dt} \\ \spacer \frac{dY}{dt} \\ \spacer \frac{dZ}{dt}
	 		\end{array}\right] = \left[\begin{array}{c}
	 			0 \\ \spacer 0 \\ \spacer 0
	 		\end{array}\right] = \left[\begin{array}{ccc}
	 			0 & \sigma & 0 \\
	 			\spacer r & 0 & -X \\
	 			\spacer 0 & X & 0
 			\end{array}\right]\left[\begin{array}{c}
 				X \\ \spacer Y \\ Z
 			\end{array}\right]\]
 			This results in the critical points $\boldsymbol{(0,0,B), (C, 0, r)}$ $\forall B,C \in \mathbb{R}$.
 			\\
	 		\item  Find the Jacobian matrix at critical points(s).
	 		\\ \\
	 		Let $f_1 = \frac{dX}{dt}$, $f_2 = \frac{dY}{dt}$, $f_3 = \frac{dZ}{dt}$.  Notice the following Jacobian:
	 		\[J(X,Y,Z) = \left[\begin{array}{ccc}
	 			\frac{\partial f_1}{\partial X} & \frac{\partial f_1}{\partial Y} & \frac{\partial f_1}{\partial Z} \\
	 			\spacer\frac{\partial f_2}{\partial X} & \frac{\partial f_2}{\partial Y} & \frac{\partial f_2}{\partial Z} \\
	 			\spacer\frac{\partial f_3}{\partial X} & \frac{\partial f_3}{\partial Y} & \frac{\partial f_3}{\partial Z} \\
	 		\end{array}\right] = \left[ \begin{array}{ccc}
	 			0 & \sigma & 0 \\
	 			\spacer -Z + r & 0 & -X \\
	 			\spacer Y & X & 0
	 		\end{array}\right]\]
 			Now we get the following Jacobian at the critical points:
 			\[J(0,0,B) = \left[ \begin{array}{ccc}
		 		0 & \sigma & 0 \\
		 		\spacer -B + r & 0 & 0 \\
		 		\spacer 0 & 0 & 0
	 		\end{array}\right], \qquad J(C,0,r) = \left[ \begin{array}{ccc}
		 		0 & \sigma & 0 \\
		 		\spacer 0 & 0 & -C \\
		 		\spacer 0 & C & 0
	 		\end{array}\right]\]
 			\newpage
	 		\item Perform a linear stability analysis at each of the critical points.
			\\ \\
			 For $J(0,0,B)$, we get the following eigenvalues:
			\begin{align*}
				-\lambda^3 - \sigma (-B + r)(-\lambda) &= 0 \\
				-\lambda \bigg(\lambda^2 - \sigma (-B + r)\bigg) &= 0 \\
				\lambda_1 = 0, \quad \lambda_{2,3} &= \pm \sqrt{\sigma (-B + r)}
			\end{align*}
			If $\sigma (-B + r) > 0$, then we get 1 zero eigenvalue, 1 negative eigenvalue, and 1 positive eigenvalue.  This would lead to a \textbf{Saddle Around a Line of Equilibrium}.
			\\ \\
			If $\sigma (-B + r) < 0$, then we get 1 zero eigenvalue, and two complex eigenvalues with zero real parts.  This would lead to a \textbf{Center}
			\\ \\ \\
 			For $J(C,0,r)$, we get the following eigenvalues
 			\begin{align*}
 				-\lambda\bigg(\lambda^2 + C^2\bigg) &= 0 \\
 				\lambda_1 = 0, \quad \lambda_{2,3} &= \pm i |C|
 			\end{align*}
 			If $C \not = 0$, we get 1 zero eigenvalue, and 2 complex eigenvalues with zero real parts.  This would lead to a \textbf{Center}
 			\\ \\
 			If $C = 0$, we get 3 zero eigenvalues with \textbf{No Phase Portrait}.
	 	\end{enumerate}
	\end{problem}

	\begin{problem}{3}
		Consider the following harmonic oscillators:
		\begin{align*}
			\frac{d^2x_1}{dt^2} &= -k_1x_1 \\
			\frac{d^2x_2}{dt^2} &= -k_2x_2
		\end{align*}
		Let $k_1 = 4\omega_1^2$ and $k_2 = \omega_2^2$.
		\begin{enumerate}[label = (\alph*)]
			\item Convert the above equations into a linear system with four first-order
			differential equations. Find the matrix $A$ that represents the 4D system.
			\\ \\
			Let $\frac{dx_1}{dt} = y_1$ and $\frac{dx_2}{dt} = y_2$ such that:
			\[y_1 = \frac{dx_1}{dt}, \qquad y_1' = \frac{d^2x_1}{dt^2}, \qquad \qquad y_2 = \frac{dx_2}{dt}, \qquad y_2' = \frac{d^2x_2}{dt^2}\]
			Using this and letting $k_1 = 4\omega_1^2$  $k_2 = \omega_2^2$ , we get the following:
			\[X' = \left[\begin{array}{c}
				\frac{dx_1}{dt} \\ \spacer \frac{d^2x_1}{dt^2} \\ \spacer \frac{dx_2}{dt} \\ \spacer \frac{d^2x_2}{dt^2}
			\end{array}\right] = \left[\begin{array}{cccc}
				0 & 1 & 0 & 0 \\
				\spacer -4\omega_1^2 & 0 & 0 & 0 \\
				\spacer 0 & 0 & 0 & 1 \\
				\spacer 0 & 0 & -\omega_2^2 & 0
			\end{array}\right]\left[\begin{array}{c}
				x_1 \\ \spacer y_1 \\ \spacer x_2 \\ \spacer y_2
			\end{array}\right] = Ax\]
			\item Find the eigenvalues and eigenvectors of $A$ in the 4-D phase space.
			\\ \\
			Notice we can find the eigenvalues by solving $det(J - \lambda I) = 0$:
			\begin{align*}
				-\lambda\bigg(-\lambda(\lambda^2 + \omega_2^2)\bigg) + 4\omega_1^2(\lambda^2 + \omega_2^2) &= 0 \\
				(\lambda^2 + 4\omega_1^2)(\lambda^2 + \omega_2^2) &= 0
			\end{align*}
			Thus we get $\boldsymbol{\lambda_{1,2} = \pm 2i\omega_1}$ and $\boldsymbol{\lambda_{3,4} = \pm i\omega_2}$.  
			\\ \\
			Notice we can find $V_1$ by solving $(A - \lambda I)V_1 = 0$ with $\lambda = 2i\omega_1$
			\[\begin{pmatrix}
				-2i\omega_1 & 1 & 0 & 0 \\
				-4\omega_1^2 & -2i\omega_1 & 0 & 0 \\
				0 & 0 & -2i\omega_1 & 1 \\
				0 & 0 & -\omega_2^2 & -2i\omega_1
			\end{pmatrix}\begin{pmatrix}
				x_1 \\ y_1 \\ x_2 \\ y_2	
			\end{pmatrix} = \begin{pmatrix}
				0 \\ 0 \\ 0 \\ 0
			\end{pmatrix}\]
			Notice we get that $\boldsymbol{V_1 = (1,\,2i\omega_1,\,0,\,0)^T}$.  Because $\lambda_2 = -2i\omega_1$ is the conjugate of $\lambda_1$, then $V_2$ is the conjugate of $V_1$, such that $\boldsymbol{V_2 = (1,\,-2i\omega_1,\,0,\,0)^T}$.
			\\ \\
			Notice we can find $V_3$ by solving $(A - \lambda I)V_3 = 0$ with $\lambda = i\omega_2$
			\[\begin{pmatrix}
				-i\omega_2 & 1 & 0 & 0 \\
				-4\omega_1^2 & -i\omega_2 & 0 & 0 \\
				0 & 0 & -i\omega_2 & 1 \\
				0 & 0 & -\omega_2^2 & -i\omega_2
			\end{pmatrix}\begin{pmatrix}
				x_1 \\ y_1 \\ x_2 \\ y_2	
			\end{pmatrix} = \begin{pmatrix}
				0 \\ 0 \\ 0 \\ 0
			\end{pmatrix}\]
			Notice we get that $\boldsymbol{V_3 = (0,\,0,\,1,\,i\omega_2)^T}$.  Because $\lambda_4 = -i\omega_2$ is the conjugate of $\lambda_3$, then $V_4$ is the conjugate of $V_3$, such that $\boldsymbol{V_4 = (0,\,0,\,1,\,-i\omega_2)^T}$.
			\newpage
			\item Find the linear map $T$ using (b) and compute $T^{-1}AT$.
			\\ \\
			Notice the following:
			\[T = (V_1, V_2, V_3, V_4) = \begin{pmatrix}
					1 & 1 & 0 & 0 \\
					2i\omega_1 & -2i\omega_1 & 0 & 0 \\
					0 & 0 & 1 & 1 \\
					0 & 0 & i\omega_2 & -i\omega_2
				\end{pmatrix} \]
			\begin{align*}
				T^{-1}AT &= \begin{pmatrix}
					1/2 & 1/4i\omega_1 & 0 & 0 \\
					1/2 & -1/4i\omega_1 & 0 & 0 \\
					0 & 0 & 1/2 & 1/2i\omega_2 \\
					0 & 0 & 1/2 & -1/2i\omega_2
				\end{pmatrix}\begin{pmatrix}
					0 & 1 & 0 & 0 \\
					-4\omega_1^2 & 0 & 0 & 0 \\
					0 & 0 & 0 & 1 \\
					0 & 0 & -\omega_2^2 & 0	
				\end{pmatrix}\begin{pmatrix}
					1 & 1 & 0 & 0 \\
					2i\omega_1 & -2i\omega_1 & 0 & 0 \\
					0 & 0 & 1 & 1 \\
					0 & 0 & i\omega_2 & -i\omega_2
				\end{pmatrix} \\
				&= \begin{pmatrix}
					i\omega_1 & 1/2 & 0 & 0 \\
					-i\omega_1 & 1/2 & 0 & 0 \\
					0 & 0 & i\omega_2 / 2 & 1/2 \\
					0 & 0 & -i\omega_2 / 2 & 1/2 
				\end{pmatrix}\begin{pmatrix}
					1 & 1 & 0 & 0 \\
					2i\omega_1 & -2i\omega_1 & 0 & 0 \\
					0 & 0 & 1 & 1 \\
					0 & 0 & i\omega_2 & -i\omega_2
				\end{pmatrix} \\
				&= \begin{pmatrix}
					2i\omega_1 & 0 & 0 & 0 \\
					0 & -2i\omega_1 & 0 & 0 \\
					0 & 0 & i\omega_2 & 0 \\
					0 & 0 & 0 & -i\omega_2
				\end{pmatrix}
			\end{align*}
			Notice this is in the form of $T^{-1}AT$ being a matrix with its diagonal consisting of its eigenvalues and the other entries being zero.
		\end{enumerate}
	\end{problem}

	\begin{problem}{4}
		Consider the following matrix:
		\[A = \begin{pmatrix}
			2 & 3 & 0 \\ 0 & 2 & -1 \\ 0 & 0 & 2
		\end{pmatrix}\]
		\begin{enumerate}[label = (\alph*)]
			\item  Find the eigenvector(s) and generalized eigenvector(s) associated with
			the matrix $A$.
			\\ \\
			Notice that A is an upper triangular matrix.  This means that we get a repeated eigenvalue of $\boldsymbol{\lambda = 2}$.
			\\ \\
			Notice, we can find $V_1$ by solving $(A - \lambda I)V_1 = 0$:
			\[\begin{pmatrix}
				0 & 3 & 0 \\ 0 & 0 & -1 \\ 0 & 0 & 0
			\end{pmatrix}\begin{pmatrix}
				x \\ y \\ z
			\end{pmatrix} = \begin{pmatrix}
				0 \\ 0 \\ 0
			\end{pmatrix}\]
			Because of the second and third row, we get that $y = z = 0$, so we let $x = 1$, such that $\boldsymbol{V_1 = (1, 0, 0)^T}$.
			\\ \\
			Notice, we can find $V_2$ by solving $(A - \lambda I)V_2 = V_1$:
			\[\begin{pmatrix}
				0 & 3 & 0 \\ 0 & 0 & -1 \\ 0 & 0 & 0
			\end{pmatrix}\begin{pmatrix}
				x \\ y \\ z
			\end{pmatrix} = \begin{pmatrix}
				1 \\ 0 \\ 0
			\end{pmatrix}\]
			Because we need $3y = 1$, we get $y = 1/3$.  Because of the second row, we get that $z = 0$.  Lastly, we can let $x = 1$, such that $\boldsymbol{V_2 = (1 , 1/3, 0)^T}$.
			\\ \\
			Notice, we can find $V_3$ by solving $(A - \lambda I)V_3 = V_2$:
			\[\begin{pmatrix}
				0 & 3 & 0 \\ 0 & 0 & -1 \\ 0 & 0 & 0
			\end{pmatrix}\begin{pmatrix}
				x \\ y \\ z
			\end{pmatrix} = \begin{pmatrix}
				1 \\ 1/3 \\ 0
			\end{pmatrix}\]
			Because we need $3y = 1$, we get $y = 1/3$.  Because we need $-z = 1 /3$, we get $z = -1/3$.  Lastly, we can let $x = 1$, such that $\boldsymbol{V_2 = (1 , 1/3, -1/3)^T}$.
			\\ 
			\item Construct a linear map $T$ using the eigenvector(s) and generalized eigenvector(s) in (a) and compute $T^{-1}AT$. 
			\\ \\
			Notice the linear map $T$ and $T^{-1}AT$:
			\begin{align*}
				T &= \left(\begin{array}{ccc}
					V_1 & V_2 & V_3
				\end{array}\right) = \begin{pmatrix}
					1 & 1 & 1 \\ 0 & 1/3 & 1/3 \\ 0 & 0 & -1/3	
				\end{pmatrix} \\
				T^{-1}AT &= \begin{pmatrix}
					1 & -3 & 0 \\ 0 & 3 & 3 \\ 0 & 0 & -3	
				\end{pmatrix}\begin{pmatrix}
					2 & 3 & 0 \\ 0 & 2 & -1 \\ 0 & 0 & 2
				\end{pmatrix}\begin{pmatrix}
					1 & 1 & 1 \\ 0 & 1/3 & 1/3 \\ 0 & 0 & -1/3	
				\end{pmatrix} = \begin{pmatrix}
					2 & 1 & 0 \\ 0 & 2 & 1 \\ 0 & 0 & 2	
				\end{pmatrix}
			\end{align*}
			Notice that $T^{-1}AT$ is in the form of case (iii) with its diagonal consisting of the repeated eigenvalue, $\lambda = 2$ and its superdiagonal consisting of ones.
		\end{enumerate}
	\end{problem}






\end{document}
