\documentclass[11pt]{article}
\usepackage[margin = 1in]{geometry}
\usepackage{amsmath}
\usepackage{amssymb}
\usepackage{amsthm}
\usepackage{graphicx}
\usepackage{subfig}
\usepackage{enumitem}
\usepackage{url}
\usepackage[parfill]{parskip}
\newcommand{\skipline}{\vspace{\baselineskip}}
\newenvironment{problem}[1]{\textbf{Problem #1: }}{\newpage}


\begin{document}
	
	\begin{center}
		\textbf{Midterm Part B} \\
		\textbf{Ordinary Differential Equations} \\
		\textbf{Math 537} \\
		\textbf{Stephen Giang RedID: 823184070} \\
		\skipline \skipline
	\end{center}

	\begin{problem}{1 (9:00am - 9:05am), (9:37am - 9:50am)}
		Consider the following two systems, including a single first order ODE:
		\[\frac{dz}{dt} = f(z) = \alpha z^2 + \beta z \tag{1.1}\]
		and a system of two first-order ODEs:
		\begin{align*}
			\frac{dx}{dt} &= ax, \\
			\frac{dy}{dt} &= by. \tag{1.2}
		\end{align*}
		\begin{enumerate}[label = (\alph*)]
			\item  The trivial critical point $z = 0$ in Eq. (1.1) may be a source,
			a sink, or a saddle point. Select a pair of $(\alpha, \beta)$ that produces each of
			the three types of critical points. [Hint: Either $\alpha$ or $\beta$ can be zero for
			simplicity.]
			\\ \\
			Let $(\alpha, \beta) = (0, 1)$.  Notice that we get:
			\[\frac{dz}{dt} = z\]
			From $z < 0$, we get $\frac{dz}{dt} < 0$, and from $z > 0$, we get $\frac{dz}{dt} > 0$.  This means we get a \textbf{sink}.
			\\ \\
			Let $(\alpha, \beta) = (0, -1)$.  Notice that we get:
			\[\frac{dz}{dt} = -z\]
			From $z < 0$, we get $\frac{dz}{dt} > 0$, and from $z > 0$, we get $\frac{dz}{dt} < 0$.  This means we get a \textbf{source}.
			\\ \\
			Let $(\alpha, \beta) = (1, 0)$.  Notice that we get:
			\[\frac{dz}{dt} = z^2\]
			From $z < 0$, we get $\frac{dz}{dt} > 0$, and from $z > 0$, we get $\frac{dz}{dt} > 0$.  This means we get a \textbf{saddle}. 
			\newpage
			\item Within the planar system in Eq. (1.2), discuss the characteristics of the trivial critical point $(x, y) = (0, 0)$ in $a - b$ space.

			\skipline
			Notice we get an upper triangular matrix such that $\lambda_1 = a$, $\lambda_2 = b$. \\ \\
			
			When ($a > 0$, $b > 0$), we get two positive eigenvalues leading to a \textbf{source} \\ \\
			When ($a < 0$, $b > 0$), we get two positive eigenvalues leading to a \textbf{saddle} \\ \\
			When ($a < 0$, $b < 0$), we get two positive eigenvalues leading to a \textbf{sink} \\ \\
			When ($a > 0$, $b < 0$), we get two positive eigenvalues leading to a \textbf{saddle} 
		\end{enumerate}
	\end{problem}

	\begin{problem}{3 (9:05am - 9:37am)}
		Consider the following system:
		\[X' = AX, \tag{3.1}\]
		where
		\[A = \begin{pmatrix}
			-0.1 & 1.1 \\
			1.1 & -0.1
		\end{pmatrix} \text{ and } X = \begin{pmatrix}
			x \\ y
		\end{pmatrix}\]
		\begin{enumerate}[label = (\alph*)]
			\item  Solve for eigenvalue(s) and eigenvector(s).
			\\ \\
			Notice we can get the characteristic equation from $A - \lambda I$:
			\begin{align*}
				(\lambda + 0.1)(\lambda + 0.1) - 1.1(1.1) &= 0\\
				\lambda^2 + 0.2\lambda + .01 - 1.21 &= 0 \\
				\lambda^2 + 0.2\lambda - 1.2 &= 0 
			\end{align*}
			Notice the following eigenvalues:
			\begin{align*}
				\lambda_1 &= \frac{-0.2 + \sqrt{.04 - 4(-1.2)}}{2} = \frac{-0.2 + 2.2}{2} = 1 \\
				\lambda_2 &= \frac{-0.2 - \sqrt{.04 - 4(-1.2)}}{2} = \frac{-0.2 - 2.2}{2} = -1.2 \\
			\end{align*}
			Notice the eigenvectors found from $A - \lambda I$ \\
			\begin{align*}
				\begin{pmatrix}
					-0.1 - \lambda_1 & 1.1 \\
					1.1 & -0.1 - \lambda_1
				\end{pmatrix}\begin{pmatrix}
					x \\ y
				\end{pmatrix} &=
				\begin{pmatrix}
				-1.1 & 1.1 \\
				1.1 & -1.1
				\end{pmatrix}\begin{pmatrix}
				x \\ y
				\end{pmatrix} = \begin{pmatrix}
				 0 \\ 0
				\end{pmatrix}, &
				v_1 = \begin{pmatrix}
					x \\ y
				\end{pmatrix} = \begin{pmatrix}
					1 \\ 1
				\end{pmatrix} \\
				\begin{pmatrix}
				-0.1 - \lambda_2 & 1.1 \\
				1.1 & -0.1 - \lambda_2
				\end{pmatrix}\begin{pmatrix}
				x \\ y
				\end{pmatrix} &=
				\begin{pmatrix}
				1.1 & 1.1 \\
				1.1 & 1.1
				\end{pmatrix}\begin{pmatrix}
				x \\ y
				\end{pmatrix} = \begin{pmatrix}
				0 \\ 0
				\end{pmatrix}, &
				v_2 = \begin{pmatrix}
				x \\ y
				\end{pmatrix} = \begin{pmatrix}
				1 \\ -1
				\end{pmatrix} 
			\end{align*}
			\item  Construct $T$ using the results from problem (3a) and calculate $T^{-1}AT$
			\\ \\
			Notice that the eigenvalues were real and different.  So we can construct T from the eigenvectors, such that:
			\[T = \begin{pmatrix}
				1 & 1 \\
				1 & -1
			\end{pmatrix}\]
			Notice $T^{-1}AT$:
			\begin{align*}
				T^{-1}AT &= \begin{pmatrix}
					\frac{-1}{2} & \frac{1}{2} \\
					\frac{1}{2} & \frac{1}{2}
				\end{pmatrix}
				\begin{pmatrix}
					-0.1 & 1.1 \\
					1.1 & -0.1
				\end{pmatrix}\begin{pmatrix}
					1 & 1 \\
					1 & -1
				\end{pmatrix} \\
				&= 
				\begin{pmatrix}
					1 & 0 \\
					0 & -1.2
				\end{pmatrix}
			\end{align*}
			\newpage
			\item Let $X = T Y$ . Show
			\[Y' = (T^{-1}AT)Y,\tag{3.2}\]
			Here $Y$ is a column vector and its transpose is defined as $Y^T = (u, w)$. 
			\\ \\
			Notice the following:
			\[\begin{pmatrix}
				u' \\ w'
			\end{pmatrix} = \begin{pmatrix}
				1 & 0 \\
				0 & -1.2
			\end{pmatrix}\begin{pmatrix}
				u \\ w
			\end{pmatrix} = \begin{pmatrix}
				u \\ -1.2w
			\end{pmatrix}\]
			Because we have that $u' = \lambda_1 u $ and $w' =\lambda_2 w$, we have shown the above statement to be true.
			\\ \\
			\item Solve Eq. (3.2) for Y.
			\\ \\
			We can see the eigenvalues because $T^{-1}AT$ is an upper triangular matrix. So we get that $\lambda_1 = 1$ and $\lambda_2 = -1.2$.  We can also easily see the eigenvectors being $v_1 = \begin{pmatrix}
				1 \\ 0
			\end{pmatrix}, v_2 = \begin{pmatrix}
				0 \\ 1
			\end{pmatrix}$.
			\\ \\
			So we get 
			\[Y = Ae^{t}\begin{pmatrix}
				1 \\ 0
			\end{pmatrix} + Be^{-1.2t}\begin{pmatrix}
				0 \\ 1
			\end{pmatrix}\]
			\item Find the solution $X$ to Eq. (3.1).
			\\ \\
			\[X = TY = \begin{pmatrix}
				1 & 1 \\
				1 & -1
			\end{pmatrix}\begin{pmatrix}
				Ae^{t} & 0 \\
				0 & Be^{-1.2t}
			\end{pmatrix} = \begin{pmatrix}
				Ae^t & Be^{-1.2t} \\
				Ae^{t} & -Be^{-1.2t}
			\end{pmatrix}\]
		\end{enumerate}

\end{problem}

	\begin{problem}{6 (9:50am - 10am)}
		\begin{enumerate}[label = (\alph*)]
			\item Notice the characteristic equation:
			\begin{align*}
				(\lambda - 2)(\lambda - 1) + 1/4 &= 0 \\
				\lambda^2 - 3\lambda + 9/4 &= 0
			\end{align*}
			So we get the eigenvalues:
			\begin{align*}
				\lambda &= \frac{3 \pm \sqrt{9 - 4(9/4)}}{2} = \frac{3}{2}
			\end{align*}
			So we can get the eigenvalues from $A - \lambda I$
			\[v_1 = \begin{pmatrix}
				1 \\ -1/2
			\end{pmatrix}, v_2 = \begin{pmatrix}
				0 \\ 1
			\end{pmatrix}\].
			\item \[T = \begin{pmatrix}
				1 & 0 \\
				-1/2 & 1
			\end{pmatrix},\hspace{2cm} T^{-1}AT = \begin{pmatrix}
				3/2 & 1 \\
				0 &  3/2
			\end{pmatrix}\]
		\end{enumerate}
	\end{problem}

\end{document}
