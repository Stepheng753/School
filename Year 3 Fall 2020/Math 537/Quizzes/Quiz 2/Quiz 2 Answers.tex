\documentclass[11pt]{article}
\usepackage[margin = 1in]{geometry}
\usepackage{amsmath}
\usepackage{amssymb}
\usepackage{amsthm}
\usepackage{graphicx}
\usepackage{subfig}
\usepackage{enumitem}
\usepackage{url}
\usepackage[parfill]{parskip}
\newcommand{\skipline}{\vspace{\baselineskip}}
\newenvironment{problem}[1]{\textbf{Problem #1: }}{\newpage}


\begin{document}
	
	\begin{center}
		\textbf{Quiz 2} \\
		\textbf{Ordinary Differential Equations} \\
		\textbf{Math 537} \\
		\textbf{Stephen Giang} \\
	\end{center}

	\begin{problem}{1}
		Introduce a new time variable $\tau$ to convert the following ODE:
		$$\frac{dy}{dt} = \sigma y$$
		into
		$$\frac{dy}{d\tau} = y$$
		Notice the following:
		\begin{align*}
			\frac{dy}{dt} &= \frac{\frac{dy}{d\tau}}{\frac{dt}{d\tau}} = \sigma y \\
			\frac{dy}{d\tau} &= \frac{dt}{d \tau} \sigma y = y \\
		\end{align*}
		From this, we can conclude that 
		$$\frac{dt}{d \tau} \sigma = 1, \qquad \frac{dt}{d \tau} = \frac{1}{\sigma}$$
		So it is now clear that if we introduce a variable $\tau = t\sigma$, we can convert the original equation into the equation we want.
	\end{problem}

	\begin{problem}{2}
		Consider the following Logistic equation
		$$\frac{dy}{dt} = \alpha y - \beta y^2$$
		Convert the above ODE into the following ODE
		$$\frac{dz}{d \tau} = z - z^2$$
		by introducing a new time variable $\tau$ and a new time-dependent variable $z$.
		Find $\tau$ and $z$. 
		\\ \\
		Notice the following:
		\begin{align*}
			\frac{dy}{dt} &= \alpha y - \beta y^2 = \alpha y \left(1 - \frac{\beta}{\alpha}y\right)
		\end{align*}
		Let $z = \frac{\beta}{\alpha}y$.  This means that $y = \frac{\alpha}{\beta}z, \frac{dy}{dz} = \frac{\alpha}{\beta}$. Let $\frac{d\tau}{dt} = \alpha$
		\begin{align*}
			\frac{dy}{dt} &= \frac{\frac{dy}{dz}}{\frac{dt}{dz}} = \frac{dy}{dz} \frac{dz}{dt} \\
			&= \frac{dy}{dz} \frac{\frac{dz}{d\tau}}{\frac{dt}{d\tau}} = \frac{dy}{dz}\frac{dz}{d\tau}\frac{d\tau}{dt} \\
			&= \frac{\alpha}{\beta}\frac{dz}{d\tau}\alpha = \frac{\alpha^2}{\beta}\frac{dz}{d\tau} \\
			\alpha y \left(1 - \frac{\beta}{\alpha}y\right) &= \frac{\alpha^2}{\beta}z(1- z) = \frac{\alpha^2}{\beta}\frac{dz}{d\tau} \\
			\frac{dz}{d\tau} &= z(1-z) = z - z^2
		\end{align*}
		\newpage
		Now we can solve for $\tau$ and $z$:
			\begin{align*}
				\frac{dz}{d \tau} - z &= -z^2 
			\end{align*}
			Let $\mu(\tau) = \frac{1}{z}$ and $\frac{d\mu}{d\tau} = \frac{-1}{z^2}\frac{dz}{d\tau}$, and multiply both sides of the equation by $\frac{-1}{z^2}$:
			\begin{align*}
				\frac{-1}{z^2}\frac{dz}{d \tau} + \frac{1}{z} &= 1 \\
				\frac{d\mu}{d\tau} + \mu &= 1
			\end{align*}
			This now becomes a separable equation and we get:
			\begin{align*}
				\frac{d\mu}{1 - \mu} &= d\tau \\
				-\ln(1 - \mu) &= \tau + C	\\
				\mu &= Ce^{-\tau} + 1 \\
				\frac{1}{z} &= Ce^{-\tau} + 1 \\
				z &= \frac{1}{Ce^{-\tau} + 1}
			\end{align*}
			We can solve for $C$ by finding $z(0) = z_0 = \frac{1}{C + 1} \Longleftrightarrow C = \frac{1}{z_0} - 1$.  Furthermore, we get:
			\begin{align*}
				z = \frac{1}{\left(\frac{1}{z_0} - 1\right)e^{-\tau} + 1}
			\end{align*}
			We also find $\tau$ by integrating our let statement, and get $\tau = \alpha t$
	\end{problem}

	\begin{problem}{3}
		 Consider the improper integral
		 \[\int_{-1}^{1} \frac{1}{x} \, dx \tag{3}\]
		 \begin{enumerate}[label = (\alph*)]
		 	\item  Verify whether the following derivations are correct.
		 	The above has the following two parts: 
		 	$$\int_{-1}^{0} \frac{1}{x}\,dx + \int_{0}^{1} \frac{1}{x}\,dx$$
		 	By introducing $y = -x$ for the first part, we have
		 	$$\int_{1}^{0} \frac{1}{y}\,dy + \int_{0}^{1} \frac{1}{x}\,dx$$
		 	which becomes
		 	$$-\int_{0}^{1} \frac{1}{y}\,dy + \int_{0}^{1} \frac{1}{x}\,dx = 0.$$
		 	\\ \\
		 	We can verify the first step by noticing:
		 	\[\int_{a}^{b} f(x)\,dx = \int_{a}^{c} f(x)\,dx + \int_{c}^{b} f(x)\,dx \tag{by Prop. of Integrals} \]
		 	We can verify the second step by noticing:
		 	\begin{align*}
		 		y &= -x, \qquad x = -y \\
		 		dy &= -dx, \qquad dx = -dy \\
		 		\int_{x = -1}^{x = 0} \frac{1}{x}\,dx &= \int_{-y = -1}^{-y = 0} \frac{1}{-y}(-1)\,dy \tag{by Subsitution}\\
		 		&= \int_{1}^{0} \frac{1}{y}\,dy \tag{Simplify}
		 	\end{align*}
		 	We can verify the third step by noticing:
		 	\begin{align*}
		 		\int_{1}^{0} \frac{1}{y}\,dy &= -\int_{0}^{1} \frac{1}{y}\,dy \tag{by the F.T.C}\\
		 		\lim\limits_{a \rightarrow 0}\left[ -\int_{a}^{1} \frac{1}{y}\,dy + \int_{a}^{1} \frac{1}{x}\right] &= \lim\limits_{a \rightarrow 0} \left[\left.-\ln(y)\right|_a^1 + \left.\ln(x)\right|_a^1\right] \tag{Use Limits for Improper Int}\\
		 		&= \lim\limits_{a \rightarrow 0}\left[ -\ln(1) + \ln(a) + \ln(1) - \ln(a) \right] = 0  \tag{Evaluate}
		 	\end{align*}
		 	\newpage
		 	\item Represent Eq. (3) as follows:
			$$\lim\limits_{\epsilon \rightarrow 0} \int_{-1}^{-\epsilon} \frac{1}{x}\,dx + \lim\limits_{\epsilon \rightarrow 0} \int_{\epsilon}^{1} \frac{1}{x}\,dx$$
			Complete the above integrals.
			\\ \\
			Notice we can represent Eq. (3) as follows:
				\begin{align*}
					\int_{-1}^{1} \frac{1}{x} \, dx = \int_{-1}^{0} \frac{1}{x}\,dx + \int_{0}^{1} \frac{1}{x}\,dx = \lim\limits_{\epsilon \rightarrow 0} \int_{-1}^{-\epsilon} \frac{1}{x}\,dx + \lim\limits_{\epsilon \rightarrow 0} \int_{\epsilon}^{1} \frac{1}{x}\,dx
				\end{align*}
			Notice the following:
				\begin{align*}
					\lim\limits_{\epsilon \rightarrow 0} \int_{-1}^{-\epsilon} \frac{1}{x}\,dx &= \lim\limits_{\epsilon \rightarrow 0} \int_{-y = -1}^{-y = -\epsilon} \frac{1}{-y}(-1)\,dy \tag{Subsitute $x = -y$}\\
					&= \lim\limits_{\epsilon \rightarrow 0} \int_{1}^{\epsilon} \frac{1}{y}\,dy \\
					&= \lim\limits_{\epsilon \rightarrow 0} -\int_{\epsilon}^{1} \frac{1}{y}\,dy \\
					\lim\limits_{\epsilon \rightarrow 0} \int_{-1}^{-\epsilon} \frac{1}{x}\,dx + \lim\limits_{\epsilon \rightarrow 0} \int_{\epsilon}^{1} \frac{1}{x}\,dx &= \lim\limits_{\epsilon \rightarrow 0} -\int_{\epsilon}^{1} \frac{1}{y}\,dy + \lim\limits_{\epsilon \rightarrow 0} \int_{\epsilon}^{1} \frac{1}{x}\,dx  \\
					&= \lim\limits_{\epsilon \rightarrow 0} \left[- \ln(1) + \ln(\epsilon) + \ln(1) - \ln(\epsilon)\right] = 0
				\end{align*}
			\item Represent Eq. (3) as follows:
			$$\lim\limits_{\epsilon \rightarrow 0} \int_{-1}^{-2\epsilon} \frac{1}{x}\,dx + \lim\limits_{\epsilon \rightarrow 0} \int_{\epsilon}^{1} \frac{1}{x}\,dx$$
			Complete the above integrals.
			\\ \\
			Notice we can represent Eq. (3) as follows:
				\begin{align*}
					\int_{-1}^{1} \frac{1}{x} \, dx = \int_{-1}^{0} \frac{1}{x}\,dx + \int_{0}^{1} \frac{1}{x}\,dx = \lim\limits_{\epsilon \rightarrow 0} \int_{-1}^{-2\epsilon} \frac{1}{x}\,dx + \lim\limits_{\epsilon \rightarrow 0} \int_{\epsilon}^{1} \frac{1}{x}\,dx
				\end{align*}
			Following the same steps as (3b), we can skip to the following step:
				\begin{align*}
					\lim\limits_{\epsilon \rightarrow 0} \int_{-1}^{-2\epsilon} \frac{1}{x}\,dx + \lim\limits_{\epsilon \rightarrow 0} \int_{\epsilon}^{1} \frac{1}{x}\,dx &= \lim\limits_{\epsilon \rightarrow 0} -\int_{2\epsilon}^{1} \frac{1}{y}\,dy + \lim\limits_{\epsilon \rightarrow 0} \int_{\epsilon}^{1} \frac{1}{x}\,dx  \\
					&= \lim\limits_{\epsilon \rightarrow 0} \left[ - \ln(1) + \ln(2\epsilon) + \ln(1) - \ln(\epsilon) \right] = \ln(2)
				\end{align*}
			\item Compare the answers in (b) and (c) to provide justifications to your analysis in (a).
			\\ \\
			I was able to get the same result in (b) and (a).  The answers both equal 0, which makes sense graphically because the original equation is symmetric about the origin $(0,0)$, meaning the integral on each side of its vertical asymptote would cancel each other.  In (c), because the middle bounds were not the same, one being $\epsilon$ and the other being $2\epsilon$, this results in a different answer.  Because $\epsilon$ only approaches 0 and not actually equals 0, we cannot say that $2\epsilon = \epsilon$, so (c) cannot be the same result as (b) and (a).
		 \end{enumerate}
	\end{problem}


\end{document}
