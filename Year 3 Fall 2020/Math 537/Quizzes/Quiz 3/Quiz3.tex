\documentclass[11pt]{article}
\usepackage[margin = 1in]{geometry}
\usepackage{amsmath}
\usepackage{amssymb}
\usepackage{amsthm}
\usepackage{graphicx}
\usepackage{subfig}
\usepackage{enumitem}
\usepackage{url}
\usepackage[parfill]{parskip}
\DeclareMathOperator{\sech}{sech}
\newcommand{\skipline}{\vspace{\baselineskip}}
\newenvironment{problem}[1]{\textbf{Problem #1: }}{\newpage}


\begin{document}
	
	\begin{center}
		\textbf{Quiz 3} \\
		\textbf{Ordinary Differential Equations} \\
		\textbf{Math 537} \\
		\textbf{Stephen Giang RedID: 823184070} \\
		\skipline \skipline
	\end{center}

	\textbf{Goal:} Based on Eq. (1.3) of the Mid-Term Part A, we will derive a second order ODE with its solution as a hyperbolic secant squared function ($\sech^2$). The second-order ODE is mathematically identical to the Korteweg-de Vries (KdV) equation in a traveling-wave coordinate. The solution is known as a solitary wave or logistic distribution. It also represents the solution ($I$) of the simplified SIR model under the condition of ”weak outbreak”.
	\skipline 
	
	\begin{problem}{1}
		In the Mid-term Part A, we have completed the following:
		\begin{itemize}[label = (*)]
			\item Consider the following logistic equation: 
			\[\frac{df}{dt} = f(1-f). \tag{MT - 1.2}\]
			Introduce a new dependent variable ($g$) to transform Eq. (MT-1.2) into
			the following ODE:
			\[\frac{dg}{dt} = \frac{1}{4} - g^2. \tag{MT - 1.3}\]
			\item Express the solutions of Eqs. (MT-1.2) and (MT-1.3) in terms of the
			sigmoid and hyperbolic tangent functions, respectively
		\end{itemize}
		Here, by defining $Z = dg/dt$, please derive the following ODE from Eq. (MT-1.3)
		\[\frac{d^2 Z}{dt^2} - Z + 6Z^2 = 0, \tag{1.1}\]
		which can be written using a new time variable ($\tau$) as follows:
		\[\frac{d^2 Z}{d\tau^2} - \frac{Z}{2} + 3Z^2 = 0. \tag{1.2}\]
		Eq. (1.2) is mathematically identical to the KdV equation in the travelingwave coordinate (Shen 2020, IJBC, in press).
		\newpage
		Notice the following, when we let $Z = dg/dt$:
		\begin{align*}
			Z &= \frac{dg}{dt} = \frac{1}{4} - g^2 \\
			\frac{dZ}{dt} &= \frac{d^2 g}{dt^2} = -2g\frac{dg}{dt} = -2g\left(\frac{1}{4} - g^2 \right) = -\frac{g}{2} + 2g^3 \\
			\frac{d^2Z}{dt^2} &= \frac{d^3 g}{dt^3} = \left(-\frac{1}{2} + 6g^2\right)\frac{dg}{dt} = \left(-\frac{1}{2} + 6g^2\right)\left(\frac{1}{4} - g^2 \right) = -\frac{1}{8} + 2g^2 - 6g^4
		\end{align*}
		Now Notice the following:
		\begin{align*}
			\frac{d^2 Z}{dt^2} - Z + 6Z^2 &= \left(-\frac{1}{8} + 2g^2 - 6g^4\right) - \left(\frac{1}{4} - g^2\right) + 6\left(\frac{1}{4} - g^2\right)^2 \\
			&= \left(-\frac{2}{16} + 2g^2 - 6g^4\right) - \left(\frac{4}{16} - g^2\right) + 6\left(\frac{1}{16} - \frac{1}{2}g^2 + g^4\right)^2 \\
			&= -\frac{2}{16} + 2g^2 - 6g^4 - \frac{4}{16} + g^2 + \frac{6}{16} - 3g^2 + 6g^4 \\
			&= \left(-\frac{2}{16} - \frac{4}{16} + \frac{6}{16} \right) + \left(2g^2 + g^2 - 3g^2\right) + (-6g^4 + 6g^4) \\
			&= 0 \\
		\end{align*}
		Notice the following with a new time variable $\tau$, with $\frac{dg}{d\tau} = \frac{dg}{dt}\frac{dt}{d\tau}$:
		\begin{align*}
			Z = \frac{dg}{dt} &= \frac{1}{4} - g^2 \\
			\frac{dZ}{d\tau} = \frac{d^2g}{d\tau\,dt} &= -2g\frac{dg}{d\tau} = -2g\frac{dg}{dt}\frac{dt}{d\tau} = -2g\left(\frac{1}{4} - g^2 \right)\frac{dt}{d\tau} = \left(-\frac{g}{2} + 2g^3\right)\frac{dt}{d\tau} \\
			\frac{d^2Z}{d\tau^2} = \frac{d^3 g}{d\tau^2\,dt} &= \left(-\frac{1}{2} + 6g^2\right)\frac{dg}{d\tau}\frac{dt}{d\tau} = \left(-\frac{1}{2} + 6g^2\right)\frac{dg}{dt}\frac{dt}{d\tau}\frac{dt}{d\tau} = \left(-\frac{1}{2} + 6g^2\right)\left(\frac{1}{4} - g^2 \right)\left(\frac{dt}{d\tau}\right)^2 \\
			&= \left(-\frac{1}{8} + 2g^2 - 6g^4\right)\left(\frac{dt}{d\tau}\right)^2
		\end{align*}
		Notice that we have the following:
		\begin{align*}
			\frac{d^2 Z}{dt^2} &= Z - 6Z^2 & \left(-\frac{1}{8} + 2g^2 - 6g^4\right)\left(\frac{dt}{d\tau}\right)^2 &= \frac{1}{2}\left(-\frac{1}{8} + 2g^2 - 6g^4\right)\\
			\frac{d^2 Z}{d\tau^2} &= \frac{Z}{2} - 3Z^2 & \left(\frac{dt}{d\tau}\right)^2 &= \frac{1}{2}\\
			&= \frac{1}{2}\left(Z - 6Z^2\right) & \frac{dt}{d\tau} &= \frac{\pm 1}{\sqrt{2}}\\
			&= \frac{1}{2}\frac{d^2 Z}{dt^2} & \int dt &= \int \frac{\pm 1}{\sqrt{2}}d\tau
		\end{align*}
		So we get that $t + C = \frac{\pm 1}{\sqrt{2}}\tau$, such that $\tau = \pm \sqrt{2}t + C$
	\end{problem}
	


\end{document}
