\documentclass[11pt]{article}
\usepackage[margin = 1in]{geometry}
\usepackage{amsmath}
\usepackage{amssymb}
\usepackage{amsthm}
\usepackage{graphicx}
\usepackage{enumitem}
\usepackage{url}
\usepackage[parfill]{parskip}
\usepackage{listings}
\usepackage{caption}
\usepackage{subcaption}
\usepackage[utf8]{inputenc}
\usepackage{xcolor}
\definecolor{codegreen}{rgb}{0,0.6,0}
\definecolor{codegray}{rgb}{0.5,0.5,0.5}
\definecolor{codepurple}{rgb}{0.58,0,0.82}
\definecolor{backcolour}{rgb}{0.95,0.95,0.92}
\lstdefinestyle{mystyle}{
	backgroundcolor=\color{backcolour},   
	commentstyle=\color{codegreen},
	keywordstyle=\color{magenta},
	numberstyle=\tiny\color{codegray},
	stringstyle=\color{codepurple},
	basicstyle=\ttfamily\footnotesize,
	breakatwhitespace=false,         
	breaklines=true,                 
	captionpos=b,                    
	keepspaces=true,                 
	numbers=left,                    
	numbersep=5pt,                  
	showspaces=false,                
	showstringspaces=false,
	showtabs=false,                  
	tabsize=2
}
\lstset{style=mystyle}
\newcommand{\skipline}{\vspace{\baselineskip}}
\newcommand{\spacer}{\noalign{\medskip}}
\newcommand{~}{\sim}
\newcommand{\qrarrow}{\quad \rightarrow \quad}
\newcommand{\qqrarrow}{\qquad \rightarrow \qquad}
\newcommand{\partiald}[2]{\frac{\partial #1}{\partial #2}}
\newenvironment{problem}[1]{\textbf{Problem #1: }}{\newpage}

\begin{document}
	\begin{center}
		\textbf{Homework 2} \\
		\textbf{Partial Differential Equations} \\
		\textbf{Math 531} \\
		\textbf{Stephen Giang RedID: 823184070} \\
		\skipline \skipline
	\end{center}

	\begin{problem}{1}
		Consider the partial differential equation for heat in a one-dimensional rod with temperature $u(x,t)$:
		\[\frac{\partial u}{\partial t} = k \frac{\partial^2 u}{\partial x^2}\]
		Assume the initial condition: 
		\[u(x,0) = f(x),\]
		and boundary conditions:
		\[u(0,t) = 16 \qquad u(3,t) = 0\]
		Determine the equilibrium temperature distribution:
		\\ \\
		If we assume that $u(x,t)$ does not dependent on $t$, then we get the following:
		\[\frac{\partial u}{\partial t} = 0 =k\frac{\partial^2 u}{\partial x^2} \]
		From this we get the following:
		\[\frac{\partial^2 u}{\partial x^2} = 0 \qqrarrow \frac{\partial u}{\partial x} = C_1 \qqrarrow u(x) = C_1x + C_2\]
		Using the boundary conditions, we get the following:
		\[u(x = 0) = C_2 = 16 \qqrarrow u(x = 3) = 3C_1 + 16 = 0 \qqrarrow C_1 = -\frac{16}{3}\] 
		Thus we get the equilibrium temperature distribution:
		\[\boldsymbol{u(x) = -\frac{16}{3}x + 16}\]
	\end{problem}

	\begin{problem}{2}
		Consider the partial differential equation for heat in a one-dimensional rod with temperature $u(x,t)$:
		\[\frac{\partial u}{\partial t} = k \frac{\partial^2 u}{\partial x^2}\]
		Assume the initial condition: 
		\[u(x,0) = f(x),\]
		and boundary conditions:
		\[\partiald{u}{x}(0,t) = 0 \qquad u(2,t) = 14\]
		Determine the equilibrium temperature distribution:
		\\ \\
		If we assume that $u(x,t)$ does not dependent on $t$, then we get the following:
		\[\frac{\partial u}{\partial t} = 0 =k\frac{\partial^2 u}{\partial x^2} \]
		From this we get the following:
		\[\frac{\partial^2 u}{\partial x^2} = 0 \qqrarrow \frac{\partial u}{\partial x} = C_1 \qqrarrow u(x) = C_1x + C_2\]
		Using the boundary conditions, we get the following:
		\[\partiald{u}{x}(x = 0) = C_1 = 0 \qqrarrow u(x = 2) = C_2 = 14\] 
		Thus we get the equilibrium temperature distribution:
		\[\boldsymbol{u(x) = 14}\]
	\end{problem}

	\begin{problem}{3}
		Consider the partial differential equation for heat in a one-dimensional rod with temperature $u(x,t)$:
		\[\frac{\partial u}{\partial t} = k \frac{\partial^2 u}{\partial x^2} + Q\]
		with $Q/k = 3$. \\ \\
		Assume the initial condition: 
		\[u(x,0) = f(x),\]
		and boundary conditions:
		\[u(0,t) = 14 \qquad u(1,t) = 3\]
		Determine the equilibrium temperature distribution:
		\\ \\
		If we assume that $u(x,t)$ does not dependent on $t$, then we get the following:
		\[\frac{\partial u}{\partial t} = 0 =k\left(\frac{\partial^2 u}{\partial x^2} + 3\right) \]
		From this we get the following:
		\[\frac{\partial^2 u}{\partial x^2} = -3 \qqrarrow \frac{\partial u}{\partial x} = -3x + C_1 \qqrarrow u(x) = -\frac{3}{2}x^2 + C_1x + C_2\]
		Using the boundary conditions, we get the following:
		\[u(x = 0) = C_2 = 14 \qqrarrow u(x = 1) = -\frac{3}{2} + C_1 + 14 = 3 \qqrarrow C_1 = -\frac{19}{2}\] 
		Thus we get the equilibrium temperature distribution:
		\[\boldsymbol{u(x) = -\frac{3}{2}x^2 + -\frac{19}{2}x + 14}\]
	\end{problem}

	\begin{problem}{4}
		Determine the equilibrium temperature distribution for a one-dimensional rod composed of two different materials in perfect thermal contact at $x = 1$.  For $0 < x < 1$, there is one material $(c\rho = 1, K_0 = 0.6)$ with a constant source $(Q = 2.5)$, whereas for the other $1 < x < 2$ there are no sources $(Q = 0, c\rho = 1.8, K_0 = 1.9)$ with $u(0) = 0$ and $u(2) = 0$. (Hint: See Exercise 1.3.2.)
		\\ \\
		Firstly, we notice the following equalities from the definition of Thermal Diffusivity and Perfect Thermal Contact:
		\begin{align*}
			k &= \frac{K_0}{c\rho}, \qquad
			u(x_0-,t) = u(x_0+,t), \qquad  
			K_0(x_0-)\partiald{u}{x}(x_0-,t) = K_0(x_0+)\partiald{u}{x}(x_0+,t) 
		\end{align*}
		\skipline \skipline
		Determine the equilibrium temperature distribution for each segment of the rod:
		\\ \\
		If we assume that $u(x,t)$ does not dependent on $t$ on $0 < x < 1$, then we get the following:
		\[\frac{\partial u}{\partial t} = 0 =k\left(\frac{\partial^2 u}{\partial x^2} + \frac{Q}{k}\right) = \frac{K_0}{c\rho}\left(\frac{\partial^2 u}{\partial x^2} + \frac{c\rho Q}{K_0}\right) = 0.6\left(\frac{\partial^2 u}{\partial x^2} + \frac{2.5}{0.6} \right)\]
		From this we get the following:
		\[\frac{\partial^2 u}{\partial x^2} = -\frac{2.5}{0.6} \qqrarrow \frac{\partial u}{\partial x} = -\frac{2.5}{0.6}x + C_1 \qqrarrow u(x) = -\frac{2.5}{1.2}x^2 + C_1x + C_2\]
		Using the boundary conditions, we get the following: $u(0) = 0 = C_2$
		\\ \\
		Thus we get the general solution for $0 < x < 1$:
		\[u(x) = -\frac{2.5}{1.2}x^2 + C_1x\]
		\\ \\
		If we assume that $u(x,t)$ does not dependent on $t$ on $1 < x < 2$, then we get the following:
		\[\frac{\partial u}{\partial t} = 0 =k\frac{\partial^2 u}{\partial x^2} = \frac{K_0}{c\rho}\,\frac{\partial^2 u}{\partial x^2} = \frac{1.9}{1.8}\,\frac{\partial^2 u}{\partial x^2}\]
		From this we get the following:
		\[\frac{\partial^2 u}{\partial x^2} = 0 \qqrarrow \frac{\partial u}{\partial x} = D_1 \qqrarrow u(x) = D_1x + D_2\]
		Using the boundary conditions, we get the following:
		\[u(2) = 0 = 2D_1 + D_2 \qrarrow D_2 = -2D_1\]
		Thus we get the general solution for $1 < x < 2$:
		\[u(x) = D_1(x-2)\]
		\newpage
		Because of the Perfect Thermal Contact at $x = 1$, we get the following:
		\[u(1^-) = -\frac{2.5}{1.2} + C_1 = -D_1 = u(1^+)\]
		\[K_0(1^-)\partiald{u}{x}(1^-,t) = 0.6\left(-\frac{2.5}{0.6} + C_1\right) = 1.9D_1 =  K_0(1^+)\partiald{u}{x}(1^+,t)\]
		Now we can solve for the missing constants:
		\begin{align*}
			C_1 &= \frac{2.5}{1.2} - D_1 \\
			0.6\left(-\frac{2.5}{0.6} + \frac{2.5}{1.2} - D_1\right) &= 1.9D_1 \\
			-\frac{2.5}{2} - 0.6D_1 &= 1.9D_1 \\
			-\frac{1}{2} &= D_1 \\
			C_1 &= \frac{3.1}{1.2}
		\end{align*}
		Thus we get the following:
		\[\textbf{For } \boldsymbol{0 \leq x \leq 1 \quad u(x) = -\frac{2.5}{1.2}x^2 + \frac{3.1}{1.2}x}\]
		\[\textbf{For } \boldsymbol{1 \leq x \leq 2 \quad u(x) = -\frac{1}{2}(x-2)}\]
		
	\end{problem}

	\begin{problem}{5}
		Consider the partial differential equation for heat in a one-dimensional rod with temperature $u(x,t)$:
		\[\frac{\partial u}{\partial t} = \frac{\partial^2 u}{\partial x^2} + 2\]
		Assume the initial condition: 
		\[u(x,0) = 4e^{-x}\sin\left(\frac{\pi x}{3}\right),\]
		and boundary conditions:
		\[\frac{\partial u}{\partial x}(0,t) = 3 \qquad \frac{\partial u}{\partial x}(3,t) = \beta\]
		For what values of $\beta$ are there solutions to this heat equation? Determine the equilibrium temperature distribution:
		\\ \\
		If we assume that $u(x,t)$ does not dependent on $t$ on $0 < x < 1$, then we get the following:
		\[\frac{\partial u}{\partial t} = 0 = \frac{\partial^2 u}{\partial x^2} + 2\]
		From this we get the following:
		\[\frac{\partial^2 u}{\partial x^2} = -2 \qqrarrow \frac{\partial u}{\partial x} = -2x + C_1 \qqrarrow u(x) = -x^2 + C_1x + C_2\]
		Using the boundary conditions, we get the following:
		\[\frac{\partial u}{\partial x}(0,t) = 3 = C_1 \qqrarrow \frac{\partial u}{\partial x}(3,t) = \boldsymbol{\beta = -3}\]
		From this, we get that as $t \rightarrow \infty$, the thermal energy becomes:
		\[u(x) = -x^2 + 3x + C_2\]
		From here we get that the initial temperature is equal to its final equilibrium temperature:
		\[c\rho\int_{0}^{3}4e^{-x}\sin\left(\frac{\pi x}{3}\right) = -c\rho\,\left(\dfrac{12{e}^{-x}\left(3\sin\left(\frac{{\pi}x}{3}\right)+{\pi}\cos\left(\frac{{\pi}x}{3}\right)\right)}{{\pi}^2+9}\right) \bigg|_0^3 = c\rho\,\left(\dfrac{12{\pi}{e}^{-3}\left({e}^3+1\right)}{{\pi}^2+9}\right)\]
		\[c\rho\int_0^3 -x^2 + 3x + C_2 = c\rho\left(-\frac{1}{3}x^3 + \frac{3}{2}x^2 + C_2x\right)\bigg|_0^3 = c\rho\left(\frac{9}{2} + 3C_2\right) \]
		\[c\rho\,\left(\dfrac{12{\pi}{e}^{-3}\left({e}^3+1\right)}{{\pi}^2+9}\right) =  c\rho\left(\frac{9}{2} + 3C_2\right) \qqrarrow C_2 = \dfrac{4{\pi}{e}^{-3}\left({e}^3+1\right)}{{\pi}^2+9} - \frac{3}{2}\]
		Thus we get the following:
		\[\boldsymbol{u(x) = -x^2 + 3x + \dfrac{4{\pi}{e}^{-3}\left({e}^3+1\right)}{{\pi}^2+9} - \frac{3}{2}}\]
		In your homework assignment write a brief paragraph explaining what is occurring physically to allow the unique equilibrium solution.
		\\ \\
		This unique equilibrium solution is an insulated rod problem. We know this rod is insulated from its boundary conditions.  $Q = 2$ means that from the interval of $x = [0,3]$, there are 6 units of heat energy going into the rod.  Because the change of heat energy goes from 3 to -3 means that 6 units of heat energy is leaving the rod.  Thus the heat energy in equals the heat energy out.
	\end{problem}

	\begin{problem}{6}
		Consider the partial differential equation for heat in a one-dimensional rod with temperature $u(x,t)$:
		\[\frac{\partial u}{\partial t} = \frac{\partial^2 u}{\partial x^2}\]
		Assume the initial condition: 
		\[u(x,0) = 6xe^{-x/1},\]
		and boundary conditions:
		\[\frac{\partial u}{\partial x}(0,t) = 1 \qquad \frac{\partial u}{\partial x}(1,t) = \beta\]
		For what values of $\beta$ are there solutions to this heat equation? Determine the equilibrium temperature distribution:
		\\ \\
		If we assume that $u(x,t)$ does not dependent on $t$, then we get the following:
		\[\frac{\partial u}{\partial t} = 0 =\frac{\partial^2 u}{\partial x^2} \]
		From this we get the following:
		\[\frac{\partial^2 u}{\partial x^2} = 0 \qqrarrow \frac{\partial u}{\partial x} = C_1 \qqrarrow u(x) = C_1x + C_2\]
		Using the boundary conditions, we get the following:
		\[\frac{\partial u}{\partial x}(0,t) = 1 =  C_1 \qqrarrow \frac{\partial u}{\partial x}(1,t) = \boldsymbol{\beta = C_1 = 1}\]
		From this, we get that as $t \rightarrow \infty$, the thermal energy becomes:
		\[u(x) = x + C_2\]
	    From here we get that the initial temperature is equal to its final equilibrium temperature:
	    \[c\rho\int_{0}^{1} 6xe^{-x} = c\rho\left(-6e^{-x}(x+1)\right)\bigg|_0^1 = c\rho\,(6-12e^{-1})\]
	    \[c\rho\int_{0}^{1} x + C_2 = c\rho\left(\frac{1}{2}x^2 + C_2x\right)\bigg|_0^1 = c\rho\,\left(\frac{1}{2} + C_2\right)\]
	    \[c\rho\,(6-12e^{-1}) = c\rho\,\left(\frac{1}{2} + C_2\right) \qqrarrow C_2 = \frac{11}{2}-\frac{12}{e}\]
		Thus we get the following:
		\[\boldsymbol{u(x) = x + \frac{11}{2}-\frac{12}{e}}\]
		In your homework assignment write a brief paragraph explaining what is occurring physically to allow the unique equilibrium solution.
		\\ \\
		We know this rod is insulated from its boundary conditions.  $Q = 0$ means that there is no units of heat energy going into the rod.  Because the change of heat energy goes from 1 to 1 means that no units of heat energy is leaving the rod.  Thus the heat energy in equals the heat energy out.
	\end{problem}

	\begin{problem}{7}
		Consider the partial differential equation for heat in a one-dimensional rod with temperature $u(x,t)$:
		\[\frac{\partial u}{\partial t} = \frac{\partial^2 u}{\partial x^2} + 1.1x - \beta \]
		Assume the initial condition: 
		\[u(x,0) = 0.2x^2\sin\left(\frac{\pi x}{2}\right)\]
		and boundary conditions:
		\[\frac{\partial u}{\partial x}(0,t) = 0 \qquad \frac{\partial u}{\partial x}(2,t) = 0\]
		For what values of $\beta$ are there solutions to this heat equation? \\
		Determine the equilibrium temperature distribution:
		\\ \\
		If we assume that $u(x,t)$ does not dependent on $t$, then we get the following:
		\[\frac{\partial u}{\partial t} = 0 =\frac{\partial^2 u}{\partial x^2} + 1.1x - \beta \]
		From this we get the following:
		\[\frac{\partial^2 u}{\partial x^2} = -1.1x + \beta \qqrarrow \frac{\partial u}{\partial x} = -\frac{1.1}{2}x^2 + \beta x + C_1 \qqrarrow u(x) = -\frac{1.1}{6}x^3 + \frac{\beta}{2} x^2 + C_1x + C_2\]
		Using the boundary conditions, we get the following:
		\[\frac{\partial u}{\partial x}(0,t) = 0 =  C_1 \qqrarrow \frac{\partial u}{\partial x}(2,t) = 0 = -2.2 + 2\beta \qrarrow \boldsymbol{\beta = 1.1}\]
		From this, we get that as $t \rightarrow \infty$, the thermal energy becomes:
		\[u(x) = -\frac{1.1}{6}x^3 + \frac{1.1}{2} x^2 + C_2\]
		From here we get that the initial temperature is equal to its final equilibrium temperature:
		\[c\rho \int_{0}^{2}0.2x^2\sin\left(\frac{\pi x}{2}\right)\,dx = c\rho\,\left( \dfrac{8{\pi}x\sin\left(\frac{{\pi}x}{2}\right)+\left(16-2{\pi}^2x^2\right)\cos\left(\frac{{\pi}x}{2}\right)}{5{\pi}^3}\right) \bigg|_0^2 = c\rho\,\left(\dfrac{8{\pi}^2-32}{5{\pi}^3}\right)  \]
		\[c\rho \int_{0}^{2} -\frac{1.1}{6}x^3 + \frac{1.1}{2} x^2 + C_2\,dx = c\rho\left( -\frac{1.1}{24}x^4 + \frac{1.1}{6}x^3 + C_2x\right)\bigg|_0^2 = c\rho\,\left(\frac{2.2}{3} + 2C_2\right) \]
		\[c\rho\,\left(\dfrac{8{\pi}^2-32}{5{\pi}^3}\right) = c\rho\,\left(\frac{2.2}{3} + 2C_2\right) \qqrarrow C_2 = \dfrac{4{\pi}^2-16}{5{\pi}^3} - \frac{1.1}{3}\]
		Thus we get the following:
		\[\boldsymbol{u(x) = -\frac{1.1}{6}x^3 + \frac{1.1}{2} x^2 + \dfrac{4{\pi}^2-16}{5{\pi}^3} - \frac{1.1}{3}}\]
		In your homework assignment write a brief paragraph explaining what is occurring physically to allow the unique equilibrium solution.
		\\ \\
		We know this rod is insulated from its boundary conditions.  $Q = 0$ means that there is no units of heat energy going into the rod.  Because the change of heat energy goes from 0 to 0 means that no units of heat energy is leaving the rod.  Thus the heat energy in equals the heat energy out.  
	\end{problem}
	
	\begin{problem}{8}
		Consider a thin one-dimensional rod without sources of thermal energy whose lateral surface area is not insulated.
		\begin{enumerate}[label = (\alph*)]
			\item Assume that the heat energy flowing out of the lateral sides per unit surface area per unit time is $w(x,t)$.  Derive the partial differential equation for the temperature $u(x,t)$.
			\\ \\
			With an insulated system, we have the following:
			\[c\rho\partiald{u}{t} = \partiald{}{x}\left(K_0\partiald{u}{x}\right)\]
			Because we are given that the lateral surface is not insulated: we can see that we lose heat energy in a part of the rod.  We take an arbitrary slice of the rod and examine the rate of change of heat energy at that slice.  We can let it equal to the heat energy flowing out of the lateral sides per unit time. 
			\[c\rho\partiald{u}{t}A\,dx = -w(x,t)A\]
			From here we can see that $A = P\,dx$, thus giving us the following:
			\[c\rho\partiald{u}{t} = -\frac{P}{A}w(x,t)\]
			Now we can take our insulated PDE and include into it the equation that denotes the portion of where heat energy is lost.  Thus we get the following:
			\[\boldsymbol{c\rho\partiald{u}{t} = \partiald{}{x}\left(K_0\partiald{u}{x}\right)-\frac{P}{A}w(x,t)}\] 
			\item Assume that $w(x,t)$ is proportional to the temperature difference between the rod $u(x,t)$ and a known outside temperature $\gamma(x,t)$.  Derive that
			\[c\rho\frac{\partial u}{\partial t} = \frac{\partial}{\partial x}\left(K_0\frac{\partial u}{\partial x}\right) - \frac{P}{A}\left[u(x,t) - \gamma(x,t)\right]h(x), \tag{1.2.15}\] 
			where $h(x)$ is a positive $x$-dependent proportionality, $P$ is the lateral perimeter, and A is the cross-sectional area.
			\\ \\
			Notice that we get the following from our assumption that $w(x,t)$ is proportional to the temperature difference between the rod $u(x,t)$ and a known outside temperature $\gamma(x,t)$ with $h(x)$ being a positive $x$-dependent proportionality:
			\[w(x,t) = \left[u(x,t) - \gamma(x,t)\right]h(x)\]
			Thus, we resubstitute this into our PDE from part (a), we get:
			 \[\boldsymbol{c\rho\frac{\partial u}{\partial t} = \frac{\partial}{\partial x}\left(K_0\frac{\partial u}{\partial x}\right) - \frac{P}{A}\left[u(x,t) - \gamma(x,t)\right]h(x)}\] 
			\newpage
			\item Compare (1.2.15) to the equation for a one-dimensional rod whose lateral surfaces are insulated, but with heat sources.
			\\ \\
			Notice the equation for a one-dimensional rod whose lateral surfaces are insulated with heat sources:
			\[c\rho\partiald{u}{t} = \partiald{}{x}\left(K_0\partiald{u}{x}\right) + Q(x)\]
			\textbf{\boldmath This is equation is compared to equation (1.2.15) in the sense that $Q$ is a heat source flowing energy into the rod, and $\frac{P}{A}\left[u(x,t) - \gamma(x,t)\right]h(x)$ is a heat sink flowing energy out of the rod.}
			\item Specialize (1.2.15) to a rod of circular cross section with constant thermal properties and $0^\circ$ outside temperature.
			\\ \\
			Let $r$ be the radius of the circular cross section:
			\[\boldsymbol{c\rho\frac{\partial u}{\partial t} = \frac{\partial}{\partial x}\left(K_0\frac{\partial u}{\partial x}\right) - \frac{2}{r}u(x,t)h(x)}\]
			\item Consider the assumptions in part (d).  Suppose that the temperature in the rod is uniform [i.e $u(x,t) = u(t)$].  Determine $u(t)$ if initially $u(0) = u_0$.
			\\ \\
			We can write the equation from part (d) as follows:
			\[\partiald{u}{t} = k\partiald{^2u}{x^2} -  \frac{2}{r}u(t)h(x)\] 
			Because $u(x,t) = u(t)$, then we get the following:
			\[\partiald{^2u}{x^2} = 0\]
			such that we get:
			\[\frac{du}{dt} = -\frac{2h(x)}{r}u \qqrarrow \int \frac{du}{u} = \int -\frac{2h(x)}{r}\,dt\]
			From this, we get the following:
			\[\boldsymbol{u(t) = u_0e^{-\frac{2h(x)}{r}t}}\]
		\end{enumerate}
	\end{problem}

	\begin{problem}{9}
		Suppose the concentration $u(x,t)$ of a chemical satisfies Fick's law (1.2.13), and the initial concentration is given $u(x,0) = f(x)$.  Consider a region $0 < x < L$ in which the flow is specified at both ends $-k \frac{\partial u}{\partial x}(0,t) = \alpha$ and $-k \frac{\partial u}{\partial x}(L,t) = \beta$.  Assume that $\alpha$ and $\beta$ are constants.
		\begin{enumerate}[label = (\alph*)]
			\item Express the conservation law for the entire region.
			\\ \\
			Notice the following equality from the assumption that the concentration $u(x,t)$ of a chemical satisfies Fick's law:
			\[\partiald{u}{t} = k\partiald{^2u}{x^2}\]
			\item Determine the total amount of chemical in the region as a function of time (using the initial condition).
			Notice the given conditions:
			\[u(x,0) = f(x) \qqrarrow -k  \partiald{u}{x}(0,t) = \alpha \qqrarrow -k \partiald{u}{x}(L,t) = \beta\]
			With these conditions, we get the following:
			\[\frac{d}{dt}\int_{0}^{L}u\,dx = k\int_{0}^{L}\partiald{^2u}{x^2}\,dx = k\,\frac{du}{dx}(L,t) - k\frac{du}{dt}(0,t) = -\beta + \alpha\]
			With this, we can solve for the total amount of chemical in the region as a function of time:
			\[\boldsymbol{u(t) = \int \left(\frac{d}{dt}\int_{0}^{L}u\,dx\right)\,dt = \int (-\beta + \alpha)\,dt = (-\beta + \alpha)t} + C\]
			\item Under what conditions is there an equilibrium chemical concentration and what is it?
			\\ \\
			We can find its equilibrium chemical concentration as long as $\alpha = \beta$.  This way, we can get that the chemical is contained entirely in its own space.  From there we can find the equilibrium chemical concentration by letting the initial condition equal our equation from part (b).
			\[c\rho\int_{0}^{L}f(x)\,dx = c\rho\int_{0}^{L}(-\beta + \alpha)t + C\,dx = c\rho\int_{0}^{L} C\,dx = c\rho CL\]
			\[\boldsymbol{u(x) = C = \frac{1}{L} \int_{0}^{L}f(x)\,dx}\]
		\end{enumerate}
	\end{problem}


\end{document}
