\documentclass[11pt]{article}
\usepackage[margin = 1in]{geometry}
\usepackage{amsmath}
\usepackage{amssymb}
\usepackage{amsthm}
\usepackage{graphicx}
\usepackage{enumitem}
\usepackage{url}
\usepackage[parfill]{parskip}
\usepackage{listings}
\usepackage{caption}
\usepackage{subcaption}
\usepackage[utf8]{inputenc}
\usepackage{xcolor}
\definecolor{codegreen}{rgb}{0,0.6,0}
\definecolor{codegray}{rgb}{0.5,0.5,0.5}
\definecolor{codepurple}{rgb}{0.58,0,0.82}
\definecolor{backcolour}{rgb}{0.95,0.95,0.92}
\lstdefinestyle{mystyle}{
	backgroundcolor=\color{backcolour},   
	commentstyle=\color{codegreen},
	keywordstyle=\color{magenta},
	numberstyle=\tiny\color{codegray},
	stringstyle=\color{codepurple},
	basicstyle=\ttfamily\footnotesize,
	breakatwhitespace=false,         
	breaklines=true,                 
	captionpos=b,                    
	keepspaces=true,                 
	numbers=left,                    
	numbersep=5pt,                  
	showspaces=false,                
	showstringspaces=false,
	showtabs=false,                  
	tabsize=2
}
\lstset{style=mystyle}
\newcommand{\skipline}{\vspace{\baselineskip}}
\newcommand{\spacer}{\noalign{\medskip}}
\newcommand{~}{\sim}
\newcommand{\approches}{\rightarrow}
\newcommand{\qarrow}{\quad \rightarrow \quad}
\newcommand{\qqarrow}{\qquad \rightarrow \qquad}
\newcommand{\qqtext}[1]{\qquad \text{ #1 } \qquad}
\newcommand{\pard}[2]{\frac{\partial #1}{\partial #2}}
\newcommand{\answer}[1]{\textbf{\boldmath #1}}
\newenvironment{problem}[1]{\textbf{Exercise #1: }}{\newpage}

\begin{document}
	
	\begin{center}
		\textbf{Homework 4} \\
		\textbf{Partial Differential Equations} \\
		\textbf{Math 531} \\
		\textbf{Stephen Giang RedID: 823184070} \\
		\skipline \skipline
	\end{center}

	% Page 87
	\begin{problem}{2.5.1a}
		Solve Laplace’s equation inside a rectangle $0 \leq x \leq L, 0 \leq y \leq H$, with the following
		boundary conditions [\textit{Hint}: Separate variables. If there are two homogeneous boundary
		conditions in y, let $u(x, y) = h(x)\phi(y)$, and if there are two homogeneous boundary
		conditions in x, let $u(x, y) = \phi(x)h(y)$.]:
		\[\pard{u}{x}(0,y) = 0, \qquad \pard{u}{x}(L,y) = 0, \qquad u(x,0) = 0, \qquad u(x,H) = f(x)\]
		\\ 
		Let the following be true:
		\[\pard{^2u}{x^2} + \pard{^2u}{y^2} = 0 \qquad 0 \leq x \leq L, \qquad 0 \leq y \leq H\]
		with 
		\[u(x,y) = \phi(x)h(y), \qquad \phi'(0) = 0, \quad \phi'(L) = 0, \quad h(0) = 0\]
		Taking Laplace's Equation, we get the following:
		\[\phi''(x)h(y) + \phi(x)h(y)'' = 0\]
		\[\frac{\phi''(x)}{\phi(x)} = -\frac{h''(y)}{h(y)} = -\lambda\]
		From this, we get the following:
		\[\phi'' + \lambda \phi = 0, \qquad h'' - \lambda h = 0\]
		\newpage
		From this, we can see this is an eigenvalue problem:
		\begin{enumerate}[label = (\alph*)]
			\item $(\lambda = 0)$:
			\[\phi'' = 0 \qqarrow \phi' = c_1 \qqarrow \phi = c_1x + c_2\]
			\[h'' = 0 \qqarrow h' = d_1 \qqarrow h = d_1y + d_2\]
			Substituting in our BC's, we get:
			\[\phi'(0) = \phi'(L) = c_1 = 0 \qqarrow \phi(x) = c_2\]
			So now we have our first eigenfunction:
			\[\phi(x) = c_2 \text{ with } \lambda = 0\]
			Now we can solve for $h(y)$:
			\[h(0) = d_2 = 0 \qarrow h(y) = d_1y \]
			From here, we get the following:
			\[u(x,y) = c_2d_1y\]
			Now we can simply set the following and get the first product solution:
			\[\boldsymbol{u_0(x,y) = A_0y}\]
			\item $(\lambda < 0)$:
			\[\phi'' - |\lambda|\phi = 0\]
			Using the characteristic equation, we get:
			\[\phi(x) = c_1\cosh(\sqrt{|\lambda|}x) + c_2\sinh(\sqrt{|\lambda|}x) \qquad \phi'(x) = c_1\sqrt{|\lambda|}\sinh(\sqrt{|\lambda|}x) + c_2\sqrt{|\lambda|}\cosh(\sqrt{|\lambda|}x) \]
			Using the BC's, we get:
			\[\phi'(0) = c_2\sqrt{|\lambda|} = 0 \qarrow \sqrt{|\lambda|} > 0 \qarrow c_2 = 0\]
			\[\phi'(L) = c_1\sqrt{|\lambda|}\sinh(\sqrt{|\lambda|}L) = 0 \qarrow \sqrt{|\lambda|}\sinh(\sqrt{|\lambda|}L) \not = 0 \qarrow c_1 = 0\]
			\[\phi(x) = 0\]
			Thus we get the following trivial solution:
			\[\boldsymbol{u(x,y) = 0}\]
			\item $(\lambda > 0)$:
			\[\phi'' + \lambda\phi = 0\]
			Using the characteristic equation, we get:
			\[\phi(x) = c_1\cos(\sqrt{\lambda}x) + c_2\sin(\sqrt{\lambda}x) \qquad \phi'(x) = -c_1\sqrt{\lambda}\sin(\sqrt{\lambda}x) + c_2\sqrt{\lambda}\cos(\sqrt{\lambda}x)  \]
			Using the BC's, we get:
			\[\phi'(0) = c_2\sqrt{\lambda} = 0 \qqarrow \sqrt{\lambda} > 0 \qqarrow c_2 = 0\]
			\[\phi'(L) = -c_1\sqrt{\lambda}\sin(\sqrt{\lambda}L) = 0\]
			\begin{enumerate}[label = (\roman*)]
				\item $(c_1 = 0)$:
				\[\phi(x) = 0\]
				Thus we get the following trivial solution:
				\[\boldsymbol{u(x,y) = 0}\]
				\item $(\sqrt{\lambda}\sin(\sqrt{\lambda}L) = 0)$:
				\[\sin(\sqrt{\lambda}L) = 0 \qqarrow \sqrt{\lambda}L = n\pi \qqarrow \lambda = \frac{n^2\pi^2}{L^2}\]
				So now we have our $n$ eigenfunctions:
				\[\phi_n(x) = c_1\cos\left(\frac{n\pi x}{L}\right)\]
				we can now substitute our eigenvalues into the other ODE, and we get:
				\[h'' - \frac{n^2\pi^2}{L^2} h = 0\]
				When solving this ODE, we get the following linear independent solutions:
				\[h_n(y) = d_1\cosh\left(\frac{n\pi y}{L}\right) + d_2\sinh\left(\frac{n\pi y}{L}\right)\]
				If we substitute our BC $(h(0) = 0)$ in, we get
				\[h_n(0) = 0 = d_1 \qqarrow h_n(y) = d_2\sinh\left(\frac{n\pi y}{L}\right)\]
				From here, we get the following $n$ product solutions:
				\[\boldsymbol{u_n(x,y) = A_n\cos\left(\frac{n\pi x}{L}\right)\sinh\left(\frac{n\pi y}{L}\right)}\] 
			\end{enumerate}		
		\end{enumerate}
		\skipline
		By the Principle of Superposition, we get the following:
		\begin{align*}
			u(x,y) &= u_0(x,y) + u_1(x,y) + ... + u_n(x,y) \\
			&= A_0y + \sum_{n=1}^{\infty} A_n\cos\left(\frac{n\pi x}{L}\right)\sinh\left(\frac{n\pi y}{L}\right)
		\end{align*}
		We can now include our nonhomogeneous solution and get the following:
		\[u(x,H) = f(x) = A_0H + \sum_{n=1}^{\infty} A_n\cos\left(\frac{n\pi x}{L}\right)\sinh\left(\frac{n\pi H}{L}\right)\]
		Using the orthogonality of cosines, we get:
		\[A_0 = \frac{1}{LH}\int_{0}^{L} f(x)\,dx \qquad A_n = \frac{2}{L \sinh\left(\frac{n\pi H}{L}\right)} \int_{0}^{L} f(x)\cos\left(\frac{n\pi x}{L}\right)\,dx \]
		Thus we get our desired solution:
		\begin{align*}
			\boldsymbol{u(x,y)} & \boldsymbol{= \frac{1}{LH}\int_{0}^{L} f(x)\,dx} \\
			&\boldsymbol{+ \sum_{n=1}^{\infty} \left[\frac{2}{L \sinh\left(\frac{n\pi H}{L}\right)}\int_{0}^{L} f(x)\cos\left(\frac{n\pi x}{L}\right)\,dx\right]\cos\left(\frac{n\pi x}{L}\right)\sinh\left(\frac{n\pi y}{L}\right)}
		\end{align*}
	\end{problem}

	\begin{problem}{2.5.1g}
		Solve Laplace’s equation inside a rectangle $0 \leq x \leq L, 0 \leq y \leq H$, with the following
		boundary conditions [\textit{Hint}: Separate variables. If there are two homogeneous boundary
		conditions in y, let $u(x, y) = h(x)\phi(y)$, and if there are two homogeneous boundary
		conditions in x, let $u(x, y) = \phi(x)h(y)$.]:
		\[\pard{u}{x}(0,y) = 0, \qquad \pard{u}{x}(L,y) = 0, \qquad u(x,0) = \begin{cases}
			0 & x > L/2 \\ 1 & x < L/2
		\end{cases}, \qquad \pard{u}{y}(x,H) = 0\]
		\\ 
		Let the following be true:
		\[\pard{^2u}{x^2} + \pard{^2u}{y^2} = 0 \qquad 0 \leq x \leq L, \qquad 0 \leq y \leq H\]
		with 
		\[u(x,y) = \phi(x)h(y), \qquad \phi'(0) = 0, \quad \phi'(L) = 0, \quad h'(H) = 0.\]
		Also let the following be true:
		\[u(x,0) = g(x) = \begin{cases}
			0 & x > L/2 \\ 1 & x < L/2
		\end{cases}\]
		Taking Laplace's Equation, we get the following:
		\[\phi''(x)h(y) + \phi(x)h(y)'' = 0\]
		\[\frac{\phi''(x)}{\phi(x)} = -\frac{h''(y)}{h(y)} = -\lambda\]
		From this, we get the following:
		\[\phi'' + \lambda \phi = 0, \qquad h'' - \lambda h = 0\]
		\newpage
		From this, we can see this is an eigenvalue problem:
		\begin{enumerate}[label = (\alph*)]
			\item $(\lambda = 0)$:
			\[\phi'' = 0 \qqarrow \phi' = c_1 \qqarrow \phi = c_1x + c_2\]
			\[h'' = 0 \qqarrow h' = d_1 \qqarrow h = d_1y + d_2\]
			Substituting in our BC's, we get:
			\[\phi'(0) = \phi'(L) = c_1 = 0 \qqarrow \phi(x) = c_2\]
			So now we have our first eigenfunction:
			\[\phi(x) = c_2 \text{ with } \lambda = 0\]
			Now we can solve for $h(y)$:
			\[h'(H) = d_1 = 0 \qarrow h(y) = d_2 \]
			From here, we get our first product solution:
			\[\boldsymbol{u_0(x,y) = A_0}\]
			\item $(\lambda < 0)$:
			\[\phi'' - |\lambda|\phi = 0\]
			Using the characteristic equation, we get:
			\[\phi(x) = c_1\cosh(\sqrt{|\lambda|}x) + c_2\sinh(\sqrt{|\lambda|}x) \qquad \phi'(x) = c_1\sqrt{|\lambda|}\sinh(\sqrt{|\lambda|}x) + c_2\sqrt{|\lambda|}\cosh(\sqrt{|\lambda|}x) \]
			Using the BC's, we get:
			\[\phi'(0) = c_2\sqrt{|\lambda|} = 0 \qarrow \sqrt{|\lambda|} > 0 \qarrow c_2 = 0\]
			\[\phi'(L) = c_1\sqrt{|\lambda|}\sinh(\sqrt{|\lambda|}L) = 0 \qarrow \sqrt{|\lambda|}\sinh(\sqrt{|\lambda|}L) \not = 0 \qarrow c_1 = 0\]
			\[\phi(x) = 0\]
			Thus we get the following trivial solution:
			\[\boldsymbol{u(x,y) = 0}\]
			\item $(\lambda > 0)$:
			\[\phi'' + \lambda\phi = 0\]
			Using the characteristic equation, we get:
			\[\phi(x) = c_1\cos(\sqrt{\lambda}x) + c_2\sin(\sqrt{\lambda}x) \qquad \phi'(x) = -c_1\sqrt{\lambda}\sin(\sqrt{\lambda}x) + c_2\sqrt{\lambda}\cos(\sqrt{\lambda}x)  \]
			Using the BC's, we get:
			\[\phi'(0) = c_2\sqrt{\lambda} = 0 \qqarrow \sqrt{\lambda} > 0 \qqarrow c_2 = 0\]
			\[\phi'(L) = -c_1\sqrt{\lambda}\sin(\sqrt{\lambda}L) = 0\]
			\begin{enumerate}[label = (\roman*)]
				\item $(c_1 = 0)$:
				\[\phi(x) = 0\]
				Thus we get the following trivial solution:
				\[\boldsymbol{u(x,y) = 0}\]
				\item $(\sqrt{\lambda}\sin(\sqrt{\lambda}L) = 0)$:
				\[\sin(\sqrt{\lambda}L) = 0 \qqarrow \sqrt{\lambda}L = n\pi \qqarrow \lambda = \frac{n^2\pi^2}{L^2}\]
				So now we have our $n$ eigenfunctions:
				\[\phi_n(x) = c_1\cos\left(\frac{n\pi x}{L}\right)\]
				we can now substitute our eigenvalues into the other ODE, and we get:
				\[h'' - \frac{n^2\pi^2}{L^2} h = 0\]
				When solving this ODE, we get the following linear independent solutions:
				\[h_n(y) = d_1\cosh\left(\frac{n\pi (H - y)}{L}\right) + d_2\sinh\left(\frac{n\pi (H - y)}{L}\right)\]
				\[h_n'(y) = d_1\frac{-n\pi}{L}\sinh\left(\frac{n\pi (H - y)}{L}\right) + d_2\frac{-n\pi}{L}\cosh\left(\frac{n\pi (H - y)}{L}\right)\]
				If we substitute our BC $(h'(H) = 0)$ in, we get
				\[h_n(H) = 0 = d_2 \qqarrow h_n(y) = d_1\cosh\left(\frac{n\pi (H - y)}{L}\right)\]
				From here, we get the following $n$ product solution:
				\[\boldsymbol{u_n(x,y) = A_n\cos\left(\frac{n\pi x}{L}\right)\cosh\left(\frac{n\pi (H - y)}{L}\right)}\]
			\end{enumerate}
		\end{enumerate}
		\newpage
		By the Principle of Superposition, we get the following:
		\begin{align*}
			u(x,y) &= u_0(x,y) + u_1(x,y) + ... + u_n(x,y) \\
			&= A_0 + \sum_{n=1}^{\infty} A_n\cos\left(\frac{n\pi x}{L}\right)\cosh\left(\frac{n\pi (H - y)}{L}\right)
		\end{align*}
		We can now include our nonhomogeneous solution and get the following:
		\[u(x,0) = g(x) = A_0 + \sum_{n=1}^{\infty} A_n\cos\left(\frac{n\pi x}{L}\right)\cosh\left(\frac{n\pi H}{L}\right)\]
		Using the orthogonality of cosines, we get:
		\[A_0 = \frac{1}{L}\int_{0}^{L} g(x)\,dx = \frac{1}{L}\left( \int_{0}^{L/2} 1\,dx + \int_{L/2}^{L} 0\,dx
		\right) = \frac{1}{L}\left(\frac{L}{2}\right) = \frac{1}{2} \]
		\begin{align*}
			A_n &= \frac{2}{L \cosh\left(\frac{n\pi H}{L}\right)} \int_{0}^{L} g(x)\cos\left(\frac{n\pi x}{L}\right)\,dx \\
			&= \frac{2}{L \cosh\left(\frac{n\pi H}{L}\right)} \left(\int_{0}^{L/2} \cos\left(\frac{n\pi x}{L}\right)\,dx + \int_{L/2}^{L} 0\,dx \right) \\
			&= \frac{2}{L \cosh\left(\frac{n\pi H}{L}\right)} \left(\frac{L}{n\pi}\sin\left(\frac{n\pi x}{L}\right)\bigg|_0^{L/2}\right) \\
			&= \frac{2}{n\pi \cosh\left(\frac{n\pi H}{L}\right)} \sin\left(\frac{n\pi}{2}\right)
		\end{align*}
		Thus we get our desired solution:
		\[\boldsymbol{u(x,y) = \frac{1}{2} + \sum_{n=1}^{\infty} \left[\frac{2}{n\pi \cosh\left(\frac{n\pi H}{L}\right)} \sin\left(\frac{n\pi}{2}\right)\right]\cos\left(\frac{n\pi x}{L}\right)\cosh\left(\frac{n\pi (H-y)}{L}\right)} \]
	\end{problem}
	
	\begin{problem}{2.5.2}
		Consider $u(x, y)$ satisfying Laplace’s equation inside a rectangle $(0 < x < L, 0 <y<
		H)$ subject to the boundary conditions
		\[\pard{u}{x}(0,y) = 0, \qquad \pard{u}{y}(x,0) = 0\]
		\[\pard{u}{x}(L,y) = 0, \qquad \pard{u}{y}(x,H) = f(x)\]
		\begin{enumerate}[label = (\alph*)]
			\item \textit{Without} solving this problem, briefly explain the physical condition under which
			there is a solution to this problem.
			\\ \\
			We know that this rectangle is insulated on 3 sides: $\pard{u}{x}(0,y) = 0$ (no change in x at (0,y), so left side of rectangle is insulated), $\pard{u}{x}(L,y) = 0$ (no change in x at (L,y), so right side of rectangle is insulated), $\pard{u}{y}(x,0) = 0$ (no change in y at (x,0), so bottom side of rectangle is insulated).  Thus the only way for this to have a solution is for the top of the rectangle to be insulated:
			\[\boldsymbol{\int_{0}^{L} \pard{u}{y}(x,H)\,dx =\int_{0}^{L} f(x)\,dx = 0}\]
			We take the integral from 0 to $L$ to denote that the net change in heat energy between the left and right endpoints is 0. 
			\\ \\
			\item Solve this problem by the method of separation of variables. Show that the method
			works only under the condition of part (a). [\textit{Hint}: You may use (5.16) without
			derivation.]
			\\ \\
			Let the following be true:
			\[\pard{^2u}{x^2} + \pard{^2u}{y^2} = 0 \qquad 0 \leq x \leq L, \qquad 0 \leq y \leq H\]
			with 
			\[u(x,y) = \phi(x)h(y), \qquad \phi'(0) = 0, \quad \phi'(L) = 0, \quad h'(0) = 0.\]
			Taking Laplace's Equation, we get the following:
			\[\phi''(x)h(y) + \phi(x)h(y)'' = 0\]
			\[\frac{\phi''(x)}{\phi(x)} = -\frac{h''(y)}{h(y)} = -\lambda\]
			From this, we get the following:
			\[\phi'' + \lambda \phi = 0, \qquad h'' - \lambda h = 0\]
			\newpage
			From this, we can see this is an eigenvalue problem:
			\begin{enumerate}[label = (\alph*)]
				\item $(\lambda = 0)$:
				\[\phi'' = 0 \qqarrow \phi' = c_1 \qqarrow \phi = c_1x + c_2\]
				\[h'' = 0 \qqarrow h' = d_1 \qqarrow h = d_1y + d_2\]
				Substituting in our BC's, we get:
				\[\phi'(0) = \phi'(L) = c_1 = 0 \qqarrow \phi(x) = c_2\]
				So now we have our first eigenfunction:
				\[\phi(x) = c_2 \text{ with } \lambda = 0\]
				Now we can solve for $h(y)$:
				\[h'(0) = d_1 = 0 \qarrow h(y) = d_2 \]
				From here, we get our first product solution:
				\[\boldsymbol{u_0(x,y) = A_0}\]
				\item $(\lambda < 0)$:
				\[\phi'' - |\lambda|\phi = 0\]
				Using the characteristic equation, we get:
				\[\phi(x) = c_1\cosh(\sqrt{|\lambda|}x) + c_2\sinh(\sqrt{|\lambda|}x) \qquad \phi'(x) = c_1\sqrt{|\lambda|}\sinh(\sqrt{|\lambda|}x) + c_2\sqrt{|\lambda|}\cosh(\sqrt{|\lambda|}x) \]
				Using the BC's, we get:
				\[\phi'(0) = c_2\sqrt{|\lambda|} = 0 \qarrow \sqrt{|\lambda|} > 0 \qarrow c_2 = 0\]
				\[\phi'(L) = c_1\sqrt{|\lambda|}\sinh(\sqrt{|\lambda|}L) = 0 \qarrow \sqrt{|\lambda|}\sinh(\sqrt{|\lambda|}L) \not = 0 \qarrow c_1 = 0\]
				\[\phi(x) = 0\]
				Thus we get the following trivial solution:
				\[\boldsymbol{u(x,y) = 0}\]
				\item $(\lambda > 0)$:
				\[\phi'' + \lambda\phi = 0\]
				Using the characteristic equation, we get:
				\[\phi(x) = c_1\cos(\sqrt{\lambda}x) + c_2\sin(\sqrt{\lambda}x) \qquad \phi'(x) = -c_1\sqrt{\lambda}\sin(\sqrt{\lambda}x) + c_2\sqrt{\lambda}\cos(\sqrt{\lambda}x)  \]
				Using the BC's, we get:
				\[\phi'(0) = c_2\sqrt{\lambda} = 0 \qqarrow \sqrt{\lambda} > 0 \qqarrow c_2 = 0\]
				\[\phi'(L) = -c_1\sqrt{\lambda}\sin(\sqrt{\lambda}L) = 0\]
				\begin{enumerate}[label = (\roman*)]
					\item $(c_1 = 0)$:
					\[\phi(x) = 0\]
					Thus we get the following trivial solution:
					\[\boldsymbol{u(x,y) = 0}\]
					\item $(\sqrt{\lambda}\sin(\sqrt{\lambda}L) = 0)$:
					\[\sin(\sqrt{\lambda}L) = 0 \qqarrow \sqrt{\lambda}L = n\pi \qqarrow \lambda = \frac{n^2\pi^2}{L^2}\]
					So now we have our $n$ eigenfunctions:
					\[\phi_n(x) = c_1\cos\left(\frac{n\pi x}{L}\right)\]
					we can now substitute our eigenvalues into the other ODE, and we get:
					\[h'' - \frac{n^2\pi^2}{L^2} h = 0\]
					When solving this ODE, we get the following linear independent solutions:
					\[h_n(y) = d_1\cosh\left(\frac{n\pi y}{L}\right) + d_2\sinh\left(\frac{n\pi y}{L}\right)\]
					\[h_n'(y) = d_1\frac{n\pi}{L}\sinh\left(\frac{n\pi y}{L}\right) + d_2\frac{n\pi}{L}\cosh\left(\frac{n\pi y}{L}\right)\]
					If we substitute our BC $(h'(0) = 0)$ in, we get
					\[h_n(0) = 0 = d_2 \qqarrow h_n(y) = d_1\cosh\left(\frac{n\pi y}{L}\right)\]
					From here, we get the following $n$ product solutions:
					\[\boldsymbol{u_n(x,y) =  A_n\cos\left(\frac{n\pi x}{L}\right)\cosh\left(\frac{n\pi y}{L}\right)}\]
				\end{enumerate}
			\end{enumerate}
			\newpage
			By the Principle of Superposition, we get the following:
			\begin{align*}
				u(x,y) &= u_0(x,y) + u_1(x,y) + ... + u_n(x,y) \\
				&= A_0 + \sum_{n=1}^{\infty} A_n\cos\left(\frac{n\pi x}{L}\right)\cosh\left(\frac{n\pi y}{L}\right)
			\end{align*}
			We can now include our nonhomogeneous solution and get the following:
			\[\pard{u}{y}(x,H) = f(x) = \sum_{n=1}^{\infty} A_n\cos\left(\frac{n\pi x}{L}\right)\left(\frac{n\pi}{L}\sinh\left(\frac{n\pi H}{L}\right)\right)\]
			Notice we can verify our assertion of part (a):
			\begin{align*}
				\int_{0}^{L} f(x)\,dx &= \int_{0}^{L}\sum_{n=1}^{\infty} A_n\cos\left(\frac{n\pi x}{L}\right)\left(\frac{n\pi}{L}\sinh\left(\frac{n\pi H}{L}\right)\right)\,dx \\
				&= \sum_{n=1}^{\infty} A_n\left[\int_{0}^{L} \cos\left(\frac{n\pi x}{L}\right)\,dx\right]\left(\frac{n\pi}{L}\sinh\left(\frac{n\pi H}{L}\right)\right)\,dx \\
				&= \sum_{n=1}^{\infty} A_n\left[\frac{L}{n\pi}\sin\left(\frac{n\pi x}{L}\right)\bigg|_0^L\right]\left(\frac{n\pi}{L}\sinh\left(\frac{n\pi H}{L}\right)\right)\,dx \\
				&= 0
			\end{align*}
			Using the orthogonality of cosines, we get:
			\[A_n = \frac{2}{n\pi\sinh\left(\frac{n\pi H}{L}\right)} \int_{0}^{L} f(x)\cos\left(\frac{n\pi x}{L}\right)\,dx \]
			Thus we get our desired solution:
			\begin{align*}
				\boldsymbol{u(x,y)}& \boldsymbol{= A_0} \\
				& \boldsymbol{+ \sum_{n=1}^{\infty} \left[\frac{2}{n\pi\sinh\left(\frac{n\pi H}{L}\right)} \int_{0}^{L} f(x)\cos\left(\frac{n\pi x}{L}\right)\,dx\right]\cos\left(\frac{n\pi x}{L}\right)\cosh\left(\frac{n\pi y}{L}\right)}
			\end{align*}
			\newpage
			\item The solution [part (b)] has an arbitrary constant. Determine it by consideration
			of the time-dependent heat equation subject to the initial condition
			\[u(x,y,0) = g(x,y)\]
			Because we know that three sides of the rectangle are insulated, we get that the total energy in the rectangle $\mathcal{D}$ must be constant, so we get the following:
			\begin{align*}
				\int\int_\mathcal{D} u(x,y,t)\,dx\,dy &= \int\int_\mathcal{D} u(x,y,0)\,dx\,dy \\
				&= \int\int_\mathcal{D} g(x,y)\,dx\,dy \\
				&= \int\int_\mathcal{D} u(x,y)\,dx\,dy \\ 
				&= \int\int_\mathcal{D} \bigg(A_0 + \sum_{n=1}^{\infty} \left[\frac{2}{n\pi\sinh\left(\frac{n\pi H}{L}\right)} \int_{0}^{L} f(x)\cos\left(\frac{n\pi x}{L}\right)\,dx\right]... \\
				&\qquad \qquad \qquad \qquad\cos\left(\frac{n\pi x}{L}\right)\cosh\left(\frac{n\pi y}{L}\right)\bigg)\,dx\,dy \\
				&= A_0\int_0^L\int_0^H\,dxdy + \sum_{n=1}^{\infty}\left[\frac{2}{n\pi\sinh\left(\frac{n\pi H}{L}\right)} \int_{0}^{L} f(x)\cos\left(\frac{n\pi x}{L}\right)\,dx\right]...\\
				&\qquad \qquad \qquad \qquad \left(\int_0^{L}\cos\left(\frac{n\pi x}{L}\right)\,dx\right)\left(\int_{0}^H\cosh\left(\frac{n\pi y}{L}\right)\,dy\right) \\
				&= A_0LH + \sum_{n=1}^{\infty}\left[\frac{2}{n\pi\sinh\left(\frac{n\pi H}{L}\right)} \int_{0}^{L} f(x)\cos\left(\frac{n\pi x}{L}\right)\,dx\right]...\\
				&\qquad \qquad \qquad \qquad \left(\int_0^{L}\cos\left(\frac{n\pi x}{L}\right)\,dx\right)\left(\int_{0}^H\cosh\left(\frac{n\pi y}{L}\right)\,dy\right) 
			\end{align*} 
			Notice we get the following: 
			\[\int_0^{L}\cos\left(\frac{n\pi x}{L}\right)\,dx = \frac{L}{n\pi}\sin\left(\frac{n\pi x}{L}\right)\bigg|_0^L = 0\]
			Thus, we get the following:
			\[\int\int_\mathcal{D} g(x,y)\,dx\,dy = A_0 LH\]
			Which gives us our arbitrary constant:
			\[\boldsymbol{A_0 = \frac{1}{LH} \int\int_\mathcal{D} g(x,y)\,dx\,dy}\] 
		\end{enumerate}
	\end{problem}
	
	\begin{problem}{2.5.6b}
		Solve Laplace’s equation inside a semicircle of radius $a(0 < r < a, 0 < \theta < \pi)$ subject
		to the boundary conditions [Hint: In polar coordinates,
		\[\nabla^2u = \frac{1}{r}\pard{}{r}\left(r \pard{u}{r}\right) + \frac{1}{r^2}\pard{^2u}{\theta^2} = 0,\]
		it is known that if $u(r,\theta) = \phi(\theta)G(r)$, then $\frac{r}{G}\frac{d}{dr}\left(r \frac{dG}{dr}\right) = -\frac{1}{\phi} \frac{d^2\phi}{d\theta^2}$.]: 
		\\ \\
		The diameter is insulated and $u(a,\theta) = g(\theta)$.
		\\ \\ 
		Let the following be true:
		\[\frac{1}{r}\pard{}{r}\left(r \pard{u}{r}\right) + \frac{1}{r^2}\pard{^2u}{\theta^2} = 0 \qquad 0 < r < a, \qquad 0 < \theta < \pi \]
		with
		\[u(r,\theta) = \phi(\theta)G(r)\]
		Notice the boundary conditions:
		\[\phi'(0) = 0 \qquad \phi'(\pi) = 0 \qquad u(a,\theta) = g(\theta) \]
		Taking Laplace's Equation, we get the following:
		\[\frac{r}{G}\frac{d}{dr}\left(r \frac{dG}{dr}\right) = -\frac{\phi''}{\phi} = \lambda\]
		From this, we get the following:
		\[\phi'' + \lambda\phi = 0 \qquad r\frac{d}{dr}\left(r \frac{dG}{dr}\right) - \lambda G = r^2G'' + rG' - \lambda G = 0\]
		\newpage
		From this, we can see this is an eigenvalue problem:
		\begin{enumerate}[label = (\alph*)]
			\item $(\lambda = 0)$:
			\[\phi'' = 0 \qqarrow \phi' = c_1 \qqarrow \phi = c_1 \theta + c_2\]
			\[r\frac{d}{dr}\left(r \frac{dG}{dr}\right) = 0 \qqarrow r \frac{dG}{dr} = d_1 \qqarrow G =  d_1\ln\,r + d_2 \]
			Substituting in our BC’s, we get:
			\[\phi'(0) = \phi'(\pi) = c_1 = 0\]
			So now we have our first eigenfunction:
			\[\phi(\theta) = c_2 \text{ with } \lambda = 0\]
			Now we can solve for $G(r)$ using the boundedness condition. This implies that $d_1 = 0$, thus we get:
			\[G(r) = d_2\]
			From here, we get our first product solution:
			\[\boldsymbol{u_0(r,\theta) = A_0}\]
			\item $(\lambda < 0)$:
			\[\phi'' - |\lambda|\phi = 0\]
			Using the characteristic equation, we get:
			\[\phi(\theta) = c_1\cosh(\sqrt{|\lambda|}\theta) + c_2\sinh(\sqrt{|\lambda|}\theta) \qquad \phi'(\theta) = c_1\sqrt{|\lambda|}\sinh(\sqrt{|\lambda|}\theta) + c_2\sqrt{|\lambda|}\cosh(\sqrt{|\lambda|}\theta) \]
			Using the BC's, we get:
			\[\phi'(0) = c_2\sqrt{|\lambda|} = 0 \qqarrow c_2 = 0\]
			\[\phi'(\pi) = c_1\sqrt{|\lambda|}\sinh(\sqrt{|\lambda|}\pi) = 0 \qquad \sqrt{|\lambda|}\sinh(\sqrt{|\lambda|}\pi) \not = 0 \qqarrow c_1 = 0 \]
			Thus we get the following trivial solution:
			\[\boldsymbol{u(r,\theta) = 0}\]
			\newpage
			\item $(\lambda > 0)$:
			\[\phi'' + \lambda\phi = 0\]
			Using the characteristic equation, we get:
			\[\phi(\theta) = c_1\cos(\sqrt{\lambda}\theta) + c_2\sin(\sqrt{\lambda}\theta) \qquad \phi'(\theta) = -c_1\sqrt{\lambda}\sin(\sqrt{\lambda}\theta) + c_2\sqrt{\lambda}\cos(\sqrt{\lambda}\theta)  \]
			Using the BC's, we get:
			\[\phi'(0) = c_2\sqrt{\lambda} = 0 \qqarrow c_2 = 0\]
			\[\phi'(\pi) = -c_1\sqrt{\lambda}\sin(\sqrt{\lambda}\pi)\]
			\begin{enumerate}[label = (\roman*)]
				\item $(c_1 = 0)$:
				\[\phi(\theta) = 0\]
				Thus we get the following trivial solution:
				\[\boldsymbol{u(r,\theta) = 0}\]
				\item $(\sin(\sqrt{\lambda}\pi) = 0)$:
				\[\sin(\sqrt{\lambda}\pi) = 0 \qqarrow \sqrt{\lambda}\pi = n\pi \qqarrow \lambda = n^2\]
				So now we have our $n$ eigenfunctions:
				\[\phi_n(\theta) = c_1\cos(n\theta) \]
				we can now substitute our eigenvalues into the other ODE, and we get:
				\[r^2G'' + rG' - n^2 G = 0\]
				Let the following be true:
				\[G = cr^\alpha \qqarrow G' = \alpha c r^{\alpha - 1} \qqarrow G'' = (\alpha^2 - \alpha) c r^{\alpha - 2}\]
				\begin{align*}
					r^2(\alpha^2 - \alpha)cr^{\alpha - 2} + r\alpha c r^{\alpha - 1} - n^2cr^\alpha &= 0 \\
					r^{\alpha}c \left( \alpha^2 - \alpha  + \alpha  - n^2 \right) &= 0 \\
					\alpha &= \pm n
				\end{align*}
				When solving this ODE, we get the following linear independent solutions:
				\[G(r) = d_1 r^{-n} + d_2 r^n\]
				ow we can solve for $G(r)$ using boundedness condition. This implies that $d_1 = 0$, thus we get:
				\[G_n(r) = d_2r^n\]
				From here, we get the following $n$ product solutions:
				\[\boldsymbol{u_n(r,\theta) = A_nr^n\cos(n\theta) }\]
			\end{enumerate}
		\end{enumerate}
		\newpage
		By the Principle of Superposition, we get the following:
		\begin{align*}
			u(r,\theta) &= u_0(r,\theta) + u_1(r,\theta) + ... + u_n(r,\theta) \\
			&= A_0 + \sum_{n=1}^{\infty}  A_nr^n\cos(n\theta)
		\end{align*}
		We can now include our nonhomogeneous solution and get the following:
		\[u(a,\theta) = g(\theta) = A_0 + \sum_{n=1}^{\infty}  A_n\,a^n\cos(n\theta)\]
		Using the orthogonality of cosines, we get:
		\[A_0 = \frac{1}{\pi} \int_{0}^{\pi} g(\theta)\,d\theta \qquad A_n = \frac{2}{a^n\pi}\int_{0}^{\pi} g(\theta)\cos(n\theta)\,d\theta\]
		Thus we get our desired solution:
		\[\boldsymbol{u(x,y) = \frac{1}{\pi} \int_{0}^{\pi} g(\theta)\,d\theta + \sum_{n=1}^{\infty}  \left[\frac{2}{a^n\pi}\int_{0}^{\pi} g(\theta)\cos(n\theta)\,d\theta\right]r^n\cos(n\theta)}\]
	\end{problem}
	
	\begin{problem}{2.5.8b}
		Solve Laplace’s equation inside a circular annulus $(a<r<b)$ subject to the boundary
		conditions [Hint: In polar coordinates,
		\[\nabla^2u = \frac{1}{r}\pard{}{r}\left(r \pard{u}{r}\right) + \frac{1}{r^2}\pard{^2u}{\theta^2} = 0,\]
		it is known that if $u(r,\theta) = \phi(\theta)G(r)$, then $\frac{r}{G}\frac{d}{dr}\left(r \frac{dG}{dr}\right) = -\frac{1}{\phi} \frac{d^2\phi}{d\theta^2}$.]:
		\[\pard{u}{r}(a,\theta) = 0, \qquad  u(b,\theta) = g(\theta) \]
		Let the following be true:
		\[\frac{1}{r}\pard{}{r}\left(r \pard{u}{r}\right) + \frac{1}{r^2}\pard{^2u}{\theta^2} = 0 \qquad 0 < r < a, \qquad 0 < \theta < \pi \]
		with
		\[u(r,\theta) = \phi(\theta)G(r)\]
		Notice the boundary conditions:
		\[\phi(-\pi) = \phi(\pi) \qquad \phi'(-\pi) = \phi'(\pi) \qquad u'(a,\theta) = 0 \qquad u(b,\theta) = g(\theta) \]
		Taking Laplace's Equation, we get the following:
		\[\frac{r}{G}\frac{d}{dr}\left(r \frac{dG}{dr}\right) = -\frac{\phi''}{\phi} = \lambda\]
		From this, we get the following:
		\[\phi'' + \lambda\phi = 0 \qquad r\frac{d}{dr}\left(r \frac{dG}{dr}\right) - \lambda G = r^2G'' + rG' - \lambda G = 0\]
		\newpage
		From this, we can see this is an eigenvalue problem:
		\begin{enumerate}[label = (\alph*)]
			\item $(\lambda = 0)$:
			\[\phi'' = 0 \qqarrow \phi' = c_1 \qqarrow \phi = c_1 \theta + c_2\]
			\[r\frac{d}{dr}\left(r \frac{dG}{dr}\right) = 0 \qqarrow r \frac{dG}{dr} = d_1 \qqarrow G =  d_1\ln\,r + d_2 \]
			Substituting in our BC’s, we get:
			\[\phi(-\pi) = -c_1\pi + c_2 = c_1\pi + c_2 = \phi(\pi) \qqarrow c_1 = 0 \]
			\[\phi'(-\pi) = c_1 = 0 = \phi'(\pi)\]
			So now we have our first eigenfunction:
			\[\phi(\theta) = c_2 \text{ with } \lambda = 0\]
			From here, we do not have any conditions on $G(r)$, so we get our first product solution:
			\[\boldsymbol{u_0(r,\theta) = c_2G(r) = A_0\ln\,r + B_0}\]
			\item $(\lambda < 0)$:
			\[\phi'' - |\lambda|\phi = 0\]
			Using the characteristic equation, we get:
			\[\phi(\theta) = c_1\cosh(\sqrt{|\lambda|}\theta) + c_2\sinh(\sqrt{|\lambda|}\theta) \qquad \phi'(\theta) = c_1\sqrt{|\lambda|}\sinh(\sqrt{|\lambda|}\theta) + c_2\sqrt{|\lambda|}\cosh(\sqrt{|\lambda|}\theta) \]
			Using the BC's, we get:
			\[\phi(-\pi) = c_1\cosh(\sqrt{|\lambda|}\pi) - c_2\sinh(\sqrt{|\lambda|}\pi) = c_1\cosh(\sqrt{|\lambda|}\pi) + c_2\sinh(\sqrt{|\lambda|}\pi) = \phi(\pi) \]
			\[c_2 = 0\]
			\[\phi'(-\pi) = -c_1\sqrt{|\lambda|}\sinh(\sqrt{|\lambda|}\pi) = c_1\sqrt{|\lambda|}\sinh(\sqrt{|\lambda|}\pi) = \phi'(\pi)\]
			\[c_1 = 0\]
			Thus we get the following trivial solution:
			\[\boldsymbol{u(r,\theta) = 0}\]
			\newpage
			\item $(\lambda > 0)$:
			\[\phi'' + \lambda\phi = 0\]
			Using the characteristic equation, we get:
			\[\phi(\theta) = c_1\cos(\sqrt{\lambda}\theta) + c_2\sin(\sqrt{\lambda}\theta) \qquad \phi'(\theta) = -c_1\sqrt{\lambda}\sin(\sqrt{\lambda}\theta) + c_2\sqrt{\lambda}\cos(\sqrt{\lambda}\theta)  \]
			Using the BC's, we get:
			\[\phi(-\pi) = c_1\cos(\sqrt{\lambda}\pi) - c_2\sin(\sqrt{\lambda}\pi) = c_1\cos(\sqrt{\lambda}\pi) + c_2\sin(\sqrt{\lambda}\pi) = \phi(\pi) \]
			\[\phi'(-\pi) = c_1\sqrt{\lambda}\sin(\sqrt{\lambda}\pi) + c_2\sqrt{\lambda}\cos(\sqrt{\lambda}\pi) = -c_1\sqrt{\lambda}\sin(\sqrt{\lambda}\pi) + c_2\sqrt{\lambda}\cos(\sqrt{\lambda}\pi) = \phi'(\pi)\]
			\begin{enumerate}[label = (\roman*)]
				\item $(c_1 = c_2 = 0)$:
				\[\phi(\theta) = 0\]
				Thus we get the following trivial solution:
				\[\boldsymbol{u(r,\theta) = 0}\]
				\item $(\sin(\sqrt{\lambda}\pi) = 0)$:
				\[\sin(\sqrt{\lambda}\pi) = 0 \qqarrow \sqrt{\lambda}\pi = n\pi \qqarrow \lambda = n^2\]
				So now we have our $n$ eigenfunctions:
				\[\phi_n(\theta) = c_1\cos(n\theta) + c_2\sin(n\theta) \]
				we can now substitute our eigenvalues into the other ODE, and we get:
				\[r^2G'' + rG' - n^2 G = 0\]
				Let the following be true:
				\[G = cr^\alpha \qqarrow G' = \alpha c r^{\alpha - 1} \qqarrow G'' = (\alpha^2 - \alpha) c r^{\alpha - 2}\]
				\begin{align*}
					r^2(\alpha^2 - \alpha)cr^{\alpha - 2} + r\alpha c r^{\alpha - 1} - n^2cr^\alpha &= 0 \\
					r^{\alpha}c \left( \alpha^2 - \alpha  + \alpha  - n^2 \right) &= 0 \\
					\alpha &= \pm n
				\end{align*}
				When solving this ODE, we get the following linear independent solutions:
				\[G(r) = d_1 r^{-n} + d_2 r^n\]
				From here, we do not have any conditions on $G(r)$, so we get our first product solution:
				\[G_n(r) = d_1 r^{-n} + d_2 r^{n}\]
				From here, we get the following $n$ product solutions:
				\[\boldsymbol{u_n(r,\theta) = \bigg(A_n r^{-n} + B_n r^{n}\bigg)\cos(n\theta) + \bigg(C_n r^{-n} + D_n r^{n}\bigg)\sin(n\theta)}\]
			\end{enumerate}
		\end{enumerate}
		\newpage
		By the Principle of Superposition, we get the following:
		\begin{align*}
			u(r,\theta) &= u_0(r,\theta) + u_1(r,\theta) + ... + u_n(r,\theta) \\
			&= A_0\ln\,r + B_0 + \sum_{n = 1}^{\infty}\bigg[ \bigg(A_n r^{-n} + B_n r^{n}\bigg)\cos(n\theta) + \bigg(C_n r^{-n} + D_n r^{n}\bigg)\sin(n\theta) \bigg]
		\end{align*}
		Now we can take the partial derivative in respect to $r$, so we can use our homogeneous BC:
		\[\pard{}{r}u(r,\theta) = \frac{A_0}{r} + \sum_{n = 1}^{\infty}\bigg[ \bigg(-n\,A_n r^{-n-1} + n\,B_n r^{n-1}\bigg)\cos(n\theta) + \bigg(-n\,C_n r^{-n-1} + n\,D_n r^{n-1}\bigg)\sin(n\theta) \bigg] \]
		\[\pard{}{r}u(a,\theta) = \frac{A_0}{a} + \sum_{n = 1}^{\infty}\bigg[ \bigg(-n\,A_n a^{-n-1} + n\,B_n a^{n-1}\bigg)\cos(n\theta) + \bigg(-n\,C_n a^{-n-1} + n\,D_n a^{n-1}\bigg)\sin(n\theta) \bigg] = 0 \]
		Because $\cos(n\theta)$ and $\sin(n\theta)$ oscillate, it is impossible to solve this inequality unless we set their coefficients to 0.  Then we get the following:
		\[A_0 = 0\]
		\[nA_na^{-n-1} = nB_na^{n-1} \qarrow A_n = B_na^{2n} \qquad nC_na^{-n-1} = nD_na^{n-1} \qarrow C_n = D_na^{2n}\]
		Now we can include our nonhomogeneous BC:
		\begin{align*}
			u(b,\theta) = g(\theta) &= A_0\ln\,b + B_0 + \sum_{n = 1}^{\infty}\bigg[ \bigg(A_n b^{-n} + B_n b^{n}\bigg)\cos(n\theta) + \bigg(C_n b^{-n} + D_n b^{n}\bigg)\sin(n\theta) \bigg] \\
			&= B_0 + \sum_{n = 1}^{\infty}\bigg[ \bigg(B_na^{2n} b^{-n} + B_n b^{n}\bigg)\cos(n\theta) + \bigg(D_na^{2n} b^{-n} + D_n b^{n}\bigg)\sin(n\theta) \bigg] \\
			&= B_0 + \sum_{n = 1}^{\infty} B_n\bigg(a^{2n} b^{-n} + b^{n}\bigg)\cos(n\theta) + \sum_{n = 1}^{\infty} D_n\bigg(a^{2n} b^{-n} + b^{n}\bigg)\sin(n\theta)  
		\end{align*}
		Using the orthogonality of cosines and sines, we get:
		\[B_0 = \frac{1}{2\pi} \int_{-\pi}^{\pi} g(\theta)\,d\theta \]
		\[B_n = \frac{1}{\pi(a^{2n} b^{-n} + b^{n})}\int_{-\pi}^{\pi} g(\theta)\cos(n\theta)\,d\theta \qquad D_n = \frac{1}{\pi(a^{2n} b^{-n} + b^{n})}\int_{-\pi}^{\pi} g(\theta)\sin(n\theta)\,d\theta\]
		Thus we get our desired solution:
		\begin{align*}
			\boldsymbol{u(r,\theta)} &\boldsymbol{ = \frac{1}{2\pi} \int_{-\pi}^{\pi} g(\theta)\,d\theta} \\
			&\boldsymbol{+ \sum_{n = 1}^{\infty} \bigg(\frac{1}{\pi(a^{2n} b^{-n} + b^{n})}\int_{-\pi}^{\pi} g(\theta)\cos(n\theta)\,d\theta\bigg)\bigg(a^{2n} b^{-n} + b^{n}\bigg)\cos(n\theta)} \\
			&\boldsymbol{+ \sum_{n = 1}^{\infty} \bigg(\frac{1}{\pi(a^{2n} b^{-n} + b^{n})}\int_{-\pi}^{\pi} g(\theta)\sin(n\theta)\,d\theta\bigg)\bigg(a^{2n} b^{-n} + b^{n}\bigg)\sin(n\theta)} 
		\end{align*}
	\end{problem}
	
	\begin{problem}{2.5.15b}
		Solve Laplace’s equation inside a semi-infinite strip $(0 < x < \infty, 0 < y < H)$ subject to
		the boundary conditions [Hint: In Cartesian coordinates, $\nabla^2 u = \pard{^2u}{x^2} + \pard{^2u}{y^2} = 0$, inside a semi infinite strip $(0 \leq y \leq H \text{ and } 0 \leq x \leq \infty)$, it is known that if $u(x,y) = F(x)G(y)$, then $\frac{1}{F}\frac{d^2F}{dx^2} = -\frac{1}{G}\frac{d^2G}{dy^2}.$]:
		\[u(x,0) = 0 \qquad u(x,H) = 0, \qquad u(0,y) = f(y)\]
		Let the following be true:
		\[\pard{^2u}{x^2} + \pard{^2u}{y^2} = 0 \qquad 0 \leq x \leq \infty, \qquad 0 \leq y \leq H\]
		with 
		\[u(x,y) = h(x)\phi(y), \qquad \phi(0) = 0, \quad \phi(H) = 0, \quad h(0) = f(y)\]
		Taking Laplace's Equation, we get the following:
		\[\phi''(y)h(x) + \phi(y)h(x)'' = 0\]
		\[\frac{\phi''(y)}{\phi(y)} = -\frac{h''(x)}{h(x)} = -\lambda\]
		From this, we get the following:
		\[\phi'' + \lambda \phi = 0, \qquad h'' - \lambda h = 0\]
		\newpage
		From this, we can see this is an eigenvalue problem:
		\begin{enumerate}[label = (\alph*)]
			\item $(\lambda = 0)$:
			\[\phi'' = 0 \qqarrow \phi' = c_1 \qqarrow \phi = c_1y + c_2\]
			Substituting in our BC's, we get:
			\[\phi(0) = c_2 = 0 = c_1H + c_2 = \phi(H) \]
			So now we have our first eigenfunction:
			\[\phi(y) = 0 \text{ with } \lambda = 0\]
			From here, we get the following trivial solution:
			\[\boldsymbol{u(x,y) = 0}\]
			\item $(\lambda < 0)$:
			\[\phi'' - |\lambda|\phi = 0\]
			Using the characteristic equation, we get:
			\[\phi(y) = c_1\cosh(\sqrt{|\lambda|}y) + c_2\sinh(\sqrt{|\lambda|}y)  \]
			Using the BC's, we get:
			\[\phi(0) = c_1 = 0 \qquad \phi(H) = c_2\sinh(\sqrt{|\lambda|}H) = 0 \qqarrow c_2 = 0\]
			Thus we get the following trivial solution:
			\[\boldsymbol{u(x,y) = 0}\]
			\item $(\lambda > 0)$:
			\[\phi'' + \lambda\phi = 0\]
			Using the characteristic equation, we get:
			\[\phi(y) = c_1\cos(\sqrt{\lambda}y) + c_2\sin(\sqrt{\lambda}y)  \]
			Using the BC's, we get:
			\[\phi(0) = c_1 = 0 \qquad \phi(H) = c_2\sin(\sqrt{\lambda}H) = 0\]
			\begin{enumerate}[label = (\roman*)]
				\item $(c_2 = 0)$:
				\[\phi(x) = 0\]
				Thus we get the following trivial solution:
				\[\boldsymbol{u(x,y) = 0}\]
				\item $(\sqrt{\lambda}\sin(\sqrt{\lambda}H) = 0)$:
				\[\sin(\sqrt{\lambda}H) = 0 \qqarrow \sqrt{\lambda}H = n\pi \qqarrow \lambda = \frac{n^2\pi^2}{H^2}\]
				So now we have our $n$ eigenfunctions:
				\[\phi_n(y) = c_2\sin\left(\frac{n\pi y}{H}\right)\]
				we can now substitute our eigenvalues into the other ODE, and we get:
				\[h'' - \frac{n^2\pi^2}{H^2} h = 0\]
				When solving this ODE, we get the following linear independent solutions:
				\[h_n(x) = d_1e^{-\frac{n\pi x}{H}} + d_2e^{\frac{n\pi x}{H}}\]
				By the boundedness condition, we set $d_2 = 0$, so that as $x$ grows, the function is still bounded.
				\[h_n(x) = d_1e^{-\frac{n\pi x}{H}}\]
				From here, we get the following $n$ product solutions:
				\[\boldsymbol{u_n(x,y) = B_n\sin\left(\frac{n\pi y}{H}\right)e^{-\frac{n\pi x}{H}}} \] 
			\end{enumerate}		
		\end{enumerate}
		\skipline
		By the Principle of Superposition, we get the following:
		\begin{align*}
			u(x,y) &= u_0(x,y) + u_1(x,y) + ... + u_n(x,y) \\
			&= \sum_{n=1}^{\infty} B_n\sin\left(\frac{n\pi y}{H}\right)e^{-\frac{n\pi x}{H}}
		\end{align*}
		We can now include our nonhomogeneous solution and get the following:
		\[u(0,y) = f(y) = \sum_{n=1}^{\infty} B_n\sin\left(\frac{n\pi y}{H}\right)\]
		Using the orthogonality of sines, we get:
		\[B_n = \frac{2}{H} \int_{0}^{H} f(y)\sin\left(\frac{n\pi y}{H}\right)\,dy  \]
		Thus we get our desired solution:
		\[\boldsymbol{u(x,y) = \sum_{n=1}^{\infty} \left[\frac{2}{H} \int_{0}^{H} f(y)\sin\left(\frac{n\pi y}{H}\right)\,dy\right]\sin\left(\frac{n\pi y}{H}\right)e^{-\frac{n\pi x}{H}}}\]
	\end{problem}


\end{document}
