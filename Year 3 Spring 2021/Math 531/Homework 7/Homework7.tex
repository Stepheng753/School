\documentclass[11pt]{article}
\usepackage[margin = 1in]{geometry}
\usepackage{amsmath}
\usepackage{amssymb}
\usepackage{amsthm}
\usepackage{graphicx}
\usepackage{enumitem}
\usepackage{url}
\usepackage[parfill]{parskip}
\usepackage{listings}
\usepackage{caption}
\usepackage{subcaption}
\usepackage[utf8]{inputenc}
\usepackage{xcolor}
\definecolor{codegreen}{rgb}{0,0.6,0}
\definecolor{codegray}{rgb}{0.5,0.5,0.5}
\definecolor{codepurple}{rgb}{0.58,0,0.82}
\definecolor{backcolour}{rgb}{0.95,0.95,0.92}
\lstdefinestyle{mystyle}{
	backgroundcolor=\color{backcolour},   
	commentstyle=\color{codegreen},
	keywordstyle=\color{magenta},
	numberstyle=\tiny\color{codegray},
	stringstyle=\color{codepurple},
	basicstyle=\ttfamily\footnotesize,
	breakatwhitespace=false,         
	breaklines=true,                 
	captionpos=b,                    
	keepspaces=true,                 
	numbers=left,                    
	numbersep=5pt,                  
	showspaces=false,                
	showstringspaces=false,
	showtabs=false,                  
	tabsize=2
}
\lstset{style=mystyle}
\newcommand{\skipline}{\vspace{\baselineskip}}
\newcommand{\spacer}{\noalign{\medskip}}
\newcommand{~}{\sim}
\newcommand{\approches}{\rightarrow}
\newcommand{\qcomma}{, \quad}
\newcommand{\qqcomma}{, \qquad}
\newcommand{\qarrow}{\quad \rightarrow \quad}
\newcommand{\qqarrow}{\qquad \rightarrow \qquad}
\newcommand{\qtext}[1]{\quad \text{ #1 } \quad}
\newcommand{\qqtext}[1]{\qquad \text{ #1 } \qquad}
\newcommand{\pard}[2]{\frac{\partial #1}{\partial #2}}
\newcommand{\answer}[1]{\textbf{\boldmath #1}}
\newenvironment{problem}[1]{\textbf{Excersise #1: }}{\newpage}

\begin{document}
	
	\begin{center}
		\textbf{Homework 7} \\
		\textbf{Partial Differential Equations} \\
		\textbf{Math 531} \\
		\textbf{Stephen Giang RedID: 823184070} \\
		\skipline \skipline
	\end{center}

	\begin{problem}{7.3.1d}
		Consider the heat equation in a two-dimensional rectangular region
		$0 < x < L, 0 < y < H$,
		\[\pard{u}{t} = k\left(\pard{^2u}{x^2} + \pard{^2u}{y^2}\right)\]
		subject to the initial condition
		\[u(x,y,0) = \alpha(x,y).\]
		[Hint: You many assume without derivation that product solutions $u(x, y, t) =
		\phi(x, y)h(t) = f(x)g(y)h(t)$ satisfy $\frac{dh}{dt} = -\lambda kh$, the two-dimensional eigenvalue problem $\nabla^2\phi + \lambda\phi = 0$ with further separation
		\[\frac{d^2f}{dx^2} = -\mu f \qqcomma \frac{d^2g}{dy^2} + (\lambda - \mu)g = 0,\]
		or you may use results of the two-dimensional eigenvalue problem.]
		Solve the initial value problem and analyze the temperature as $t \approches \infty$ if the boundary	conditions are
		\[u(0,y,t) = 0, \qquad \pard{u}{x}(L,y,t) = 0, \qquad \pard{u}{y}(x,0,t) = 0, \qquad \pard{u}{y}(x,H,t) = 0\]
		\newpage
		Let the following be true:
		\[u(x,y,t) = f(x)g(y)h(t) \qtext{with} f(0) = 0 \qcomma f'(L) = 0 \qcomma g'(0) = 0 \qcomma g'(H) = 0 \]
		Also let the following be true throughout the rest of this assignment:
		\[n,m,\ell \in \mathbb{Z}^+\]
		Notice the following ODE, and different values for $\mu$:
		\[\frac{d^2f}{dx^2} = -\mu f\]
		($\mu < 0$):
		\[f(x) = c_1\cosh(\sqrt{\mu}x) + c_2\sinh(\sqrt{\mu}x) \qcomma f'(x) = c_1\sqrt{\mu}\sinh(\sqrt{\mu}x) + c_2\sqrt{\mu}\cosh(\sqrt{\mu}x)\]
		Using the boundary conditions, we get:
		\[f(0) = c_1 = 0 \qqarrow f'(L) = c_2\sqrt{\mu}\cosh(\sqrt{\mu}L) \qarrow c_2 = 0\]
		Thus, we get the trivial solution:
		\[f(x) = 0 \qqarrow u(x,y,t) = 0\]
		($\mu = 0$):
		\[f''(x) = 0 \qcomma f'(x) = c_1 \qcomma f(x) = c_1x + c_2\]
		Using the boundary conditions, we get:
		\[f(0) = c_2 = 0 \qqarrow f'(L) = c_1 = 0\]
		Thus, we get the trivial solution:
		\[f(x) = 0 \qqarrow u(x,y,t) = 0\]
		($\mu > 0$):
		\[f(x) = c_1\cos(\sqrt{\mu}x) + c_2\sin(\sqrt{\mu}x) \qcomma f'(x) = -c_1\sqrt{\mu}\sin(\sqrt{\mu}x) + c_2\sqrt{\mu}\cos(\sqrt{\mu}x)\]
		Using the boundary conditions, we get:
		\[f(0) = c_1 = 0 \qqarrow f'(L) = c_2\sqrt{\mu}\cos(\sqrt{\mu}L)\]
		If we let $c_2 = 0$, we get the trivial solution. Notice the other condition to allow a non-trivial solution:
		\[\cos(\sqrt{\mu}L) = 0 \qarrow \sqrt{\mu}L = \frac{(2n + 1)\pi}{2} \qarrow \mu = \left( \frac{(2n + 1)\pi}{2L} \right)^2 \]
		Thus, we get this ODE's $n$ eigenfunctions and $n$ eigenvalues:
		\[\mu_n = \left( \frac{(2n + 1)\pi}{2L} \right)^2 \qquad f_n(x) = \sin \left( \frac{(2n + 1)\pi x}{2L} \right)\]
		\newpage
		Notice the following ODE:
		\[\frac{d^2g}{dy^2} + (\lambda - \mu)g = 0 \qqarrow \frac{d^2g}{dy^2} = - (\lambda - \mu)g \]
		We can see that this is similar to our previous ODE, so we can infer that when $\lambda - \mu \leq 0$, we get the trivial solution, so notice the following:
		($\lambda > \mu$):
		\[g(y) = d_1\cos(\sqrt{\lambda - \mu}y) + d_2\sin(\sqrt{\lambda - \mu}y) \qquad g'(y) = -d_1\sqrt{\lambda - \mu}\sin(\sqrt{\lambda - \mu}y) + d_2\sqrt{\lambda - \mu}\cos(\sqrt{\lambda - \mu}y)\]
		Using the boundary conditions, we get:
		\[g'(0) = d_2\sqrt{\lambda - \nu} = 0 \qarrow d_2 = 0 \qqarrow g'(H) = -d_1\sqrt{\lambda - \mu}\sin(\sqrt{\lambda - \mu}H)\]
		If we let $d_1 = 0$, we get the trivial solution. Notice the other condition to allow a non-trivial solution:
		\[\sin(\sqrt{\lambda - \mu}H) = 0 \qarrow \sqrt{\lambda - \mu}H = m\pi \qarrow \lambda - \mu = \left( \frac{m\pi}{H}\right)^2\]
		Using the result from the previous ODE, we get this ODE's $mn$ eigenfunctions and $mn$ eigenvalues:
		\[\lambda_m - \mu_n = \left( \frac{m\pi}{H}\right)^2 \qquad g_{mn}(y) = \cos\left( \frac{m\pi y}{H}\right) \]
		\\ \\
		Notice the eigenvalues and eigenfunctions for $\phi(x,y)$:
		\[\lambda_{mn} = \left( \frac{m\pi}{H}\right)^2 + \mu_n = \left( \frac{m\pi}{H}\right)^2 + \left( \frac{(2n + 1)\pi}{2L} \right)^2 \qquad \phi_{mn}(x,y) = \sin \left( \frac{(2n + 1)\pi x}{2L} \right)\cos\left( \frac{m\pi y}{H}\right)\]
		\\ \\
		Notice the following ODE:
		\[\frac{dh}{dt} = -\lambda kh\]
		We get the following solution:
		\[h_{mn}(t) = Ce^{-\lambda_{mn} kt}\]
		\\
		From here, we get the following solution for $u(x,y,t)$ using the Principle of Superposition:
		\[\boldsymbol{u(x,y,t) = \sum_{m = 1}^{\infty}\sum_{n = 1}^{\infty} A_{mn}\sin \left( \frac{(2n + 1)\pi x}{2L} \right)\cos\left( \frac{m\pi y}{H}\right)e^{-\lambda_{mn} kt}}\]
		where $\lambda_{mn}$ are the eigenvalues of the spatial component.
		\\ \\
		Now we solve for the coefficients using the initial condition:
		\[u(x,y,0) = \alpha(x,y) = \sum_{m = 1}^{\infty}\sum_{n = 1}^{\infty} A_{mn}\sin \left( \frac{(2n + 1)\pi x}{2L} \right)\cos\left( \frac{m\pi y}{H}\right)\]
		Using the orthogonality of sines and cosines, we get:
		\[\boldsymbol{A_{mn} = \frac{4}{LH}\int_{0}^{L}\int_{0}^H \alpha(x,y)\sin \left( \frac{(2n + 1)\pi x}{2L} \right)\cos\left( \frac{m\pi y}{H}\right)\,dy\,dx}\]
	\end{problem}

	\begin{problem}{7.3.2a}
		Consider the heat equation in a three-dimensional box-shaped region, $0 < x < L, 0< y < H, 0 < z < W$,
		\[\pard{u}{t} = k\left(\pard{^2u}{x^2} + \pard{^2u}{y^2} + \pard{^2u}{z^2}\right)\]
		subject to the initial condition
		\[u(x,y,z,0) = \alpha(x,y,z)\]
		[Hint: You many assume without derivation that the product solutions $u(x, y, z, t) = \phi(x, y, z)h(t)$ satisfy $\frac{dh}{dt} = -\lambda kh$ and satisfy the three-dimensional eigenvalue problem $\nabla^2\phi + \lambda\phi = 0$, or you may use results of the three-dimensional eigenvalue problem.]
		Solve the initial value problem and analyze the temperature as $t \approches \infty$ if the boundary	conditions are
		\[u(0,y,z,t) = 0, \qquad \pard{u}{y}(x,0,z,t) = 0, \qquad \pard{u}{z}(x,y,0,t) = 0,\]
		\[u(L,y,z,t) = 0, \qquad \pard{u}{y}(x,H,z,t) = 0, \qquad u(x,y,W,t) = 0\]
		\newpage
		Let the following be true:
		\[u(x,y,z,t) = \phi(x,y,z)h(t) = f(x)g(y)q(z)h(t)\]
		with the following boundary conditions:
		\[f(0) = 0 \qcomma f(L) = 0 \qcomma g'(0) = 0 \qcomma g'(H) = 0 \qcomma q'(0) = 0 \qcomma q(W) = 0\]
		Now we notice the following ODE's:
		\[f'' = -\mu f \qquad g'' = -\nu g \qquad q'' + (\lambda - \mu - \nu) = 0 \qquad h' = -\lambda kh \]
		To avoid the trivial solution, we will only notice the ODE's when the following is true:
		\[\mu > 0 \qquad \nu > 0 \qquad \lambda > \mu + \nu\]
		Thus we get the following:
		\[f(x) = c_1\cos(\sqrt{\mu}x) + c_2\sin(\sqrt{\mu}x) \qcomma f'(x) = -c_1\sqrt{\mu}\sin(\sqrt{\mu}x) + c_2\sqrt{\mu}\cos(\sqrt{\mu}x)\]
		\[g(y) = d_1\cos(\sqrt{\nu}y) + d_2\sin(\sqrt{\nu}y) \qquad g'(y) = -d_1\sqrt{\nu}\sin(\sqrt{\nu}y) + d_2\sqrt{\nu}\cos(\sqrt{\nu}y)\]
		\[q(z) = b_1\cos(\sqrt{\lambda - \mu - \nu}z) + b_2\sin(\sqrt{\lambda - \mu - \nu}z)\]
		\[q'(y) = -b_1\sqrt{\lambda - \mu - \nu}\sin(\sqrt{\lambda - \mu - \nu}z) + b_2\sqrt{\lambda - \mu - \nu}\cos(\sqrt{\lambda - \mu - \nu}z)\]
		Now we use our boundary conditions, to get the following:
		\[f(0) = c_1 = 0 \qquad f(L) = c_2\sin(\sqrt{\mu}L) = 0 \qquad \sqrt{\mu}L = n\pi \qquad \mu = \left( \frac{n\pi}{L}\right)^2 \]
		\[g'(0) = d_2\sqrt{\nu} = 0 \qquad d_2 = 0 \qquad g'(H) = -d_1\sqrt{\nu}\sin(\sqrt{\nu}H) = 0 \qquad \sqrt{\nu}H = m\pi \qquad \nu = \left( \frac{m\pi}{H}\right)^2 \]
		\[q'(0) =  b_2\sqrt{\lambda - \mu - \nu} = 0 \qquad b_2 = 0\]
		\[q(W) = b_1\cos(\sqrt{\lambda - \mu - \nu}W) \qquad \sqrt{\lambda - \mu - \nu}W = \frac{(2\ell + 1)\pi}{2} \qquad \lambda_\ell - \mu_n - \nu_m = \left( \frac{(2\ell + 1)\pi}{2W} \right)^2 \]
		So thus we get the following eigenvalues and eigenfunctions:
		\[\mu_n = \left( \frac{n\pi}{L}\right)^2 \qquad f_n(x) = \sin\left( \frac{n\pi x}{L}\right) \]
		\[\nu_m = \left( \frac{m\pi}{H}\right)^2 \qquad g_m(x) = \cos\left( \frac{m\pi y}{H}\right) \]
		\[\lambda_\ell - \mu_n - \nu_m = \left( \frac{(2\ell + 1)\pi}{2W} \right)^2 \qquad q_\ell(x) = \cos\left( \frac{(2\ell + 1)\pi z}{2W}\right) \]
		Thus we get the eigenvalues and eigenfunctions for $\phi(x,y,z)$
		\[\lambda_{mn\ell} = \mu_n + \nu_m + \left( \frac{(2\ell + 1)\pi}{2W} \right)^2 =  \left( \frac{n\pi}{L}\right)^2 + \left( \frac{m\pi}{H}\right)^2 + \left( \frac{(2\ell + 1)\pi}{2W}\right) \]
		\[\phi(x,y,z) = \sin\left( \frac{n\pi x}{L}\right)\cos\left( \frac{m\pi y}{H}\right)\cos\left( \frac{(2\ell + 1)\pi z}{2W}\right)\]
		\newpage
		Now we can solve for the time dependent ODE:
		\[h' = -\lambda kh \qqarrow h(t) = Ce^{-\lambda k t}\]
		From here, we get the following solution for $u(x,y,z,t)$ using the Principle of Superposition:
		\[\boldsymbol{u(x,y,z,t) = \sum_{m = 1}^{\infty}\sum_{n = 1}^{\infty}\sum_{\ell = 1}^{\infty} A_{mn\ell}\sin\left( \frac{n\pi x}{L}\right)\cos\left( \frac{m\pi y}{H}\right)\cos\left( \frac{(2\ell + 1)\pi z}{2W}\right)e^{-\lambda_{mn\ell} k t}}\]
		where $\lambda_{mn\ell}$ are the eigenvalues of the spatial component.
		\\ \\
		Now we solve for the coefficients using the initial condition:
		\[u(x,y,z,0) = \alpha(x,y,z) = \sum_{m = 1}^{\infty}\sum_{n = 1}^{\infty}\sum_{\ell = 1}^{\infty} A_{mn\ell}\sin\left( \frac{n\pi x}{L}\right)\cos\left( \frac{m\pi y}{H}\right)\cos\left( \frac{(2\ell + 1)\pi z}{2W}\right)\]
		Using the orthogonality of sines and cosines, we get:
		\[\boldsymbol{A_{mn} = \frac{8}{LHW}\int_{0}^{L}\int_{0}^H\int_{0}^W \alpha(x,y)\sin\left( \frac{n\pi x}{L}\right)\cos\left( \frac{m\pi y}{H}\right)\cos\left( \frac{(2\ell + 1)\pi z}{2W}\right)\,dz\,dy\,dx}\]
	\end{problem}

	\begin{problem}{7.3.4a}
		Consider the wave equation for a vibrating rectangular membrane $(0 < x < L, 0 <
		y<H)$
		\[\pard{^2u}{t^2} = c^2\left(\pard{^2u}{x^2} + \pard{^2u}{y^2}\right)\]
		subject to the initial conditions
		\[u(x,y,0) = 0 \qqtext{and} \pard{u}{t}(x,y,0) = \alpha(x,y)\]
		[Hint: You many assume without derivation that the product solutions $u(x, y, t) =
		\phi(x, y)h(t)$ satisfy $\frac{d^2h}{dt^2} = -\lambda c^2 h$ and the two-dimensional eigenvalue problem $\nabla^2\phi + \lambda\phi = 0$, and you may use results of the two-dimensional eigenvalue problem.]
		\\ \\
		Solve the initial value problem if
		\[u(0,y,t) = 0, \qquad u(L,y,t) = 0, \qquad \pard{u}{y}(x,0,t) = 0, \qquad \pard{u}{y}(x,H,t) = 0\]
		\newpage
		Let the following be true:
		\[u(x,y,t) = \phi(x,y)h(t) = f(x)g(y)h(t)\]
		with the following boundary conditions:
		\[f(0) = 0 \qcomma f(L) = 0 \qcomma g'(0) = 0 \qcomma g'(H) = 0\]
		Now we notice the following ODE's:
		\[f'' = -\mu f \qquad g'' + (\lambda - \mu)g = 0 \qquad h'' = -\lambda c^2h \]
		To avoid the trivial solution, we will only notice the ODE's when the following is true:
		\[\mu > 0 \qquad \lambda > \mu \]
		Thus we get the following:
		\[f(x) = c_1\cos(\sqrt{\mu}x) + c_2\sin(\sqrt{\mu}x) \qcomma f'(x) = -c_1\sqrt{\mu}\sin(\sqrt{\mu}x) + c_2\sqrt{\mu}\cos(\sqrt{\mu}x)\]
		\[g(y) = d_1\cos(\sqrt{\lambda - \mu}y) + d_2\sin(\sqrt{\lambda - \mu}y) \qquad g'(y) = -d_1\sqrt{\lambda - \mu}\sin(\sqrt{\lambda - \mu}y) + d_2\sqrt{\lambda - \mu}\cos(\sqrt{\lambda - \mu}y)\]
		Now we use out boundary conditions to get the following:
		\[f(0) = c_1 = 0 \qquad f(L) = c_2\sin(\sqrt{\mu}L) = 0 \qquad \sqrt{\mu}L = n\pi \qquad \mu = \left( \frac{n\pi}{L} \right)^2\]
		\[g'(0) =  d_2\sqrt{\lambda - \mu} \qquad d_2 = 0 \qquad g'(H) = -d_1\sqrt{\lambda - \mu}\sin(\sqrt{\lambda - \mu}H) \qquad \sqrt{\lambda - \mu}H = m\pi \qquad \lambda - \mu = \left( \frac{m\pi}{H} \right)^2\]
		So thus we get the following eigenvalues and eigenfunctions:
		\[\mu_n = \left( \frac{n\pi}{L}\right)^2 \qquad f_n(x) = \sin\left( \frac{n\pi x}{L}\right) \]
		\[\lambda_m - \mu_n = \left( \frac{m\pi}{H} \right)^2 \qquad g_m(y) = \cos\left( \frac{m\pi y}{H}\right) \]
		Thus we get the eigenvalues and eigenfunctions for $\phi(x,y)$:
		\[\lambda_{mn} = \left( \frac{m\pi}{H} \right)^2 + \mu_n = \left( \frac{m\pi}{H} \right)^2 + \left( \frac{n\pi}{L}\right)^2 \qquad \phi(x,y) = \sin\left( \frac{n\pi x}{L}\right)\cos\left( \frac{m\pi y}{H}\right) \]
		\newpage
		Now we can solve for the time dependent ODE:
		\[h'' = -\lambda c^2h\]
		Notice that we solved for $\lambda = \lambda_{mn} > 0$:
		\[h(t) = b_1\cos(\sqrt{-\lambda c^2}t) + b_2\sin(\sqrt{-\lambda c^2}t) \qarrow h'(t) = -b_1\sqrt{-\lambda c^2}\sin(\sqrt{-\lambda c^2}t) + b_2\sqrt{-\lambda c^2}\cos(\sqrt{-\lambda c^2}t)\]
		From here, we get the following solution for $u(x,y,t)$ using the Principle of Superposition:
		\begin{align*}
			\boldsymbol{u(x,y,t)} &\boldsymbol{= \sum_{m = 1}^{\infty}\sum_{n = 1}^{\infty} A_{mn}\sin\left( \frac{n\pi x}{L}\right)\cos\left( \frac{m\pi y}{H}\right)\cos(\sqrt{-\lambda c^2}t) } \\
			&\boldsymbol{+ B_{mn}\sin\left( \frac{n\pi x}{L}\right)\cos\left( \frac{m\pi y}{H}\right)\sin(\sqrt{-\lambda c^2}t)}
		\end{align*}
		\begin{align*}
			\boldsymbol{\pard{u}{t}(x,y,t)} &\boldsymbol{= \sum_{m = 1}^{\infty}\sum_{n = 1}^{\infty} -A_{mn}\sqrt{-\lambda c^2}\sin\left( \frac{n\pi x}{L}\right)\cos\left( \frac{m\pi y}{H}\right)\sin(\sqrt{-\lambda c^2}t) } \\
			&\boldsymbol{+ B_{mn}\sqrt{-\lambda c^2}\sin\left( \frac{n\pi x}{L}\right)\cos\left( \frac{m\pi y}{H}\right)\cos(\sqrt{-\lambda c^2}t)}
		\end{align*}
		where $\lambda_{mn}$ are the eigenvalues of the spatial component.
		\\ \\
		Now we solve for the coefficients using the initial conditions:
		\begin{align*}
			\boldsymbol{u(x,y,0) = 0} &\boldsymbol{= \sum_{m = 1}^{\infty}\sum_{n = 1}^{\infty} A_{mn}\sin\left( \frac{n\pi x}{L}\right)\cos\left( \frac{m\pi y}{H}\right)} \\
			\boldsymbol{A_{mn}} & \boldsymbol{= 0}
		\end{align*}
		\begin{align*}
			\boldsymbol{\pard{u}{t}(x,y,0) = \alpha(x,y)} &\boldsymbol{= \sum_{m = 1}^{\infty}\sum_{n = 1}^{\infty} B_{mn}\sqrt{-\lambda c^2}\sin\left( \frac{n\pi x}{L}\right)\cos\left( \frac{m\pi y}{H}\right) } \\
			\boldsymbol{B_{mn}} & \boldsymbol{= \frac{4}{LH}\int_{0}^L \int_{0}^H \sin\left( \frac{n\pi x}{L}\right)\cos\left( \frac{m\pi y}{H}\right)\,dy\,dx}
		\end{align*}
	\end{problem}

	\begin{problem}{7.3.5}
		Consider
		\[\pard{^2u}{t^2} = c^2\left(\pard{^2u}{x^2} + \pard{^2u}{y^2}\right) - k\pard{u}{t} \qqtext{with} k > 0.\]
		\begin{enumerate}[label = (\alph*)]
			\item Give a \textit{brief} physical interpretation of this equation.
			\\ \\
			This is a 2 dimensional vibrating membrane that is being damped over time. 
			\item Suppose that $u(x, y, t) = f(x)g(y)h(t)$. What ordinary differential equations are satisfied by f, g, and h?
			\\ \\
			Notice we can rearrange the given ODE:
			\begin{align*}
				fgh'' &= c^2f''gh + c^2fg''h - kfgh' \\
				fg(h'' -kh') &= h(c^2f''g + c^2fg'') \\
				\frac{1}{h}(h'' - kh') &= c^2\frac{f''}{f} + c^2\frac{g''}{g} = -\lambda
			\end{align*}
			Thus we get the following ODE's:
			\[\boldsymbol{h'' - kh' + \lambda h = 0 \qquad f'' = -\mu f \qquad g'' + \left( \frac{\lambda}{c^2} + \mu\right) = 0} \]
		\end{enumerate}
	\end{problem}

	\begin{problem}{7.5.1}
		The vertical displacement of a nonuniform membrane satisfies
		\[\pard{^2u}{t^2} = c^2\left(\pard{^2u}{x^2} + \pard{^2u}{y^2}\right),\]
		where $c$ depends on $x$ and $y$. Suppose that $u = 0$ on the boundary of an irregularly shaped membrane. 
		\begin{enumerate}[label = (\alph*)]
			\item Show that the time variable can be separated by assuming that
			\[u(x,y,t) = \phi(x,y)h(t).\]
			Show that $\phi(x, y)$ satisfies the eigenvalue problem
			\[\nabla^2\phi + \lambda \sigma(x,y)\phi = 0 \qqtext{with} \phi = 0 \qqtext{on the boundary}\]
			What is $\sigma(x, y)$?
			\\ \\
			Notice we can rearrange the given ODE:
			\begin{align*}
				\phi h'' &= c^2\left( \pard{^2\phi}{x^2}h + \pard{^2\phi}{y^2}h\right)  \\
				\frac{h''}{h} &= \frac{c^2}{\phi}\left( \pard{^2\phi}{x^2} + \pard{^2\phi}{y^2}\right) \\
				\frac{h''}{h} &= \frac{c^2}{\phi}\nabla^2\phi = -\lambda
			\end{align*}
			From here we get the following:
			\[\boldsymbol{\nabla^2\phi(x,y) + \frac{\lambda \phi(x,y)}{c^2(x,y)} = \nabla^2\phi(x,y) + \lambda \sigma(x,y)\phi(x,y) = 0 \qquad \sigma = \frac{1}{c^2(x,y)}}\]
			\item If the eigenvalues are known (and $\lambda > 0$), determine the frequencies of vibration.
			\\ \\
			Notice if we know the eigenvalues, we can solve the time dependent ODE:
			\[h'' + \lambda_n h = 0 \qquad h(t) = c_1\cos(\sqrt{\lambda_n}t) + c_2\sin(\sqrt{\lambda_n}t)\]
			Thus we get that the frequencies of vibration is:
			\[\boldsymbol{\sqrt{\lambda_n}}\]
		\end{enumerate}
	\end{problem}


\end{document}
